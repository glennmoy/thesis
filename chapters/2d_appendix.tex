
\section[Deriviation of the Voigt-Reuss-Hill Equations]{Deriviation of the Voigt-Reuss-Hill \break Equations}
\label{sec:vrh_derivation}

%GENERAL HOOKES LAW
In this Appendix, we shall re-derive 
the formulae 
for the elastic properties 
of a planar material
 as a function 
of orientation of the in-plane stress 
as outlined in Ref.~\cite{jones1975mechanics}.
%
The general form of Hooke's law for a non-isotropic material 
is expressed as the linear mapping 
$\varepsilon_{ij}=C_{ijkl}\sigma_{kl}$ 
between the stress tensor $\boldsymbol{\sigma}$ 
and strain tensor $\boldsymbol{\varepsilon}$ 
by a fourth-order stiffness tensor $\boldsymbol{C}$.
%
In cartesian coordinates the stress and strain 
tensors are $3\times3$ matrices, given by 
%
\begin{equation}
\boldsymbol{\sigma}=\left[\begin{array}{ccc}
\sigma_{11} & \sigma_{12} & \sigma_{13} \\
\sigma_{21} & \sigma_{22} & \sigma_{21} \\
\sigma_{31} & \sigma_{32} & \sigma_{33} 
\end{array}
\right]
\quad
\mbox{and}
\quad
\boldsymbol{\varepsilon}=\left[\begin{array}{ccc}
\varepsilon_{11} & \varepsilon_{12} & \varepsilon_{13} \\
\varepsilon_{21} & \varepsilon_{22} & \varepsilon_{21} \\
\varepsilon_{31} & \varepsilon_{32} & \varepsilon_{33} 
\end{array}
\right].
\end{equation}

%
This relation may be transformed 
into a rotated reference frame 
via a rotational matrix $\boldsymbol{R}$ 
applied to each of the tensors.
%
If we consider rotations about the z-axis, 
and ignore contributions 
to the stresses in that direction, 
the rotation matrix is 
%
\begin{equation}
\boldsymbol{R_z}(\theta)=\left[\begin{array}{ccc}
\cos{\theta} & -\sin{\theta} & 0 \\
\sin{\theta} & \cos{\theta} & 0 \\
0 & 0&1
\end{array}
\right],
\end{equation}
%
where the transformed stress-tensor 
as a function of $\theta$ is now
%
\begin{equation}
\boldsymbol{\sigma^\prime}(\theta)=\boldsymbol{R_z}(\theta)\boldsymbol{\sigma}\boldsymbol{R_z}^T(\theta),
\end{equation}
%
and likewise for the strain-tensor.
%
% Hooks law for orthotropic materials
For orthotropic materials 
subject to in-plane stress, 
Hooke's law may be re-expressed in Voigt notation, 
where the stiffness tensor is a 
symmetric $3\times3$ matrix given by
%
\begin{equation}
C_{2D}=\left[\begin{array}{ccc}
c_{11} & c_{12}  & 0 \\
c_{12} & c_{22}  & 0 \\
0 & 0  & c_{66}
\end{array}\right]
\quad
\text{with}
\quad
S_{2D}=C_{2D}^{-1},
\label{eq:2dmatrix}
\end{equation} 
where $S$ is the compliance matrix. 
%
Here, the reduced stress and strain vectors are 
%
\begin{equation}
\sigma^T=\left[\sigma_{11},\sigma_{22},\sigma_{12}\right]
\quad\mbox{and}\quad 
\varepsilon^T=\left[\varepsilon_{11},\varepsilon_{22},\varepsilon_{12}\right],
\end{equation}
arising from planar symmetries.
%
The reduced stress-vector 
of a rotated stress-tensor 
relates the new coordinate system 
to the original via a matrix operation  
$\sigma^\prime=\boldsymbol{M}\sigma$, 
with
\begin{equation}
%
\boldsymbol{M}
=
\left[\begin{array}{ccc}
\cos^2\theta& \sin^2\theta & \cos\theta\sin\theta\\
\sin^2\theta& \cos^2\theta&-\cos\theta\sin\theta\\
-2\cos\theta\sin\theta& 2\cos\theta\sin\theta & \cos^2\theta-\sin^2\theta
\end{array}\right],
\end{equation}
%
and similarly for the strain tensor $\boldsymbol{\varepsilon}$.
%
From this, Hooke's law in a rotated frame 
can be expressed in the original basis as 
%
\begin{align}
{\varepsilon^\prime}&=\boldsymbol{C^\prime}{\sigma^\prime}
\Rightarrow
\boldsymbol{M}{\varepsilon}=\boldsymbol{C^\prime}\boldsymbol{M}{\sigma}\\
\Rightarrow
{\varepsilon}&=\boldsymbol{M}^T\boldsymbol{C^\prime}\boldsymbol{M}{\sigma}\ 
\Rightarrow
{\varepsilon}=\boldsymbol{C}{\sigma},
\end{align}
%
with  
$\boldsymbol{C}=\boldsymbol{M}^T\boldsymbol{C}^\prime \boldsymbol{M}$ 
and similarly 
$\boldsymbol{S}=\boldsymbol{M}^T\boldsymbol{S}^\prime \boldsymbol{M}$.
%
Shown below in Eq.~\ref{eq:rot2dmatrix} 
are the elements of the rotated $C_{2D}$, 
where the Reuss equations $S_{ij}\left(\theta\right)$ 
are obtained by 
substituting the elements $c_{ij}\to s_{ij}$:
%
\begin{equation}
\begin{aligned}
C_{11}\left(\theta\right)&=c_{11}\cos^4\theta+c_{22}\sin^4\theta
                +(2c_{12}+c_{66})\cos^2\theta\sin^2\theta\\
C_{22}\left(\theta\right)&=c_{11}\sin^4\theta+c_{22}\cos^4\theta
                +(2c_{12}+c_{66})\cos^2\theta\sin^2\theta\\
C_{66}\left(\theta\right)&=(4c_{11}+4c_{22}-8c_{12}-2c_{66})\cos^2\theta\sin^2\theta+c_{66}(\cos^4\theta+\sin^4\theta)\\
C_{12}\left(\theta\right)&=c_{12}(\cos^4\theta+\sin^4\theta)+(c_{11}+c_{22}-c_{66})\cos^2\theta\sin^2\theta.
\end{aligned}
\label{eq:rot2dmatrix}
\end{equation}
%
From these, 
one can resolve the 
in-plane elastic properties 
as a function of orientation and, 
by integrating over $\theta$, 
the average in-plane 
properties can be extracted
from the fully isotropic tensors.

Finally, 
the above derivation can be 
easily extended to three dimensions.
%
A general rotation matrix 
$\boldsymbol{R}=\boldsymbol{R_1}(\alpha)\boldsymbol{R_2}(\beta)\boldsymbol{R_3}(\gamma)$ 
about the coordinate axes 
can be constructed 
by compounding three rotation matrices, 
parameterised by three angles 
of rotation 
$\alpha$, $\beta$, $\gamma$.
%
The elastic matrices for any rotated reference 
frame may be constructed 
by applying this general transformation  
to the stress and strain tensors.
%
One arrives exactly at the Voigt~\cite{ZAMM:ZAMM19290090104,cook1999advanced} 
and Reuss~\cite{ANDP:ANDP18892741206} 
equations from the which 
the Hill-averages~\cite{0370-1298-65-5-307} 
are considered reliable estimates 
for physical values 
by integrating these matrices over 
each of the angles 
$\alpha$, $\beta$, $\gamma$.
%
Presented here are the equations for the 
Voigt estimate for the Young's $Y_V$ 
and shear modulus $G_V$:
%
\begin{equation}
Y_V=\frac{(A-B+3C)(A+2B)}{2A+3B+C} 
\quad \mbox{and} \quad
G_V=\frac{A-B+3C}{5}, 
\label{eq:voigt1}
\end{equation}
%
where the parameters $A$, $B$ and $C$ 
are expressed in terms of the stiffness tensor elements
%
\begin{equation}
A=\frac{c_{11}+c_{22}+c_{33}}{3},
\quad
B=\frac{c_{23}+c_{13}+c_{12}}{3},
\quad\mbox{and}\quad
C=\frac{c_{44}+c_{55}+c_{66}}{3}.
\label{eq:voigt2}
\end{equation}
%
Similarly, the equations for the 
Reuss estimate for the Young's $Y_R$ 
and shear modulus $G_R$ are given by
%
\begin{equation}
Y_R=\frac{5}{3A^\prime+2B^\prime+C^\prime}
\quad \mbox{and} \quad
G_R = \frac{5}{4A^\prime-4B^\prime+3C^\prime},
\label{eq:reuss1}
\end{equation}
%
where the parameters $A^\prime$, $B^\prime$ and $C^\prime$ 
are expressed in terms of the compliance tensor elements
%
\begin{equation}
A^\prime=\frac{s_{11}+s_{22}+s_{33}}{3},
\quad
B^\prime=\frac{s_{23}+s_{13}+s_{12}}{3},
\quad \mbox{and} \quad
C^\prime=\frac{s_{44}+s_{55}+s_{66}}{3}.
\label{eq:reuss2}
\end{equation}
%%%%%%%%%%%%%%%%%%%%%%%%%%%%%%%%%%%%%%%%%%%%%%%%%%

\clearpage
\section{Calculated elastic tensor elements}
\label{sec:calculated_elastic_tensor_elements}

Presented in Table~\ref{table:cmatrix_data} 
are the calculated stiffness tensor elements 
for each of the phases of P, As and Sb with 
available experimental data from
Refs.~\cite{doi:10.1143/JPSJ.60.1612,doi:10.1143/JPSJ.55.1196}
given in parentheses.

\begin{table}[htb!]
\centering
\resizebox{\textwidth}{!}{%
\begin{tabular}{lrrrrrrrrr}
\hline\hline
&$c_{11}$&$c_{22}$&$c_{33}$&$c_{44}$&$c_{55}$&$c_{66}$&$c_{23}$&$c_{13}$&$c_{12}$\\  
\hline
P\textsubscript{mono}   &79.48&25.33& -&-&-&20.86&-&-&16.50\\
P\textsubscript{bi}     &95.16&30.44&-&-&-&28.71&-&-&20.95\\
P\textsubscript{bulk}   &187.92&55.06&67.64&4.55&22.04&64.85&1.52&10.37&42.15\\
\hline
Ref.~\cite{doi:10.1143/JPSJ.60.1612}&178&55.1&53.6&5.5&11.1-21.3&14.5-15.6\\
Ref.~\cite{doi:10.1143/JPSJ.55.1196}&284&80&74&10.8&7.2&59.4\\
\hline
As\textsubscript{mono}  &40.35&8.04&-&-&-&11.31&-&-&11.68\\
As\textsubscript{bi}        &71.25&24.91&-&-&-&23.58&-&-&23.70\\
As\textsubscript{bulk} &134.24&41.34&93.24&11.18&20.13&49.26&15.27&23.37&43.64\\
\hline
Sb\textsubscript{mono}  &31.23&12.93&-&-&-&13.60&-&-&14.53\\
Sb\textsubscript{bi}        &36.78&14.10&-&-&-&19.47&-&-&15.36\\
Sb\textsubscript{bulk}&70.79&29.56&111.52&15.82&22.06&37.63&18.08&26.99&26.47\\
\hline\hline
\end{tabular}
}
\caption{
Elements of stiffness tensor in GPa of each structure.
Experimental values for bulk P from Refs.~\cite{doi:10.1143/JPSJ.60.1612,doi:10.1143/JPSJ.55.1196}
are also presented. 
The stiffness in general increases with the number of layers and decreases as 
we move from P to As to Sb, whereas the anisotropy increases for both cases.
}
\label{table:cmatrix_data}
\end{table}

\clearpage
\section{Group-V in-plane elastic extrema}
\label{sec:2d_profiles_extrema}

\begin{table}[htb!]
\centering
\resizebox{\textwidth}{!}{%
\begin{tabular}{lrrrrr|rrrrr|rrrrr}
\hline\hline
&$Y_{\min}$&$\theta$&$Y_{\max}$&$\theta$&$\left<Y\right>$
&$G_{\min}$&$\theta$&$G_{\max}$&$\theta$&$\left<G\right>$
&$\nu_{\min}$&$\theta$&$\nu_{\max}$&$\theta$&$\left<\nu\right>$\\
\hline
P\textsubscript{mono}   &17.1 & $69^{\degree}$ &72.4 & $0^{\degree}$ &36.4
&20.9& $90^{\degree}$&42.2 &$45^{\degree}$&31.0&0.3& $31^{\degree}$&0.5
&$63^{\degree}$&0.5\\
P\textsubscript{bi}        &20.5 &$75^{\degree}$&85.6 &$0^{\degree}$&44.4
&28.1& $90^{\degree}$&49.2& $45^{\degree}$&37.8
&0.31& $34^{\degree}$&0.5& $68^{\degree}$&0.4\\
P\textsubscript{bulk}       &34.2& $87^{\degree}$&166.4& $0^{\degree}$&85.7
&64.9 &$87^{\degree}$&92.4 &$45^{\degree}$&74.5
&0.2& $36^{\degree}$ &0.5& $75^{\degree}$&0.3\\
\hline
As\textsubscript{mono}  &-2.1 &$90^{\degree}$&30.5& $18^{\degree}$&12.7
&10.4 &$77^{\degree}$&13.8 &$45^{\degree}$&11.2
&0.2 &$39^{\degree}$&0.9 &$90^{\degree}$&0.4\\
As\textsubscript{bi}    &9.7& $90^{\degree}$&56.0 &$0^{\degree}$&31.0
&23.0 &$78^{\degree}$&28.6& $45^{\degree}$&24.3
&0.4 &$40^{\degree}$&0.6& $90^{\degree}$&0.5\\
As\textsubscript{bulk}  &10.9 &$90^{\degree}$&105.1 &$20^{\degree}$ &55.0
&44.4&$73^{\degree}$&51.1& $45^{\degree}$&66.7
&0.3 &$40^{\degree}$&0.7& $90^{\degree}$ &0.4\\
\hline
Sb\textsubscript{mono}&1.4 &$90^{\degree}$&24.4 &$22^{\degree}$&12.6
&8.9 &$45^{\degree}$&13.6 &$90^{\degree}$&9.4
&0.3 &$43^{\degree}$&0.8& $90^{\degree}$&0.6\\
Sb\textsubscript{bi}        &2.6 &$90^{\degree}$&31.6&$33^{\degree}$&16.3
&11.8 &$45^{\degree}$&19.5 &$90^{\degree}$&12.8
&0.2&$42^{\degree}$&0.8& $90^{\degree}$&0.5\\
Sb\textsubscript{bulk}  &12.4 &$90^{\degree}$&58.9& $30^{\degree}$&35.4
&28.1& $59^{\degree}$&28.2& $45^{\degree}$&63.2
&0.2& $41^{\degree}$&0.6& $90^{\degree}$&0.4\\
\hline\hline
\end{tabular}
}
\caption{
Summary of the minima and maxima 
of the Hill-averaged in-plane 
Young's modulus $Y\left(\theta\right)$ (GPa),  
shear modulus $G\left(\theta\right)$ (GPa), 
and Poisson's ratio $\nu\left(\theta\right)$ 
as well as the angle $\theta$ 
with respect to the $\vec{x}$-direction (zigzag) 
at which they occur in degrees,
and their in-plane averages.
}
\label{table:2d_extrema}
\end{table}

%BP EXP DATA
\clearpage
\section{Black phosphorus experimental data}
\label{sec:bp_experimental_data}


\begin{table}[htb!]
\centering
\begin{tabular}{ll}
\hline\hline
Volume fraction (\%) & Young's Modulus (MPa) \\
$0 	\pm0$ 		&$551\pm118$\\
$0.037\pm0.002$ 	&$418\pm152$\\
$0.055\pm0.003$ 	&$629\pm123$\\
$0.073\pm0.004$ 	&$466\pm78$\\
$0.110\pm0.005$	&$635\pm53$\\
\hline
$0.037\pm0.002$	&$632\pm217$\\
$0.055\pm0.003$	&$712\pm85.2$\\
$0.073\pm0.004$ 	&$757\pm120$\\
$0.11\pm0.01$ 		&$824\pm194$\\
$0.183\pm0.01$ 	&$726\pm139$\\
$0.294\pm0.01$ 	&$930\pm135$\\
\hline\hline
\end{tabular}
\caption{
Summary of the experimental data associated with 
the analysis in section~\ref{sec:comparison_to_exp}. 
%
The Young's modulus (MPa) of the BP:PVC 
composite was measured while varying the 
volume fraction ($\%$) of the BP nano-flakes.
%
The data on top were acquired from the small nano-flakes 
with mean-length $L_s=130\pm111$~nm, 
and the data on the bottom were from the large nano-flakes 
with mean-length $L_l=2260\pm1000$~nm. 
}
\label{table:bp_exp_data}
\end{table}


%SUMMARY OF ELECTRONIC PROPERTIES
\clearpage
\section{Summary of electronic properties}
\label{sec:elec_properties}

\begin{table}[htb!]
\centering
\begin{tabular}{lrrrrr}
\hline\hline
&$E_g$ (eV) &$Me_{\Gamma-X}$&$Me_{\Gamma-Y}$&$Mh_{\Gamma-X}$&$Mh_{\Gamma-Y}$\\
\hline
P\textsubscript{mono}	&0.9			&1.25(1)&0.16(1)&2.8(2)&0.14(1)\\
As\textsubscript{mono}	&$0.15^\star$	&1.16(1)&0.26(1)&1.09(1)&0.18(2)\\
Sb\textsubscript{mono}	&$0.2^\star$	&1.10(1)&0.28(1)&1.04(1)&0.19(2)\\
\hline
P\textsubscript{bi}		&0.4			&1.41(1)&0.19(1)&1.21(3)&0.15(2)\\
As\textsubscript{bi}		&0.45		&1.15(1)&0.24(1)&0.94(4)&0.17(2)\\
Sb\textsubscript{bi}		&$0^\dagger$	&1.16(1)&0.39(1)&0.99(4)&0.33(4)\\
\hline
P\textsubscript{bulk}		&$0^\ddagger$	&-&0.37(2)&-&0.21(2)\\
As\textsubscript{bulk}	&$0^\ddagger$	&-&0.36(1)&-&0.30(3)\\
Sb\textsubscript{bulk}	&$0^\ddagger$	&-&-&-&-\\
\hline\hline
\end{tabular}
\caption{ 
Kohn-Sham band gaps (eV), 
indicating the indirect semiconducting ($\star$), 
semi-metallic ($\dagger$) 
and metallic ($\ddagger$) states, 
as well as the charge-carrier 
effective masses ($m_0$) 
for each phase of P, As, Sb.
}
\label{table:bandgaps}
\end{table}

\clearpage
\section{Summary of electronic transitions}
\label{sec:elec_transitions}

\begin{table}[htb!]
\centering
\resizebox{0.75\textwidth}{!}{%
\begin{tabular}{lllr}
\hline\hline
&Transition&Direction&Strain (\%)\\
\hline
P\textsubscript{bi}     &D Gap $\to$ SM\textsuperscript{$\triangle$}
&$xx$, $yy$&-5\\
                    &D Gap $\to$ ID Gap     &$xx$, $yy$&+2\\
P\textsubscript{bulk}       &SM
                    $\to$ D Gap $\to$ ID Gap &$xx$, $yy$&+1$\to$+3\\
                    &{SM$\to$SM\textsuperscript{$\triangledown$}}
&{$xx$, $yy$}&{+2}\\
\hline
As\textsubscript{mono}  &ID $\to$ D Gap     &$xx$&-3\\
                    &ID Gap $ \to$ SM       &$xx$&+2\\
                    &ID Gap $ \to$ SM       &$xx$&+2\\
                    &SM $\to$ SM\textsuperscript{$\triangle$}       &$xx$&+5\\
As\textsubscript{bi}        &D Gap $\to$ ID Gap $\to$ D Gap &$xx$&+2$\to$+3\\
                    &D Gap $\to$ ID Gap     & $yy$&-3,+2\\
As\textsubscript{bulk}  &SM 
                    $\to$ D Gap $\to $ ID Gap&$xx$&0$\to$+3\\
                    &SM  
                    $\to$ D Gap& $yy$&0$\to$+3\\
                    &{SM$\to$SM\textsuperscript{$\triangledown$}}
                    &{$xx$, $yy$}&{+1}\\
\hline
Sb\textsubscript{mono}  &ID Gap $\to$ SM        &$xx$&-2\\
                    &ID Gap $\to$ D Gap $\to$
SM\textsuperscript{$\triangle$}&$xx$&+2$\to$+4\\
                    &ID Gap $\to$ SM        & $yy$&-2\\
Sb\textsubscript{bi}        &SM $\to$ SM\textsuperscript{$\triangle$}   &$xx$&+4\\
                    &Structural Transition  & $yy$&-3\\
                    &SM $\to$ ID Gap        & $yy$&+3\\
\hline\hline
\end{tabular}
}
\caption{
Summary of the  
band gap transitions  
including direct (D); 
indirect (ID); metallic (M);
and semi-metallic (SM), 
in particular those that 
indicate potential Dirac {states} ($\triangle$), 
{Weyl states ($\triangledown$)}, 
and the structural phase-transition.
}
\label{table:transitions}
\end{table}





\clearpage
\section{Band structures figures}
\label{sec:band_structure_figs}

In this section we present 
the band structures of the relaxed 
P, As and Sb discussed in Chapter~\ref{ch:elec_prop_2d_mater}, 
which were calculated without spin-orbit coupling (SOC).
%
Band structures of the strained phases may be 
found in the supplemental material of Ref.~\cite{2053-1583-4-4-045018}.
%
%The primary figure on each page 
%is the band structure 
%of the fully relaxed phase, 
%following which we present the 
%band structures for 
%increasing strain along 
%$\varepsilon_{xx}$, 
%$\varepsilon_{yy}$ 
%and $\varepsilon_{xy}$.
%%
%Beginning with P 
%we first present the monolayer ({Fig.~\ref{fig:monopbs}}), 
%then the bilayer  ({Fig.~\ref{fig:bipbs}}), 
%and finally the bulk  ({Fig.~\ref{fig:bulkpbs}}) 
%band structures, 
%following which we present 
%those for As  ({Figs.~\ref{fig:monoasbs}-\ref{fig:bulkasbs}}) 
%and Sb  ({Figs.~\ref{fig:monosbbs}-\ref{fig:bulksbbs}}) 
%in the same fashion.
%%
%We highlight the potential 
%Dirac or Weyl states in each case, 
%which are recalculated including SOC 
%and presented in Chapter~\ref{ch:elec_prop_2d_mater}.
 
 
\begin{figure}[htb!]
\centering
\begin{subfloat}[Monolayer P]{
\includegraphics[width=0.31\textwidth]{images/p/mono/xx/bs.0.00.pdf}}
\end{subfloat}
\begin{subfloat}[Bilayer P]{
\includegraphics[width=0.31\textwidth]{images/p/bi/xx/bs.0.00.pdf}}
\end{subfloat}
\begin{subfloat}[Bulk P]{
\includegraphics[width=0.31\textwidth]{images/p/bulk/xx/bs.0.00.pdf}}
\end{subfloat}
\\
\begin{subfloat}[Monolayer As]{
\includegraphics[width=0.31\textwidth]{images/as/mono/xx/bs.0.00.pdf}}
\end{subfloat}
\begin{subfloat}[Bilayer As]{
\includegraphics[width=0.31\textwidth]{images/as/bi/xx/bs.0.00.pdf}}
\end{subfloat}
\begin{subfloat}[Bulk As]{
\includegraphics[width=0.31\textwidth]{images/as/bulk/xx/bs.0.00.pdf}}
\end{subfloat}
\\
\begin{subfloat}[Monolayer Sb]{
\includegraphics[width=0.31\textwidth]{images/sb/mono/xx/bs.0.00.pdf}}
\end{subfloat}
\begin{subfloat}[Bilayer Sb]{
\includegraphics[width=0.31\textwidth]{images/sb/bi/xx/bs.0.00.pdf}}
\end{subfloat}
\begin{subfloat}[Bulk Sb]{
\includegraphics[width=0.31\textwidth]{images/sb/bulk/xx/bs.0.00.pdf}}
\end{subfloat}
\label{fig:relaxed_bandstructures}
\caption[Band structures of relaxed group-V phases]
{Band structures of relaxed group-V materials 
in the monolayer, bilayer and bulk phases.}
\end{figure}
 
%
%\clearpage
%\begin{figure}[htb!]
%\caption[Band structures of strained monolayer phosphorus]
%{Band structures of strained {\bf monolayer phosphorus}.}
%\centering
%\includegraphics[width=0.20\textwidth]{images/p/mono/xx/bs.0.00.pdf}
%\\
%\includegraphics[width=0.19\textwidth]{images/p/mono/xx/bs.-0.05.pdf}
%\includegraphics[width=0.19\textwidth]{images/p/mono/xx/bs.-0.04.pdf}
%\includegraphics[width=0.19\textwidth]{images/p/mono/xx/bs.-0.03.pdf}
%\includegraphics[width=0.19\textwidth]{images/p/mono/xx/bs.-0.02.pdf}
%\includegraphics[width=0.19\textwidth]{images/p/mono/xx/bs.-0.01.pdf}
%\\
%\includegraphics[width=0.19\textwidth]{images/p/mono/xx/bs.0.01.pdf}
%\includegraphics[width=0.19\textwidth]{images/p/mono/xx/bs.0.02.pdf}
%\includegraphics[width=0.19\textwidth]{images/p/mono/xx/bs.0.03.pdf}
%\includegraphics[width=0.19\textwidth]{images/p/mono/xx/bs.0.04.pdf}
%\includegraphics[width=0.19\textwidth]{images/p/mono/xx/bs.0.05.pdf}
%\\
%\includegraphics[width=0.19\textwidth]{images/p/mono/yy/bs.-0.05.pdf}
%\includegraphics[width=0.19\textwidth]{images/p/mono/yy/bs.-0.04.pdf}
%\includegraphics[width=0.19\textwidth]{images/p/mono/yy/bs.-0.03.pdf}
%\includegraphics[width=0.19\textwidth]{images/p/mono/yy/bs.-0.02.pdf}
%\includegraphics[width=0.19\textwidth]{images/p/mono/yy/bs.-0.01.pdf}
%\\
%\includegraphics[width=0.19\textwidth]{images/p/mono/yy/bs.0.01.pdf}
%\includegraphics[width=0.19\textwidth]{images/p/mono/yy/bs.0.02.pdf}
%\includegraphics[width=0.19\textwidth]{images/p/mono/yy/bs.0.03.pdf}
%\includegraphics[width=0.19\textwidth]{images/p/mono/yy/bs.0.04.pdf}
%\includegraphics[width=0.19\textwidth]{images/p/mono/yy/bs.0.05.pdf}
%\\
%\includegraphics[width=0.19\textwidth]{images/p/mono/xy/bs.-0.05.pdf}
%\includegraphics[width=0.19\textwidth]{images/p/mono/xy/bs.-0.04.pdf}
%\includegraphics[width=0.19\textwidth]{images/p/mono/xy/bs.-0.03.pdf}
%\includegraphics[width=0.19\textwidth]{images/p/mono/xy/bs.-0.02.pdf}
%\includegraphics[width=0.19\textwidth]{images/p/mono/xy/bs.-0.01.pdf}
%\\
%\includegraphics[width=0.19\textwidth]{images/p/mono/xy/bs.0.01.pdf}
%\includegraphics[width=0.19\textwidth]{images/p/mono/xy/bs.0.02.pdf}
%\includegraphics[width=0.19\textwidth]{images/p/mono/xy/bs.0.03.pdf}
%\includegraphics[width=0.19\textwidth]{images/p/mono/xy/bs.0.04.pdf}
%\includegraphics[width=0.19\textwidth]{images/p/mono/xy/bs.0.05.pdf}
%\label{fig:monopbs}
%\end{figure}
%
%\clearpage
%\begin{figure}[htb!]
%\caption[Band structures of strained bilayer phosphorus]
%{Band structures of strained {\bf bilayer phosphorus} 
%with the predicted $\Gamma$-point Dirac states at 
%$\varepsilon_{xx}=-5\%$ and $\varepsilon_{yy}=-5\%$ 
%compressive strains circled in green.}
%\centering
%%
%\includegraphics[width=0.20\textwidth]{images/p/bi/xx/bs.0.00.pdf}
%\\
%\includegraphics[width=0.19\textwidth]{images/p/bi/xx/bs.-0.05.ed.pdf}
%\includegraphics[width=0.19\textwidth]{images/p/bi/xx/bs.-0.04.pdf}
%\includegraphics[width=0.19\textwidth]{images/p/bi/xx/bs.-0.03.pdf}
%\includegraphics[width=0.19\textwidth]{images/p/bi/xx/bs.-0.02.pdf}
%\includegraphics[width=0.19\textwidth]{images/p/bi/xx/bs.-0.01.pdf}
%\includegraphics[width=0.19\textwidth]{images/p/bi/xx/bs.0.01.pdf}
%\includegraphics[width=0.19\textwidth]{images/p/bi/xx/bs.0.02.pdf}
%\includegraphics[width=0.19\textwidth]{images/p/bi/xx/bs.0.03.pdf}
%\includegraphics[width=0.19\textwidth]{images/p/bi/xx/bs.0.04.pdf}
%\includegraphics[width=0.19\textwidth]{images/p/bi/xx/bs.0.05.pdf}
%\\
%\includegraphics[width=0.19\textwidth]{images/p/bi/yy/bs.-0.05.ed.pdf}
%\includegraphics[width=0.19\textwidth]{images/p/bi/yy/bs.-0.04.pdf}
%\includegraphics[width=0.19\textwidth]{images/p/bi/yy/bs.-0.03.pdf}
%\includegraphics[width=0.19\textwidth]{images/p/bi/yy/bs.-0.02.pdf}
%\includegraphics[width=0.19\textwidth]{images/p/bi/yy/bs.-0.01.pdf}
%\includegraphics[width=0.19\textwidth]{images/p/bi/yy/bs.0.01.pdf}
%\includegraphics[width=0.19\textwidth]{images/p/bi/yy/bs.0.02.pdf}
%\includegraphics[width=0.19\textwidth]{images/p/bi/yy/bs.0.03.pdf}
%\includegraphics[width=0.19\textwidth]{images/p/bi/yy/bs.0.04.pdf}
%\includegraphics[width=0.19\textwidth]{images/p/bi/yy/bs.0.05.pdf}
%\\
%\includegraphics[width=0.19\textwidth]{images/p/bi/xy/bs.-0.05.pdf}
%\includegraphics[width=0.19\textwidth]{images/p/bi/xy/bs.-0.04.pdf}
%\includegraphics[width=0.19\textwidth]{images/p/bi/xy/bs.-0.03.pdf}
%\includegraphics[width=0.19\textwidth]{images/p/bi/xy/bs.-0.02.pdf}
%\includegraphics[width=0.19\textwidth]{images/p/bi/xy/bs.-0.01.pdf}
%\includegraphics[width=0.19\textwidth]{images/p/bi/xy/bs.0.01.pdf}
%\includegraphics[width=0.19\textwidth]{images/p/bi/xy/bs.0.02.pdf}
%\includegraphics[width=0.19\textwidth]{images/p/bi/xy/bs.0.03.pdf}
%\includegraphics[width=0.19\textwidth]{images/p/bi/xy/bs.0.04.pdf}
%\includegraphics[width=0.19\textwidth]{images/p/bi/xy/bs.0.05.pdf}
%\label{fig:bipbs}
%\end{figure}
%
%\clearpage
%\begin{figure}[htb!]
%\caption[Band structures of strained bulk phosphorus]
%{Band structures of strained {\bf bulk phosphorus} 
%with the potential Weyl  {states} at $+2\%$ uniaxial tensile strain 
%circled in green.
%%
%{See Fig.~\ref{fig:p_bulk_highres} for more detail}.}
%\centering
%\includegraphics[width=0.20\textwidth]{images/p/bulk/xx/bs.0.00.pdf}
%\\
%\includegraphics[width=0.19\textwidth]{images/p/bulk/xx/bs.-0.05.pdf}
%\includegraphics[width=0.19\textwidth]{images/p/bulk/xx/bs.-0.04.pdf}
%\includegraphics[width=0.19\textwidth]{images/p/bulk/xx/bs.-0.03.pdf}
%\includegraphics[width=0.19\textwidth]{images/p/bulk/xx/bs.-0.02.pdf}
%\includegraphics[width=0.19\textwidth]{images/p/bulk/xx/bs.-0.01.pdf}
%\includegraphics[width=0.19\textwidth]{images/p/bulk/xx/bs.0.01.pdf}
%\includegraphics[width=0.19\textwidth]{images/p/bulk/xx/bs.0.02.ed.pdf}
%\includegraphics[width=0.19\textwidth]{images/p/bulk/xx/bs.0.03.pdf}
%\includegraphics[width=0.19\textwidth]{images/p/bulk/xx/bs.0.04.pdf}
%\includegraphics[width=0.19\textwidth]{images/p/bulk/xx/bs.0.05.pdf}
%\\
%\includegraphics[width=0.19\textwidth]{images/p/bulk/yy/bs.-0.05.pdf}
%\includegraphics[width=0.19\textwidth]{images/p/bulk/yy/bs.-0.04.pdf}
%\includegraphics[width=0.19\textwidth]{images/p/bulk/yy/bs.-0.03.pdf}
%\includegraphics[width=0.19\textwidth]{images/p/bulk/yy/bs.-0.02.pdf}
%\includegraphics[width=0.19\textwidth]{images/p/bulk/yy/bs.-0.01.pdf}
%\includegraphics[width=0.19\textwidth]{images/p/bulk/yy/bs.0.01.pdf}
%\includegraphics[width=0.19\textwidth]{images/p/bulk/yy/bs.0.02.ed.pdf}
%\includegraphics[width=0.19\textwidth]{images/p/bulk/yy/bs.0.03.pdf}
%\includegraphics[width=0.19\textwidth]{images/p/bulk/yy/bs.0.04.pdf}
%\includegraphics[width=0.19\textwidth]{images/p/bulk/yy/bs.0.05.pdf}
%\\
%\includegraphics[width=0.19\textwidth]{images/p/bulk/xy/bs.-0.05.pdf}
%\includegraphics[width=0.19\textwidth]{images/p/bulk/xy/bs.-0.04.pdf}
%\includegraphics[width=0.19\textwidth]{images/p/bulk/xy/bs.-0.03.pdf}
%\includegraphics[width=0.19\textwidth]{images/p/bulk/xy/bs.-0.02.pdf}
%\includegraphics[width=0.19\textwidth]{images/p/bulk/xy/bs.-0.01.pdf}
%\includegraphics[width=0.19\textwidth]{images/p/bulk/xy/bs.0.01.pdf}
%\includegraphics[width=0.19\textwidth]{images/p/bulk/xy/bs.0.02.pdf}
%\includegraphics[width=0.19\textwidth]{images/p/bulk/xy/bs.0.03.pdf}
%\includegraphics[width=0.19\textwidth]{images/p/bulk/xy/bs.0.04.pdf}
%\includegraphics[width=0.19\textwidth]{images/p/bulk/xy/bs.0.05.pdf}
%\label{fig:bulkpbs}
%\end{figure}
%
%\clearpage
%\begin{figure}[htb!]
%\caption[Band structures of strained monolayer arsenic]
%{Band structures of strained {\bf monolayer arsenic} 
%with the predicted $\Gamma$-point Dirac  {state} at 
%$\varepsilon_{xx}=+5\%$ tensile strain circled in green.
%%
%See Fig.~\ref{fig:as_bi_highres} for more detail.}
%\centering
%\includegraphics[width=0.20\textwidth]{images/as/mono/xx/bs.0.00.pdf}
%\\
%\includegraphics[width=0.19\textwidth]{images/as/mono/xx/bs.-0.05.pdf}
%\includegraphics[width=0.19\textwidth]{images/as/mono/xx/bs.-0.04.pdf}
%\includegraphics[width=0.19\textwidth]{images/as/mono/xx/bs.-0.03.pdf}
%\includegraphics[width=0.19\textwidth]{images/as/mono/xx/bs.-0.02.pdf}
%\includegraphics[width=0.19\textwidth]{images/as/mono/xx/bs.-0.01.pdf}
%\includegraphics[width=0.19\textwidth]{images/as/mono/xx/bs.0.01.pdf}
%\includegraphics[width=0.19\textwidth]{images/as/mono/xx/bs.0.02.pdf}
%\includegraphics[width=0.19\textwidth]{images/as/mono/xx/bs.0.03.pdf}
%\includegraphics[width=0.19\textwidth]{images/as/mono/xx/bs.0.04.pdf}
%\includegraphics[width=0.19\textwidth]{images/as/mono/xx/bs.0.05.ed.pdf}
%\\
%\includegraphics[width=0.19\textwidth]{images/as/mono/yy/bs.-0.05.pdf}
%\includegraphics[width=0.19\textwidth]{images/as/mono/yy/bs.-0.04.pdf}
%\includegraphics[width=0.19\textwidth]{images/as/mono/yy/bs.-0.03.pdf}
%\includegraphics[width=0.19\textwidth]{images/as/mono/yy/bs.-0.02.pdf}
%\includegraphics[width=0.19\textwidth]{images/as/mono/yy/bs.-0.01.pdf}
%\includegraphics[width=0.19\textwidth]{images/as/mono/yy/bs.0.01.pdf}
%\includegraphics[width=0.19\textwidth]{images/as/mono/yy/bs.0.02.pdf}
%\includegraphics[width=0.19\textwidth]{images/as/mono/yy/bs.0.03.pdf}
%\includegraphics[width=0.19\textwidth]{images/as/mono/yy/bs.0.04.pdf}
%\includegraphics[width=0.19\textwidth]{images/as/mono/yy/bs.0.05.pdf}
%\\
%\includegraphics[width=0.19\textwidth]{images/as/mono/xy/bs.-0.05.pdf}
%\includegraphics[width=0.19\textwidth]{images/as/mono/xy/bs.-0.04.pdf}
%\includegraphics[width=0.19\textwidth]{images/as/mono/xy/bs.-0.03.pdf}
%\includegraphics[width=0.19\textwidth]{images/as/mono/xy/bs.-0.02.pdf}
%\includegraphics[width=0.19\textwidth]{images/as/mono/xy/bs.-0.01.pdf}
%\includegraphics[width=0.19\textwidth]{images/as/mono/xy/bs.0.01.pdf}
%\includegraphics[width=0.19\textwidth]{images/as/mono/xy/bs.0.02.pdf}
%\includegraphics[width=0.19\textwidth]{images/as/mono/xy/bs.0.03.pdf}
%\includegraphics[width=0.19\textwidth]{images/as/mono/xy/bs.0.04.pdf}
%\includegraphics[width=0.19\textwidth]{images/as/mono/xy/bs.0.05.pdf}
%\label{fig:monoasbs}
%\end{figure}
%
%\clearpage
%\begin{figure}[htb!]
%\caption[Band structures of strained bilayer arsenic]
%{Band structures of strained {\bf bilayer arsenic}
%with the predicted $\Gamma$-point Dirac  {state} at 
%$\varepsilon_{xx}=-4\%$ compressive strain circled in green.
%%
%See Fig.~\ref{fig:as_bi_highres} for more detail.}
%\centering
%%
%\includegraphics[width=0.20\textwidth]{images/as/bi/xx/bs.0.00.pdf}
%\\
%\includegraphics[width=0.19\textwidth]{images/as/bi/xx/bs.-0.05.pdf}
%\includegraphics[width=0.19\textwidth]{images/as/bi/xx/bs.-0.04.ed.pdf}
%\includegraphics[width=0.19\textwidth]{images/as/bi/xx/bs.-0.03.pdf}
%\includegraphics[width=0.19\textwidth]{images/as/bi/xx/bs.-0.02.pdf}
%\includegraphics[width=0.19\textwidth]{images/as/bi/xx/bs.-0.01.pdf}
%\includegraphics[width=0.19\textwidth]{images/as/bi/xx/bs.0.01.pdf}
%\includegraphics[width=0.19\textwidth]{images/as/bi/xx/bs.0.02.pdf}
%\includegraphics[width=0.19\textwidth]{images/as/bi/xx/bs.0.03.pdf}
%\includegraphics[width=0.19\textwidth]{images/as/bi/xx/bs.0.04.pdf}
%\includegraphics[width=0.19\textwidth]{images/as/bi/xx/bs.0.05.pdf}
%\\
%\includegraphics[width=0.19\textwidth]{images/as/bi/yy/bs.-0.05.pdf}
%\includegraphics[width=0.19\textwidth]{images/as/bi/yy/bs.-0.04.pdf}
%\includegraphics[width=0.19\textwidth]{images/as/bi/yy/bs.-0.03.pdf}
%\includegraphics[width=0.19\textwidth]{images/as/bi/yy/bs.-0.02.pdf}
%\includegraphics[width=0.19\textwidth]{images/as/bi/yy/bs.-0.01.pdf}
%\includegraphics[width=0.19\textwidth]{images/as/bi/yy/bs.0.01.pdf}
%\includegraphics[width=0.19\textwidth]{images/as/bi/yy/bs.0.02.pdf}
%\includegraphics[width=0.19\textwidth]{images/as/bi/yy/bs.0.03.pdf}
%\includegraphics[width=0.19\textwidth]{images/as/bi/yy/bs.0.04.pdf}
%\includegraphics[width=0.19\textwidth]{images/as/bi/yy/bs.0.05.pdf}
%\\
%\includegraphics[width=0.19\textwidth]{images/as/bi/xy/bs.-0.05.pdf}
%\includegraphics[width=0.19\textwidth]{images/as/bi/xy/bs.-0.04.pdf}
%\includegraphics[width=0.19\textwidth]{images/as/bi/xy/bs.-0.03.pdf}
%\includegraphics[width=0.19\textwidth]{images/as/bi/xy/bs.-0.02.pdf}
%\includegraphics[width=0.19\textwidth]{images/as/bi/xy/bs.-0.01.pdf}
%\includegraphics[width=0.19\textwidth]{images/as/bi/xy/bs.0.01.pdf}
%\includegraphics[width=0.19\textwidth]{images/as/bi/xy/bs.0.02.pdf}
%\includegraphics[width=0.19\textwidth]{images/as/bi/xy/bs.0.03.pdf}
%\includegraphics[width=0.19\textwidth]{images/as/bi/xy/bs.0.04.pdf}
%\includegraphics[width=0.19\textwidth]{images/as/bi/xy/bs.0.05.pdf}
%\label{fig:biasbs}
%\end{figure}
%
%\clearpage
%\begin{figure}[htb!]
%\caption[Band structures of strained bulk arsenic]
%{Band structures of strained {\bf bulk arsenic} 
%with the potential Weyl  {states} at $\varepsilon_{xx}=+1\%$ 
%and $\varepsilon_{yy}=+1\%-2\%$ tensile strains circled in green.
%%
%See Fig.~\ref{fig:as_bulk_highres} for more detail.}
%\centering
%\includegraphics[width=0.20\textwidth]{images/p/bulk/xx/bs.0.00.pdf}
%\\
%\includegraphics[width=0.19\textwidth]{images/as/bulk/xx/bs.-0.05.pdf}
%\includegraphics[width=0.19\textwidth]{images/as/bulk/xx/bs.-0.04.pdf}
%\includegraphics[width=0.19\textwidth]{images/as/bulk/xx/bs.-0.03.pdf}
%\includegraphics[width=0.19\textwidth]{images/as/bulk/xx/bs.-0.02.pdf}
%\includegraphics[width=0.19\textwidth]{images/as/bulk/xx/bs.-0.01.pdf}
%\includegraphics[width=0.19\textwidth]{images/as/bulk/xx/bs.0.01.ed.pdf}
%\includegraphics[width=0.19\textwidth]{images/as/bulk/xx/bs.0.02.pdf}
%\includegraphics[width=0.19\textwidth]{images/as/bulk/xx/bs.0.03.pdf}
%\includegraphics[width=0.19\textwidth]{images/as/bulk/xx/bs.0.04.pdf}
%\includegraphics[width=0.19\textwidth]{images/as/bulk/xx/bs.0.05.pdf}
%\\
%\includegraphics[width=0.19\textwidth]{images/as/bulk/yy/bs.-0.05.pdf}
%\includegraphics[width=0.19\textwidth]{images/as/bulk/yy/bs.-0.04.pdf}
%\includegraphics[width=0.19\textwidth]{images/as/bulk/yy/bs.-0.03.pdf}
%\includegraphics[width=0.19\textwidth]{images/as/bulk/yy/bs.-0.02.pdf}
%\includegraphics[width=0.19\textwidth]{images/as/bulk/yy/bs.-0.01.pdf}
%\includegraphics[width=0.19\textwidth]{images/as/bulk/yy/bs.0.01.ed.pdf}
%\includegraphics[width=0.19\textwidth]{images/as/bulk/yy/bs.0.02.ed.pdf}
%\includegraphics[width=0.19\textwidth]{images/as/bulk/yy/bs.0.03.pdf}
%\includegraphics[width=0.19\textwidth]{images/as/bulk/yy/bs.0.04.pdf}
%\includegraphics[width=0.19\textwidth]{images/as/bulk/yy/bs.0.05.pdf}
%\\
%\includegraphics[width=0.19\textwidth]{images/as/bulk/xy/bs.-0.05.pdf}
%\includegraphics[width=0.19\textwidth]{images/as/bulk/xy/bs.-0.04.pdf}
%\includegraphics[width=0.19\textwidth]{images/as/bulk/xy/bs.-0.03.pdf}
%\includegraphics[width=0.19\textwidth]{images/as/bulk/xy/bs.-0.02.pdf}
%\includegraphics[width=0.19\textwidth]{images/as/bulk/xy/bs.-0.01.pdf}
%\includegraphics[width=0.19\textwidth]{images/as/bulk/xy/bs.0.01.pdf}
%\includegraphics[width=0.19\textwidth]{images/as/bulk/xy/bs.0.02.pdf}
%\includegraphics[width=0.19\textwidth]{images/as/bulk/xy/bs.0.03.pdf}
%\includegraphics[width=0.19\textwidth]{images/as/bulk/xy/bs.0.04.pdf}
%\includegraphics[width=0.19\textwidth]{images/as/bulk/xy/bs.0.05.pdf}
%\label{fig:bulkasbs}
%\end{figure}
%
%\clearpage
%\begin{figure}[htb!]
%\caption[Band structures of strained monolayer antimony]
%{Band structures of strained {\bf monolayer antimony}
%with the predicted non-symmetry-point Dirac states at 
%$\varepsilon_{xx}\geq+3\%$ tensile strains circled in green.}
%\centering
%%
%\includegraphics[width=0.20\textwidth]{images/sb/mono/xx/bs.0.00.pdf}
%\\
%\includegraphics[width=0.19\textwidth]{images/sb/mono/xx/bs.-0.05.pdf}
%\includegraphics[width=0.19\textwidth]{images/sb/mono/xx/bs.-0.04.pdf}
%\includegraphics[width=0.19\textwidth]{images/sb/mono/xx/bs.-0.03.pdf}
%\includegraphics[width=0.19\textwidth]{images/sb/mono/xx/bs.-0.02.pdf}
%\includegraphics[width=0.19\textwidth]{images/sb/mono/xx/bs.-0.01.pdf}
%\includegraphics[width=0.19\textwidth]{images/sb/mono/xx/bs.0.01.pdf}
%\includegraphics[width=0.19\textwidth]{images/sb/mono/xx/bs.0.02.pdf}
%\includegraphics[width=0.19\textwidth]{images/sb/mono/xx/bs.0.03.ed.pdf}
%\includegraphics[width=0.19\textwidth]{images/sb/mono/xx/bs.0.04.ed.pdf}
%\includegraphics[width=0.19\textwidth]{images/sb/mono/xx/bs.0.05.ed.pdf}
%\\
%\includegraphics[width=0.19\textwidth]{images/sb/mono/yy/bs.-0.05.pdf}
%\includegraphics[width=0.19\textwidth]{images/sb/mono/yy/bs.-0.04.pdf}
%\includegraphics[width=0.19\textwidth]{images/sb/mono/yy/bs.-0.03.pdf}
%\includegraphics[width=0.19\textwidth]{images/sb/mono/yy/bs.-0.02.pdf}
%\includegraphics[width=0.19\textwidth]{images/sb/mono/yy/bs.-0.01.pdf}
%\includegraphics[width=0.19\textwidth]{images/sb/mono/yy/bs.0.01.pdf}
%\includegraphics[width=0.19\textwidth]{images/sb/mono/yy/bs.0.02.pdf}
%\includegraphics[width=0.19\textwidth]{images/sb/mono/yy/bs.0.03.pdf}
%\includegraphics[width=0.19\textwidth]{images/sb/mono/yy/bs.0.04.pdf}
%\includegraphics[width=0.19\textwidth]{images/sb/mono/yy/bs.0.05.pdf}
%\\
%\includegraphics[width=0.19\textwidth]{images/sb/mono/xy/bs.-0.05.pdf}
%\includegraphics[width=0.19\textwidth]{images/sb/mono/xy/bs.-0.04.pdf}
%\includegraphics[width=0.19\textwidth]{images/sb/mono/xy/bs.-0.03.pdf}
%\includegraphics[width=0.19\textwidth]{images/sb/mono/xy/bs.-0.02.pdf}
%\includegraphics[width=0.19\textwidth]{images/sb/mono/xy/bs.-0.01.pdf}
%\includegraphics[width=0.19\textwidth]{images/sb/mono/xy/bs.0.01.pdf}
%\includegraphics[width=0.19\textwidth]{images/sb/mono/xy/bs.0.02.pdf}
%\includegraphics[width=0.19\textwidth]{images/sb/mono/xy/bs.0.03.pdf}
%\includegraphics[width=0.19\textwidth]{images/sb/mono/xy/bs.0.04.pdf}
%\includegraphics[width=0.19\textwidth]{images/sb/mono/xy/bs.0.05.pdf}
%\label{fig:monosbbs}
%\end{figure}
%
%\clearpage
%\begin{figure}[htb!]
%\caption[Band structures of strained bilayer antimony]
%{Strained {\bf bilayer antimony} 
%with the predicted non-symmetry-point Dirac  {states} at 
%$\varepsilon_{xx}\geq+3\%$ tensile strains circled in green.
%%
%See Fig.~\ref{fig:sb_bi_highres} for more detail.}
%\centering
%\includegraphics[width=0.20\textwidth]{images/sb/bi/xx/bs.0.00.pdf}
%\\
%\setlength\tabcolsep{1.5pt}
%\begin{tabular}{ccccc}
%\includegraphics[width=0.19\textwidth]{images/sb/bi/xx/bs.-0.05.pdf}&
%\includegraphics[width=0.19\textwidth]{images/sb/bi/xx/bs.-0.04.pdf}&
%\includegraphics[width=0.19\textwidth]{images/sb/bi/xx/bs.-0.03.pdf}&
%\includegraphics[width=0.19\textwidth]{images/sb/bi/xx/bs.-0.02.pdf}&
%\includegraphics[width=0.19\textwidth]{images/sb/bi/xx/bs.-0.01.pdf}\\
%\includegraphics[width=0.19\textwidth]{images/sb/bi/xx/bs.0.01.pdf}&
%\includegraphics[width=0.19\textwidth]{images/sb/bi/xx/bs.0.02.pdf}&
%\includegraphics[width=0.19\textwidth]{images/sb/bi/xx/bs.0.03.ed.pdf}&
%\includegraphics[width=0.19\textwidth]{images/sb/bi/xx/bs.0.04.ed.pdf}&
%\includegraphics[width=0.19\textwidth]{images/sb/bi/xx/bs.0.05.ed.pdf}
%\\
%\raisebox{1.2cm}{Phase}  &\raisebox{1.2cm}{Transition}&
%\includegraphics[width=0.19\textwidth]{images/sb/bi/yy/bs.-0.03.pdf}&
%\includegraphics[width=0.19\textwidth]{images/sb/bi/yy/bs.-0.02.pdf}&
%\includegraphics[width=0.19\textwidth]{images/sb/bi/yy/bs.-0.01.pdf}\\
%\includegraphics[width=0.19\textwidth]{images/sb/bi/yy/bs.0.01.pdf}&
%\includegraphics[width=0.19\textwidth]{images/sb/bi/yy/bs.0.02.pdf}&
%\includegraphics[width=0.19\textwidth]{images/sb/bi/yy/bs.0.03.pdf}&
%\includegraphics[width=0.19\textwidth]{images/sb/bi/yy/bs.0.04.pdf}&
%\includegraphics[width=0.19\textwidth]{images/sb/bi/yy/bs.0.05.pdf}
%\\
%\includegraphics[width=0.19\textwidth]{images/sb/bi/xy/bs.-0.05.pdf}&
%\includegraphics[width=0.19\textwidth]{images/sb/bi/xy/bs.-0.04.pdf}&
%\includegraphics[width=0.19\textwidth]{images/sb/bi/xy/bs.-0.03.pdf}&
%\includegraphics[width=0.19\textwidth]{images/sb/bi/xy/bs.-0.02.pdf}&
%\includegraphics[width=0.19\textwidth]{images/sb/bi/xy/bs.-0.01.pdf}\\
%\includegraphics[width=0.19\textwidth]{images/sb/bi/xy/bs.0.01.pdf}&
%\includegraphics[width=0.19\textwidth]{images/sb/bi/xy/bs.0.02.pdf}&
%\includegraphics[width=0.19\textwidth]{images/sb/bi/xy/bs.0.03.pdf}&
%\includegraphics[width=0.19\textwidth]{images/sb/bi/xy/bs.0.04.pdf}&
%\includegraphics[width=0.19\textwidth]{images/sb/bi/xy/bs.0.05.pdf}
%\end{tabular}
%\label{fig:bisbbs}
%\end{figure}
%
%\clearpage
%\begin{figure}[htb!]
%\caption[Band structures of strained bulk antimony]
%{Band structures of strained {\bf bulk antimony}. 
%Here we see an appreciable effect of shear-strain on the 
%valence and conduction bands despite not opening a gap.}
%\centering
%\includegraphics[width=0.20\textwidth]{images/p/bulk/xx/bs.0.00.pdf}
%\\
%\includegraphics[width=0.19\textwidth]{images/sb/bulk/xx/bs.-0.05.pdf}
%\includegraphics[width=0.19\textwidth]{images/sb/bulk/xx/bs.-0.04.pdf}
%\includegraphics[width=0.19\textwidth]{images/sb/bulk/xx/bs.-0.03.pdf}
%\includegraphics[width=0.19\textwidth]{images/sb/bulk/xx/bs.-0.02.pdf}
%\includegraphics[width=0.19\textwidth]{images/sb/bulk/xx/bs.-0.01.pdf}
%\includegraphics[width=0.19\textwidth]{images/sb/bulk/xx/bs.0.01.pdf}
%\includegraphics[width=0.19\textwidth]{images/sb/bulk/xx/bs.0.02.pdf}
%\includegraphics[width=0.19\textwidth]{images/sb/bulk/xx/bs.0.03.pdf}
%\includegraphics[width=0.19\textwidth]{images/sb/bulk/xx/bs.0.04.pdf}
%\includegraphics[width=0.19\textwidth]{images/sb/bulk/xx/bs.0.05.pdf}
%\\
%\includegraphics[width=0.19\textwidth]{images/sb/bulk/yy/bs.-0.05.pdf}
%\includegraphics[width=0.19\textwidth]{images/sb/bulk/yy/bs.-0.04.pdf}
%\includegraphics[width=0.19\textwidth]{images/sb/bulk/yy/bs.-0.03.pdf}
%\includegraphics[width=0.19\textwidth]{images/sb/bulk/yy/bs.-0.02.pdf}
%\includegraphics[width=0.19\textwidth]{images/sb/bulk/yy/bs.-0.01.pdf}
%\includegraphics[width=0.19\textwidth]{images/sb/bulk/yy/bs.0.01.pdf}
%\includegraphics[width=0.19\textwidth]{images/sb/bulk/yy/bs.0.02.pdf}
%\includegraphics[width=0.19\textwidth]{images/sb/bulk/yy/bs.0.03.pdf}
%\includegraphics[width=0.19\textwidth]{images/sb/bulk/yy/bs.0.04.pdf}
%\includegraphics[width=0.19\textwidth]{images/sb/bulk/yy/bs.0.05.pdf}
%\\
%\includegraphics[width=0.19\textwidth]{images/sb/bulk/xy/bs.-0.05.pdf}
%\includegraphics[width=0.19\textwidth]{images/sb/bulk/xy/bs.-0.04.pdf}
%\includegraphics[width=0.19\textwidth]{images/sb/bulk/xy/bs.-0.03.pdf}
%\includegraphics[width=0.19\textwidth]{images/sb/bulk/xy/bs.-0.02.pdf}
%\includegraphics[width=0.19\textwidth]{images/sb/bulk/xy/bs.-0.01.pdf}
%\includegraphics[width=0.19\textwidth]{images/sb/bulk/xy/bs.0.01.pdf}
%\includegraphics[width=0.19\textwidth]{images/sb/bulk/xy/bs.0.02.pdf}
%\includegraphics[width=0.19\textwidth]{images/sb/bulk/xy/bs.0.03.pdf}
%\includegraphics[width=0.19\textwidth]{images/sb/bulk/xy/bs.0.04.pdf}
%\includegraphics[width=0.19\textwidth]{images/sb/bulk/xy/bs.0.05.pdf}
%\label{fig:bulksbbs}
%\end{figure}
%





