
\lett{T}{he search for graphene} alternatives 
that are capable of supporting 
ballistic electron transport 
has driven intense research 
toward the development of a 
new class of materials termed   
Weyl and Dirac semi-metals~\cite{Xuaaa9297,PhysRevLett.102.166803,Shekhar2015,PhysRevB.83.205101,NIELSEN1983389,turner2013beyond,PhysRevX.4.031035,PhysRevLett.108.140405,PhysRevB.85.195320,PhysRevB.88.125427,C3CP53257G,doi:10.1093/nsr/nwu080,Weyl1929,doi:10.1146/annurev-conmatphys-031016-025458,C3CP53257G,doi:10.1093/nsr/nwu080}.
%
Weyl and Dirac states 
in semi-metals  
result from the 
manifestation of massless quasi-particle states in 
bulk or low-dimensional 
condensed matter systems, respectively.
%
These states lead to unique topological properties 
characterised, primarily, 
by ballistic conduction.

In this Chapter, 
we continue our analysis  
of the calculation results of 
puckered phosphorus (P), arsenic (As) and antimony (Sb)
in relation to the electronic properties, 
with particular emphasis 
on the identification 
of qualitative electronic transitions.
%
Specifically, 
we investigate the Kohn-Sham band gaps and 
charge-carrier effective masses
from the computed electronic band structures, 
insofar as the Kohn-Sham eigenvalues 
are physically relevant, 
which, strictly speaking, 
they are not.
%
Nonetheless, 
reasonable confidence can be afforded 
in the reproduction of 
electronic trends~\cite{1402-4896-2004-T109-001,doi:10.1021/cr200107z}
and, thus, we make tentative predictions 
of strain-induced Dirac and  
Weyl states in particular cases, 
numerous band gap transitions,  
one-dimensional conductivity, 
and a structural phase-transition 
for moderate in-plane stresses.
%
%{These phases are predicted to support charge velocities 
%up to $10^6$~$\textrm{ms}^{-1}$ 
%and, in some highly anisotropic cases, 
%permit one-dimensional ballistic conductivity 
%in the puckered direction.}
%
Our predicted results contribute to the 
mounting evidence for the 
utility of these materials, 
as outlined in the previous Chapter, 
but are undermined by 
the failure of DFT to accurately resolve 
specific electronic properties 
which, as we shall observe, 
can differ dramatically from 
experimental observations.


% WEYL AND DIRAC SEMI-METALS
\section{Weyl and Dirac semi-metals}
\label{sec:weyl_and_dirac}

In 1929, Herman Weyl derived    
a solution to the Dirac equation~\cite{dirac1928quantum} 
that proposed the existence of 
relativistic, massless fermions~\cite{Weyl1929},
which have come to bear his name.
%
As of yet, 
Weyl fermions have not been 
observed as fundamental particles in nature, 
but persistent breakthroughs in 
topological insulators have lead 
to their recent discovery     
by Xu {\it et al.}~\cite{Xuaaa9297}
as emergent quasi-particles 
in tantalum arsenide (TaAs), 
among other  
predicted candidates~\cite{PhysRevB.83.205101,PhysRevLett.107.186806,PhysRevB.89.081106}.
%
Weyl semi-metals are primarily 
characterised by the crossing 
of linearly-dispersive energy bands 
through nodes at the Fermi energy, 
termed Weyl {\it points} or {\it cones}, 
which support ballistic charge 
transport~\cite{doi:10.1146/annurev-conmatphys-031016-025458}.
%
Weyl points emerge as the result of 
underlying symmetries of the crystal lattice,
namely time-reversal and 
inversion or reflection symmetries~\cite{doi:10.1093/nsr/nwu080}, 
but can also be produced by mirror symmetries~\cite{PhysRevB.93.035401}, 
which affect the solutions of the Hamiltonian.
%
As such, 
Weyl cones are highly robust 
and always occur in pairs 
of opposite chirality~\cite{doi:10.1146/annurev-conmatphys-031016-025458}.
%

A Dirac semi-metal, 
on the other had, 
is the two-dimensional 
analogue of a Weyl semi-metal, 
the most famous exhibition of which 
is seen in graphene~\cite{PhysRevLett.115.126803}. 
%
The particular difficulty in attaining Dirac points 
in two-dimensional materials  
is well documented~\cite{C3CP53257G,doi:10.1093/nsr/nwu080} 
and is due to the difficulty attaining  
the required symmetry-protection mechanisms 
in reduced dimensions.
%
Nonetheless, 
a Dirac semi-metal in few-layer  BP 
was observed experimentally by Kim \emph{et al.}~\cite{Kim723}, 
while arsenic~\cite{doi:10.1063/1.4943548}, 
antimony~\cite{Yao2013,Lu2016,Zhao2015}, 
and bismuth~\cite{Lu2016} 
%and P~\cite{Lu2016,PhysRevB.93.245303,Zhao2015,2053-1583-4-2-025071} 
are also predicted candidates.
%
The ambitious development 
of Weyl and Dirac semi-metals 
is driven by highly anticipated applications 
in high-speed electronics and computing. 
%
We refer the reader to Refs.~\cite{Weyl1929,Xuaaa9297,PhysRevLett.108.140405,doi:10.1146/annurev-conmatphys-031016-025458,C3CP53257G,doi:10.1093/nsr/nwu080} 
for further information on 
Weyl and Dirac semi-metals.

%%%%%%%%%%%%%%%%%%%%%%%%%%%%%%%%%%%%%%%%%%%%%%%%%%
% METHODOLOGY
%%%%%%%%%%%%%%%%%%%%%%%%%%%%%%%%%%%%%%%%%%%%%%%%%%
\section{Methodology}
\label{sec:methodology2}

The run-time calculation parameters 
and straining procedure for this Chapter  
were the same as those outlined 
in section~\ref{sec:calc_details}.
%
Calculations without SOC 
were initially performed for convenience 
and those strains that exhibit potential Dirac or Weyl states 
were reassessed at representative strains 
including non-perturbative SOC~\cite{PhysRevB.71.115106}.
%
Electronic band structures 
were calculated along the high-symmetry points 
of the Brillouin zone 
{\{$\Gamma$, $X$, $S$, $Y$, $Z$\}}, 
shown in Fig.~\ref{fig:bz} below, 
from which the Kohn-Sham band gaps 
were determined.

\begin{figure}[th!]
\centering
\includegraphics[height=0.494\textwidth]{images/3dbz.jpg}
\caption[Brillouin zone of orthorhombic phosphorus]
{The Brillouin zone for orthorhombic phosphorus 
of the Cmca space group and D$_{2h}$ point group 
with high-symmetry points 
\{$\Gamma$, $X$, $S$, $Y$, $Z$\} labelled.}
\label{fig:bz}
\end{figure}
%
Under shear strain, the Brillouin zone 
deforms into an asymmetric honeycomb 
in the $x$-$y$ plane, 
yet we retained sampling along the original  
path since the deformation 
up to 5\% strain is negligible 
and the effective masses were all
calculated at the $\Gamma$-point.
%
We determined 
charge-carrier effective masses 
according to the nearly-free electron model 
$m_{ij}^\star=\hbar^2\left(\partial^2E/\partial k_i\partial k_j \right)^{-1}$ 
using a cubic spline fit to 
9 data points about the $\Gamma$-point.
%
The charge velocities were similarly 
determined according to the dispersion relation 
$v=\hbar^{-1}dE(k)/dk$ 
from the linear fit to 
the Dirac and Weyl states.
%
To confirm the existence of linear-dispersion, 
we also plotted three-dimensional band structures 
in the region of the Dirac and Weyl points with SOC, 
as well as lines intersecting the points 
at $0^\circ$, $30^\circ$, $60^\circ$, and $90^\circ$ 
with respect to the $\Gamma-X$ line.
%
Following the convention of terminology 
found in the Refs.~\cite{doi:10.1021/acs.nanolett.5b04106,Kim723,Shekhar2015}, and other sources, 
we henceforth classify Dirac or Weyl states 
as those associated with regions 
of sustained linear dispersion 
in the band structure, 
at or near the Fermi level, 
in at least one direction.


%%%%%%%%%%%%%%%%%%%%%%%%%%%%%%%%%%%%%%%%%%%%%%%%%%
% RESULTS
%%%%%%%%%%%%%%%%%%%%%%%%%%%%%%%%%%%%%%%%%%%%%%%%%%
\section{Electronic properties}
\label{sec:electronic_results}

All of the band structures 
pertaining to the following analysis 
%may be found in the supplemental material 
%for Ref.~\cite{2053-1583-4-4-045018}.
are presented in 
the supplemental material of Ref.~\cite{2053-1583-4-4-045018}.
%
This section is divided 
into nine parts detailing 
the electronic properties  
derived from the plane-wave DFT calculations 
for each species at each layer-number. 
%
We discuss the trends 
in Kohn-Sham band gaps 
and charge carrier effective masses, 
and identify the qualitative transitions that occur, 
which are summarised in Table~\ref{table:transitions}.
%
Where applicable, 
we discuss the predicted 
existence of Dirac and Weyl states 
and reassess those examples 
with high resolution 
two- and three-dimensional band structures, 
including SOC.

In the following discussion, 
directions along $\Gamma-Y$ 
pertain to the zig-zag direction in the crystal structure, 
while those along $\Gamma-X$ 
pertain to the puckered direction.

%MONOLAYER PHOSPHORUS
\subsubsection{Monolayer Phosphorus}

For the relaxed P monolayer, 
our calculations produced  
a direct band gap of 0.88~eV 
at the $\Gamma$-point, 
as shown in Fig.~\ref{fig:pmonobg}, 
which falls within range of other gaps 
determined by DFT: 0.7~eV from DFT-PBEsol~\cite{PhysRevLett.112.176801}
and 1.0~eV from DFT-HSE06~\cite{doi:10.1021/nn501226z}.
%
On the other hand, quasi-particle calculations~\cite{PhysRevB.89.235319}, 
which treat electronic addition and removal events on an explicit footing, 
predict a larger band gap of 2~eV
with significant exciton binding~\cite{PhysRevB.90.205421} 
(between 0.4~eV to 0.83~eV).

This discrepancy is a classic example of 
the systematic 
band gap problem inherent to approximate 
semi-local functionals
such as PBE~\cite{PhysRevB.77.115123}.
%
These errors not only misrepresent  
band gaps and other electronic properties, 
but may also adversely 
affect the critical strains of the identified transitions 
and, thus, must be treated cautiously.
%
Nevertheless, 
it is important to emphasise that
band alignments and rates of change 
are quite often reliably reproduced~\cite{PhysRevB.90.155405}.
%as are direct-indirect transitions
%in two-dimensional materials~\cite{PhysRevB.90.085402}. 

While absolute band gaps 
are therefore not expected to be accurate,  
we can expect reasonable agreement 
with trends in electronic and 
mechanical behaviour~\cite{1402-4896-2004-T109-001,doi:10.1021/cr200107z}.
%
While discussing the remaining results, 
we shall omit further discussion of these errors 
until section~\ref{sec:2d_conclusion2}.

The application of uniaxial in-plane strain 
was found to open the band gap for tensile strain 
and diminished it for compressive strain,  
while shear strain had a negligible effect.
%
The electron and hole effective masses 
(Figs.~\ref{fig:pmonomeff}~\&~\ref{fig:pmonomhff}), 
compare well to the figures computed in Ref.~\cite{Wang2015},
where, at $\varepsilon_{xx}=+5\%$ tensile strain,  
the electron and hole effective masses   
coincide at 0.9~$m_0$ while, 
for compressive strains, 
the hole effective mass along 
$\Gamma-X$ rises significantly.

\begin{figure}[th!]
\begin{subfloat}[Mono P band gap (eV)]{
\includegraphics[height=0.315\textwidth]{images/p_mono_bandgap.pdf}
  \label{fig:pmonobg}}
\end{subfloat}
%
\begin{subfloat}[Monolayer P $m_e/m_0$]{
\includegraphics[height=0.315\textwidth]{images/p_mono_meff.pdf}
  \label{fig:pmonomeff}}
\end{subfloat}
%
\begin{subfloat}[Monolayer P $m_h/m_0$]{
\includegraphics[height=0.315\textwidth]{images/p_mono_mhff.pdf}
  \label{fig:pmonomhff}}
\end{subfloat}
%
\caption[Electronic properties of monolayer phosphorus for in-plane strains]
{The relationships for monolayer P between 
the applied in-plane strains $\varepsilon_{xx}$ (blue),
$\varepsilon_{yy}$ (red) and $\varepsilon_{xy}$ (green) against 
%
\protect\subref{fig:pmonobg}
the direct $E$ (solid  squares) and 
indirect $E^\star$ (dashed triangles) band gaps (eV);
%
\protect\subref{fig:pmonomeff}
the effective electron masses $m_e/m_0$ 
along $\Gamma-X$ (solid squares) and 
$\Gamma-Y$ (dashed triangles);
%
\protect\subref{fig:pmonomhff} 
the effective hole masses $m_h/m_0$ 
along $\Gamma-X$ (solid squares) and 
$\Gamma-Y$ (dashed triangles).}
\label{fig:p_mono_elec_properties}
\end{figure}


%BILAYER PHOSPHORUS
\subsubsection{Bilayer Phosphorus}
Relaxed bilayer P was also found 
to have a direct band gap of 
0.43~eV (Fig.~\ref{fig:pbibg}), 
and exhibits broadly the same 
electronic behaviour as the monolayer.
%
This includes 
the convergence of the electron effective masses 
(Fig.~\ref{fig:pbimeff}) 
at $+4\%$ strain, 
and increasing hole effective masses 
(Fig.~\ref{fig:pbimhff})
for compressive strains.
%
A direct-indirect band gap transition 
was also observed at 
$+2\%$ uniaxial tensile strain. 

\begin{figure}[th!]
\begin{subfloat}[Bilayer P band gap (eV)]{
\includegraphics[height=0.315\textwidth]{images/p_bi_bandgap.pdf}
  \label{fig:pbibg}}
\end{subfloat}
%
\begin{subfloat}[Bilayer P $m_e/m_0$]{
\includegraphics[height=0.315\textwidth]{images/p_bi_meff.pdf}
  \label{fig:pbimeff}}
\end{subfloat}
%
\begin{subfloat}[Bilayer P $m_h/m_0$]{
\includegraphics[height=0.315\textwidth]{images/p_bi_mhff.pdf}
  \label{fig:pbimhff}}
\end{subfloat}
%
\caption[Electronic properties of bilayer phosphorus for in-plane strains]
{The relationships for bilayer P between 
the applied in-plane strains $\varepsilon_{xx}$ (blue),
$\varepsilon_{yy}$ (red) and $\varepsilon_{xy}$ (green) against 
%
\protect\subref{fig:pmonobg}
the direct $E$ (solid  squares) and 
indirect $E^\star$ (dashed triangles) band gaps (eV);
%
\protect\subref{fig:pmonomeff}
the effective electron masses $m_e/m_0$
 along $\Gamma-X$ (solid squares) and 
$\Gamma-Y$ (dashed triangles);
%
\protect\subref{fig:pmonomhff} 
the effective hole masses $m_h/m_0$ 
along $\Gamma-X$ (solid squares) and 
$\Gamma-Y$ (dashed triangles).}
\label{fig:p_bi_elec_properties}
\end{figure}


Incidentally, the band gap closes 
around $-5\%$ uniaxial compressive strain, 
at which point a Dirac {state} emerges 
at the $\Gamma$-point. 
%
{The effect of SOC 
on the band structure 
(Fig.~\ref{fig:pbixxdiracbs}) 
induces no qualitative difference 
in the bands 
and the three-dimensional 
bands plotted about the Dirac points 
(Fig.~\ref{fig:pbixx3ddiracbs})
confirm the linear-dispersion,  
albeit in only one direction.
%
A linear fit to the 
surface of the bands 
(Fig.~\ref{fig:pbixxcutdiracbs}) 
 returns a maximum charge velocity 
of $v=3.80(1)\times 10^6$~$\textrm{ms}^{-1}$ 
along $\Gamma-Y$, 
while, in the orthogonal direction, 
the bands are flat with a  
a charge velocity that is relatively negligible.
%
This high anisotropy in charge velocities, 
dominated by ballistic conduction along $\Gamma-Y$, 
is indicative of effective one-dimensional conductivity 
that is supported by 
the large disparity in effective 
masses at $\varepsilon_{xx}=-5\%$, 
evident in Figs.~\ref{fig:pbimeff}~\&~\ref{fig:pbimhff}.}
%
The Dirac state for $\varepsilon_{yy}=-5\%$, 
which is due to band inversion, 
occurs off the $\Gamma-Y$ symmetry line
at a point $X^\prime$, 
as illustrated in Figs.~\ref{fig:pbixxdiracbs}~-~\ref{fig:pbixxcutdiracbs}.
%
Here the maximum charge velocity is
$v=3.22(1)\times 10^6$~$\textrm{ms}^{-1}$.
%
These results are supported by 
the work of Doh \emph{et al.}~\cite{2053-1583-4-2-025071}, 
who demonstrated the effect of strain 
on hopping parameters 
can lead to a Dirac semi-metallic state in bilayer P.
%
Similarly, Baik \emph{et al.}~\cite{doi:10.1021/acs.nanolett.5b04106} 
found that the SOC 
did not induce a band gap 
in potassium-doped multi-layer P,  
but did lift the spin-degeneracy of the Dirac points.

\begin{figure}[th!]
\centering
\begin{tabular}{ccc}
\begin{subfloat}[Bi P band structure $\varepsilon_{xx}=-5\%s$]{
  \centering
\includegraphics[height=0.247\textwidth]{images/p_bi_xx_diracbs.pdf}
  \label{fig:pbixxdiracbs}}
\end{subfloat}
&
%
\begin{subfloat}[]{
  \centering
\includegraphics[height=0.247\textwidth]{images/p_bi_xx_3ddirac_cut.pdf}
  \label{fig:pbixx3ddiracbs}}
\end{subfloat}
&
%
\begin{subfloat}[$\Gamma$-point Dirac state]{
  \centering
\includegraphics[height=0.247\textwidth]{images/p_bi_xx_dirac_cut.pdf}
  \label{fig:pbixxcutdiracbs}}
\end{subfloat}
\\
%
\begin{subfloat}[Bi P band structure $\varepsilon_{yy}=-5\%$]{
  \centering
\includegraphics[width=0.4\textwidth]{images/p_bi_yy_diracbs.pdf}
    \label{fig:pbiyydiracbs}}
\end{subfloat}
&
%
\begin{subfloat}[]{
  \centering
\includegraphics[width=0.18\textwidth]{images/p_bi_yy_3ddirac_cut.pdf}
  \label{fig:pbiyy3ddiracbs}}
\end{subfloat}
&
%
\begin{subfloat}[$X^\prime$-point Dirac state]{
  \centering
\includegraphics[width=0.27\textwidth]{images/p_bi_yy_dirac_cut.pdf}
  \label{fig:pbiyycutdiracbs}}
\end{subfloat}
\end{tabular}
\caption[Predicted Dirac states of strained bilayer phosphorus]
{\protect\subref{fig:pbixxdiracbs} Band structure of bilayer P at $\varepsilon_{xx}=-5\%$ 
with SOC (thick lines), 
and without SOC (thin lines) 
%
\protect\subref{fig:pbixx3ddiracbs} 
three-dimensional bands about 
the predicted Dirac point at $\Gamma$
\protect\subref{fig:pbixxcutdiracbs} 
slices through the Dirac point 
at $0^\circ$, $30^\circ$, $60^\circ$ and $90^\circ$ 
relative to the $\Gamma-X$ line 
that indicate highly anisotropic conduction. 
%
[\protect\subref{fig:pbiyydiracbs}~-~\protect\subref{fig:pbiyycutdiracbs}] 
Illustrates the same for bilayer P at $\varepsilon_{yy}=-5\%$, 
where the Dirac state occurs at the non-symmetry point $X^\prime$ 
along $\Gamma-X$.
}
\label{fig:p_dirac_states}
\end{figure}


%BULK PHOSPHORUS
\subsubsection{Bulk Phosphorus}

In the bulk, however, 
we found the band gap to be completely 
closed (Fig.~\ref{fig:pbulkbg}), 
and therefore the system is predicted to be metallic, 
in contrast to 
numerous experiments~\cite{Morita1986,doi:10.1063/1.1729699,NARITA1983422,MARUYAMA198199},
which have measured a direct gap in 
the range of 0.31-0.36~eV. 
%
We concluded after investigation 
that this was, in part, 
due to an effect  
of the Fermi surface smearing functionality~\cite{PhysRevLett.82.3296} 
in the ionic geometry relaxation procedure.
%
When relaxed under fixed-occupancy 
conditions, instead, a band gap of 
0.35~eV was produced.


The electron effective masses 
were found to be quite responsive to strain 
(Fig.~\ref{fig:pbulkmeff}), 
where those along 
$\Gamma-Y$ rose for 
both tensile strain along $\varepsilon_{xx}$, 
due to falling conduction bands,  
and compressive strain along $\varepsilon_{yy}$, 
due to flattening bands along $\Gamma-Y$.
%
In comparison, the hole effective masses 
were found to vary to a lesser extent 
(Fig.~\ref{fig:pbulkmhff}).
%
The effective masses along $\Gamma-X$ 
were not computed once 
the band gap closed below $+2\%$ strain.
%
Finally, shear strain was seen  
to have a negligible effect on the gap.

\begin{figure}[th!]
\begin{subfloat}[Bulk P band gap (eV)]{
\includegraphics[height=0.315\textwidth]{images/p_bulk_bandgap.pdf}
  \label{fig:pbulkbg}}
\end{subfloat}
%
\begin{subfloat}[Bulk P $m_e/m_0$]{
\includegraphics[height=0.315\textwidth]{images/p_bulk_meff.pdf}
  \label{fig:pbulkmeff}}
\end{subfloat}
%
\begin{subfloat}[Bulk P $m_h/m_0$]{
\includegraphics[height=0.315\textwidth]{images/p_bulk_mhff.pdf}
  \label{fig:pbulkmhff}}
\end{subfloat}
%
\caption[Electronic properties of bulk phosphorus for in-plane strains]
{The relationships for bulk P between 
the applied in-plane strains $\varepsilon_{xx}$ (blue),
$\varepsilon_{yy}$ (red) and $\varepsilon_{xy}$ (green) against 
%
\protect\subref{fig:pmonobg}
the direct $E$ (solid  squares) and 
indirect $E^\star$ (dashed triangles) band gaps (eV);
%
\protect\subref{fig:pmonomeff}
the effective electron masses $m_e/m_0$ 
along $\Gamma-X$ (solid squares) and 
$\Gamma-Y$ (dashed triangles);
%
\protect\subref{fig:pmonomhff} 
the effective hole masses $m_h/m_0$ 
along $\Gamma-X$ (solid squares) and 
$\Gamma-Y$ (dashed triangles).}
\label{fig:p_bulk_elec_properties}
\end{figure}


While the PBE gap remained 
incorrectly closed in the relaxed state, 
a possible Weyl {state} was observed 
at 2\% uniaxial strain 
(Fig.~\ref{fig:pbulkxxdiracbs}), 
before a direct gap opened 
that subsequently transitioned to an 
indirect gap at $+3\%$.
%
The SOC slightly 
reduced the band gap by approximately $0.05$~eV 
but did not qualitatively 
affect the overall results.
%
The three-dimensional band structure with SOC 
is shown in Fig.~\ref{fig:pbulkxx3ddiracbs}, 
in which a pair of potential Weyl points occur  
on an off-symmetry point $X^\prime$ 
along $\Gamma-X$.
%
{
The maximum charge velocity, 
computed from Fig.~\ref{fig:pbulkxxdiracbs},
is $v=2.40(1)\times 10^6$~$\textrm{ms}^{-1}$  
for both $\varepsilon_{xx}=+2\%$
and  $\varepsilon_{yy}=+2\%$ 
(Figs.~\ref{fig:pbulkyydiracbs}~-~\ref{fig:pbulkyycutdiracbs})
and occurs along a line parallel to $\Gamma-Y$.}
%
This band-inversion may 
also lead to further Weyl {states}
under greater compression,    
which have been predicted 
at similar pressures~\cite{PhysRevLett.115.186403,PhysRevB.93.195434,PhysRevB.91.195319}.

\begin{figure}[th!]
\centering
\begin{tabular}{ccc}
%
\smallskip
\begin{subfloat}[Bulk P band structure $\varepsilon_{xx}=2\%$]{
\centering
\includegraphics[width=0.4\textwidth]{images/p_bulk_xx_diracbs.pdf}
\label{fig:pbulkxxdiracbs}}
\end{subfloat}
&
\begin{subfloat}[]{
\centering
\includegraphics[width=0.18\textwidth]{images/p_bulk_xx_3ddirac_cut.pdf}
\label{fig:pbulkxx3ddiracbs}}
\end{subfloat}
&
\begin{subfloat}[$X^\prime$-point Dirac state]{
\centering
\includegraphics[width=0.27\textwidth]{images/p_bulk_xx_dirac_cut.pdf}
\label{fig:pbulkxxcutdiracbs}}
\end{subfloat} 
\\
%
\smallskip
\begin{subfloat}[Bulk P band structure $\varepsilon_{yy}=2\%$]{
\centering
\includegraphics[width=0.4\textwidth]{images/p_bulk_yy_diracbs.pdf}
\label{fig:pbulkyydiracbs}}
\end{subfloat}
&
\begin{subfloat}[]{
\centering
\includegraphics[width=0.18\textwidth]{images/p_bulk_yy_3ddirac_cut.pdf}
\label{fig:pbulkyy3ddiracbs}}
\end{subfloat}
&
\begin{subfloat}[$X^\prime$-point Weyl state]{
\centering
\includegraphics[width=0.27\textwidth]{images/p_bulk_yy_dirac_cut.pdf}
\label{fig:pbulkyycutdiracbs}}
\end{subfloat} 
%
\end{tabular}
\caption[Predicted Weyl states of strained bulk phosphorus]{
\protect\subref{fig:pbulkxxdiracbs} 
Band structure of  bulk P at $\varepsilon_{xx}=+2\%$ 
with SOC (thick lines), 
and without SOC (thin lines) 
where the off-symmetry point $X^\prime$, 
located along $\Gamma-X$, 
is indicated by the dashed vertical line 
%
\protect\subref{fig:pbulkxx3ddiracbs}  
three-dimensional bands with SOC about 
the potential Weyl points  
%
\protect\subref{fig:pbulkxxcutdiracbs} 
slices through one Weyl point 
at $0^\circ$, $30^\circ$, $60^\circ$ and $90^\circ$ 
relative to the $\Gamma-X$ line, also with SOC.
%
[\protect\subref{fig:pbulkyydiracbs}~-~\protect\subref{fig:pbulkyycutdiracbs} ] 
The same is illustrated for $\varepsilon_{yy}=+2\%$.}
\label{fig:p_bulk_highres}
\end{figure}

%SUMMARY P
To summarise, 
we predict the onset of $\Gamma$-point 
Dirac {states} in bilayer P 
at -5\% uniaxial compression, 
which is supported by previous 
studies~\cite{2053-1583-4-2-025071,doi:10.1021/acs.nanolett.5b04106}.
%
We also predict effective one-dimensional conductivity 
at $\varepsilon_{xx}=-5\%$, 
and a direct-indirect band gap transition at 
$+2\%$ tensile strain. 
%
We observe possible Weyl {states} 
at $+2\%$ tensile strain in bulk P, 
followed by a direct-indirect 
band gap transition at +3\%.
%
Finally, effective masses  
are found to vary widely 
between phases and strain directions 
and seem strongly susceptible to tuning. 


%MONOLAYER ARSENIC
\subsubsection{Monolayer Arsenic}
In contrast to P, 
we identifed an indirect 
band gap of 0.15~eV along the 
$\Gamma-Y$ direction
in the relaxed As monolayer (Fig.~\ref{fig:asmonobg}),
which was significantly lower than the  
predicted DFT-HSE06 gap of 0.83~eV~\cite{1882-0786-8-5-055201}.
%
However, 
the relaxed band structure and band gap profiles 
closely resemble the band structures in 
Refs.~\cite{PhysRevB.91.085423,doi:10.1063/1.4943548}.
 
\begin{figure}[th!]
\begin{subfloat}[Mono As band gap (eV)]{
\includegraphics[height=0.315\textwidth]{images/as_mono_bandgap.pdf}
  \label{fig:asmonobg}}
\end{subfloat}
%
\begin{subfloat}[Monolayer As $m_e/m_0$]{
\includegraphics[height=0.315\textwidth]{images/as_mono_meff.pdf}
  \label{fig:asmonomeff}}
\end{subfloat}
%
\begin{subfloat}[Monolayer As $m_h/m_0$]{
\includegraphics[height=0.315\textwidth]{images/as_mono_mhff.pdf}
  \label{fig:asmonomhff}}
\end{subfloat}
\caption[Electronic properties of monolayer arsenic for in-plane strains]
{The relationships for bulk As between 
the applied in-plane strains $\varepsilon_{xx}$ (blue),
$\varepsilon_{yy}$ (red) and $\varepsilon_{xy}$ (green) against 
%
\protect\subref{fig:asmonobg}
the direct (solid  squares) and indirect 
(dashed triangles) band gaps (eV);
%
\protect\subref{fig:asmonomeff}
the effective electron masses $m_e/m_0$ 
along $\Gamma-X$ (solid squares) and 
$\Gamma-Y$ (dashed triangles);
%
\protect\subref{fig:asmonomhff}
the effective hole masses $m_h/m_0$ 
along $\Gamma-X$ (solid squares) and 
$\Gamma-Y$ (dashed triangles).}
\label{fig:as_mono_elec_properties}
\end{figure}

The band gap diminishes for tensile  
strain along $\varepsilon_{xx}$ 
and at $+2\%$ 
the material becomes semi-metallic 
with a Dirac {state} at the $\Gamma$-point  
emerging at $\varepsilon_{xx}=+5\%$ 
{accompanied by an electron pocket 
above the Fermi-level 
(Fig.~\ref{fig:as_mono_highres}), 
which is unaffected by the SOC.}
% 
{
The maximum charge velocity here 
is $v=3.01(1)\times10^6$~$\textrm{ms}^{-1}$ 
and lies along $\Gamma-Y$.
%
In the orthogonal direction 
the bands are flat, 
similarly to monolayer P, 
with a relatively small charge velocity.
%
This high anisotropy in charge velocities, 
dominated by the ballistic conduction along $\Gamma-Y$, 
is again indicative of effective one-dimensional conductivity 
and is further supported by 
the large disparity in effective 
masses at $\varepsilon_{xx}=5\%$, 
shown in Figs.~\ref{fig:asmonomeff}~\&~\ref{fig:asmonomhff}.}

%
For compressive strain along $\varepsilon_{xx}$ 
an indirect-direct transition occurs~\cite{PhysRevB.91.085423} 
at $\varepsilon_{xx}$ =-2\%.  
%
For tensile strain along $\varepsilon_{yy}$ 
the band gap opens, where at  $\varepsilon_{yy}$=-3\%, 
the indirect band gap closes along $\Gamma-Y$.
%
Similar to monolayer phosphorus, 
there is no appreciable effect due to shear-strain.
%
Meanwhile, the charge-carrier effective masses 
(Figs.~\ref{fig:asmonomeff}~\&~\ref{fig:asmonomhff}) 
respond linearly to uniaxial strain and compare 
well to other works~\cite{doi:10.1080/14786437508229285}, 
where, in particular valence band broadening along $\Gamma-X$
leads to an increasing hole effective mass.

\begin{figure}[th!]
\centering
\begin{tabular}{ccc}
%
\smallskip
\begin{subfloat}[Mono As band structure $\varepsilon_{xx}=5\%$]{
\centering
\includegraphics[height=0.247\textwidth]{images/as_mono_xx_diracbs.pdf}
\label{fig:asmonoxxdiracbs}}
\end{subfloat}
&
\begin{subfloat}[]{
\centering
\includegraphics[height=0.247\textwidth]{images/as_mono_xx_3ddirac_cut.pdf}
\label{fig:asmonoxx3ddiracbs}}
\end{subfloat}
&
\begin{subfloat}[$\Gamma$-point Dirac state]{
\centering
\includegraphics[height=0.247\textwidth]{images/as_mono_xx_dirac_cut.pdf}
\label{fig:asmonoxxcutdiracbs}}
\end{subfloat} 
%
\end{tabular}
\caption[Predicted Dirac states of strained monolayer arsenic]{
\protect\subref{fig:asmonoxxdiracbs} 
Band structure of  monolayer As at $\varepsilon_{xx}=+5\%$ 
with SOC (thick lines), 
and without SOC (thin lines)
%
\protect\subref{fig:asmonoxx3ddiracbs} 
 three-dimensional bands  with SOC about 
the predicted Dirac point at $\Gamma$ 
%
\protect\subref{fig:asmonoxxcutdiracbs} 
slices through the Dirac point 
at $0^\circ$, $30^\circ$, $60^\circ$ and $90^\circ$ 
relative to the $\Gamma-X$ line, also with SOC.
}
\label{fig:as_mono_highres}
\end{figure}

%BILAYER ARSENIC
\subsubsection{Bilayer Arsenic}
For bilayer As, we identify a 
direct band gap of $0.45$~eV 
(Fig.~\ref{fig:asbibg}), 
in contrast to the indirect band gap 
observed in the monolayer.
%
Here, the band gap opens 
for uniaxial tensile strain 
and diminishes for compressive strain.
%
The direct band gap transitions
to an indirect gap at both 
$\varepsilon_{yy}=-3\%$ and 
$\varepsilon_{yy}=+2\%$, 
while at 
$\varepsilon_{xx}=+2\%$ 
it also transitions to an indirect gap 
before resuming to a 
direct gap again at $\varepsilon_{xx}=+3\%$.

\begin{figure}[th!]
\begin{subfloat}[Bilayer As band gap (eV)]{
  \centering
\includegraphics[height=0.315\textwidth]{images/as_bi_bandgap.pdf}
  \label{fig:asbibg}}
\end{subfloat}
%
\begin{subfloat}[Bilayer As $m_e/m_0$]{
\includegraphics[height=0.315\textwidth]{images/as_bi_meff.pdf}
  \label{fig:asbimeff}}
\end{subfloat}
%
\begin{subfloat}[Bilayer As $m_h/m_0$]{
\includegraphics[height=0.315\textwidth]{images/as_bi_mhff.pdf}
  \label{fig:asbimhff}}
\end{subfloat}
\caption[Electronic properties of bilayer arsenic for in-plane strains]
{The relationships for bulk As between 
the applied in-plane strains $\varepsilon_{xx}$ (blue),
$\varepsilon_{yy}$ (red) and $\varepsilon_{xy}$ (green) against 
%
\protect\subref{fig:asbibg} 
the direct (solid  squares) and indirect 
(dashed triangles) band gaps (eV);
%
\protect\subref{fig:asbimeff}
the effective electron masses $m_e/m_0$ 
along $\Gamma-X$ (solid squares) and 
$\Gamma-Y$ (dashed triangles);
%
\protect\subref{fig:asbimhff}
the effective hole masses $m_h/m_0$ 
along $\Gamma-X$ (solid squares) and 
$\Gamma-Y$ (dashed triangles).}
\label{fig:as_bi_elec_properties}
\end{figure}

{
Moreover, we predict a Dirac state 
at the $\Gamma$-point 
at $\varepsilon_{xx}=-4\%$ compression 
(Fig.~\ref{fig:as_bi_highres}), 
for which the maximum charge velocity 
is $v=2.62(2)\times10^6$~$\textrm{ms}^{-1}$ 
along $\Gamma-Y$.
%
The bands are also flat 
along $\Gamma-X$, 
similar to the monolayer,
and have a relatively negligible charge velocity.
%
The high anisotropy in charge velocities, 
again suggests effective one-dimensional conductivity, 
dominated by the ballistic conduction along $\Gamma-Y$,
and is further supported by 
the large disparity in effective 
masses at $\varepsilon_{xx}=5\%$, 
shown in Figs.~\ref{fig:asbimeff}~\&~\ref{fig:asbimhff}.}
%
Here again, the SOC has no appreciable 
effect on the bands.
%
The electron and hole effective masses 
(Figs.~\ref{fig:asbimeff}~\&~\ref{fig:asbimhff})
respond approximately linearly 
to the applied strain, 
where conduction band broadening 
leads to increased 
effective electron masses, 
and valence band flattening at $\Gamma$
leads to increasing hole effective masses. 

\begin{figure}[th!]
\centering
\begin{tabular}{ccc}
%
\smallskip
\begin{subfloat}[Bi As band structure $\varepsilon_{xx}=5\%$]{
\centering
\includegraphics[height=0.247\textwidth]{images/as_bi_xx_diracbs.pdf}
\label{fig:asbixxdiracbs}}
\end{subfloat}
&
\begin{subfloat}[]{
\centering
\includegraphics[height=0.247\textwidth]{images/as_bi_xx_3ddirac_cut.pdf}
\label{fig:asbixx3ddiracbs}}
\end{subfloat}
&
\begin{subfloat}[$\Gamma$-point Dirac state]{
\centering
\includegraphics[height=0.247\textwidth]{images/as_bi_xx_dirac_cut.pdf}
\label{fig:asbixxcutdiracbs}}
\end{subfloat} 
\\
%
\end{tabular}
\caption[Predicted Dirac states of strained bilayer arsenic]{
\protect\subref{fig:asbixxdiracbs} 
Band structure of  bilayer As at $\varepsilon_{xx}=+5\%$ 
with SOC (thick lines), 
and without SOC (thin lines)
%
\protect\subref{fig:asbixx3ddiracbs} 
 three-dimensional bands  with SOC about 
the predicted Dirac point at $\Gamma$ 
%
\protect\subref{fig:asbixxcutdiracbs} 
slices through the Dirac point 
at $0^\circ$, $30^\circ$, $60^\circ$ and $90^\circ$ 
relative to the $\Gamma-X$ line, also with SOC.
}
\label{fig:as_bi_highres}
\end{figure}



%BULK ARSENIC
\subsubsection{Bulk Arsenic}
Finally, no band gap is determined 
in the relaxed bulk phase (Fig.~\ref{fig:asbulkbg}), 
again contrary to experiments~\cite{doi:10.1080/00018737900101355}, 
where a small direct band gap of 
approximately 0.3~eV is observed.
%
The electron and 
hole effective masses 
along $\Gamma-Y$ 
(Figs.~\ref{fig:asbulkmeff}~\&~\ref{fig:asbulkmhff}) 
are found to increase rapidly 
for compressive strains, 
as the band peaks rapidly flatten at 
the $\Gamma$-point.

\begin{figure}[th!]
\begin{subfloat}[Bulk As band gap (eV)]{
\includegraphics[height=0.315\textwidth]{images/as_bulk_bandgap.pdf}
  \label{fig:asbulkbg}}
\end{subfloat}
%
\begin{subfloat}[Bulk As $m_e/m_0$]{
\includegraphics[height=0.315\textwidth]{images/as_bulk_meff.pdf}
  \label{fig:asbulkmeff}}
\end{subfloat}
%
\begin{subfloat}[Bulk As $m_h/m_0$]{
\includegraphics[height=0.315\textwidth]{images/as_bulk_mhff.pdf}
  \label{fig:asbulkmhff}}
\end{subfloat}
\caption[Electronic properties of bulk arsenic for in-plane strains]
{The relationships for bulk As between 
the applied in-plane strains $\varepsilon_{xx}$ (blue),
$\varepsilon_{yy}$ (red) and $\varepsilon_{xy}$ (green) against 
%
\protect\subref{fig:asbulkbg} 
the direct (solid  squares) and indirect 
(dashed triangles) band gaps (eV);
%
\protect\subref{fig:asbulkmeff}  
the effective electron masses $m_e/m_0$ 
along $\Gamma-X$ (solid squares) and 
$\Gamma-Y$ (dashed triangles);
%
\protect\subref{fig:asbulkmhff} 
the effective hole masses $m_h/m_0$ 
along $\Gamma-X$ (solid squares) and 
$\Gamma-Y$ (dashed triangles)}
\label{fig:as_bulk_elec_properties}
\end{figure}

%
At $\varepsilon_{xx}=+1\%$ strain, 
a potential Weyl {state} is briefly observed 
on an off-symmetry point $X^\prime$
(Fig.~\ref{fig:as_bulk_highres})
before a direct gap opens 
that subsequently transitions 
to an indirect one 
at $\varepsilon_{xx}=+3\%$, 
after which it reduces again.
%
Another potential Weyl {state} 
around the same off-symmetry point $X^\prime$
is also predicted to occur between 
$+1\%\leq\varepsilon_{yy}\leq+2\%$, 
after which a direct band gap also appears.
%
The recalculated band structure 
with the SOC for $\varepsilon_{yy}=+1\%$
(Fig.~\ref{fig:as_bulk_highres}) 
confirms the linear-dispersion, 
where the maximum charge velocity in both cases  
occurs along a line parallel to $\Gamma-Y$ 
and is $v=1.38(1)\times10^6$~$\textrm{ms}^{-1}$.

\begin{figure}[th!]
\centering
\begin{tabular}{ccc}
%
\smallskip
\begin{subfloat}[Bulk As band structure $\varepsilon_{xx}=1\%$]{
\centering
\includegraphics[height=0.247\textwidth]{images/as_bulk_xx_diracbs.pdf}
\label{fig:asbulkxxdiracbs}}
\end{subfloat}
&
\begin{subfloat}[]{
\centering
\includegraphics[height=0.247\textwidth]{images/as_bulk_xx_3ddirac_cut.pdf}
\label{fig:asbulkxx3ddiracbs}}
\end{subfloat}
&
\begin{subfloat}[$X^\prime$-point Weyl state]{
\centering
\includegraphics[height=0.247\textwidth]{images/as_bulk_xx_dirac_cut.pdf}
\label{fig:asbulkxxcutdiracbs}}
\end{subfloat} 
%
\\
\smallskip
\begin{subfloat}[Bulk As band structure $\varepsilon_{xx}=1\%$]{
\centering
\includegraphics[height=0.247\textwidth]{images/as_bulk_yy_diracbs.pdf}
\label{fig:asbulkyydiracbs}}
\end{subfloat}
&
\begin{subfloat}[]{
\centering
\includegraphics[height=0.247\textwidth]{images/as_bulk_yy_3ddirac_cut.pdf}
\label{fig:asbulkyy3ddiracbs}}
\end{subfloat}
&
\begin{subfloat}[$X^\prime$-point Weyl state]{
\centering
\includegraphics[height=0.247\textwidth]{images/as_bulk_yy_dirac_cut.pdf}
\label{fig:asbulkyycutdiracbs}}
\end{subfloat} 
%
\end{tabular}
\caption[Predicted Weyl states of strained bulk arsenic]{ 
\protect\subref{fig:asbulkxxdiracbs}
Band structure of bulk As at $\varepsilon_{xx}=+1\%$ 
with SOC (thick lines), 
and without SOC (thin lines) 
where the off-symmetry point $X^\prime$, 
located along $\Gamma-X$, 
is indicated by the dashed vertical line 
%
\protect\subref{fig:asbulkxx3ddiracbs}
three-dimensional bands with SOC about 
the potential Weyl points  
%
\protect\subref{fig:asbulkxxcutdiracbs}
slices through one Weyl point 
at $0^\circ$, $30^\circ$, $60^\circ$ and $90^\circ$ 
relative to the $\Gamma-X$ line, also with SOC.
%
[\protect\subref{fig:asbulkyydiracbs}~-~\protect\subref{fig:asbulkyycutdiracbs}]~The same is illustrated for $\varepsilon_{yy}=+1\%$.}
\label{fig:as_bulk_highres}
\end{figure}


In summary, 
we predict  
$\Gamma$-point Dirac {states} 
in the monolayer and  bilayer of As, 
{which support one-dimensional ballistic conduction}, 
as well as possible 
Weyl {states} 
on off-symmetry points in the bulk 
at moderate levels of in-plane stress.
%
These states were found to be 
unaffected by SOC.
%
We also observe several 
band gap transitions, 
in particular in the monolayer phase, 
which also include 
semi-conducting-metallic transitions.
%
Finally, the effective masses 
respond approximately linearly 
with respect to uniaxial strain, 
except in the bulk,
which exhibits quadratic behaviour.



%MONOLAYER ANTIMONY
\subsubsection{Monolayer Antimony}

The relaxed Sb monolayer  
is found to possess 
an indirect band gap of $0.21$~eV 
along $\Gamma-Y$ (Fig.~\ref{fig:sbmonobg}), 
which is reasonably comparable to other 
PBE values 
0.28~\cite{doi:10.1021/acsami.5b02441} 
and 0.37~\cite{PhysRevB.91.235446}~eV, 
where these values 
were obtained including SOC.
%
For tensile strain along $\varepsilon_{yy}$
the band gap opens, 
suggesting an indirect-direct 
transition for strains above $6-7\%$,  
and it diminishes for compressive strains 
before finally closing at $\varepsilon_{yy}=-2\%$, 
where the material becomes a semi-metal.
%
Similarly, the indirect gap closes along $\Gamma-X$ 
at a compressive strain of 
$\varepsilon_{xx}=-2\%$ 
 at which monolayer Sb 
 again becomes semi-metallic.

\begin{figure}[th!]
\begin{subfloat}[Mono Sb band gap (eV)]{
\includegraphics[height=0.315\textwidth]{images/sb_mono_bandgap.pdf}
  \label{fig:sbmonobg}}
\end{subfloat}
%
\begin{subfloat}[Monolayer Sb $m_e/m_0$]{
\includegraphics[height=0.315\textwidth]{images/sb_mono_meff.pdf}
  \label{fig:sbmonomeff}}
\end{subfloat}
%
\begin{subfloat}[Monolayer Sb $m_h/m_0$]{
\includegraphics[height=0.315\textwidth]{images/sb_mono_mhff.pdf}
  \label{fig:sbmonomhff}}
\end{subfloat}
%
\caption[Electronic properties of monolayer antimony for in-plane strains]
{The relationships for monolayer Sb between 
the applied in-plane strains $\varepsilon_{xx}$ (blue),
$\varepsilon_{yy}$ (red) and $\varepsilon_{xy}$ (green) against 
%
\protect\subref{fig:sbmonobg} 
the direct (solid  squares) and indirect 
(dashed triangles) band gaps (eV);
%
\protect\subref{fig:sbmonomeff}  
the effective electron masses $m_e/m_0$ 
along $\Gamma-X$ (solid squares) and 
$\Gamma-Y$ (dashed triangles);
%
\protect\subref{fig:sbmonomhff} 
the effective hole masses $m_h/m_0$ 
along $\Gamma-X$ (solid squares) and 
$\Gamma-Y$ (dashed triangles).}
\label{fig:sb_mono_elec_properties}
\end{figure}


The indirect gap transitions to a direct gap
at $\varepsilon_{xx}=+1\%$ tensile strain 
and remains so until finally 
closing at $\varepsilon_{xx}=+4\%$, 
at which point we predict a {potential}
Dirac {state} along $\Gamma-Y$ 
at an off-symmetry point~\cite{PhysRevLett.102.166803}  
$Y^\prime$
(Fig.~\ref{fig:sbmonoxxdiracbs})
that has also been predicted 
in Ref.~\cite{Lu2016}.
%
Fig.~\ref{fig:sbmonoxxdiracbs} depicts the 
calculated band structure, 
in which it is shown that SOC 
preserves the Dirac {state}  
but not does not open the band gap.
%
{
The three-dimensional band structure 
about the Dirac point 
is shown in Fig.~\ref{fig:sbmonoxx3ddiracbs} 
in which the maximum charge velocity 
is  $v=4.31(1)\times10^6$~$\textrm{ms}^{-1}$ 
and occurs along a line parallel to the $\Gamma-Y$ direction 
(Fig.~\ref{fig:sbmonoxxcutdiracbs}).}
%
Moreover, 
the valence band at the $X$-point 
undergoes a Rashba 
splitting~\cite{1367-2630-17-5-050202} 
due to SOC, 
which is also predicted to occur in 
the monolayers of $\alpha$-P~\cite{PhysRevB.92.035135}, 
and $\beta$-Sb~\cite{Zhao2015,C6RA13101H}.
%
Finally, the electron effective masses  
experience a weak linear-response to strain 
(Figs.~\ref{fig:sbmonomeff}~\&~\ref{fig:sbmonomhff}), 
while the hole effective masses along $\Gamma-X$ 
respond much more strongly 
to a rapid broadening or flattening 
of the valence band.


%Dirac States
\begin{figure}[th!]
\centering
\begin{tabular}{ccc}
%
\begin{subfloat}[Mono Sb band structure $\varepsilon_{xx}=3\%$]{
  \centering
\includegraphics[height=0.247\textwidth]{images/sb_mono_xx_diracbs.pdf}
  \label{fig:sbmonoxxdiracbs}}
\end{subfloat}
&
%
\begin{subfloat}[]{
  \centering
\includegraphics[height=0.247\textwidth]{images/sb_mono_xx_3ddirac_cut.pdf}
  \label{fig:sbmonoxx3ddiracbs}}
\end{subfloat}
&
%
\begin{subfloat}[$Y^\prime$-point Dirac state]{
  \centering
\includegraphics[height=0.247\textwidth]{images/sb_mono_xx_dirac_cut.pdf}
  \label{fig:sbmonoxxcutdiracbs}}
\end{subfloat}
\end{tabular}
\caption[Predicted Dirac states of monolayer antimony]{
\protect\subref{fig:sbmonoxxdiracbs} 
Band structure of monolayer Sb 
at $\varepsilon_{xx}=-5\%$ 
with SOC (thick lines), 
and without SOC (thin lines) 
%
\protect\subref{fig:sbmonoxx3ddiracbs} 
three-dimensional bands about 
the predicted Dirac point at $\Gamma$
%
\protect\subref{fig:sbmonoxxcutdiracbs} 
slices through the Dirac point 
at $0^\circ$, $30^\circ$, $60^\circ$ and $90^\circ$ 
relative to the $\Gamma-X$ line 
that indicate highly anisotropic conduction. 
}
\label{fig:sb_dirac_states}
\end{figure}

%BILAYER ANTIMONY
\subsubsection{Bilayer Antimony}
The relaxed bilayer phase 
is found to be semi-metallic, 
where an indirect band gap 
opens at $\varepsilon_{yy}=+3\%$
tensile strain, 
and band-inversion at the $\Gamma$-point 
leads to to a 
fully-metallic state for 
uniaxial strains $<-1\%$.
%
The effective masses  
(Figs.~\ref{fig:sbbimeff} \& \ref{fig:sbbimhff})
experience soft linear-response to strains 
prior to the transition to full metallicity, 
at which point a rapid flattening of the bands at the 
$\Gamma$-point suggesting strong electron localisation.

%Just Antimony
\begin{figure}[th!]
\begin{subfloat}[Bi Sb band gap (eV)]{
\includegraphics[height=0.315\textwidth]{images/sb_bi_bandgap.pdf}
  \label{fig:sbbibg}}
\end{subfloat}
%
\begin{subfloat}[Bilayer Sb $m_e/m_0$]{
\includegraphics[height=0.315\textwidth]{images/sb_bi_meff.pdf}
  \label{fig:sbbimeff}}
\end{subfloat}
%
\begin{subfloat}[Bilayer Sb $m_h/m_0$]{
\includegraphics[height=0.315\textwidth]{images/sb_bi_mhff.pdf}
  \label{fig:sbbimhff}}
\end{subfloat}
%
\caption[Electronic properties of antimony for in-plane strains]
{The relationships for bilayer Sb between 
the applied in-plane strains $\varepsilon_{xx}$ (blue),
$\varepsilon_{yy}$ (red) and $\varepsilon_{xy}$ (green) against 
%
\protect\subref{fig:sbbibg}
the direct (solid  squares) and indirect 
(dashed triangles) band gaps (eV);
%
\protect\subref{fig:sbbimeff}
the effective electron masses $m_e/m_0$ 
along $\Gamma-X$ (solid squares) and 
$\Gamma-Y$ (dashed triangles);
%
\protect\subref{fig:sbbimhff}
the effective hole masses $m_h/m_0$ 
along $\Gamma-X$ (solid squares) and 
$\Gamma-Y$ (dashed triangles).}
\label{fig:sb_elec_properties}
\end{figure}
%


In addition, a possible Dirac {state} emerges 
at a non-symmetry-point 
$Y^\prime$ along $\Gamma-Y$ 
for $\varepsilon_{xx}=+2\%$ 
tensile strain 
(Fig.~\ref{fig:sb_bi_highres})
and remains in place 
up to at least $+5\%$ strain.
%
The maximum charge velocity 
$v=4.47(3)\times10^6$~$\textrm{ms}^{-1}$ 
is also along $\Gamma-Y$ 
and is approximately 
the same as that of the monolayer.

\begin{figure}[th!]
\begin{tabular}{ccc}
\smallskip
\begin{subfloat}[Bi Sb band structure]{
\includegraphics[height=0.247\textwidth]{images/sb_bi_xx_diracbs.pdf}
\label{fig:sbbixxdiracbs}}
\end{subfloat}
&
\begin{subfloat}[]{
\includegraphics[height=0.247\textwidth]{images/sb_bi_xx_3ddirac_cut.pdf}
\label{fig:sbbixx3ddiracbs}}
\end{subfloat}
&
\begin{subfloat}[$Y^\prime$-point Dirac state]{
\includegraphics[height=0.247\textwidth]{images/sb_bi_xx_dirac_cut.pdf}
\label{fig:sbbixxcutdiracbs}}
\end{subfloat} 
%
\end{tabular}
\caption[Predicted Dirac states of bilayer antimony]{ 
\protect\subref{fig:sbbixxdiracbs}
Band structure of bilayer Sb at $\varepsilon_{xx}=+3\%$ 
with SOC (thick lines), 
and without SOC (thin lines) 
where the off-symmetry point $Y^\prime$, 
located along $\Gamma-Y$, 
is indicated by the dashed vertical line 
%
\protect\subref{fig:sbbixx3ddiracbs}
three-dimensional bands with SOC about 
the predicted Dirac point 
%
\protect\subref{fig:sbbixxcutdiracbs}
slices through the Dirac point 
at $0^\circ$, $30^\circ$, $60^\circ$ and $90^\circ$ 
relative to the $\Gamma-X$ line, also with SOC.}
\label{fig:sb_bi_highres}
\end{figure}

%
Finally, beyond a compressive 
strain of  $\varepsilon_{yy}=-3\%$,
at a stress of approximately 0.3~GPa, 
we observed a structural phase-transition 
in the bilayer, 
which buckled in the puckered ($\vec{y}$) direction.
%
This buckled structure 
has a total-energy 1.7~meV/atom 
lower than that of the relaxed state 
of the unperturbed $\alpha$-bilayer 
and 3.0~meV/atom lower when 
allowed to fully-relax,  
as shown in Fig.~\ref{fig:buckled}.
%
This suggests the possible existence of a 
new phase of Sb that is attainable via strain.

\begin{figure}[th!]
\centering
\includegraphics[height=0.494\textwidth]{images/buckled1.jpg}
\caption[Strain-induced buckled phase in bilayer antimony]
{Side-view of the strain-induced buckled phase of Sb
at $\varepsilon_{yy}=-4\%$ compression.}
\label{fig:buckled}
\end{figure}



%BULK ANTIMONY
\subsubsection{Bulk Antimony}
Finally, bulk Sb is 
found to be completely metallic 
for all levels of strain explored in this work.
%
However, shear strains in this case 
do appear to have a significant
effect on the bands despite not opening a gap.


In summary, 
we predict possible non-symmetry-point 
Dirac {states} in the strained monolayer {and bilayer} of Sb, 
which are qualitatively unaffected by SOC, 
as well as Rashba splitting at the $X$-point 
{in the monolayer}.
%
We also predict 
indirect-direct and indirect-semi-metallic 
transitions in the monolayer phase 
and a band gap opening in the bilayer phase.
%
Finally, we observe a 
buckled phase induced in bilayer Sb 
at $-4\%$ compressive strain.
%
Bulk Sb was found to be metallic 
at all levels of strain explored.

We provide a brief summary  
of the band gaps and qualitative phase-transitions 
in Table~\ref{table:transitions}, 
and in Table~\ref{table:bandgaps} 
we outline the calculated 
band gaps and effective 
charge carrier masses 
for the relaxed phases of each structure.
%

%%%%%%%%%%%%%%%%%%%%%%%%%%%%%%%%%%%%%%%%%%%%%%%%%
% CONCLUSIONS
%%%%%%%%%%%%%%%%%%%%%%%%%%%%%%%%%%%%%%%%%%%%%%%%%
\section{Conclusion}
\label{sec:2d_conclusion2}

Continuing the discussion in the previous Chapter, 
we have provided the accompanying electronic properties 
from the extensive calculations 
of few-layer and bulk P, As and Sb, 
where we have compared and contrasted 
these results with experimental studies.
%
Our findings predict several band gap transitions, 
numerous Dirac and Weyl states, 
one-dimensional conductivity 
driven by ballistic conduction 
on the order of $10^6$~$\textrm{ms}^{-1}$, 
and a notable strain-induced 
buckled phase in bilayer Sb.
%
We also note that the critical stresses 
at which these transitions occur 
are expected to be experimentally 
accessible and highly switchable, 
paving the way for possible 
verification.
%
With the ongoing enthusiastic research 
in group-V layered materials, 
they are poised to become central to 
the development of next generation materials 
comprising a vast portfolio of applications.


We would like to conclude this 
work on 2D materials with a brief note on 
the conclusions of these calculations
 within a wider context, 
in particular with regards to the general 
predictive power of Kohn-Sham DFT.
%
For this work, we used the 
PBE exchange-correlation functional 
supplemented by vdW corrections 
and, where warranted, spin-orbit coupling, 
which form a highly robust  
computational framework.
%
Nonetheless, 
throughout this Chapter, 
we made note of several instances 
where computed properties differed, 
or even outright contradicted, 
experimental observations, 
for example in the band gaps of 
bulk phosphorus~\cite{Morita1986,doi:10.1063/1.1729699,NARITA1983422,MARUYAMA198199} 
and bulk arsenic~\cite{doi:10.1080/00018737900101355}.
%
Ostensibly, the best agreement one can 
reasonably expect to obtain 
from conventional DFT calculations 
are in band alignments~\cite{PhysRevB.90.155405}, 
lattice constants, 
and general trends in 
mechanical and electronic 
properties~\cite{PhysRevB.70.125426,PhysRevB.90.085402}.
%
%The reason for this is that these quantities 
%are determined by the behaviour in kinetic energy, 
%which does not suffer from errors to 
%the same degree as the Hartree potential. 
%
However, the accurate computation 
of specific electronic properties 
is, patently, not guaranteed.

{
This lack of agreement 
with basic experimental measurements 
for relatively simple structures 
is concerning, 
and draws attention to a pressing issue 
pervading approximate DFT - 
that of the self-interaction error (SIE).
%
Despite decades of 
theoretical, algorithmic and computational advances, 
the SIE, to this day, 
leads to qualitative inaccuracies
 in some of the most basic DFT 
 calculations.
% \footnote{{However, 
% approximate XC functionals are not directly responsible 
%for the inaccurate KS band gap, which has more do with the 
%behaviour of the total-energy around integer particle number.}}.
%
This SIE originates from 
the approximation of  
exchange-correlation functionals, 
{and requires that calculation results 
be interpreted with caution} 
to avoid coming to erroneous conclusions.
%
Rather than depend on 
one's own intuition or experience to decide 
if SIE might be present in a system 
(and in absence of relevant experimental data), 
there exist strategies with which one can quantify 
and treat it, 
at least approximately.
%
%Indeed, 
%there are several examples of systems 
%that are known to suffer significant SIE, 
%
In the next Chapter, 
we shall explore the SIE in detail, 
and discuss some techniques 
developed over the years toward its correction.}










