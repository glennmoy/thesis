%2D MATERIALS
\lett{D}{ensity-functional theory} (DFT), 
{
in its approximate Kohn-Sham formalism, 
is a highly-acclaimed computational tool that 
affords the practical and expeditious  
calculation of ground-state properties 
of molecules and solids, 
often with a very reasonable accuracy.
%
It finds routine application 
in the fields of chemistry, 
physics, 
materials science, 
and biochemistry, 
where it now contributes 
%to research 
in both 
a descriptive and predictive capacity.}

{
It is not, in practice, without systematic errors 
such as those defined by self-interaction and static correlation.
%
These errors undermine the accurate description 
of particular systems that are beyond the scope   
of the approximate exchange-correlation functionals, 
particularly for those comprising 
so-called {\it strongly-correlated} electrons. 
%
The effective treatment of these errors 
is laid down in a number of formative works 
now adopted within the canon of Kohn-Sham DFT. 
%
Many of the most popular and affordable correction schemes 
entail the calculation of external parameters 
to diagnose and treat these pervasive errors 
on a per-electron basis, 
such as the DFT+Hubbard $U$ method.}

{
A possibility that has not yet been explored, however, 
is the automation of these correction schemes  
for the provision of greater efficiency, 
versatility and comparability between 
DFT calculations.
%
An automated procedure   
would enable the correction process to be self-contained, 
thereby circumventing the need for human input,  
and establish a standardised approach  
between the various softwares and electronic systems.
%
Of particular interest is the application 
in high-throughput materials design, 
and the comparability of DFT+$U$ 
total-energies for the calculation  
of thermodynamical quantities.}

{
In this dissertation, 
we present a comprehensive account 
of our work in pursuit of this goal.
%some initial advances  
%in pursuit of this goal from first-principles.
%
We motivate and describe an 
efficient self-contained approach 
for correcting the many-body self-interaction error 
in strongly-correlated systems 
from ground-state quantities 
within the DFT+$U$ framework.
%
Moreover, 
we implement this procedure in a linear-scaling code, 
which extends its applicability to large-scale systems.}

{
Specifically, 
we develop a highly accurate 
variational linear-response approach  
for calculating the 
Hubbard $U$ and Hund's $J$ parameters, 
for which a unique criterion for 
their self-consistency is identified.
% 
Our results demonstrate that this scheme 
is accurate and versatile, 
and facilitates the correction of 
many-body self-interaction error 
for various systems.
%
Moreover, 
we propose the novel construction of a 
generalised DFT+$U$ functional 
that resolves Koopmans' condition exactly 
in a one-electron system 
when supplied with the appropriate 
self-consistent $U$ value.}

{
Our research provides insight into important questions 
about the practice and consequences 
of calculating corrective parameters 
for approximate DFT self-consistently, 
and opens up several new avenues 
for future developments.}

%oeuvre
%As an attest to the utility of DFT,  
%we perform comprehensive calculations on 
%the nascent class of two-dimensional pnictogen materials 
%characterised by the puckered phase of black phosphorus, 
%for which we make qualitative predictions of  
%their electro-mechanical properties 
%in various states of mechanical strain. 
%%electronic transitions, 
%%and states exhibiting ballistic conduction 
%%are predicted to occur.
%%
%%SELF-INTERACTION ERROR
%In so doing, we reveal the 
%tendency of DFT to exhibit systematic errors, 
%such as the pervasive self-interaction error (SIE), 
%which are incurred by approximate exchange-correlation functionals.
%%used to incorporate the complex many-body interactions.
%%
%These errors are known to be particularly disruptive 
%in systems comprising strongly-correlated electrons.
%%which includes a growing number systems 
%%that require large-scale simulation.
%
%Adequate treatment of these systems 
%is typically sought by the administration of 
%an appropriate corrective scheme, 
%such as the established DFT+$U$ method,  
%which targets SIE-prone electrons 
%in highly-localised orbitals.
%%
%However, 
%the calculation of the Hubbard $U$ parameters 
%governing the correction can be challenging to compute, 
%and expertise in the matter is limited to a small community.
%%
%The development of methods that address SIE 
%from an automated approach, 
%wherein the calculation of Hubbard parameters 
%is delegated to a self-contained procedure, 
%thus presents an attractive and expedient prospect 
%for many practitioners, 
%especially when readily extended to large-scale systems.
%%
%We are therefore motivated towards the design 
%of efficient, self-contained methods  
%for the treatment of SIE-prone systems 
%within a linear-scaling framework.
%
%
%%NON-LINEAR CONSTRAINTS
%We first elect to exploit the proven efficiency 
%of constrained-DFT (cDFT) 
%to enforce non-linear constraints 
%on the occupancies of the localised orbitals, 
%the interaction for which emulates 
%that found in the SIE-targeting DFT+$U$ functional.
%%
%However, 
%we prove via inductive proof 
%and stringent numerical calculations  
%that such a procedure is strictly impossible.
%%due to non-vanishing response functions.  
%%
%%GENERALISED DFT+U FUNCTIONAL
%%
%However, we suggest a possible workaround 
%by introducing a generalised DFT+$U$ functional 
%that simultaneously targets the correction of 
%the ground-state total-energy and eigenvalue.
%%
%The scheme requires only the input of 
%ground-state DFT properties to operate, 
%including the Hubbard $U$ parameter, 
%the calculation of which in we proceed to develop.
% 
%%CALCULATING U
%We build upon the established success 
%of the computationally convenient linear-response approach 
%and present a modification based, instead, 
%on the variation of the ground-state density, 
%whereupon  the $U$ parameter 
%is defined in terms of ground-state quantities.
%%
%Similarly, 
%we devise an original and equivalent self-consistency scheme 
%to calculate the exchange parameter $J$.
%
%%We demonstrate for the first time that DFT+$U$ 
%%precisely corrects the one-electron total-energy 
%%for self-interaction error under ideal conditions.
%%
%Accordingly, 
%the variational approach provides a convenient framework 
%in which to address the incipient practice of parameter self-consistency, 
%which hitherto remains unresolved in spite of 
%a number of proposed schemes.
%%
%Stringent numerical tests on various systems reveal  
%a clear best choice for the self-consistency condition, 
%and to effectuate the correction for multiple electrons 
%we find the corresponding $J$-term is not only advantageous, 
%but required.
%%
%Moreover, we suggest this scheme as a possible 
%remedy for static-correlation (SCE) error, 
%for which preliminary results on dissociating H$_2$ 
%are encouraging.
%
%
%












 
