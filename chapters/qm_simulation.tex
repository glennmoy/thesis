%%%%%%%%%%%%%%%%%%%%%%
% INTRODUCTION
%%%%%%%%%%%%%%%%%%%%%%

\lett{R}{ecent technological} advances, 
{
in conjunction with an \break 
increasing community of active contributors, 
have succeeded in establishing Kohn-Sham   
density-functional theory (DFT) as one of 
the most widespread and utilised tools 
in modern science.
%
{
DFT has evolved significantly 
since its inception over 50 years ago~\cite{doi:10.1063/1.4869598};
following the {award of the Nobel prize in 1998
to Walter Kohn}~\cite{RevModPhys.71.1253,RevModPhys.71.1267} 
it has continued to flourish to this day under 
the auspicious stewardship of countless contributors.
%
Presently, it sees routine application 
in academic and industrial research 
and has imparted immeasurable 
scientific~\cite{doi:10.1021/cr200107z, RevModPhys.87.897, doi:10.1063/1.4704546, doi:10.1146/annurev-physchem-032511-143803}
and economic~\cite{goldbeck2012economic} value.}
%
In this Chapter,
we review the pioneering developments 
in condensed matter theory that have culminated 
in contemporary DFT, which is now routinely applied 
to electronic structure calculations.}

We begin our discussion with the Schr\"odinger equation, 
central to all quantum mechanical calculations, 
and the seminal theorems of 
Hohenberg and Kohn~\cite{PhysRev.136.B864} 
who pioneered the field.
%
{
We then discuss how Kohn and Sham 
subsequently laid the foundations 
for the solution of the ground-state exactly 
via self-consistent field equations, 
incorporating an exchange-correlation (XC) functional~\cite{PhysRev.140.A1133}.
%
The inaccessibility of the exact XC functional, 
however, prompted the development of 
approximate local functionals  
that enable 
the computational solution to the equations.
%
These were originally based on the local-density 
of a uniform electron gas~\cite{PhysRev.140.A1133,PhysRevLett.45.566}, 
which was later extended to incorporate 
spin~\cite{0022-3719-5-13-012,PhysRevB.7.1912,PhysRevB.45.13244} 
and self-interaction corrections~\cite{PhysRevB.23.5048}, 
with further development involving a 
generalised gradient~\cite{PhysRevB.54.16533,PhysRevLett.77.3865} 
approximation.
%
These incremental improvements in theory  
were bolstered by simultaneous advances in hardware 
that {preceded} the proliferation of DFT applications 
across various fields~\cite{doi:10.1063/1.4704546,RevModPhys.61.689}.}

{
In spite of the theoretical and algorithmic breakthroughs 
that contributed to the popularity of DFT 
{toward the end of} the last century, 
calculations of very large systems 
remained decidedly out of reach~\cite{guerra1998towards}.
%
The restrictive computational bottle-neck 
involved in diagonalising the dense Hamiltonian 
{incurred} a compute-time 
that scaled cubically with system size~\cite{PhysRevLett.66.1438}.
% 
The quest for a linear-scaling formalism 
was marked by the proposal of Kohn himself 
to exploit the {\it near-sightedness} of quantum mechanical 
interactions~\cite{PhysRevLett.76.3168} 
in the DFT procedure - 
thereby enabling the calculation 
of large-scale systems.
%
In the final section, 
we outline the procedure for 
{constructing a} linear-scaling DFT, 
which has risen to prominence over the last 25 years~\cite{0953-8984-22-7-074207,RevModPhys.71.1085}.
%
{A suite of packages now boast linear-scaling functionality},
which includes  
{\sc ONETEP}~\cite{skylaris2005introducing,hine2009linear,hine2010linear}, 
{\sc CONQUEST}~\cite{Gillan200714,0953-8984-14-11-303},
{\sc SIESTA}~\cite{0953-8984-14-11-302,0953-8984-20-6-064208}, 
{\sc BigDFT}~\cite{Genovese2011149}, 
{\sc OpenMX}~\cite{PhysRevB.67.155108,PhysRevB.72.045121}, 
and {\sc CP2K}~\cite{doi:10.1063/1.2841077}.}
%



%%%%%%%%%%%%%%%%%%%%%%
% SCHRODINGER EQUATION
%%%%%%%%%%%%%%%%%%%%%%
\section{The Schr{\" o}dinger equation}
\label{sec:schrodinger_equation}

The basis of many quantum mechanical {systems}   
lies in the solution of the many-body, time-independent, 
Sch{\"o}dinger equation~\cite{PhysRev.28.1049}
%
\begin{equation}
\hat H \Psi \left(\br,t\right)
%=i\frac{\partial}{\partial t}\Psi \left(\br,t\right)
=E\Psi \left(\br,t\right),
\end{equation}
%
where the complex-valued $N$-body wave function 
$\Psi \left(\br,t\right)$ 
encodes all information about the quantum state 
that is {propagated in time} by the Hamiltonian $\hat H$. 
%
In an organised configuration of $N$  atoms 
in a crystal lattice or molecule, 
with nuclear positions $\bf R_\alpha$ 
and $n$ electrons at positions ${\br}_i$, 
the governing non-relativistic, 
time-independent Hamiltonian  
is given by 
%
\begin{align}
\hat H &= -\frac{1}{2}\left[\sum_{i=1}^n\nabla^2_i+\sum_{\alpha=1}^N\frac{1}{m_\alpha}\nabla^2_\alpha\right] 
- \sum_{i=1}^n\sum_{\alpha=1}^N\frac{Z_\alpha}{|{\br}_i-R_\alpha|}\nonumber\\[0.5em]
&+\frac{1}{2}\sum_{i=1}^n\sum_{j\neq i}^n\frac{1}{|{\br}_i-r_j|} 
+ \frac{1}{2}\sum_{\alpha=1}^N\sum_{\beta\neq\alpha}^N\frac{Z_\alpha Z_\beta}{|\bf R_\alpha-R_\beta|},
\label{eq:full_hamiltonian}
\end{align}
%
with nuclear masses and charges 
denoted by $m_\alpha$ and $Z_\alpha$, {respectively}.
%
Hartree units are invoked throughout this dissertation 
such that $\hbar=c=m_e=1$.

For an atomic system 
{that does not exchange energy with its environment}, 
the wave function $\Psi$ is the steady-state 
solution of a time-independent Hamiltonian, 
{the Ansatz for which 
is constructed from a separation of variables}
%
\begin{equation}
\Psi \left(\{\br_i\},\{{\bf R_\alpha}\},t\right) 
= \Phi \left(\{\br_i\},\{{\bf R_\alpha}\}\right)\Theta\left(t\right) 
%= \Phi \left(\{\br_i\},\{{\bf R_\alpha}\}\right)e^{-iEt} 
\end{equation}
%
into a component that is time-dependent 
$\Theta\left(t\right)$ 
and one that is not
$\Phi \left(\{\br_i\},\{{\bf R_\alpha}\}\right)$.
%
Thus the constituents are independently solvable 
by the Hamiltonian and coupled 
by the common energy eigenvalue $\varepsilon$
 %
 \begin{equation}
 \hat H \Phi \left(\{\br_i\},\{{\bf R}_\alpha\}\right) 
 = E \Phi \left(\{\br_i\},\{{\bf R}_\alpha\}\right); 
 \quad 
i\frac{\partial}{\partial t}\Theta\left(t\right) = \varepsilon \Theta\left(t\right).
 \end{equation}

For low-temperature atomic systems, 
governed by a Hamiltonian such as Eq.~\eqref{eq:full_hamiltonian}, 
the time-scale of nuclear motion, 
tied to the available thermal energy, 
is generally orders of magnitude {longer} 
than that of the electrons 
due to the much heavier nuclear masses.
%
Since the electrons relax into equilibrium 
{on a time-scale much {shorter} than that} of the nuclei, 
their behaviour is modelled  
%{the rigour of a full quantum mechanical treatment} 
by an electronically-relevant Hamiltonian $\hat{H}_\textrm{el}$
%
\begin{align}
\hat {H}_\textrm{el}
\approx \left(-\frac{1}{2}\sum_{i=1}^n\nabla^2_i\right.
&- \sum_{i=1}^n\sum_{\alpha=1}^N\frac{Z_\alpha}{|{{\br}_i}-{\bf R_\alpha}|} \left.+ \frac{1}{2}\sum_{i=1}^n\sum_{j\neq i}^n\frac{1}{|{\br}_i-{\br}_j|}\right)
\label{eq:electron_hamiltonian}
\end{align}
%
in which the nuclei are assumed to move so little 
that they can be effectively described 
as classical point charges. 
%
This renowned Born-Oppenheimer, 
or adiabatic, approximation~\cite{ANDP:ANDP19273892002}, 
is invoked to make 
quantum mechanical calculations  
under Eq.~\eqref{eq:full_hamiltonian} 
more feasible.
%
{However, 
advances in computational modelling, 
combined with the improved experimental apparatus~\cite{PhysRevX.7.031035}, 
have permitted the application of Ehrenfest molecular dynamics 
for some time~\cite{0953-8984-16-7-L03,0953-8984-16-46-012}.}

%
Eigenvalue solutions to Eq.~\eqref{eq:full_hamiltonian} 
span a multi-dimensional adiabatic potential energy surface 
parameterised by the nuclear coordinates $\{{\bf R_\alpha}\}$, 
which may be optimised classically 
using molecular dynamics methods~\cite{PhysRevLett.69.1077,PhysRevLett.55.2471}. 
%
The neglect of explicitly nuclear-dependent terms 
{leads to the failure} to predict more complex phenomena 
such as superconductivity, 
photochemistry and vibrational spectroscopy, 
however, for most systems, 
the Born-Oppenheimer approximation 
is a very reasonable simplification. 



%%%%%%%%%%%%%%%%%%%%%%
% HOHENBERG KOHN THEOREMS
%%%%%%%%%%%%%%%%%%%%%%
\section{The Hohenberg-Kohn theorems}
\label{sec:hk_theorems}

Despite the benefits of the adiabatic approximation, 
computationally solving the 
many-body Schr\"{o}dinger equation 
for all but the simplest systems is 
practically impossible.
%
Consider a real-valued, $N$-electron wave function 
discretised on a grid of $M$ points 
(ignoring spin).
% 
A full description of this wave function 
will require $M^{3N}$ scalars~\cite{CAPELLE2006} 
and is unquestionably beyond the 
scope of modern processing power 
for any realistic system of interest.
%
Even if the exact wave function were known, 
the memory required to store it would 
present an impasse.


{For an $N$-electron system,
Hohenberg and Kohn~\cite{PhysRev.136.B864} (HK)  
showed in their seminal work 
that the many-electron 
ground-state density distribution 
of a system $n_0(\br)$ 
(subscript to denote ground-state)
is sufficient for providing a full description 
of the ground-state properties, 
in principle.}
%
{Thus, for a} many-body quantum state vector $\ket{\Phi}$, 
the electron density is defined as 
%
\begin{equation}
n(\br)=\bra{\Phi}\hat n\ket{\Phi}=\int\prod_{i=2}^N\ d{{\br}_i} \left|\Phi({\bf r_1,\cdots,r_N})\right|^2,
\end{equation}
%
where $\ket{\Phi}$ is assumed to be 
normalised $\braket{\Phi}{\Phi}=N$ 
and antisymmetric under particle exchange 
so as to obey the Pauli exclusion principle~\cite{PhysRev.58.716}.
%
Furthermore, they proved   
that the ground-state was identified 
by variational minimisation 
of the total-energy with respect to the density~\cite{PhysRev.136.B864}. 
%
Promoting the density to the central quantity 
of the theory drastically simplifies the problem 
of computationally solving Eq.~\eqref{eq:electron_hamiltonian}, 
as the discretised wave function 
is now parameterised by only $M^3$ scalars.
%

{
Faithful to the motivation behind the HK formalism,
the Hamiltonian in Eq.~\eqref{eq:electron_hamiltonian} 
may be re-expressed in terms of contributions stemming from 
potentials that are intrinsic and extrinsic to the system 
$\hat{H}_\textrm{el}=\hat{F}+\hat{V}$.}
%
Here $\hat{F}$ is sum of the kinetic energy operator 
and electron-electron Coulomb potential 
\begin{equation}
\hat {F}= -\frac{1}{2}\sum_{i=1}^n\nabla^2_i
+ \frac{1}{2}\sum_{i=1}^n\sum_{j\neq i}^n\frac{1}{|{\br}_i-r_j|},
\end{equation} 
and is a universal function 
of the number of electrons $n$.
%
The term $\hat{V}$ describes the static 
potential arising from the lattice of nuclei 
and defines the external potential specific to the system 
%
\begin{equation}
\hat{V} =  \sum_{i=1}^n V_\textrm{ext} \left({{\br}_i}\right) 
= - \sum_{i=1}^n\sum_{\alpha=1}^N\frac{Z_\alpha}{|{{\br}_i}-{\bf R_\alpha}|} 
\end{equation}
%
{but may also include other fields, 
such as the electromagnetic}.
%
The single-particle eigenstates $\{\ket{\phi_i}\}$ are found 
by solving Eq.~\eqref{eq:electron_hamiltonian}, 
for which the eigenenergies $\{\varepsilon_i\}$ 
are the corresponding expectation values with the Hamiltonian 
%
\begin{equation}
\varepsilon_i=\bra{\phi_i}{\hat{H}_\textrm{el}}\ket{\phi_i}
=\bra{\phi_i}{\hat{F}}\ket{\phi_i} 
+ \int d\br\ V_\textrm{ext}(\br)n(\br).
\label{eq:expectation_value}
\end{equation}

Thus, for a given number of electrons, 
the Hamiltonian is fully defined 
by the external potential $V_\textrm{ext}(\br)$, 
on which the resultant ground-state density depends. 
%
The first theorem sets out to prove that 
the converse of this statement is also true, 
that is, that the ground-state density 
uniquely determines the potential 
up to an additive constant.

\begin{HK1}
There is a one-to-one correspondence 
between the ground-state charge density 
of an $N$ electron system 
and the external potential acting upon it.
\end{HK1}
\noindent
We shall prove this statement by contradiction.
%
Consider two potentials 
that differ by more than an arbitrary constant
$\hat{V} - \hat{V}' \neq C$
with corresponding Hamiltonians 
$\hat{H}_\textrm{el}$ 
and $\hat{H}_\textrm{el}'$ 
that give rise to the  
eigenenergies $E_0$ and $E'_0$, 
and eigenstates $\ket{\phi_0}$ and $\ket{\phi_0'}$, 
respectively.
%
Suppose we assume 
$\ket{\phi_0}=\ket{\phi_0'}$.
%
By considering the difference in 
the Hamiltonians
\begin{equation}
(\hat{H}_\textrm{el}-\hat{H}_\textrm{el}')\ket{\phi_0}=(\hat{V}-\hat{V}')\ket{\phi_0}=(E_0-E_0')\ket{\phi_0}
\end{equation}
we demonstrate that 
$\hat{V}-\hat{V}'=E_0-E_0'$ 
and thus contradict our earlier assumption 
that the potentials differ by more than an 
additive constant.
%
Thus we have shown that the mapping 
from the space of potentials to the space 
of ground-state densities is injective.
%
It remains to be shown that the mapping 
from the space of densities to the space 
of eigenstates is also injective.
%
Consider then that two ground-states 
$\ket{\phi_0}$ and $\ket{\phi_0'}$
both give rise to the same 
ground-state density $n_0(\br)$.
%
We may then write, 
using Eq.~\eqref{eq:expectation_value}, 
that 
%
\begin{align}
E_0<\bra{\phi'_0}H\ket{\phi'_0}
=&\bra{\phi'_0}H'\ket{\phi'_0}+\bra{\Psi'_0}H-H'\ket{\phi'_0}\nonumber\\[0.3em]
=&E'_0 + \int d\br\ [V(\br)-V'(\br)] n(\br).
\label{eq:HK1}
\end{align}
%
Similarly, by computing the converse, we find {that}   
%
\begin{equation}
E'_0< E_0+\int\  n(\br) [V'(\br)-V(\br)]\ dr.
\label{eq:HK2}
\end{equation}
%
Summing \eqref{eq:HK1} and \eqref{eq:HK2} 
leads to the second contradiction
%
\begin{equation}
E'_0+ E_0<E'_0+ E_0,
\end{equation}
%
from which we conclude that the mapping 
from the space of eigenstates 
to the space of densities 
is also injective, 
thus proving the first 
HK theorem.
%
We now proceed to the second theorem.

%
\begin{HK1}
For all $V$-representable 
ground-state densities $n(\br)$, 
$E_0 \leq E_V[n]$, 
where $E_0=E_V[n_0]$ 
is the ground-state energy for $N$ electrons 
in the external potential $V(\br)$ 
with corresponding ground-state density $n_0(\br)$.
\end{HK1}
%
\noindent
By the first HK theorem, 
the ground-state density of a system $n(\br)$ 
uniquely determines the potential $V_\textrm{ext}(\br)$
and thus its wave function $\ket{\phi[n(\br)]}$, 
where the class of $V$-representable densities 
are those generated by solving the 
Schr\"{o}dinger equation with some external potential.
%
A uniqueness criterion must also apply 
to the internal-energy operator $\hat{F}$, 
which we define as the minimum possible expectation value 
under a given density $n(\br)$
by searching over the entire set of N-body anti-symmetric 
wave functions that give rise to that density.
%
{This is expressed mathematically as}
%
\begin{equation}
F[n]=\min_{n[\phi]\to n} {\bra{\phi}\hat{F}\ket{\phi}} = \bra{\phi_n}\hat{F}\ket{\phi_n}.
\label{eq:f_formula}
\end{equation}
%
Thus,  we may uniquely define the lowest total-energy 
of a given density $n(\br)$ 
subject to an external potential $V(\br)$, 
namely 
%
\begin{equation}
E_V[n]=F_V[n]+\int d\br\ V(\br)n(\br) = \bra{\phi_n}\hat{F}+\hat{V}\ket{\phi_n}.
\end{equation}
%
For an external potential, 
there exists a ground-state wave function $\ket{\phi_0}$ 
that produces the ground-state energy $E^0_V$ 
and density $n_0(\br)$, 
such that, by the variational principle, 
for any other N-representable densities, 
$E^0_V\leq E_V[n]$. 
%
Furthermore, for the ground-state density $n_0(\br)$ 
we find that, by definition, $F[n]$ 
can only be at most as large 
as the kinetic energy term 
of the ground-state total-energy 
that is determined by the 
ground-state wave function $\ket{\phi_0}$, 
such that 
%
\begin{equation}
F_V[n]=\min_{n[\phi]\to n_0} {\bra{\phi}\hat{F}\ket{\phi}} \leq \bra{\phi_0}\hat{F}\ket{\phi_0},
\end{equation}
%
which trivially implies the same for the total-energy 
%
\begin{align}
\bra{\phi_{n_0}}\hat{F}+\hat{V}\ket{\phi_{n_0}} 
&\leq \bra{\phi_0}\hat{F}+\hat{V}\ket{\phi_0} \nonumber \\[0.5em]
\Rightarrow E_V[n_0]&\leq E^0_V.
\end{align}
%
Since $E^0_V$ is the ground-state energy by definition, 
we have proven the second HK theorem, 
namely that is the ground-state density, 
subject to an external potential, 
is that which uniquely determines the 
minimum of the total-energy functional.

The Hohenberg-Kohn theorems illustrate 
that a full description of the ground-state 
of an interacting $N$ electron system  
can be distilled to the knowledge 
of the ground-state density alone.
%
Moreover, this ground-state density 
can be located via  minimisation 
of the total-energy functional $E[n]$ 
with respect to the density, 
thus drastically reducing 
the complexity of the problem. 
%
In practice, however, 
no such formalism exists to carry out 
this variational procedure 
as the exact form of 
the universal internal energy functional 
described in Eq.~\eqref{eq:f_formula} 
does not exist~\cite{CAPELLE2006} 
and approximations 
must be used in its place.
%
The formalism of Kohn and Sham, 
which we shall now describe, 
utilises intelligent approximations for 
the many-body effects contained in $\hat{F}$ 
in order to map the fully interacting $N$-particle system  
onto a fictitious system of $N$ non-interacting 
particles immersed in an effective potential.


%%%%%%%%%%%%%%%%%%%%%%
% THE KOHN-SHAM EQUATIONS
%%%%%%%%%%%%%%%%%%%%%%
\section{The Kohn-Sham equations}
\label{sec:kohn_sham_equations}

{Strategies for simplifying 
and then solving the many-body 
Schr\"{o}dinger equation},  
which strive to simultaneously capture  
the qualitative features of the system of interest 
and avoid treating many-body interactions explicitly, 
precede the HK theorems by almost four decades.
%
Indeed, 
soon after the formulation 
of the Schr{\"o}dinger equation, 
Hartree established an approximate, 
self-consistent field procedure 
for iteratively solving it 
for the atomic wave functions  
in what was the first endeavour toward 
an \emph{ab initio} approach~\cite{hartree_1928}.
%
Fock later contributed to the 
procedure~\cite{Fock1930,PhysRev.35.210.2} 
by imposing the antisymmetry 
of the wave function explicitly 
in order to satisfy the Pauli exclusion principle~\cite{PhysRev.58.716}, 
to form what is now known 
as the `Hartree-Fock' (HF) approach.
%
The exchange energy 
given in the HF theory, 
for particle exchange  
between pairs of orbitals, 
is then 
%
\begin{equation}
E_x[n]=-\frac{1}{2}\sum_{ij}\int d\br d{\bf r'}\ \frac{\psi_i^*(\br)\psi_j^*({\bf r'})\psi_i({\bf r'})\psi_j(\br)}{|{\bf r-r'}|}.
\label{eq:exact_exchange}
\end{equation}
%
Despite neglecting electron correlation effects, 
{which account for many of the 
discrepancies of the HF approach with experimental results, 
the scheme} remains to this day 
a valuable tool in quantum chemistry.
 
A far more viable approach was developed 
by Kohn and Sham (KS)~\cite{PhysRev.140.A1133}, 
who proposed mapping the interacting system onto a 
reference system comprising the same number of non-interacting particles 
whose ground-state density is identical to the interacting density  
by virtue of an effective potential $V_\textrm{KS}(\br)$.
%
The many-body problem is thereby circumvented 
by solving $N$ one-particle Schr\"{o}dinger equations 
instead of one $N$-particle equation.

Let us first consider the variational minimisation of 
the HK total-energy functional, 
subject to conservation of the particle number via
%
\begin{equation}
\delta\left[E[n]+\mu\left(N-\int d\br\ n(\br)\right)\right]=0.
\label{eq:hk_total_energy_functional}
\end{equation}
%
This gives rise to the Euler-Lagrange equation 
in terms of the functional derivative of the  
internal energy functional $F[n]$ 
with respect to the density $n(\br)$, 
the external potential $V_\textrm{ext}(\br)$, 
and the chemical potential $\mu$, namely 
%
\begin{equation}
\frac{\delta F[n]}{\delta n(\br)}+V_\textrm{ext}(\br)=\mu.
\label{eq:euler-lagrange}
\end{equation}
%

In the Kohn-Sham approach, 
the internal energy is expanded 
into its constituent single-particle terms  
%
\begin{equation}
F[n]\equiv T_s[n]+E_H[n]+E_\textrm{xc}[n]
\label{eq:ks_internal_energy}
\end{equation}
%
comprising the {non-interacting} 
kinetic energy %$T_s[n]$,
%
\begin{equation}
T_s[n]=-\frac{1}{2}\sum^{N}_{i=1}\int d\br\ \psi^*_i(\br)\nabla^2 \psi_i(\br),
\end{equation}
%
the classical Hartree energy %$E_H[n]$, 
%
\begin{equation}
E_H[n]=\frac{1}{2}\int\int d\br\ d{\bf r'}\ \frac{n(\br)n({\bf r'})}{|{ \bf r-r'}|},
\label{eq:classical_hartree_energy}
\end{equation}
%
and finally the exchange-correlation (XC) energy $E_\textrm{xc}[n]$, 
which encompasses all the quantum many-body effects 
ignored in the single-particle mapping, 
specifically the non-classical electron-electron interaction energy 
and the kinetic energy difference between the 
interacting and non-interacting systems\footnote{
see section~\ref{sec:xc_functionals} for further discussion}.
%
Substituting Eq.~\eqref{eq:ks_internal_energy} 
into Eq.~\eqref{eq:euler-lagrange} 
generates the following 
non-interacting Euler-Lagrange equation 
%
\begin{equation}
\frac{\delta T_s[n]}{\delta n(\br)}+V_\textrm{KS}(\br)=\mu, 
\label{eq:euler-lagrange2}
\end{equation} 
%
which, by virtue of the first HK theorem, 
is guaranteed to reproduce 
the exact ground-state density 
of the interacting system 
subject to the effective 
\emph{Kohn-Sham potential} 
%
\begin{equation}
V_\textrm{KS}(\br)=
\int d\br'\ \frac{n(\br')}{|\br-\br'|}
+\frac{\delta E_\textrm{xc}[n]}{\delta n(\br)}
+V_\textrm{ext}(\br)
\label{eq:ks_schroding_equation}
\end{equation}
%
with the exchange-correlation potential given by
%
\begin{equation}
V_\textrm{xc}(\br)[n]=\frac{\delta E_\textrm{xc}[n]}{\delta n(\br)}.
\end{equation}
%
{
The solution to 
the non-interacting Euler-Lagrange equation in 
Eq.~\eqref{eq:euler-lagrange2} is achieved  
by resolving $N$ independent Schr\"{o}dinger equations}
%
\begin{equation}
\hat H_{KS} \psi_i=\left[-\frac{1}{2}\nabla^2+\hat{V}_{KS}(\br)\right]\psi_i= \epsilon_i\psi_i,
\label{eq:ks_hamiltonian}
\end{equation}
%
{acting on a set of 
non-interacting eigenfunctions $\{\psi_i(\br)\}$  
termed the \emph{Kohn-Sham orbitals}.}
%
{The KS orbitals then reconstruct}  
the interacting ground-state density via 
%
\begin{equation}
n(\br)=\sum_{i=1}^N \int d\br \left| \psi_i(\br)\right|^2.
\end{equation}
%
%{with the total non-interacting kinetic energy 
%given by the summation over all contributions 
%from the orbitals} 
%%
%\begin{equation}
%T_s[n]=-\frac{1}{2}\sum^{N}_{i=1}\int d\br\ \psi^*_i(\br)\nabla^2 \psi_i(\br).
%\end{equation}


Since the KS potential produces the KS orbitals, 
which construct the density, 
which in turn determines the potential, 
the solutions to \eqref{eq:ks_hamiltonian} 
{may be found by iteratively solving for the potential and orbitals 
until self-consistency is reached~\cite{martin2004electronic}, 
or by direct-minimisation of the total-energy~\cite{skylaris2005introducing}, 
which we will describe in greater detail in section~\ref{sec:linear_scaling_dft}.} 
%
Furthermore, while the KS orbitals and eigenvalues 
may often provide an approximate representation of 
the physical electronic orbitals and related spectra, 
great care must be exercised in interpreting them as such,  
since they represent a fictitious system 
of non-interacting particles. 
% 
{We shall address this issue further 
in section~\ref{sec:koopmans_condition}.}

%%%%%%%%%%%%%%%%%%%%%%
% EXCHANGE-CORRELATION FUNCTIONALS
%%%%%%%%%%%%%%%%%%%%%%
\section{Exchange-correlation functionals}
\label{sec:xc_functionals}

Were the closed-form 
of $E_\textrm{xc}[n]$ known, 
then the KS approach described heretofore 
would be exact 
and resolve the ground-state density 
to arbitrary precision, 
thereby elevating approximate-DFT 
to the status of an exact theory. 
%
As the case may be, 
$E_\textrm{xc}(\br)$ is unknown 
except for the simplest of systems~\cite{PhysRevA.93.042511,PhysRevA.83.062512,Matito2016,PhysRevLett.109.036402}, 
and the ability of the KS approach 
to reliably incorporate many-body effects 
and reproduce realistic ground-state properties 
lies in its {appropriate approximation}~\cite{PhysRev.140.A1133}.

As the name suggests, 
$E_\textrm{xc}[n]$ 
may be {partitioned} into effects 
pertaining to exchange $E_\textrm{x}[n]$ 
and correlation $E_\textrm{c}[n]$~\cite{CAPELLE2006}.
%
The exchange term is required to satisfy 
the Pauli exclusion principle~\cite{PhysRev.58.716} 
and lowers the Coulomb repulsion 
between like-spin electrons 
by keeping them spatially separated.
%
{The correlation energy, meanwhile, 
has many interpretations~\cite{CAPELLE2006,doi:10.1021/jacs.5b12434}, 
the most tangible of which is that given 
by Perdew and Zunger in Ref.~\cite{PhysRevB.23.5048}.
%
Given the functional for exact exchange in 
Eq.~\eqref{eq:exact_exchange}, 
the correlation energy is uniquely defined 
as the remainder when all other energy terms 
have been removed from the total-energy, 
such that 
%
\begin{equation}
E_c=E[n]-T_s[n]-E_H[n]-\int d\br\ V_\textrm{ext}n(\br)-E_x[n].
\end{equation}
}\vspace{-1em}
%is the difference in total-energy 
%derived from the exact many-body wave function 
%and that obtained from the wave function 
%{constructed from} of a KS single-particle Slater determinant.
%%
%It may also be considered as the energy reduction 
%incurred by the mutual avoidance of interacting 
%electrons ignored in the non-interacting formalism.}

Initially, Kohn and Sham sought to use a 
\emph{local-density approximation} (LDA)~\cite{PhysRev.140.A1133,PhysRevLett.45.566}
where the XC potential  
is approximated to that produced by
a homogeneous electron gas (HEG), 
whose density is the same as the non-interacting system  
in each infinitesimal volume element $d\br$ 
%
\begin{align}
E_\textrm{xc}^{\mbox{\tiny LDA}}[n] 
&\equiv \int d\br\ \epsilon_{xc}^{\mbox{\tiny LDA}}(n(\br))n(\br)\nonumber \\[1em]
\Rightarrow V_\textrm{xc}^{\mbox{\tiny LDA}}(\br)
=\frac{\delta E_\textrm{xc}^{\mbox{\tiny LDA}}}{\delta n(\br)}
&=\epsilon_\textrm{xc}^{\mbox{\tiny LDA}}\left[n(\br)\right]
+ n(\br)\left.\frac{d\epsilon_\textrm{xc}^{\mbox{\tiny LDA}}\left[n(\br)\right]}{dn}\right|_{n=n(\br)},
\end{align}
%
where, $\epsilon_\textrm{xc}^{\mbox{\tiny LDA}}(n(\br))$ 
is the XC energy per-electron 
of the HEG with density $n(\br)$.
%
%It is possible to write $E_{xc}[n]$ in the form of a classical interaction between a charge-depletion (or hole) density $\bar n_{xc}({\bf r,r'})$ and the density of the inhomgeneous electron density that created it $n(\br)$ such that
%\begin{equation}
%\epsilon_{xc}[n]=\frac{1}{2}\int d{\bf r'}\ \frac{\bar n_{xc}({\bf r,r'})}{|{\bf r-r'}|}.
%\end{equation}
%$\bar n_{xc}({\bf r,r'})$ is calculated via coupling constant integration by varying the electronic coupling from $0\to e$ while keeping the density fixed, and, when integrated over ${\bf r'}$, obeys the sum rule
%\begin{equation}
%\int d{\bf r'}\ \bar n_{xc}({\bf r,r'})=-1.
%\label{sum_rule}
%\end{equation}
It was later extended to incorporate 
the local-spin density approximation 
(LSDA)~\cite{0022-3719-5-13-012,PhysRevB.7.1912,PhysRevB.45.13244}, 
for which a commonly used functional 
is that developed 
by Perdew and Zunger~\cite{0022-3719-5-13-012,PhysRevB.23.5048}.

However, the LDA  and LSDA  
suffer from systematic inaccuracies 
in insulating band gaps~\cite{PhysRevB.93.195208,PhysRevLett.111.073003}, 
charge-transfer energies~\cite{PhysRevLett.95.146402,:/content/aip/journal/jcp/126/20/10.1063/1.2743004},
activation barriers~\cite{doi:10.1021/jp049908s},
 binding and formation energies,  
as well as in spin-densities and their moments.
%
The {\it generalised gradient approximation} 
(GGA)~\cite{PhysRevLett.77.3865} 
is a semi-local extension to the LDA  
that takes into account spatial variations 
in the density 
%
\begin{equation}
E_\textrm{xc}^{\mbox{\tiny GGA}}[n]=\int d\br\ \zeta(n(\br),|\nabla n(\br)|,\nabla^2n(\br),\cdots) 
\label{eq:gga}
\end{equation}
%
and offers modest improvement 
of the LDA in many systems.
%
Indeed, the Perdew-Burke-Ernzerhof (PBE)~\cite{PhysRevLett.77.3865} 
functional is one that is widely used in calculations 
as it produces reliable results for a broad range of applications 
and is the XC functional 
primarily used in this dissertation.

%There are loads more approximations
%Jacobs ladder
Over the years there have been numerous other 
incremental improvements to the LDA functional, 
of which GGA was only the first, 
that offer increasingly accurate, 
albeit more computationally expensive, 
results.
%
The next development   
were meta-GGA functionals~\cite{PhysRevLett.91.146401}, 
which depend on the density, its gradient 
and the kinetic energy density.
%
These were followed by 
hybrid-functionals~\cite{doi:10.1063/1.464304},  
which contain a linear combination of 
traditional XC functionals with exact exchange from HF 
and are orbital dependent, 
a popular choice of which is B3LYP.
%
{The next instalment} 
admits functionals including non-local correlation effects~\cite{B608478H}, 
and finally,  
functionals currently in development include all 
occupied and unoccupied 
orbitals~\cite{JCC:JCC23391}.

%%
%This hierarchy of functionals, 
%dubbed {\it Jacob's ladder},  
%culminates in the so-called 
%`Heaven of Chemical Accuracy', 
%where it is envisioned a 
%functional with universal utility 
%may be designed~\cite{Car2016}.
%
As of yet however, 
no such functional has been developed 
that can be universally employed 
that delivers reliable results 
for minimal computational cost. 
%
A reasonable balance 
between the two must be struck 
depending on the calculation to be performed.
%
Notwithstanding the current 
selection %suite 
of XC functionals available, 
{the GGA remains widely popular 
due to its reasonable accuracy 
and minimal expense.
% 
While meta-GGAs and hybrid functionals 
provide superior accuracy, 
they do so at considerably more expense}.
%for treating molecules~\cite{doi:10.1063/1.4948636}.
%
The popular LDA and GGA functionals 
are also limited in their capacity to describe 
systems with exotic electronic behaviour, 
and are known to suffer from systematic  
self-interaction and static-correlation errors, 
which will be discussed in Chapter~\ref{ch:self_interaction_error}.


%%%%%%%%%%%%%%%%%%%%%%
% ION-CORE PSEUDOPOTENTIALS
%%%%%%%%%%%%%%%%%%%%%%
\section{Treatment of the ionic potential}
\label{sec:ion_cores}

In the process of chemical bonding, 
the role played by the valence electrons 
greatly outweighs that of the core electrons,  
whose contribution to the interaction is so little 
they may regarded as effectively inert 
to the chemical environment~\cite{RASSOLOV2001573}.
%
{
Since the changes in energy are 
largely attributed to the valence electrons, 
the removal of the core states 
will allow these changes to 
be more accurately calculated, 
as they will constitute a 
larger proportion of the total-energy.} 
%
{
Furthermore, 
the condition of orthonormality  
between all non-interacting KS eigenstates 
requires rapidly oscillating 
valence wave functions 
in the region of the spatially-localised core states.}
% 
Satisfying this condition requires 
a large number of basis functions, 
and thus computational expense, 
to accurately describe them.

%
Motivated by the above observations, 
we now proceed to describe the two methods 
used in this dissertation 
for efficiently treating the ion core states, 
namely the {\it pseudopotential} 
approximation~\cite{HEINE19701,PhysRev.112.685,PhysRev.116.287}
and {\it projector-augmented wave} method~\cite{PhysRevB.50.17953,PhysRevB.59.1758}.
%
These techniques 
{allow for greater accuracy 
in determining total-energy differences}, 
and reduce 
the computational expense in 
treating the localised core states, 
while retaining the explicit 
treatment of the more 
chemically relevant valence states.

%NC PSEUDOS
\subsection{Norm-conserving pseudopotentials}
%
The pseudopotential approximation 
is a method 
in which explicit treatment of the core states 
can be effectively avoided in the 
electronic structure problem.
% 
{Instead, 
they may be} considered part of 
the external nuclear potential.
%
Consequently, the Coulomb potential 
is mitigated by a repulsive 
term that mimics the presence of 
the core electrons, 
{and this effect produces} a much weaker 
pseudopotential.
%
The rapidly oscillating wave functions 
are thus replaced by smoothly varying 
\emph{pseudo-wave functions} 
%that matches the exact wave function 
%beyond some pre-determined cutoff radius 
that can be resolved at a significantly reduced 
computational expense.
%

Following Ref.~\cite{PhysRev.57.1169}, 
let us suppose that a valence state $\ket{\psi_\textrm{val}}$, 
which is an eigenstate of the KS Hamiltonian 
$\hat{H}$ with eigenvalue $E$, 
may be transformed into a smoothly varying pseudo-state 
$\ket{\psi_\textrm{ps}}$ 
by subtracting a linear combination of core states 
%
\begin{equation}
\ket{\psi_\textrm{val}}=\ket{\psi_\textrm{ps}} + \sum_n^\textrm{core} a_n \ket{\chi_n}.
\end{equation}
%
The set of coefficients $\{a_n\}$ 
is then determined by the condition of orthogonality 
between the valence state and all core states
%
\begin{align}
\braket{\chi_n}{\psi_\textrm{val}}&=0=\braket{\chi_n}{\psi_\textrm{ps}}+a_n\\[0.5em]
\Rightarrow a_n&=-\braket{\chi_n}{\psi_\textrm{ps}}
\end{align}
thus, we arrive at 
\begin{equation}
\ket{\psi_\textrm{val}}=\ket{\psi_\textrm{ps}} - \sum_n^\textrm{core}\ket{\chi_n} \braket{\chi_n}{\psi_\textrm{ps}}.
\end{equation}
%
Solving the eigenvalue equation 
$\hat{H}\ket{\psi_\textrm{val}}=E\ket{\psi_\textrm{val}}$ 
we find that 
%
\begin{equation}
\hat{H}\ket{\psi_\textrm{ps}} - \sum_n^\textrm{core}E_n \ket{\chi_n} \braket{\chi_n}{\psi_\textrm{ps}} 
=E\ket{\psi_\textrm{ps}} 
- E\sum_n^\textrm{core} \ket{\chi_n} \braket{\chi_n}{\psi_\textrm{ps}}, 
\end{equation}
%
which may be conveniently re-written 
into an eigenvalue equation for the pseudo-state
%
\begin{equation}
\left[\hat{H}+ \sum_n^\textrm{core}(E-E_n) \ket{\chi_n} \bra{\chi_n}\right]{\psi_\textrm{ps}} 
=E\ket{\psi_\textrm{ps}}. 
\end{equation}
%
Hence, the pseudo-state returns 
the same eigenvalue as the valence state 
when the Hamiltonian differs by a non-local potential, 
given by
%
\begin{equation}
\hat{V}_\textrm{nl}= \sum_n^\textrm{core}(E-E_n) \ket{\chi_n} \bra{\chi_n}.
\end{equation}
%
{This potential is repulsive in the region of the core 
and acts to partially cancel the Coulomb attraction.
%
It thereby provides a net, smoothly-varying potential 
for the pseudo-valence states.}

The energy argument $E$ of the pseudopotential, 
the core states $\{\ket{\chi_n}\}$, and eigenenergies $E_n$ 
are initially calculated for an isolated atom and 
generally assumed to be fixed.
%
The validity of this assumption 
largely determines 
the {\it transferability} of a pseudopotential, 
that is, the accuracy with which it performs in various 
chemical environments.
%
This greatly depends on the condition 
that any shift in the valence energy $\Delta E$, 
when transferring from the atomic to chemical environment, 
satisfies $E-E_n\gg \Delta E$.

A widely used variety of pseudopotentials 
are those that preserve the scattering properties 
of the full atomic potential, 
up to leading order in the energy, 
by use of 
{\it norm-conservation}.
%
We refer the reader to Refs.~\cite{PhysRevLett.43.1494,0022-3719-13-9-004,PhysRevB.40.2980,PhysRevB.41.1227} 
for more details.
%
These pseudopotentials 
must be independently computed for each of 
the angular momentum states $l$ but, 
for spherically symmetric potentials, 
are independent of 
the azimuthal quantum number $m$. 
%
Suppose we then construct a pseudopotential 
where the contribution due to core electrons 
vanishes beyond a cutoff radius $r_c$ 
{such that $v_\textrm{nl}(r)=v(r) $ 
for $r<r_c$ and $v_\textrm{nl}(r)=0$ 
otherwise.}
%
%\begin{equation}
%v_\textrm{nl}(r)
%=\left\{\begin{array}{c}
%v(r)  \quad\mbox{for}\quad  r<r_c \\ 
%0   \quad\mbox{for}\quad  r>r_c.
%\end{array}\right.
%\end{equation}
%
The valence wave functions may also be 
decomposed into radial $R_l$ and 
spherical harmonic $Y_{lm}$ terms, 
given by
%
\begin{equation}
\psi_{lm}(\br, E)= R_l(r, E)Y_{lm}(\theta, \phi) 
\end{equation} 
that are solutions of 
the atomic Schr{\"o}dinger equation
%
\begin{equation}
\left[-\frac{1}{2}\nabla^2+v(\br)\right]\psi_{lm}(\br, E)=E\psi_{lm}(\br, E).
\label{eq:schro_equation_reduced}
\end{equation}
%
For $r>r_c$, 
the radial solution may be expressed 
in terms of  Bessel $j_l(x)$ 
and von Neumann $n_l(x)$ 
functions
%
\begin{equation}
R_l^{>}(r,E)=A_l j_l(kr)+B_l n_l(kr) = A_l[j_l(kr)-\tan(\delta_l)n_l(kr)], 
\end{equation}
%
where $k=\sqrt{2E}$ 
and the scattering phase shift $\delta_l$ 
is defined by  
$B_l/A_l=-\tan(\delta_l)$.
%
Further stipulating that $R_l^{>}$   
and its radial derivative 
match the solution of 
Eq.~\eqref{eq:schro_equation_reduced} 
at the boundary $r=r_c$,
one can then show that
%
\begin{equation}
\left.\frac{d}{dr}\log\left[R_l(\br, E) \right]\right|_{r=r_c} 
= k\frac{j_l(kr_c)-\tan(\delta_l)n_l(kr_c)}{j_l(kr_c)-\tan(\delta_l)n_l(kr_c)}
\label{eq:norm_condition_1}
\end{equation}
%
and, moreover, that the energy derivative is proportional 
to the norm of the wave function in the core region, i.e., {that} 
%
\begin{equation}
\frac{d}{dE}\left[\frac{d}{dr}\log\left[R_l(\br, E) \right]\right] \sim 
\int_0^{r_c}dr\ r^2 R_l^2(r).
\label{eq:norm_condition_2}
\end{equation}
%
Simultaneously satisfying 
Eqs.~\eqref{eq:norm_condition_1}~\&~\eqref{eq:norm_condition_2},  
such that 
%
\begin{equation}
R_{l,\,\textrm{ps}}(l)=R_{l,\,\textrm{val}}(l) 
\quad\mbox{for}\quad
r>r_c,
\end{equation}
%
we can construct a pseudopotential 
that preserves the scattering 
properties of the full potential 
up to leading order in $E$.

{
In the construction of pseudopotentials, 
it is necessary to remove the contributions from the 
valence electron Hartree and XC potentials, 
{as these will be accounted for in the DFT calculation 
according to the specific chemical environment.}
%
Thus, we leave behind only the XC of the core states  
%
\begin{equation}
\hat{V}_\textrm{xc}^\textrm{pseudo}=\hat{V}_\textrm{xc}[n_v(\br)+n_c(\br),\zeta] -\hat{V}_\textrm{xc}[n_v(\br),\zeta_v] 
\label{eq:xc_pseudo}
\end{equation}
%
where $n_v(\br)$ and $n_c(\br)$ 
are the valence and core electron densities, respectively, 
which define the all-electron and valence magnetisations as 
%
\begin{equation}
\zeta(\br)=\frac{n^\uparrow_v(\br)-n^\downarrow_v(\br)}{n_v(\br)+n_c(\br)}
\quad\mbox{and}\quad
\zeta_v(\br)=\frac{n^\uparrow_v(\br)-n^\downarrow_v(\br)}{n_v(\br)}.
\end{equation}
%
%{Upon performing a DFT calculation, 
%the XC potential for the valence states 
%is computed explicitly 
%and remitted to the ion-core contribution 
%to compute the total XC potential}  
%%
%\begin{equation}
%\hat{V}_\textrm{xc}[n_v(\br)+n_c(\br),\zeta] = \hat{V}_\textrm{xc}^\textrm{pseudo}+\hat{V}_\textrm{xc}[n_v(\br),\zeta_v].
%\end{equation}
%%
Thus, 
in the pseudopotential approximation, 
it is important to consider 
the degree of spatial overlap 
between the valence and core electron densities, 
especially for spin-polarised systems,
{such as} the first-row transition-metals.

Since the $V_\textrm{xc}$ functional is non-linear in 
the charge density and magnetisation,   
linear approximations, (such as those in Eq.~\eqref{eq:xc_pseudo}) 
although sufficient for many cases,  
will fail when $\zeta$ and $\zeta_v$ significantly differ.
%
The non-linear core correction (NLCC), 
developed by Louie {\it et al.}~\cite{PhysRevB.26.1738}, 
circumvents this problem in a DFT calculation 
by explicitly computing the XC of both the 
valence states and so-called {\it partial core} $\tilde{n}_c(\br)$, 
which is identical to the core density 
outside some cutoff radius $r_\textrm{NLCC}$ 
and a smoothly varying function within.
%
{
Thus, the XC contribution from the pseudopotential 
and valence electrons can be effectively partitioned 
in an approximately linear fashion once again}
% 
\begin{equation}
\hat{V}_\textrm{xc}^\textrm{pseudo}=\hat{V}_\textrm{xc}[n_v(\br)+n_c(\br),\zeta] -V_\textrm{xc}[n_v(\br)+\tilde{n}_c(\br),\zeta_v].
\label{eq:xc_pseudo2}
\end{equation}
}


%PAW
\subsection{Projector-augmented wave method}
The projector-augmented wave (PAW) method, 
devised by Bl{\"o}chl~\cite{PhysRevB.50.17953}, 
is a generalisation of the 
pseudopotential~\cite{HEINE19701,PhysRev.112.685,PhysRev.116.287}
and linear-augmented plane wave (LAPW) 
methods~\cite{0305-4608-5-11-016,PhysRevB.12.3060}, 
which combines the simple formalism of the former 
with the versatility of the latter.
%
In this approach, 
instead of using a linear combination of core states, 
the all-electron KS 
wave function\footnote{not to be confused with the many-electron wave function.} 
$\ket{\Psi}$ 
is derived from the pseudo  
wave function 
$\ket{\Psi_\textrm{ps}}$ by a linear transformation $\mathcal{T}$ 
%
\begin{equation}
\ket{\Psi}=\mathcal{T}\ket{\Psi_\textrm{ps}}.
\label{eq:paw_1}
\end{equation}
%
Expectation values of 
operators $\hat{A}$ 
can be obtained from either the all-electron states 
$\langle \hat{A}\rangle=\bra{\Psi}\hat{A}\ket{\Psi}$ 
or, equivalently, from the pseudo-states 
$\langle \hat{A}\rangle=\bra{\Psi_\textrm{ps}}\hat{A}'\ket{\Psi_\textrm{ps}}$ 
with 
$\hat{A}'=\mathcal{T}^\dagger \hat{A} \mathcal{T}$, 
when $\mathcal{T}$ is known.
%

In order to restrict the transformation 
to the regions of the ion core, 
we may expand $\mathcal{T}$ in 
terms of a sum of local contributions $\mathcal{T}_R$ 
%
\begin{equation}
\mathcal{T}=1+\sum_R \mathcal{T}_R, 
\end{equation}
%
which only act within an {\it augmentation region} 
$\mathcal{V}_R$ enclosing the atom, 
analogous to the cutoff radius 
in the pseudopotential method.
%
It is useful to re-cast the pseudo wave functions 
as a linear combination of 
partial-waves $\{\ket{\phi_{i,\textrm{ps}}}\}$ 
in $\mathcal{V}_R$ %the augmentation region
%
\begin{equation}
\ket{\Psi_\textrm{ps}}=\sum_i c_i\ket{\phi_{i,\textrm{ps}}}, 
\label{eq:paw_2}
\end{equation}
%
which, themselves, may be a linear combination of 
polynomials or Bessel functions.
%
These, in turn, are related to the 
all-electron partial-waves by %$\mathcal{T}$ 
the same linear transformation 
%
\begin{equation}
\ket{\phi_i}=\mathcal{T}\ket{\phi_{i,\textrm{ps}}}, 
\label{eq:paw_3}
\end{equation}
%
and are typically the solutions 
to the atomic Kohn-Sham Schr{\" o}dinger equation.
%
Thus, $i$ is an index over the 
atomic labels $R$, 
angular momentum quantum numbers $\{l,m\}$, 
and partial-wave index $n$.
%
Combining Eqs.~\eqref{eq:paw_1}, \eqref{eq:paw_2} and~\eqref{eq:paw_3}, 
the all-electron wave function can then 
be resolved from the pseudo wave function
%
\begin{equation}
%\ket{\Psi}=\mathcal{T}\ket{\Psi_\textrm{ps}}
%=\mathcal{T}\sum_i \ket{\phi_{i,\textrm{ps}}}c_i
%=\sum_i \ket{\phi_i}c_i \nonumber \\
%\Rightarrow 
\ket{\Psi}=\ket{\Psi_\textrm{ps}} - \sum_i \ket{\phi_{i,\textrm{ps}}}c_i + \sum_i \ket{\phi_i}c_i, 
\label{eq:paw_4}
\end{equation}
%
and the expansion coefficients $\{c_i\}$, 
yet to be determined.
%
Since the transformation $\mathcal{T}$ must be linear, 
the coefficients $\{c_i\}$ 
must be derived from a scalar product 
of $\ket{\Psi_\textrm{ps}}$ 
with some projector functions $\{\ket{p_i}\}$
%
\begin{equation}
c_i=\braket{p_i}{\Psi_\textrm{ps}} 
\quad\mbox{with}\quad
\braket{p_i}{\phi_{j,\textrm{ps}}}=\delta_{ij}
\label{eq:paw_5}
\end{equation}
%
that each have a corresponding pseudo partial-wave 
and satisfies the completeness condition 
$\sum_i\ket{\phi_{i,\textrm{ps}}}\bra{p_i}=1$.
%
Thus, inserting the first term in Eq.~\eqref{eq:paw_5} 
into Eq.~\eqref{eq:paw_4} 
we arrive at the full expression for the all-electron wave function 
\begin{equation}
\ket{\Psi}=\ket{\Psi_\textrm{ps}} + \sum_i\left(\ket{\phi_i}-\ket{\phi_{i,\textrm{ps}}}\right)\braket{p_i}{\Psi_\textrm{ps}},  
\end{equation}
%
where we can now define the transformation as 
%
\begin{equation}
\mathcal{T}=1+\sum_i\left(\ket{\phi_i}-\ket{\phi_{i,\textrm{ps}}}\right)\bra{p_i}.
\label{eq:paw_6}
\end{equation}

The quantities that govern the 
transformation are 
%
\begin{enumerate}
\item
The set of all-electron partial-wave functions $\{\ket{\phi_i}\}$, 
which are solutions to the atomic Schr{\" o}dinger equation, 
%
\item
the set of pseudo partial-wave functions $\{\ket{\phi_{i,\textrm{ps}}}\}$, 
%
\item
the set of projector functions $\{\ket{p_i}\}$ 
localised in the augmentation region, 
which satisfy $\braket{p_i}{\phi_{j,\textrm{ps}}}=\delta_{ij}$.
\end{enumerate}
%
There exists a variety of strategies to 
solve for each of these terms, 
which are discussed in Refs.~\cite{PhysRevB.50.17953,PhysRevB.71.035109}

On a final note, we will briefly outline  
how the pseudopotential method 
may be derived from the PAW method.
%
For a comprehensive derivation 
see Ref.~\cite{PhysRevB.59.1758}.
%
By construction, 
the PAW method fulfils the 
same essential criteria as the 
pseudopotential approach, 
except for the condition of norm-conservation, 
although this can also be enforced.
%
Using our definition of $\mathcal{T}$ in Eq.~\eqref{eq:paw_6}, 
it can be shown that 
a pseudo operator $\hat{A}_\textrm{ps}$ 
is given by
%
\begin{align}
\hat{A}_\textrm{ps}=\hat{A} 
+ \sum_{i,j} \ket{p_i}\left(\bra{\phi_i}\hat{A}\ket{\phi_j} 
- \bra{\phi_{i,\textrm{ps}}}\hat{A}\ket{\phi_{j,\textrm{ps}}}\right)\bra{p_j}.
\end{align}
%
However, there is the option to include 
an additional term of the form 
%
\begin{equation}
\hat{B}-\sum_{i,j} \ket{p_i}\bra{\phi_{i,\textrm{ps}}}\hat{B}\ket{\phi_{j,\textrm{ps}}}\bra{p_j},
\end{equation}
%
where $\hat{B}$ is an operator 
localised to the augmentation region,  
such that the expectation value 
of this new term with 
$\ket{\Psi_\textrm{ps}}$ is zero.
%
This additional degree of freedom 
is useful for constructing 
potentials that are difficult to 
express using a plane wave expansion, 
such as the nuclear Coulomb potential, 
due to the singularity inside 
the augmentation region.
%
This new potential will inherently match  
the exact potential outside the core region 
and will be smoothly-varying inside, 
thus providing the framework 
for the pseudopotential approach in PAW.
%
As a result of reducing the PAW method 
to the pseudopotential approach, 
a contribution from an additional 
non-local potential $\hat{V}_\textrm{nl}$ 
emerges during the process.
%


In summary, the PAW method 
offers numerous advantages over the 
pseudopotential approach, 
while requiring only a modest increase 
in computational resources, 
albeit larger memory requirements.
%
It allows for increased computational 
accuracy and efficiency, 
larger core regions, 
and a reduced basis set 
if the norm-conserving condition is relaxed. 
%
{
Most importantly, 
by using both valence and core 
states in the self-consistency cycle 
and the construction of potentials, 
the PAW method bears a physical basis  
approaching that of more expensive `all-electron' models, 
but at the price of a pseudopotential calculation.}
%
{
Moreover, 
for norm-conserving pseudopotentials without NLCC, 
the {\bk}-point grid-spacing for the density
needs to be at least twice as fine 
as the grid-spacing for the wave function 
in order to guarantee adequate sampling 
and ensure that the two converge at 
a similar same rate.
%
When NLCC or PAW are used, 
this equivalence is no longer guaranteed to hold 
because of the additional complexities 
introduced by the core electron density 
or augmentation regions.}



%%%%%%%%%%%%%%%%%%%%%%
% LINEAR SCALING DFT
%%%%%%%%%%%%%%%%%%%%%%
\section{An overview of linear-scaling DFT}
\label{sec:linear_scaling_dft}

{
Despite the practicality of approximate DFT 
that has lead to its widespread adoption,  
there exists, within the electronic structure community, 
a growing repertoire of large-scale systems under study 
($N_\textrm{atoms}\gtrsim10^3$ )
that are computationally inaccessible 
by conventional codes 
and oblige a linear-scaling,
so-called $\mathcal{O}(N)$, treatment.
%
These include 
large biological molecules~\cite{skylaris2005introducing,doi:10.1021/jz3004188,Weber22042014,doi:10.1021/jz5018703,PROT:PROT24686},
defect states~\cite{PhysRevB.79.024112,hine2010linear,PhysRevB.84.035209}, 
optical absorption~\cite{PhysRevB.84.165131,C3CP52043A}, 
molecular dynamics~\cite{doi:10.1063/1.4978684,PhysRevB.83.195102}, 
molecules in solution~\cite{doi:10.1021/acs.jctc.5b00391,0295-5075-95-4-43001},  
among many other examples~\cite{PhysRevLett.108.256402,HINE20091041,doi:10.1063/1.4817001,0953-8984-28-7-074003}.}

{
In this section, 
we will proceed from 
the over-arching theory 
and techniques used in approximate DFT 
and discuss the motivation 
and technical details   
related to solving 
Eq.~\eqref{eq:ks_schroding_equation}
within a linear-scaling regime.
% 
For this dissertation, 
we primarily used the 
Order-$N$ Electronic Total-Energy Package 
(\textsc{ONETEP})~\cite{skylaris2005introducing,0953-8984-17-37-012,PSSB:PSSB200541457,0953-8984-20-29-294207,0953-8984-20-6-064209,HINE20091041,hine2009linear,PhysRevB.85.085107}, 
which employs direct-minimisation 
of the total-energy, 
rather than plane wave self-consistency, 
as a means to locate 
the ground-state energy.
%
We will describe the optimisation procedure that 
occurs within \textsc{ONETEP} 
affording it $\mathcal{O}(N)$ functionality  
(with an accuracy comparable to 
that of plane wave 
codes~\cite{skylaris2005introducing,PhysRevB.51.10157,PhysRevLett.69.3547,hine2009linear}),
and define the central quantities 
of the calculations.
%
For a comprehensive account  
of the \textsc{ONETEP} procedure, 
we refer the reader to 
Refs.~\cite{skylaris2005introducing,0953-8984-17-37-012,PSSB:PSSB200541457,0953-8984-20-29-294207,0953-8984-20-6-064209,HINE20091041,hine2009linear,PhysRevB.85.085107}.}

{
In this section,
we shall invoke the Einstein summation convention 
over repeated indices~\cite{Einstein1916}, 
where {we have denoted} 
Greek suffixes {to correspond} 
to non-orthogonal quantities, 
and Latin suffixes to orthogonal ones.}


%DENSITY MATRIX FORMULATION
\subsection{Density matrix formalism in DFT}
\label{sec:density_matrix}
{
The real-space single-particle density matrix 
is given in terms of the KS orbitals $\psi_i(\br)$ 
with occupancies $\{f_i\}$ by 
%
\begin{equation}
\rho(\br,\br')=\bra{\br}\hat{\rho}\ket{\br'}
=\sum_i f_i\; \psi_i(\br)\psi_i^*(\br')
\quad\mbox{where}\quad
f_i=\{0,1\}, 
\label{eq:1_particle_dens_matrix1}
\end{equation}
%
with the single-particle density 
$n(\br)=\rho(\br, \br)$.
%
A physically meaningful 
density-matrix must 
abide by certain properties, 
namely, it must be 
%
\begin{enumerate}
\item
Hermitian $\rho(\br,\br')=\rho^\dagger(\br,\br')$, 
to ensure real expectation values, 
\item
conserve (spin-)particle number 
$\int d\br\ \hat\rho^\sigma(\br,\br)=N^\sigma$,
\item
idempotent 
$\rho(\br,\br')=\rho(\br,\br')^2$,  
to  ensure that 
$ f_i =\{0,1\}$.
\end{enumerate}}

{
In large DFT calculations comprising $N\gtrsim10^2$ particles, 
enforcing orthogonality between the KS orbitals 
incurs an $\mathcal{O}(N^3)$ compute-time.
%
Amid a growing appetite in the 1990s 
for solutions to surmounting 
this performance limitation~\cite{PhysRevLett.66.1438,
doi:10.1063/1.470549,
PhysRevLett.69.3547,
KOHN1993167,
PhysRevB.51.1456,
PhysRevB.47.9973,
PhysRevB.47.10891}, 
Kohn postulated a {\it near-sightedness} principle, 
based on the dependence of 
the one-particle density matrix on local-potentials, 
which forecast the development of 
linear-scaling methods~\cite{PhysRevLett.76.3168}.
%
A corollary of this principle was that 
expectation values 
of local operators depend negligibly  
on spatially-distant density elements, 
which could be truncated accordingly 
beyond some cutoff distance $r_\textrm{cut}$ 
(to be determined by total-energy convergence), 
such that 
%
\begin{align}
%&\rho(\br,\br')\sim \exp(-|\br-\br'|) \nonumber\\
%\Rightarrow\ 
\rho(\br,\br')=0
\quad\mbox{for}\quad 
|\br-\br'|> r_\textrm{cut}.
\end{align}}
%
{The conditions presented above 
pose numerous challenges 
to direct-minimisation techniques 
invoking the density-matrix. 
%
We shall not discuss in detail the strategies employed  
by {\sc ONETEP} to ensure a well-behaved  
density-matrix, but we refer the reader 
to Refs.~\cite{RevModPhys.32.335,0953-8984-20-29-294207,PhysRevB.47.10891,PhysRevB.50.17611,PhysRevB.47.10895,PhysRevB.18.7165} 
for more information.}
%%%%%%%%%%%%%


%PSINC BASIS SETS
\subsection{Psinc basis sets}
{
The tractable computation 
of the KS wave functions 
relies on the appropriate 
selection of a finite set of basis functions 
spanning the Hilbert space.
%
This selection should be minimal, 
yet afford the efficient evaluation of 
ground-state densities, 
as well as the application  
of potentials and differential operators.
%
Popular choices of 
basis set 
include Gaussian functions~\cite{WCMS:WCMS1123,doi:10.1063/1.2770708,HAYNES2006345} 
in molecular systems, 
and  plane waves~\cite{0022-3719-12-21-009,0022-3719-18-21-010,KRESSE199615,PhysRevB.54.11169} 
in periodic systems, 
which are efficiently transformed 
between representations via the 
Fast-Fourier Transform (FFT)~\cite{doi:10.1137/1.9781611970999.bm}.}

%The Fourier expansion of the plane waves 
%are given in terms of the reciprocal-space vectors ${\bf G}$ 
%%
%\begin{equation}
%\psi_{n\bk}(\br)=e^{i\bk\cdot\br}u_{n\bk}
%=e^{i\bk\cdot\br}\left[\sum_{\bf G}c_{n\bk,{\bf G}}e^{i{\bf G}\cdot\br}\right] 
%=\sum_{\bf G}c_{n,\bk+{\bf G}}e^{i(\bk+{\bf G})\cdot\br}
%\end{equation}
%%
%which may be truncated to wave-vectors satisfying 
%$\frac{1}{2}\|\bk+{\bf G}\|^2<E_\textrm{cut}$.
%%
%Thus, higher-energy terms are 
%gradually included, 
%according to the kinetic-energy 
%cutoff parameter $E_\textrm{cut}$, 
%until a tolerable convergence 
%in total-energy is achieved~\cite{PhysRevB.45.1538}.
{
{\sc ONETEP}, on the other hand, 
adopts a variational basis set of {\it psinc} 
functions~\cite{doi:10.1063/1.1613633,0305-4470-19-11-013}, 
centred at positions ${\br}_{\{m\}}$, 
defined by 
%
\begin{equation}
D_{\{m\}}(\br)=\prod_{k=1}^{3}\frac{1}{N_k}\sum_{p_l=-J_k}^{J_k} 
e^{ip_l{\bf b_k}\cdot(\br-\br_{\{m\}})},
\end{equation}
where the number of grid-points 
in each lattice direction then 
determines the grid-resolution as follows 
%
\begin{align}
N_k=2J_k+1&;\quad  J_k\in \mathds{N} \nonumber\\[0.5em]
\quad\mbox{with}\quad
{\br}_{\{m\}}=\sum_{i=1}^3 \frac{m_i}{N_i}{\bf a}_i&;\quad
m_i\in\{0,1,\ldots,N_k-1\}.
\end{align}
%
Here, $\{\bf b_k\}$ are the reciprocal lattice 
basis vectors\footnote{The reciprocal lattice is discussed further in section~\ref{sec:bz_sampling}}.
%
The resolution of the grid then truncates the basis set 
and thus determines 
the maximum kinetic energy permissible.

The {\it psinc} functions are also periodic 
$D_{\{m\}}(\br)=D_{\{m\}}(\br+{\bf R})$, 
localised to a single grid-point each 
and real-valued everywhere\footnote{{As we will later demonstrate that 
${\bf b_k}\cdot{\br_j}=2\pi n\delta_{kj}$ for $n\in\mathds{Z}$}
}~\cite{doi:10.1063/1.1613633}
%
\begin{equation}
D_{\{m_1,m_2,m_3\}}(\br_{\{\mu_1,\mu_2,\mu_3\}})
=\delta_{m_1\mu_1}\delta_{m_2\mu_2},\delta_{m_3\mu_3}.
\end{equation}
%
As a result, they are an ideal basis for 
the construction of spatially-localised 
{\it representation support functions}~\cite{PhysRevB.51.10157} 
$\{\phi_\alpha(\br)\}$, 
the choice of which we will now motivate.}
%

%NGWFS
\subsection{Non-orthogonal generalised Wannier functions}
\label{sec:ngwfs}

{
In periodic systems,  
the real-space projection of 
the single-particle density-matrix 
may be generated from the integration 
of extensive periodic Bloch wave functions 
$\{\psi_{n,\bk}^\sigma(\br)\}$, 
labelled by band-index $n$ 
and momentum-vector $\bk$, 
%
\begin{equation}
\rho(\br,\br')
=\frac{1}{\Omega_{BZ}}\sum_n f_n \int_{\textrm{BZ}} d\bk\ \psi_{n,\bk}(\br) \psi_{n,\bk}^*(\br'), 
\label{eq:1_particle_dens_matrix2}
\end{equation}
%
where the $\bk$-vectors are 
restricted to the reciprocal-space 
primitive unit cell called  
the {\it first Brillouin zone} (BZ)
with volume $\Omega_\textrm{BZ}$\footnote{See section~\ref{sec:bz_sampling}.}.
%
The principle of near-sightedness, 
advocated earlier by Kohn 
to facilitate linear-scaling functionality~\cite{PhysRevLett.76.3168}, 
prescribes that our set of 
support functions for the density-matrix 
be highly spatially-localised,  
a popular choice for which 
are Wannier functions~\cite{PhysRev.52.191,
PhysRev.135.A685,
PhysRevLett.86.5341,
PhysRevLett.98.046402}.
%
The Wannier functions 
may be expressed as the 
Fourier transformations of the 
periodic Bloch wave functions 
restricted to a cell at the lattice-vector ${\bf R}$ 
%
\begin{equation}
w_{n,{\bf R}}^\sigma(\br)
=\left(\frac{\Omega_\textrm{cell}}{2\pi}\right)^\frac{1}{2}
\int_\textrm{BZ} d\bk\
\psi_{n,\bk}^\sigma(\br)\,
e^{-i\bk\cdot{\bf R}}, 
\label{eq:wannier_function}
\end{equation}
%
where $\Omega_\textrm{cell}$ 
is the real-space supercell volume.
%
The Bloch functions 
may then be reconstructed 
from the discretised inverse-Fourier transform
%
\begin{equation}
\psi_{n,\bk}^\sigma(\br)
=\left(\frac{\Omega_\textrm{cell}}{2\pi}\right)^\frac{1}{2}
\sum_{\bf R} 
w_{n,{\bf R}}^\sigma(\br)\,
e^{i\bk\cdot{\bf R}}, 
\label{eq:psi_from_wannier_function}
\end{equation}
%
%The unitarity of the Fourier transformation 
%confers the orthonormality condition 
%of the Bloch states to the Wannier functions
%%
%\begin{equation}
%\int d\br\ w^{\sigma*}_{n,{\bf R}}(\br)\, w^\sigma_{m,{\bf R'}}(\br) = 
%\delta_{nm}\delta_{{\bf R}{\bf R}'}.
%\end{equation}
which, when substituted 
into Eq.~\eqref{eq:1_particle_dens_matrix2}, 
returns the one-particle density-matrix 
in terms of the Wannier functions 
%
\begin{equation}
\rho(\br,\br')
=\sum_n\sum_{\bf R} w^{\sigma}_{n,{\bf R}}(\br)\, f_n^\sigma\, w^{\sigma*}_{n,{\bf R}}(\br').
\end{equation}}

{
Incidentally, it is often the case that 
the non-orthogonal representation of
the Wannier functions is more local  
that its orthogonal equivalent as 
shown in Ref.~\cite{doi:10.1080/00268976.2016.1173733}
%
\edit{This is because 
a set of orthogonal Wannier functions 
generated from orthogonal KS orbitals 
may require large oscillations in the former 
to ensure this condition is met, 
which increases their spatial spread 
characterised by the central second moment.
%
Thus, if we relax the orthogonality condition, 
but conserve the particle number, 
we can then facilitate a more localised representation 
that avoids large oscillations.}
%
{\sc ONETEP} implements a complete set 
of spatially-truncated non-orthogonal 
generalised Wannier functions (NGWFs) 
$\{\phi_\alpha(\br)\}$,
which are constructed from the 
KS orbitals by a unitary 
transformation ${\bf M}$
%
\begin{equation}
\psi_{i}(\br)=\phi_{\alpha}(\br)M_i^\alpha
\quad\mbox{where}\quad
M_i^\alpha=\braket{\phi^\alpha}{\psi_i}.
\label{eq:NGWF1}
\end{equation}}
%

{
The grid-point representation 
of the NGWFs $\{\phi_\alpha^D(\br)\}$, 
expanded in terms of the 
{\it psinc} basis functions 
over the simulation cell with dimension 
$N_\textrm{cell}=N_1\times N_2\times N_3$, 
is given by 
%
\begin{align}
\phi_\alpha^D(\br)&=\sum_{\{m\}}C_{\{m\},\alpha}D_{\{m\}}(\br);
\nonumber\\[0.5em]
\quad\mbox{with}\quad
C_{\{m\},\alpha}&=\int d\br\ D_{\{m\}}(\br)\phi_\alpha(\br)
=\frac{\Omega_\textrm{cell}}{N_\textrm{cell}}\phi^D_\alpha(\br_{\{m\}}).
\label{eq:support_functions}
\end{align}
%
The coefficients $C_{\{m\},\alpha}$
are determined by the integral 
overlap of the NGWFs with the {\it psinc} functions, 
which, in turn, are proportional to the 
approximate NGWFs $\{\phi^D_\alpha(\br_{\{m\}})\}$
evaluated at the corresponding grid-points.
%
Therefore, since all spatial functions 
are evaluated at discrete grid-points, 
we henceforth suppress the superscript $D$
for brevity and retain the original notation.}

{
The NGWF overlap matrix ${\bf S}$  
then defines the metric 
of the non-orthogonal basis 
spanned by the NGWFs
%
\begin{align}
S_{\alpha\beta}=\braket{\phi_\alpha}{\phi_\beta}
= \int d\br\ \phi^*_\alpha(\br)\phi_\beta(\br) = \left(\frac{\Omega_\textrm{cell}}{N_\textrm{cell}}\right) \sum_{\{m\}}C^*_{\{m\},\alpha}C_{\{m\},\beta}, 
\end{align}
%
and arises in imposing orthogonality between Kohn-Sham orbitals 
%
\begin{align}
\delta_{ij}&=\int d\br\ \psi^*_{i\bk}(\br)\psi_{j\bk}(\br) =  \int d\br\ (M^\dagger)^\alpha_i\, \phi^*_\alpha(\br)\phi_\beta(\br)\, M^\beta_j =(M^\dagger)^\alpha_i S_{\alpha\beta}M^\beta_j.
\end{align}

In the non-orthogonal regime, 
the dual functions $\{\phi^\alpha(\br)\}$ 
are orthogonal to the NGWFs 
$\braket{\phi_\alpha}{\phi^\beta}=\delta_\alpha^\beta$, 
and span the 
dual-space\footnote{{The term `duals' is perhaps a 
mathematical misnomer, but it is now part of the 
vocabulary since they were first proposed in Ref.~\cite{PhysRevA.43.5770}}.
%
Duals are discerned from the NGWFs by 
a superscript rather than a subscript.}, 
whereby we find, by the completeness relation, that
%
\begin{equation}
\phi^\alpha(\br)=\braket{\br}{\phi_\beta}\braket{\phi^\beta}{\phi^\alpha}=\phi_\beta(\br)S^{\beta\alpha}
\quad\Rightarrow\quad
S^{\beta\alpha}=(S_{\beta\alpha})^{-1}.
\label{eq:completeness_relation}
\end{equation}
%
Finally, with Eqs.~\eqref{eq:wannier_function}~-~\eqref{eq:completeness_relation}, 
it can be shown that the NGWFs 
depend on the aforementioned elements as follows
%
\begin{equation}
\ket{\phi_{\alpha{\bf R}}}=\left(\frac{\Omega_\textrm{cell}}{(2\pi)^3}\right)^{\frac{1}{2}}
\int_\textrm{BZ} d\bk\ e^{-i\bk\cdot{\bf R}} 
\sum_{i=1}^N \ket{\psi_{i\bk}}\left[M_i^{\dagger\beta}S_{\beta\alpha}\right],
\end{equation}
%
in which the the expansion coefficients 
$C_{\{m\},\alpha}$ 
of the {\it psinc} basis set 
are one of the variational parameters 
optimised {\it in situ} 
in order to minimise the total-energy.}


%In an independent-particle quantum system, 
%the electron density-matrix operator 
%is given by the weighted sum of the 
%single-particle Kohn-Sham eigenstates  
%with occupancies $\{f_i\}$
%%
%\begin{equation}
%\hat{\rho}=\sum_i f_i\; \ket{\psi_i}\bra{\psi_i}
%\quad\mbox{with}\quad
%0 \leq f_i \leq 1, 
%\end{equation}
%%
%in which expectation values of 
%operators are easily determined 
%by the trace of their 
%product with the density matrix 
%$\langle A\rangle = \textrm{Tr}[\hat{\rho}\hat{A}]$.
%%
%%\begin{equation}
%%\langle A\rangle = \textrm{Tr}[\hat{\rho}\hat{A}]
%%\end{equation}
%%

%
%Consequently, 
%the decay properties of the Wannier functions 
%are transferred to the real-space density matrix 
%whose elements may be truncated 
%beyond an appropriate cutoff 
%radius $r_\textrm{cut}$ 
%%


{
The other variational component 
in \textsc{ONETEP} procedure 
is the density-kernel ${\bf K}$~\cite{RevModPhys.32.335}, 
which is found by 
substituting the transformation in 
Eq.~\eqref{eq:NGWF1}
into the expression for the density-matrix 
in Eq.~\eqref{eq:1_particle_dens_matrix1} 
%
\begin{equation}
\rho(\br,\br')=\phi_\alpha(\br)K^{\alpha\beta}\phi_\beta(\br'). 
\label{eq:dens_mat_ONETEP}
\end{equation}
%
The density-kernel is, therefore, 
a projection of the density-matrix  
onto the space of localised NGWF duals
%
\begin{equation}
K^{\alpha\beta}=\bra{\phi^\alpha}\hat{\rho}\ket{\phi^\beta}
= \sum_i \braket{\phi^\alpha}{\psi_i} f_i \braket{\psi_i}{\phi^\beta}
= \sum_i M_i^\alpha f_i M_i^{\dagger\beta}.
\end{equation}
%
Invoking a cutoff radius $r_\textrm{cut}$ 
on the spatial extent of the NGWFs, 
such that spatial overlap between them 
vanishes beyond a certain threshold, 
thus translates to truncating the density-matrix 
thereby making $\bf K$ a sparse matrix.
%
The information contained in $\bf K$ 
then scales linearly with system size and 
enables the inexpensive computation 
of the density, 
the {non-interacting band} energy, 
as well as other quantities, 
by means of simple matrix operations
%
\begin{equation}
N=2S_{\alpha\beta}K^{\beta\alpha}
\quad\mbox{and}\quad
E=2K^{\beta\alpha}\bra{\phi_\alpha}\hat{H}_\textrm{KS}\ket{\phi_\beta}.
\label{eq:ONETEP_quantities}
\end{equation}
%
Hence, total-energy minimisation 
according to the density-matrix   
is instead performed with respect to the auxiliary variables, 
%that quantify these terms according to Eq.~\eqref{eq:ONETEP_quantities}
namely the NGWFs, 
parameterised by the expansion coefficients $C_{\{m\},\alpha}$, 
and the density-kernel ${\bf K}$.}
%

{
The \textsc{ONETEP} procedure 
thus comprises two nested optimisation loops, 
controlled by the conjugate-gradient algorithm~\cite{hestenes1952methods}, 
in which the ground-state total-energy 
is located via minimisation with respect to 
the NGWFs and density-kernel, 
such that the density-matrix 
remains idempotent, conserved 
and commutes with the Hamiltonian.
%\cite{PhysRevLett.79.1337}
%
The outer loop keeps 
the density-kernel fixed, 
while the real-space profile  
of the NGWFs is optimised 
by variational optimisiation 
of the expansion coefficients 
$C_{\{m\},\alpha}$~\cite{skylaris2005introducing,
PSSB:PSSB200541457,HINE20091041,0953-8984-20-6-064209}.
%
The inner loop, meanwhile, 
retains the form of the NGWFs 
while the energy is minimised 
with respect to the 
density-kernel matrix-elements 
using a variant of the LNV method~\cite{0953-8984-20-29-294207}.
%
The mathematical formalism 
underlying the optimisation procedure 
may be summarised by the following 
construction, adopted from Ref.~\cite{o2011optimised}, 
%
\begin{align}
E_0&=\min_n E[n] = \min_\rho E[\rho] \nonumber\\[0.25em]
&=\min_\rho E\left[\rho(K^{\alpha\beta},\{\phi_\alpha\})\right] \nonumber \\[0.25em]
&=\min_{K^{\alpha\beta},\{C_{\{m\},\alpha}\}}
		E\left[K^{\alpha\beta},\left\{C_{\{m\},\alpha}\right\}\right] \nonumber \\[0.25em]
&=\min_{\{C_{\{m\},\alpha}\}}
		\varepsilon \left[\left\{C_{\{m\},\alpha}\right\}\right]; \nonumber \\[0.25em]
\mbox{where}\quad  
\varepsilon \left[\left\{C_{\{m\},\alpha}\right\}\right]
&=\min_{K^{\alpha\beta}}
		E\left[K^{\alpha\beta}; \left\{C_{\{m\},\alpha}\right\}\right].
\end{align}}
\vspace{-1em}


%BZ SAMPLING
\subsection{Brillouin zone sampling in {\sc ONETEP}}
\label{sec:bz_sampling}

{
In the final section of this Chapter, 
we will outline the  
Brillouin zone sampling procedure 
in {\sc ONETEP}.
%
{A rapidly growing number 
of electronic structure simulations~\cite{doi:10.1063/1.4704546} 
now} involve bulk or molecular crystals 
that are defined by 
an infinite, periodic, lattice.
%
The primitive lattice vectors $\{{\bf a}_i\}$, 
which describe the primitive unit cell, 
are then repeated 
according to discrete translations 
generated by 
%
\begin{equation}
{\bf R} = \sum_{i=1}^3 n_i {\bf a}_i;
\quad\mbox{with}\quad
n_i \in \mathds{Z}, 
\end{equation}
to form the underlying Bravais lattice. 
%
By construction, 
the Kohn-Sham potential 
also has the same periodicity~\cite{ashcroft2005solid}, 
where 
%
\begin{equation}
V_\textrm{KS}(\br + {\bf R})=V_\textrm{KS}(\br).
\end{equation}
%
This inherent translational symmetry 
can be exploited 
to reduce the number of unique 
wave function solutions 
of Eq.~\eqref{eq:ks_schroding_equation}.
%
Indeed, it can be shown that, 
by virtue of the periodicity of the potential, 
the KS wave functions 
$\{\psi_n\}$ are simultaneous eigenstates 
of the Hamiltonian and 
translation operator $\hat{T}_{\bf R}$, 
with eigenvalue $c(\br)=e^{i{\bk}\cdot{\br}}$, 
such that 
%
\begin{equation}
\hat{T}_{\bf R}\psi(\br) = \psi_n(\br + {\bf R}) 
= e^{i{\bk}\cdot{\bf R}}\psi_n(\br).
\end{equation}
%
Thus, we may label the 
eigenstates $\psi_n(\br)$ 
by an additional 
set of quantum numbers ${\bk}$, 
specifying the wave-vector.
%
This forms 
the basis of Bloch's theorem~\cite{Bloch1929},
which states that, 
in a periodic crystal, 
the electron wave function 
can be written as 
the product of a plane wave, 
with momentum $\bk$, 
and a periodic function $u_n(\br)$ 
with the same periodicity as the lattice 
%
\begin{equation}
\psi_{n,\bk}(\br)=e^{i{\bk}\cdot \br}u_n(\br)
\quad\mbox{with}\quad
u_n(\br+{\bf R})=u_n(\br).
\end{equation}
%
The ${\bk}$-vector is constructed from 
a set of reciprocal lattice vectors 
$\{{\bf b}_i\}$ 
generated from the Fourier transform of 
the real-space Bravais lattice 
%
\begin{equation}
{\bk}=\sum_{i=1}^3 c_i {\bf b}_i;
\quad 
c_i \in \mathds{R}
\quad\mbox{with}\quad
{\bf a}_i\cdot{\bf b}_j=2\pi\delta_{ij}.
\end{equation}
%
It follows that wave functions that 
differ by integer multiples of 
reciprocal lattice vectors, 
which describe the first Brillouin zone (BZ), 
%
\begin{equation}
{\bf G}=\sum_{i=1}^3 m_i {\bf b}_i;
\quad
m_i \in \mathds{Z}, 
\end{equation}
%
are equal modulo ${\bf G}$, 
and therefore redundant beyond  
the first BZ, as shown by
%
\begin{align}
\psi_{n, {\bk+\bf G}}(\br)&=e^{i({\bk+{\bf G}})\cdot \br}u_n(\br)
=e^{i{\bf G}\cdot \br}e^{i{\bk}\cdot \br}u_n(\br)\nonumber \\[0.5em]
&=e^{i\sum_j{\bf a}_j\cdot {\bf b}_j}e^{i{\bk}\cdot \br}u_n(\br)  
%=e^{i6\pi}e^{i{\bk}\cdot \br}u_n(\br) 
=e^{i{\bk}\cdot \br}u_n(\br)
=\psi_{n,\bk}(\br).
\end{align}
%
Thus, an infinite number 
of occupied wave functions 
in an infinite, periodic crystal 
may be represented by 
a finite number of occupied 
wave functions parameterised 
by the wave-vectors $\bk$, 
confined to the first Brillouin zone.}

{
The number of unique $\bk$-vectors,
parameterising the solutions of the periodic system, 
scales with the number of unit cells, 
which become impossible to resolve 
in the bulk limit.
%
However, 
since the eigenvalues and wave functions 
behave smoothly over the Brillouin zone~\cite{PhysRev.50.58}, 
a weighted sampling of $\bk$-vectors 
is sufficient to approximate the 
continuous integration~\cite{PhysRevB.8.5747,PhysRevB.13.5188}. 
%
The magnitude of the 
error introduced by 
the discrete sampling 
can be reduced, 
but not necessarily {monotonically},  
by increasing the $\bk$-point resolution.}

{
For aperiodic systems, 
such as molecules or {defective} crystals,
in which translational symmetry is broken, 
Bloch's theorem becomes inapplicable.
%
To incorporate Bloch states 
in finite systems, 
it is {a common} practice to generate a 
periodic supercell, 
in which the molecule or defect 
is sufficiently separated  
(either by vacuum space or pristine crystal) 
from its periodic images~\cite{PhysRevB.51.4014}.
%
The degree of electronic and orbital isolation
is determined by an 
appropriate convergence 
of the total-energy 
with respect to the 
volume of the supercell $\Omega_\textrm{cell}$.}

{
Moreover, 
for sufficiently large supercells, 
such as those employed by {\sc ONETEP}, 
the first Brillouin zone volume, 
given by 
%
\begin{equation}
\Omega_\textrm{1\textsuperscript{st} BZ} = \frac{(2\pi)^3}{\Omega_\textrm{cell}}
\end{equation}
%
is sufficiently shrunk  
such that the bands between 
BZ boundaries flatten 
and $\Gamma$-point 
sampling alone is adequate, 
{
i.e., $\bk=0$, 
whereby it is possible 
to use real-valued wave functions~\cite{0953-8984-20-29-294207}}.
%
In this approach, 
repeating the unit cell by a factor of $N_k$ 
along each Bravais lattice direction 
is analogous to sampling the original Brillouin zone
on a regular grid of $N_k$ vectors 
along each direction in reciprocal-space, 
as is the approach prescribed 
by most plane wave codes.}


\section{Conclusion}
{
In this Chapter, 
we introduced the fundamental principles 
and incremental approximations 
underpinning the widely successful 
and highly versatile 
approximate density-functional theory.
%
We discussed the principle developments, 
namely, 
the founding Hohenberg-Kohn theorems;
the subsequent Kohn-Sham equations, 
after which the development of approximate 
exchange-correlation functionals 
permitted the tractable solution of the Schr{\"o}dinger equation; 
as well as the pseudopotential and PAW methods, 
which {permit a more convenient treatment} 
of the core states.}

{
Furthermore, 
we highlighted the motivation for 
developing linear-scaling methods 
to facilitate large-scale electronic structure calculations, 
and outlined the theoretical framework 
behind the {\sc ONETEP} code.
%
We introduced 
the fundamental convergence parameters 
that govern the simulation process, 
namely the supercell dimension   
$N_1\times N_2\times N_3$, 
which effectively emulates $\bk$-point sampling; 
the grid-point resolution ${\br}_{\{m\}}$, 
related to the kinetic energy cutoff;  
and the NGWF cutoff radius $r_\textrm{cut}$.}

{
Finally, we briefly introduced 
some important concepts 
that will be addressed in greater detail 
in later Chapters, 
such as the self-interaction error 
and the DFT+$U$ method, 
which plays an important role 
in increasing the accuracy and viability 
of commercial DFT.}


%We have so far outlined the 
%underlying concepts central to  
%linear-scaling {\it ab intio} simulation, 
%i.e.,
%the avoidance of $\mathcal{O}(N^3)$ 
%operations incurred by a truncated density-matrix;
%the $\Gamma$-point sampling 
%afforded by a supercell simulation.
%
%It is difficult to understate 
%the benefits the linear-scaling functionality 
%of \textsc{ONETEP} has afforded 
%to electronic structure calculations 
%in the past decade.
%
{
Over the years, 
DFT has risen in prominence 
and become one of the most 
ubiquitous computational tools.
%
Its widespread applicability 
is made possible due to the 
prevalence of efficient codes and powerful hardware, 
bolstered by timely, inexpensive and versatile methodologies, 
which strike a reasonable balance between 
accuracy and convenience 
to suit a wide variety of requirements.}

{
DFT has {benefitted from} 
%cut the teeth of  
some of the best minds in science 
and continues to 
produce exciting developments 
in emergent technologies, 
such as 
high-energy storage materials~\cite{Saal2013,0965-0393-21-7-074005}, 
superconductors~\cite{PhysRevB.72.024545,PhysRevLett.94.037004}, 
topological insulators~\cite{PhysRevLett.105.096404,PhysRevLett.106.016402,PhysRevB.82.235121}, 
and 
photovoltaic devices~\cite{PhysRevB.79.115126,doi:10.1021/nl0732777}, 
to name a few.
%
It also features as a centre-piece 
to ambitious scientific projects 
such as the United States 
Materials Genome Initiative~\cite{holdren2011materials}, 
and the Open Quantum 
Materials Database~\cite{Kirklin2015}.}

{
However, one of the fundamental difficulties
that still limits the predictive capabilities of DFT, 
is the many-body self-interaction error~\cite{PhysRevB.57.1505,cohen2008insights,PhysRevB.23.5048}, 
introduced in section~\ref{sec:xc_functionals}, 
which is 
produced by the approximate nature of 
the XC functionals,  
and predominantly manifest 
in systems exhibiting 
strong electron correlation. 
%
We will explore this issue in detail 
in Chapter~\ref{ch:self_interaction_error} 
and discuss {some viable} solutions.
%
For a comprehensive discussion 
on the history, development 
and future prospects of DFT, 
we refer the reader to 
Refs.~\cite{CAPELLE2006,doi:10.1063/1.4704546,RevModPhys.87.897,cohen2008insights}.}

