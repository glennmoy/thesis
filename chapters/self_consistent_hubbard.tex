

\lett{I}{n the previous Chapter}, 
we outlined the construction of a variational linear-response (LR) approach 
to calculate the Hubbard $U$ and Hund's $J$ parameters 
in linear-scaling DFT, 
with the added advantage that these quantities 
are computed as strict ground-state properties.
%
%Our calculations with $U$ on various systems, 
%such as H$_2^+$, Ni(CO)$_4$, NiO and Cr$_2$O$_3$ 
%gave excellent agreement with 
%the qualitative and quantitative 
%experimental results.
%%
%Moreover, 
%the corresponding $J$-term evaluated for NiO 
%was reasonable and afforded further refinement.
%
To progress, 
it is our intention to now address the open question of 
the rigorous comparability of 
DFT+$U$ total-energies that are
generated by calculations with different $U$ values, 
which ordinarily represent external parameters
with the same status as ionic positions.
%
This question is of considerable contemporary relevance, 
as demonstrated by recent progress in calculating  
thermodynamic quantities~\cite{PhysRevB.75.035109,PhysRevB.78.075125,PhysRevB.75.035115,1674-1056-17-4-035,doi:10.1021/cm702327g,CapdevilaCortada201558,PhysRevB.85.155208,PhysRevB.84.045115,PhysRevB.85.115104,PhysRevB.90.115105},
in high-throughput materials informatics~\cite{jain2011high,setyawan2010high,curtarolo,Curtarolo2012218,PhysRevX.5.011006}, 
 catalysis~\cite{Bliem1215,C1CP22128K,doi:10.1021/jp407736f,doi:10.1021/acscatal.6b01907,PhysRevB.88.245204},
and in the study of ion-migration in battery materials~\cite{PhysRevB.70.235121,PhysRevB.83.075112,PhysRevB.93.085135,C2TA00839D,C1EE01782A,Zhou2004181,doi:10.1021/jacs.5b04690,doi:10.1021/acs.chemmater.5b03554,Morgan20075034} 
by means of DFT+$U$.

Currently, 
the calculation of $U$ as an external, 
{albeit ground-state}, parameter 
of the system does not uniquely determine 
a ground-state density and therefore prohibits  
the comparability of the resulting DFT+$U$ total-energies.
%
If the $U$  parameter is instead calculated 
as an implicit functional of the ground-state density
$E[n(\br),U[n(\br)]]\ \equiv E[n(\br)]$, 
%
%\begin{equation}
%E[n(\br),U[n(\br)]]\ \equiv E[n(\br)].
%\end{equation}
%%
which can be uniquely defined, 
then DFT+$U$ {as a whole} becomes a 
fully self-contained, variational first-principles method. 
%
Only under the latter condition could we expect the 
fully rigorous direct comparability of the total-energies, 
and their derived thermodynamic observables,  
calculated from different DFT+$U$ calculations 
with different system-specific $U$ parameters.
 %
As a result, any external parameters, 
such as ionic positions, 
may then be varied simultaneously with their 
corresponding first-principles $U$ values.

{
In this Chapter, 
we argue that the self-consistent calculation 
of the $U$ and $J$ parameters 
as properties of the respective 
DFT+$U$ and DFT+$J$ ground-states 
may be a solution toward this goal.
%
Calculating Hubbard $U$ parameters self-consistently 
in this fashion is a recent and rather esoteric practice, 
however its application bears significant theoretical 
and conceptual consequences in this context.
%
Hence, we shall first give a brief outline 
of its operation.}

In Ref~\cite{PhysRevLett.97.103001}, 
Kulik {\it et al.} first outlined the scheme 
for the self-consistent calculation of 
Hubbard $U$ parameters 
that was later adopted by other 
works~\cite{PhysRevB.81.245113,0953-8984-22-5-055602,cococcioni4}.
% 
The motivation for this approach 
arose from the observation that, 
for certain systems,  
the nature of the electronic states  
(and response properties) 
in the DFT+$U$ ground-state differed
qualitatively from those of the DFT ground-state~\cite{PhysRevB.57.1505,PhysRevB.83.075112,PhysRevB.84.115108}, 
i.e., insulating vs metallic, 
thereby requiring a $U$ that is 
consistently with the state in which it is applied.
 
In self-consistency schemes  generally, 
incremental values of $U_\textrm{in}$ 
are applied to the subspace at hand,  
with varying ground-state orbitals and densities as a result,  
and a new first-principles $U_\textrm{out}$ 
is computed for each $U_\textrm{in}$.
%
The  numerical relationship 
$U_\textrm{out} ( U_\textrm{in} )$  
is then used to select the  
self-consistent $U$, using a pre-defined  criterion.
%
Its conceptual implications aside, a self-consistent $U$ 
has been shown to provide  
improvements in transition-metal 
chemistry~\cite{kulik2008self,:/content/aip/journal/jcp/133/11/10.1063/1.3489110,kulik2011transition,PhysRevB.84.115108,PhysRevLett.106.118501,Youmbi20141,2053-1591-3-8-086104,doi:10.1021/acs.jctc.6b00937,Hsu201019}, 
biological systems~\cite{doi:10.1021/jp070549l},
photovoltaics~\cite{PhysRevB.78.241201,PhysRevLett.105.146405,doi:10.1021/jp1041316}, 
and high-density energy storage~\cite{PhysRevB.93.085135}.

While many practitioners have used the original, 
linear-extrapolation type $U_\textrm{scf}$
in their studies~\cite{kulik2008self,:/content/aip/journal/jcp/133/11/10.1063/1.3489110,doi:10.1021/jp070549l,2053-1591-3-8-086104,Youmbi20141,doi:10.1021/acs.jctc.6b00937,PhysRevB.78.241201,PhysRevLett.105.146405,doi:10.1021/jp1041316,Hsu201019,PhysRevB.80.075102}, 
others have used the equality between  
$U_\textrm{in}$ and $U_\textrm{out}$ 
as an alternative self-consistency condition~\cite{doi:10.1021/jp070549l,0953-8984-22-5-055602,PhysRevB.84.115108,PhysRevB.93.085135,PhysRevLett.106.118501,doi:10.1063/1.4947240,PhysRevB.85.045132}.
%
The majority of published first-principles $U$ calculations 
involve no self-consistency over the  parameter at all, however, and there may 
even be a case to be made that none is ordinarily warranted. 
%
While approaching this concept  
from a different motivation and perspective, 
we find the resolution of this ambiguity, in itself, 
an intriguing open challenge in abstract DFT.
% 
In particular, 
it demands investigation in the present context of
the variational linear-response $U$ since, ideally, 
the optimal scheme to match that method 
should be established from the outset.

In this Chapter, 
we shall demonstrate that our variational approach 
provides a convenient  framework in which to analyse 
a number of different self-consistency criteria that have been proposed, 
and address this partially-unresolved question. 
%
We will again utilise H$_2^+$ as a model system, 
as we have repeatedly done so far, 
as it will allow us to study and draw firm conclusions 
regarding the numerous plausible, yet different, strategies 
currently in use for defining self-consistency over the Hubbard $U$.
%
For the particular case of the variational LR 
Coulomb parameters at least, 
we identify a well-defined best choice of self-consistency criterion  
supported by numerical results.
%
We will then extend this methodology to 
the treatment of multi-electronic systems, 
which, in the process, will necessitate the inclusion of a self-consistent $J$.
%
We will then test a variety of possible 
multi-electronic self-consistency schemes with NiO, 
incorporating both $U$ and $J$.
%
As a final demonstration, 
we will use the self-consistent $J$, 
which is itself a relatively unexplored concept~\cite{PhysRevB.85.045132}, 
as a means to correct the static-correlation error (SCE)  
beyond the Coulson-Fischer point in dissociating H$_2$.


%DIRECTLY COMPARING DFT+U ENERGIES
\section[{A note on the direct comparison of DFT+$U$ total-energies}]{A note on the direct comparison of \break DFT+$U$ total-energies}

Before we commence our analysis  
of parameter self-consistency, 
we would like to outline our 
argument in favour of adopting this approach 
for the purposes of comparing the ground-state 
total-energies of DFT+$U$ calculations.
%
%In general, 
%one cannot directly compare the total-energies 
%of DFT+$U$ calculations 
%between systems with differing 
%nuclear positions, 
%ionic states 
%or stoichiometry.
%%
%This is because once you vary the external parameters 
%of the system, 
%the new $U$ you compute is also an external parameter 
%and thus changes the Hamiltonian.

The variational LR definition
$U = d v_\textrm{int} / d N$, 
so-called as it is based on the 
variational response of the  ground-state density, 
demands that the subspace-averaged 
non-interacting response 
$\chi_0 = d N / d v_\textrm{KS} $ is calculated
using the same set of ground-state densities 
as that used for calculating 
the interacting response $\chi = d N / d \alpha$, 
which are parameterised by the external perturbation strength $\alpha$.
%
If we take the limit of 
small perturbations around the 
unperturbed KS ground-state density matrix $\hat{\rho}_0 $, 
we may write  
\begin{align}
U_\textrm{out}  = \chi_0^{-1} - \chi^{-1}
= \left. \frac{d v_\textrm{KS} \left[ \hat{\rho } \right] }{d N \left[ \hat{\rho } \right] } \right|_{\hat{\rho}_0} 
-
\left. \frac{d v_\textrm{ext}}{d N \left[ \hat{\rho } \right] } \right|_{\hat{\rho}_0}.
\end{align}
%
From this, it is clear that 
$U_\textrm{out} \left[ \hat{\rho}_0 \right] $ 
is a ground-state density-functional, 
albeit not one of an explicit algebraic form.
% 
This illustrates why the comparability of DFT+$U$ total-energies, 
using a $U$ determined from the unperturbed density, breaks down.
%
%This definition is readily adaptable to orbital-free
%DFT, in which there is no Kohn-Sham Hamiltonian to diagonalise  
%but only a density (rather than a Kohn-Sham density-matrix)
%to optimise, and where $\hat{v}_\textrm{KS}$ is replaced by the 
%total potential. 

If we instead perform a variational LR calculation 
for a given $U_\textrm{in}$, 
it follows that  the resulting  $U_\textrm{out} $ 
may be thought of as a functional of the ground-state 
density parameterised 
by $U_\textrm{in}$ such that 
$U_\textrm{out}\equiv U_\textrm{out}\left[\hat{\rho}(U_\textrm{in})\right]$.
%
If we can then uniquely determine {a particular}
$U_\textrm{in}^\prime$, 
by applying a self-consistency criterion, 
we will thereby uniquely determine the self-consistent ground-state 
DFT+$U$ density-matrix (up to  unitary transformations)
$\hat{\rho}_{\textrm{DFT}+U} ( U_\textrm{in}^\prime )$.
%
The same then can also be said for its derived
properties, such as the total-energy, 
in terms of the remaining  parameters, 
i.e., the ionic positions.
%
In this respect, the DFT+$U$ ground-state energy 
is thus a functional of the DFT+$U$ ground-state density alone.
%
The comparability {then} 
of total-energies between 
various crystallographic 
or molecular structures with  differing 
self-consistent Hubbard $U$ values directly follows, 
as does the validity of thermodynamic calculations 
based on DFT+$U$.

In this way, 
given the underlying explicit algebraic 
XC functional, such as PBE, 
together with the choice of a set of
subspaces to target for SIE correction, 
DFT+$U$ is elevated to the status of a 
self-contained  orbital-dependent 
density-functional in its own right,  
which incorporates the Hubbard $U$ as a non-algebraic 
but readily computable auxiliary ground-state variable. 
%
The task at hand is then to apply the correct 
self-consistency scheme to all relevant systems  
to facilitate their direct comparison.


%SELF-CONSISTENT U FOR ONE ELECTRON
\section{A self-consistent $U$ in a one-electron model}

In order to compute a variational linear-response 
$U_\textrm{out}$ 
for a single subspace already subject to a DFT+$U$
term of strength $U_\textrm{in}$, 
the subspace-averaged interaction potential 
$V_\textrm{int}$
must incorporate the {spin-dependent} DFT+$U$ potential 
{$\hat{V}^\sigma_U$ 
as well as the usual Hartree and exchange-correlation (XC) terms 
$\hat{V}_\textrm{Hxc}$.}
%
Furthermore, 
as described in Chapter~\ref{ch:calculating_hubbard_u}, 
each component of the interaction potential 
$ V_\textrm{int}$ 
must be appropriately averaged.
%
For the Hartree and XC potential,  
we saw that the appropriate average is simply
%
\begin{equation}
V_\textrm{Hxc}=\textrm{Tr}[\hat V_\textrm{Hxc}\hat P]/\textrm{Tr}[\hat P],
\end{equation}
%
where the operator $\hat{V}_\textrm{Hxc}$ 
may approximately scale with $N^\sigma$ 
but the averaging scheme does not.
% 
However, unlike 
$\hat{V}_\textrm{Hxc}\equiv\hat{V}_\textrm{Hxc}[n^\uparrow,n^{\downarrow}]$, 
which acts on one state but is 
generated by all occupied states, 
the  DFT+$U$ potential, given by 
%
\begin{equation}
\hat{V}_{U_\textrm{in}}^\sigma 
%= \frac{U_\textrm{in}}{2} (\hat{P}-2\hat{P}\hat{\rho}^\sigma\hat{P})
= \frac{U_\textrm{in}}{2}(\hat{1}-2\hat{n}^\sigma)
\end{equation}
%
is intrinsically both specific and due to 
each subspace occupancy matrix eigenvector individually.
%
Thus, we find that the simple trace 
\begin{equation}
V_{U_\textrm{in}}^\sigma = \textrm{Tr} [ \hat{V}_{U_\textrm{in}}^\sigma ] 
= \frac{U_\textrm{in}}{2} ( \mathrm{Tr} [ \hat{P} ] - 2 N^\sigma )
\end{equation}
is that which scales appropriately with $N^\sigma$ 
(or equivalently $N$).
% 
Put another way, $V_{U_\textrm{in}}^\sigma $ is the average
DFT+$U$ potential acting on a  subspace eigenvector, 
since there are $\textrm{Tr} [\hat{P}] $ copies of that eigenvector, 
and is therefore scalable along with $V_\textrm{Hxc}$.
%
%The factor $ \mathrm{Tr} [ \hat{P} ] $ separating   
%the definitions of $V_\textrm{Hxc}$ and $\hat{v}_{U_\textrm{in}}$
% is consistent with DFT+$U$ correcting the Hartree + xc generated 
%many-body subspace SIE, which is assumed to be
%proportional to $N^{2} \approx 
% ( \mathrm{Tr} [ \hat{P} ] \langle n_i \rangle ) ^{2} $, by 
% only $ \mathrm{Tr} [ \hat{P} ]$ 
% one-electron SIE corrector
%terms on the order of $\langle n_i \rangle^{2}$.
%%

{To first treat the test case of a 
one-electron system, 
it naturally follows that  
the interacting and Hubbard potentials 
depend on one spin channel only, 
such that $N=N^\sigma$ and 
$V_{U_\textrm{in}}=U_\textrm{in}(\textrm{Tr}[\hat{P}]-2N)/2$.}
%
Then, starting from Eq.~\eqref{eq:u_definition},
the single-site variational LR Hubbard $U_\textrm{out}$, 
in the presence of a non-zero  $U_\textrm{in}$, 
may be written as
% 
\begin{align}
U_\textrm{out}=\frac{d V_\textrm{int}}{d N} 
=\frac{d V_\textrm{Hxc} +dV_{U_\textrm{in}}}{d N}
&=\frac{d V_\textrm{Hxc} }{d N}-U_\textrm{in}
=F^{\hat{P}}_\textrm{Hxc}(U_\textrm{in})-U_\textrm{in},\nonumber \\[1em]
\mbox{where}\qquad 
F^{\hat{P}}_\textrm{Hxc} (U_\textrm{in}) &\equiv \frac{dV_\textrm{Hxc}}{d N}.
\label{eq:u_one_electron}
\end{align}
%
Here, the subspace-bare, bath-screened 
and subspace-averaged Hxc interaction \break 
$F^{\hat{P}}_\textrm{Hxc} (U_\textrm{in}) \equiv 
F^{\sigma\sigma}_\textrm{int}$ 
is calculated at the fully-relaxed 
DFT+$U_\textrm{in}$ ground-state 
and specifically due, in this example, 
to like-spin interactions only.
  

From Eq.~\ref{eq:u_one_electron} 
we may now readily identify 
three unique self-consistency criteria.
%
The first is a very plausible self-consistency criterion, 
first proposed in Ref.~\cite{0953-8984-22-5-055602} 
and later utilised in Refs.~\cite{PhysRevB.84.115108,PhysRevB.93.085135}, 
which requires that
%
\begin{equation}
 U_\textrm{out}=U_\textrm{in}
\quad\mbox{giving}\quad 
U^{(1)}\equiv F^{\hat{P}}_\textrm{Hxc}(U_\textrm{in})/2.
\end{equation}
%
This $U_\textrm{in}$, denoted here as $U^{(1)}$, 
appears to account for, or cancel away, 
precisely one-half of the subspace SIE 
that remains at that DFT+$U_\textrm{in}$ 
and is represented by the triangle 
in Fig.~\ref{fig:h2+_uin_vs_uout}.
%
The second criterion, 
denoted by $U^{(2)}$, dictates that    
%
\begin{equation}
 U_\textrm{out}=0
\quad\mbox{giving}\quad 
U^{(2)}\equiv F^{\hat{P}}_\textrm{Hxc}(U_\textrm{in}),
\end{equation}
%
and implies that $U_\textrm{in}$  
fully cancels the subspace-related SIE 
computed at the same DFT+$_\textrm{in}$ ground-state.
%
This is indicated by the diamond  
in Fig.~\ref{fig:h2+_uin_vs_uout}.

Finally, the third condition, denoted by $U^{(3)}\equiv U_\textrm{out} ( 0 )$ 
is one that matches the original 
self-consistency scheme~\cite{PhysRevLett.97.103001}
where it is denoted $U_\textrm{scf}$, 
albeit with a different underling linear-response procedure.
%
Here, the linear-extrapolation of  
$U_\textrm{out} ( U_\textrm{in} )$ 
(for sufficiently large $U_\textrm{in}$
to obtain a good fit)
is taken back to $U_\textrm{in} = 0$~eV, 
as shown in Fig.~\ref{fig:h2+_uin_vs_uout} 
by the circle.


%
\begin{figure}[th!]
\centering
\includegraphics[height=0.494\textwidth]{images/h2+_uin_vs_uout.pdf}
\caption[Example of $U_\textrm{in}$ vs $U_\textrm{out}$ profile for H$_2^+$]
{An illustration of a typical 
$U_\textrm{in}$ vs $U_\textrm{out}$ 
profile for H$_2^+$, 
indicating each of the self-consistent criteria:
$U^{(1)}$ (triangle), 
$U^{(2)}$ (diamond), 
and $U^{(1)}$ (circle).
%
We find a highly-linear profile across 
all bond-lengths for this system 
and LR methodology.}
\label{fig:h2+_uin_vs_uout}
\end{figure}
%



For our present purposes, it is reasonable to assume that 
a DFT+$U$ corrected electronic structure has been well-obtained 
at $U^{(2)}$, and thus performing the 
linear extrapolation for $U^{(3)}$ around  $U^{(2)}$, 
we find that
\begin{equation}
U^{(3)}\equiv U^{(2)} \left(1-  \left. \frac{d F^{\hat{P}}_\textrm{Hxc} }{ dU_\textrm{in}  } \right|_{U^{(2)}} \right).
\end{equation} 
%
From this, a clear interpretation of
$U^{(3)}$ as screened version of $U^{(2)}$ emerges,  
in the sense that, 
instead of an externally applied potential 
being attenuated by relaxation of the electronic structure, 
it is the externally applied \emph{interaction correction}
which is attenuated.

A normal dielectric screening operator 
measures the rate of change of 
the potential with respect to an external perturbation, taking 
the form 
%
\begin{equation}
\hat{\epsilon}^{-1}= \frac{d\hat{v}_\textrm{KS}}{d\hat{v}_\textrm{ext}} 
= \hat{1}+ \hat{f}_\textrm{Hxc} \hat{\chi}.
\end{equation}
%
A generalised screening function here instead  
measures the rate of reduction in the subspace-averaged
SIE with respect to $U_\textrm{in}$, and is given by
%
\begin{equation}
\epsilon_U^{-1}= - \frac{dU_\textrm{out}}{dU_\textrm{in}} = 1-\frac{dF^{\hat{P}}_\textrm{Hxc}}{dU_\textrm{in}}
\end{equation}
%
Therefore, while we might expect a DFT+$U$
correction with parameter $U^{(2)}$
to  cancel the subspace-averaged SIE, 
including all  self-consistent
response effects in the electronic structure, 
when we have done so we have 
in fact removed an SIE (with respect to DFT) 
of magnitude $U^{(3)}= \epsilon_U^{-1} U^{(2)}$, 
which is typically smaller in magnitude than $U^{(2)}$.
%
In other words, 
there is a numerically relevant 
 distinction between the external `bare'  $U_\textrm{in}$ 
that we  apply using DFT+$U$,  and the `screened' SIE quantifier 
$U_\textrm{out}$ that we then measure.


Moreover, 
we find the SIE measure $U^{(3)}$, 
calculated around the $U^{(2)}$ ground-state, 
of particular interest.
%
For example, 
it may find use in quantifying the change in SIE 
in a subspace in response to an
external parameter such as atomic position, 
or in comparing the SIE of an atom in two different charge states.
%
We also expect $U^{(3)}$ to be suitable as
an input Hubbard $U$ parameter
for non-self-consistent protocols such as 
in a post-processing 
DFT+$U_1$+$U_2$ procedure, 
or DFT+$U$ band-structure correction based on the DFT density, 
or a DFT + dynamical mean-field theory (DMFT) calculation
with no density self-consistency. 
%
The central message is that 
$U^{(3)}$ linearly accounts
for the resistance to SIE reduction that {\it would} be met
were density self-consistency in response to $U$ allowed.

On the basis of the above analysis, however, 
we surmise that the criterion presented by $U^{(2)}$ 
represents the appropriate self-consistency scheme 
for the variational linear-response method, 
wherever the standard self-consistent response of the
density occurs upon application of DFT+$U$.
%
Furthermore, 
the value of $U^{(2)}$ may be efficiently obtained, 
for example via the bisection method.
%
The three self-consistency conditions 
discussed heretofore are summarised 
graphically in 
Fig.~\ref{fig:h2+_uin_vs_uout} above,  and 
Table~\ref{table:self_consistent_conditions}.

%%%%%%%%%%%%%%%%%%%%%%%%%%%%%%%%%%%%%%%%%%%%%%%%%%%
%TABLE 1
\begin{table}[th!]
\begin{tabular*}{\columnwidth}{@{\extracolsep{\fill}}cll}
\hline\hline
Notation & 	Condition 		& Formula derived from Eq.~\ref{eq:u_one_electron}\\
\hline
$U^{(1)}$	&\;$U_\textrm{out}=U_\textrm{in}$ & \;$U_\textrm{in}= F^{\hat{P}}_\textrm{Hxc}(U_\textrm{in})/2$\\[0.5em]
$U^{(2)}$	&\;$U_\textrm{out}=0$	& \;$U_\textrm{in}= F^{\hat{P}}_\textrm{Hxc}(U_\textrm{in})$\\[0.5em]
$U^{(3)}$	&\;$U_\textrm{out}(0)$	& \;$U_\textrm{out}(0)=U^{(2)}(1-d F^{\hat{P}}_\textrm{Hxc}/dU_\textrm{in}|_{U^{(2)}})$\\
\hline\hline
\end{tabular*}
\caption{Summary of  three first-principles Hubbard $U$ self-consistency criteria derived from Eq.~\ref{eq:u_one_electron}.}
\label{table:self_consistent_conditions}
\end{table}
%%%%%%%%%%%%%%%%%%%%%%%%%%%%%%%%%%%%%%%%%%%%%%%%%%%


%WHICH SELF-CONSISTENCY SCHEME
\subsection{Which self-consistency scheme?: revisiting H$_2^+$}

In order to assess each of the self-consistency conditions outlined in 
Table~\ref{table:self_consistent_conditions}, 
and select the appropriate scheme  
we applied them to dissociating H$_2^+$.
%
For each bond-length, 
a $U_\textrm{in}$ versus $U_\textrm{out}$ 
profile was calculated 
according to Eq.~\ref{eq:u_one_electron},
an example of which is illustrated by Fig~\ref{fig:h2+_uin_vs_uout}, 
with due care to error accumulation.
% 
These profiles were found to remain highly linear 
across all bond-lengths 
for this particular system and LR methodology, 
and we note that the slope remained greater than $-1$ throughout,
signifying $d F_\textrm{Hxc}^{\hat{P}} / d U_\textrm{in}
> 0$ and a `resistance' to SIE reduction,
for all but the shortest bond-lengths 
($\lesssim 1.3$~a$_0$) 
strongly affected by subspace double-counting.
 
The linear fit to $U_\textrm{out} ( U_\textrm{in} )$
was then used to evaluate 
$U^{(1)}$, $U^{(2)}$, $U^{(3)}$,
according to Table~\ref{table:self_consistent_conditions}, 
and their values are depicted by 
dashed, dotted and dot-dashed lines, respectively, 
in Fig.~\ref{fig:h2+_uout_vs_r}.
%
For each bond-length, 
we also estimated, 
by interpolation, the $U_\textrm{int}$ (solid line)
required to recover the exact total-energy.
 
%
\begin{figure}[th!]
\centering
\includegraphics[height=0.494\textwidth]{images/h2+_uout_vs_r.pdf}
\caption[Self-consistent $U_\textrm{out}$ schemes for H$_2^+$ vs best estimated $U$]
{The estimated best $U$ value,
$U_\textrm{int}$ (solid),  for correcting the total-energy
SIE in H$_2^+$, shown with the
$U^{(1)}$ (dashed), 
$U^{(2)}$ (dotted), and 
$U^{(3)}$ (dot-dashed) values.
%
$U^{(2)}$ and $U^{(3)}$  
approximately equal   $U_\textrm{int}$ 
in the dissociation limit, 
while $U^{(1)}$ is serendipitously 
more successful at equilibrium and below
due to  subspace overlap and double-counting.}
\label{fig:h2+_uout_vs_r}
\end{figure}
%
 
The $U^{(2)}$ and $U^{(3)}$ 
schemes, and particularly the former, 
closely approximate the $U_\textrm{int}$ required 
to correct the SIE in the total-energy in the dissociated limit,
whereas $U^{(1)}$  clearly represents 
an underestimation by a factor of  $2$, 
as suggested by Table~\ref{table:self_consistent_conditions}.
%
The numerical situation is reversed within the equilibrium 
bond-length of approximately $2$~a$_0$,
where $U^{(1)}$ appears to perform better than the alternatives.
%
We emphasise that the latter result is 
misleading, however, 
since $U^{(1)}$ performs better
at short bond-lengths
only due to the serendipitous cancellation 
of its reduced magnitude  
with the double-counting effects of spatially
overlapping DFT+$U$ subspaces.
%
Notwithstanding, 
the breakdown of the subspace-bath separation 
underpinning  DFT+$U$ takes place in the strong-bonding regime.
%
This finding highlights a risk when assessing the relative
merits of correction formulae of this kind solely on the basis 
of numerical results gathered under equilibrium conditions, 
where bonding or overlap effects complicate the analysis.


%PBE+U total-energIES
\subsubsection{PBE+$U_\textrm{out}$ total-energies}


The total-energy based 
binding curves of H$_2^+$ were recalculated 
using the bond-length dependent 
$U^{(1)}$ (dashed), 
$U^{(2)}$ (dotted), 
and $U^{(3)}$ (dot-dashed), for comparison with
the exact total-energy (solid) in Fig.~\ref{fig:h2+_pbe+uout}.

\begin{figure}[th!]
\centering
\includegraphics[height=0.494\textwidth]{images/h2+_pbe+uout.pdf}
\caption[Binding energy of H$_2^+$ calculated 
with various $U_\textrm{out}$ schemes]
{The H$_2^+$ binding  curves 
of the exact functional (solid),
PBE+$U^{(1)}$ (dashed), 
PBE+$U^{(2)}$ (dotted), 
and PBE+$U^{(3)}$ (dot-dashed).
%
The reference energy for each curve 
is that of the isolated H atom calculated with 
the corresponding XC functional.
% 
In the dissociated limit (inset),  
the $U^{(2)}$ result tends asymptotically to the exact one,
and the  $U^{(3)}$ scheme begins to deviate from it non-negligibly.}
\label{fig:h2+_pbe+uout}
\end{figure}
%


As already suggested by Fig.~\ref{fig:h2+_uout_vs_r}, 
$U^{(1)}$ fails to correct 
the SIE in the total-energy at bond-lengths further from equilibrium, 
whereas $U^{(2)}$ and $U^{(3)}$  
provide a more universal correction  
of the total-energy, 
becoming  acceptable in the dissociation limit.
%
The inset of Fig.~\ref{fig:h2+_pbe+uout}  
illustrates, however,  that the  
PBE+$U^{(3)}$ scheme, which is numerically equivalent to no 
Hubbard  $U$ self-consistency in this particular system, 
begins to under-perform with respect to  PBE+$U^{(2)}$
in the dissociated limit.
%
The PBE+$U^{(2)}$ total-energy, 
meanwhile,  seems to converge upon 
the exact total-energy asymptotically. 
%
Our results confirm that DFT+$U$ 
is  capable of precisely correcting 
the total-energy SIE of a one-electron system 
under ideal population-analysis conditions but especially, it seems,
when using 
the appropriate self-consistency scheme, $U^{(2)}$.
%
It is clear then, that DFT+$U$ 
is an efficient and effective corrector for the
SIE  manifested in the total-energy, 
as discussed in detail in 
Refs.~\cite{PhysRevLett.97.103001,:/content/aip/journal/jcp/133/11/10.1063/1.3489110,:/content/aip/journal/jcp/145/5/10.1063/1.4959882}. 


%BINDING CURVE PARAMETERS
\subsubsection{Binding Curve Parameters}
\label{sec:binding_curve_params}
 
In order to further quantify the results  
of the various Hubbard $U$ self-consistency schemes tested, 
and draw attention to the importance 
of population analysis in the dissociation limit, 
we determined the equilibrium 
bond-length $R_e$, 
dissociation energy $E_D$, 
harmonic frequency $\omega_e$ 
and anharmonicity $\omega_e\chi_e$, 
corresponding to  each, as shown in  Table.~\ref{table2}, 
by fitting a polynomial  
about the energy  minima.
 %
Compared to the 
experimental data of 
Ref.~\cite{HERZBERG1972425}, 
the exact calculations perform very well in determining 
the bond-length and harmonicity 
in particular with errors that reflect the 
inaccuracies due to our fitting scheme, finite 
computational basis set size, core pseudisation, and
absent physical effects, as well as experimental factors.
%
\begin{table}[th!]
\begin{tabular*}{\columnwidth}{@{\extracolsep{\fill}}lrrrr}
\hline \hline
 &$R_e$  	&$E_D$	 &$\omega_e$&$\omega_e\chi_e$\\
\hline
Experiment~\cite{HERZBERG1972425}
			& 1.988		& 2.6508		& 2321.7		& 66.2 \\
Exact		& 1.997		& 2.7922		& 2323.6	& 59.9 \\
PBE			& 2.138		& 2.9893		& 1912.0	& 37.9 \\
PBE+$U^{(1)}$	& 1.963		& 2.957(3)		& 2346(6)		& 57.7(3)\\
PBE+$U^{(2)}$	& 1.827		& 2.990(3) 	& 2799(5)		& 81.2(4)\\
PBE+$U^{(3)}$	& 1.845		& 2.985(9) 	& 2721(9)		& 76.9(7)\\
PBE+$U_1 + U_2$ & 1.829 	& 2.990(2)		& 2810(3)		& 81.8(6)\\ 
\hline
\hline
\end{tabular*}
\caption{
Equilibrium bond-lengths $R_e$ (bohr), dissociation energy $E_D$ ($eV$), 
harmonic frequencies $\omega_e\ (\textrm{cm}^{-1})$, 
and anharmonicities $\omega_e\chi_e\ (\textrm{cm}^{-1})$ 
for each calculation scheme tested, for comparison with 
experimental values~\cite{HERZBERG1972425}.}
\label{table2}
\end{table}

The PBE functional, meanwhile, 
overestimates the equilibrium bond-length 
and dissociation energy,  
and underestimates the harmonic frequency and anharmonicity.
%
In order to ensure a complete comparison 
between all methods explored in this dissertation  
we have also included 
the generalised DFT+$U_1$+$U_2$ 
functional\footnote{Here we have determined $U_1$ and $U_2$ 
by setting $U= U^{(2)}$ in Eq.~\eqref{eq:U1U2}} 
which is defined in Chapter~\ref{ch:non_linear_constraints}.
%

%
The various DFT+$U$ schemes tested
generally preserve the PBE dissociation energy 
but they underestimate the equilibrium bond-length 
and overestimate the frequency and anharmonicitiy.
%
The DFT+$U^{(2)}$,  DFT+$U^{(3)}$ 
and DFT+$U_1$+$U_2$ schemes, in particular 
are found to over-correct the latter three quantities 
and predict the experimental data as 
poorly as the uncorrected PBE, 
albeit in the opposite direction.
%
We attribute this to  imperfect
DFT+$U$ population analysis at shorter bond-lengths,
featuring both  double-counting across the two
subspaces and spillage, 
as well as the breakdown of the subspace-bath separation.
%
The double-counting, in particular is 
not properly compensated by the 
self-consistently calculated  $U$ parameters, 
since the formula for $U_\textrm{out}$ 
does not take orbital overlap into account.
%
Conversely, 
the $U^{(1)}$ scheme performs well around equilibrium, 
as reflected also in Fig.~\ref{fig:h2+_uout_vs_r}, 
and approximately recovers the exact 
bond-length, harmonic frequency and anharmonicity.
%
We again emphasise that this fortuitous outcome 
is due entirely to  the $U^{(1)}$ parameter 
simply being smaller by a factor of two by definition, 
so that the over-correction due to double-counting
is approximately halved.
% 
It therefore coincides with the exact regime serendipitously, 
rather than by deliberate design.


It is therefore clear that while 
the $U^{(2)}$ scheme is the one that correctly  
eliminates SIE in the one-electron system, 
for our variational linear-response 
definition of the Hubbard $U$,
this choice hinges on the correct population 
analysis of the subspace at hand, 
as well as the assumption of a 
weak subspace-bath interaction, 
neither of which are met 
at the equilibrium bond-length of H$_2^+$.



%SELF-CONSISTNET U IN MULTIELECTRONIC SYSTEMS
\section[Coupled self-consistent parameters in multi-electronic systems]{Coupled self-consistent parameters in \break multi-electronic systems}

In spite of the simplicity 
of the self-consistent Hubbard $U$ formulation 
for a one-electron system, 
this particular formula is not expected to apply in general.
%
We are therefore obliged to derive 
the self-consistency formula again 
for multi-electronic systems, 
including those that are open-shell.
%
Nonetheless, 
the correct self-consistency condition that we previously identified, 
i.e., that which completely removes the SIE 
subject to the standard self-consistent response of the density $U^{(2)}$, 
remains valid since the +$U$ correction still applies to one electron at a time.
% 
However, we must now consider a potential constructed from both spins.
%

To begin constructing this generalisation, 
let us revisit the spin-resolved expression  
for $U$ in Eq~\eqref{eq:u_interaction}, 
including the spin-dependent 
Hubbard potentials $V_{U_\textrm{in}}^\sigma$.
%
Thus, 
by performing the same procedure as before, 
we may compute the self-consistent Hubbard $U$ 
for a multi-electronic system as follows 
%
\begin{align}
U_\textrm{out}&
=\frac{1}{2}\left(\frac{dV_\textrm{int}^\uparrow}{dN}+\frac{dV_\textrm{int}^\downarrow}{dN}+\frac{dV_{U_\textrm{in}}^\uparrow}{dN}+\frac{dV_{U_\textrm{in}}^\downarrow}{dN}\right)\nonumber \\[0.75em]
&=\frac{1}{2}\left(\frac{dV_\textrm{int}^\uparrow}{dN}+\frac{dV_\textrm{int}^\downarrow}{dN}\right)
-\frac{U_\textrm{in}}{2}\left(\frac{dN^\uparrow}{dN}+\frac{dN^\downarrow}{dN}\right)\nonumber \\[0.75em]
&=F_\textrm{Hxc}^{\hat{P}}(U_\textrm{in})-\frac{U_\textrm{in}}{2}. 
\label{eq:u_self_consistent_interaction1}
\end{align}
%
In this expression, 
the subspace-averaged Hxc interaction 
$F^{\hat{P}}_\textrm{Hxc} (U_\textrm{in})$, 
as defined in Eq.~\eqref{eq:u_interaction}, 
is now an implicit function of $U_\textrm{in}$.
%
However, 
according to this self-consistency formula, 
the condition needed to completely remove 
the total subspace SIE,
i.e. that for which $U_\textrm{out}=0$, 
now double-counts the total average interactions
%
\begin{equation}
U_\textrm{in}=2F_\textrm{Hxc}^{\hat{P}}
=\frac{1}{2}\left(F^{\uparrow\uparrow}_\textrm{int}
+F^{\uparrow\downarrow}_\textrm{int} +
F^{\downarrow\downarrow}_\textrm{int}
+F^{\downarrow\uparrow}_\textrm{int}\right).
\end{equation}
%

This result seems to suggest that the  
self-consistency scheme that 
completely removes the SIE for a
multi-electronic system is therefore 
$U^{(1)}\equiv U_\textrm{in}=U_\textrm{out}$, 
which would introduce the required factor 
of one-half in order to re-scale the interaction.
%
Indeed, 
this choice seems entirely reasonable 
and seems to have been the basis for the 
self-consistent criteria used in multiple 
studies~\cite{doi:10.1021/jp070549l,0953-8984-22-5-055602,PhysRevB.84.115108,PhysRevB.93.085135,PhysRevLett.106.118501,doi:10.1063/1.4947240}.
%
We argue, however, 
that this choice is incorrect, 
not only because it directly contradicts the 
more physically intuitive result derived in the previous section, 
but also because the expression for $U_\textrm{out}$ 
now contains a mix of like and unlike-spin interactions 
(that were naturally absent from the one-electron case).
%
{
Hence, in a multi-electronic system, 
the unlike-spin interactions are likely 
to be of a similar magnitude as the like-spin interactions 
and thus have an equal, but unwanted, effect.
%
Moreover, the former do not feature 
at all in standard DFT+$U$.}
%

Moreover, 
as outlined in discussing around Eq.~\eqref{eq:u-j_interaction}, 
a more natural formulation of $U$, 
that is to say, consistent with its implementation 
in the DFT+$U$ functional, 
is one comprised of like-spin interactions only, 
i.e., $U_\textrm{eff}\sim F^{\sigma\sigma}$.
%$U-J=(F^{\uparrow\uparrow}+F^{\downarrow\downarrow})/2$.
%
It is therefore clear 
that we will need to consider interactions 
derived from the exchange parameter $J$ 
in the context of a full, generalised $\textrm{DFT}+U+J$ 
formulation~\cite{PhysRevB.79.035103,
PhysRevB.57.1505,PhysRevB.84.115108,PhysRevB.44.943,PhysRevB.60.10763,PhysRevLett.102.226401,PhysRevB.62.16392}, 
in order to fully account for the multi-electronic 
self-consistent calculation of $U$, 
and {vice versa}.
%
In other words, 
the missing ingredient to reconnect the one-electron 
self-consistent condition with the multi-electronic one, 
is the explicit consideration of a $J_\textrm{in}$ term.


%SELF-CONSISTENT U
\subsection{Generalised, self-consistent $U$ and $J$ formulae}

If we wish to construct a generalised 
self-consistency formula for $U$ 
we must also consider potentials 
coupling unlike-spin density matrices 
$\langle\hat{n}^\sigma \hat{n}^{\bar\sigma}\rangle$, 
according to the following, 
reduced DFT+$U$+$J$ correction 
functional
%
\begin{align}
E_{U+J}[\{n^{I\sigma}\}]=\sum_{I\sigma}\left\{\frac{U^I}{2}\right.&\left.\mbox{Tr}[\hat{n}^{I\sigma}-\hat{n}^{I\sigma}\hat{n}^{I\sigma}]+\frac{J^I}{2}\textrm{Tr}[\hat{n}^{I\sigma}\hat{n}^{I\bar\sigma}]\right\}\nonumber \\[0.5em]
\mbox{where}\qquad&
\hat{V}^\sigma_{J}=\frac{\delta E_{J}}{\delta \hat{n}^{I\sigma}}
=J^I\hat{n}^{I\bar\sigma}.
\label{eq:dft+u+j_functional2}
\end{align}
%
We have arranged this functional intentionally 
so that $U$ only corresponds to interactions 
between density matrices of the same spin, 
i.e., $U_\textrm{eff}$,  
and $J$ corresponds to interactions of opposite 
spin\footnote{An alternative derivation of the 
following using the original DFT+$U$+$J$ 
functional in Eq.~\eqref{eq:dft+u+j_functional} 
is presented in Appendix~\ref{ch:self_consistency_appendix}.}.

Now, returning to Eq.~\eqref{eq:u_self_consistent_interaction1}, 
we may re-derive the self-consistency formula 
for a single-site model including contributions from $V^\sigma_{J_\textrm{in}}$ 
%
\begin{align}
U_\textrm{out}&
=\frac{1}{2}\left(\frac{dV_\textrm{int}^\uparrow}{dN}+\frac{dV_\textrm{int}^\downarrow}{dN}
+\frac{dV_{U_\textrm{in}}^\uparrow}{dN}+\frac{dV_{U_\textrm{in}}^\downarrow}{dN}
+\frac{dV_{J_\textrm{in}}^\uparrow}{dN}+\frac{dV_{J_\textrm{in}}^\downarrow}{dN}\right)
\nonumber \\[0.75em]
&=\frac{1}{2}\left(\frac{dV_\textrm{int}^\uparrow}{dN}+\frac{dV_\textrm{int}^\downarrow}{dN}\right)
-\frac{U_\textrm{in}}{2}\left(\frac{dN^\uparrow}{dN}+\frac{dN^\downarrow}{dN}\right)
+\frac{J_\textrm{in}}{2}\left(\frac{dN^\uparrow}{dN}+\frac{dN^\downarrow}{dN}\right)
\nonumber \\[0.75em]
&=F_\textrm{Hxc}^{\hat{P}}(U_\textrm{in},J_\textrm{in})-\frac{U_\textrm{in}}{2}+\frac{J_\textrm{in}}{2}. 
\label{eq:u_self_consistent_interaction2}
\end{align}
%
It is then clear that the calculation of $U_\textrm{out}$ 
depends non-trivially on the applied $J_\textrm{in}$, 
and naturally reduces to 
Eq.~\eqref{eq:u_self_consistent_interaction1} 
when the latter is set to zero.
%
However, 
the interaction kernel $F_\textrm{Hxc}$ 
is still the subspace average of {\it all} 
Coulomb and XC interactions, 
as it was in Eq.~\eqref{eq:u_interaction}, 
and so we are no closer to acquiring 
like-spin only interactions with this formulation alone.
%
To proceed, 
we must calculate the corresponding 
self-consistency equation for $J_\textrm{out}$, 
while considering an applied $U_\textrm{in}$, 
which will comprise the interaction kernel $F_J$, 
defined in Eq.~\eqref{eq:j_interaction}.
%
We therefore proceed as follows
%
\begin{align}
J_\textrm{out}&
=-\frac{1}{2}\left(\frac{dV_\textrm{int}^\uparrow}{dM}-\frac{dV_\textrm{int}^\downarrow}{dM}
+\frac{dV_{U_\textrm{in}}^\uparrow}{dM}-\frac{dV_{U_\textrm{in}}^\downarrow}{dM}
+\frac{dV_{J_\textrm{in}}^\uparrow}{dM}-\frac{dV_{J_\textrm{in}}^\downarrow}{dM}\right)
\nonumber \\[0.75em]
&=-\frac{1}{2}\left(\frac{dV_\textrm{int}^\uparrow}{dM}-\frac{dV_\textrm{int}^\downarrow}{dM}\right)
+\frac{U_\textrm{in}}{2}\left(\frac{dN^\uparrow}{dM}-\frac{dN^\downarrow}{dM}\right)
+\frac{J_\textrm{in}}{2}\left(\frac{dN^\uparrow}{dM}-\frac{dN^\downarrow}{dM}\right)
\nonumber \\[0.75em]
&=F_J^{\hat{P}}(U_\textrm{in},J_\textrm{in})+\frac{U_\textrm{in}}{2}+\frac{J_\textrm{in}}{2}. 
\label{eq:j_self_consistent_interaction2}
\end{align}
%

We have now constructed a pair of 
coupled equations expressing   
$U_\textrm{out}$ and $J_\textrm{out}$ 
in terms of the applied parameters 
$U_\textrm{in}$ and $J_\textrm{in}$, 
and the intrinsic interaction kernels 
$F_\textrm{Hxc}$ and $F_J$.
%
Computing the linear combinations of 
Eqs.~\eqref{eq:u_self_consistent_interaction2}~\&~\eqref{eq:j_self_consistent_interaction2} 
thus decouples the expressions and yields 
a set the effective Coulomb parameters 
in terms of the respective input parameters 
%
\begin{align}
U_\textrm{eff}=U_\textrm{out}-J_\textrm{out} = \left(F_\textrm{Hxc}-F_J\right)-U_\textrm{in}\nonumber \\
J_\textrm{eff}=U_\textrm{out}+J_\textrm{out} = \left(F_\textrm{Hxc}+F_J\right)+J_\textrm{in}.
\label{eq:uoutjout}
\end{align}

We may now invoke the 
self-consistency condition 
previously identified to completely remove  
the one-site interaction error, 
i.e., $U_\textrm{eff}=0$,   
such that the input parameters corresponding 
to this condition pertains to 
like-spin interactions only, as desired, 
%
\begin{align}
U_\textrm{in} = \left(F_\textrm{Hxc}-F_J\right)=\frac{1}{2}\left(F^{\uparrow\uparrow}+F^{\downarrow\downarrow}\right)
\label{eq:ueff}
\end{align}.
%
If we operate under the assumption that 
the same self-consistency condition will apply 
to $J$ then we arrive at a similar formula related 
to the correction of interactions between un-like spins
%
\begin{align}
J_\textrm{in} = -\left(F_\textrm{Hxc}+F_J\right)=-\frac{1}{2}\left(F^{\uparrow\downarrow}+F^{\downarrow\uparrow}\right).
\label{eq:jeff}
\end{align}
%
Hence, in the multi-electronic formulation, 
subject to the correction functional presented in 
Eq.~\eqref{eq:dft+u+j_functional2}, 
we predict that a combined self-consistency condition 
applied to both $U_\text{in}$ and $J_\text{in}$ 
may become satisfied 
%
\begin{equation}
U_\textrm{eff}(U_\textrm{in},J_\textrm{in})=J_\textrm{eff}(U_\textrm{in},J_\textrm{in})=0
\end{equation}
%
and are precisely those that relate to the interactions we wish to address.

We have now resolved the question we posed 
at the close of Chapter~\ref{ch:calculating_hubbard_u}.
%
The application of $U_\textrm{in}$ serves, 
as we have already seen, 
to decrease the computed $U_\textrm{out}$ 
by mitigating the delocalisation 
and reducing the spurious electronic curvature.
%
This leads to a convenient linear relationship between the two 
that forms the basis and motivation for the self-consistency regime.
%
Incidentally, 
$U_\textrm{in}$ has the opposite effect on $J_\textrm{out}$ 
and increases the resulting magnetic responses.
%
This makes sense if we recall that 
$U_\textrm{in}$ enforces localisation of both spin-channels, 
such that the energy cost of perturbing either increases 
as the electron response stiffens and, thus, 
the curvature with respect to magnetisation increases.

Meanwhile, 
$J_\textrm{in}$ increases both the computed 
$U_\textrm{out}$ and $J_\textrm{out}$ 
for similar reasons.
%
The $J$ term encourages polarisation, 
thereby promoting same-spin charge to occupy the same region, 
which increases the self-interaction error 
and hence the magnitude of $U_\textrm{out}$.
%
Similarly, $J$ is related to the 
negative of the energy with respect to magnetisation 
(c.f. Eq.~\eqref{eq:j_curvature}) 
and so a positive $J$ will only exacerbate that curvature 
and increase the calculated $J_\textrm{out}$.


Finally, 
we can speculate why the self-consistency condition 
of $U_\textrm{in}=U_\textrm{out}$,
chosen in Refs.~\cite{doi:10.1021/jp070549l,0953-8984-22-5-055602,PhysRevB.84.115108,PhysRevB.93.085135,PhysRevLett.106.118501,doi:10.1063/1.4947240}, 
produces a $U_\textrm{out}$ of the correct magnitude 
but for the wrong reason. 
%
In these multi-electronic systems 
there is no self-consistent treatment of the exchange term.
%
As such, 
the calculated $U_\textrm{out}$, 
or the average on-site interaction, 
is likely over-estimated by an approximate factor of two~\cite{doi:10.1021/jp070549l,PhysRevB.93.085135}, 
which explains $U_\textrm{out}=U_\textrm{in}$ 
as the suggested self-consistency condition.

The method presented here, 
on the other hand, 
provides a complete picture of 
the underlying interactions by 
explicitly incorporating the $J$-term.
%
Moreover, 
the interaction terms, 
to which the $U$ and $J$ parameters correspond, 
are revealed to be consistent with their 
intended application in the DFT+$U$+$J$ functional.
%
We shall now describe extensive numerical calculations on NiO, 
which showcase all plausible combinations of self-consistent $U$ and $J$, 
to illustrate that our derived scheme is indeed correct.



%REVISITING NiO
\subsection{Application to NiO}

Presented in Fig.~\ref{fig:NiO_uin_vs_uout} 
are the calculated profiles of 
$U_\textrm{in}$ vs $U_\textrm{out}$ (solid), 
$U_\textrm{in}$ vs $J_\textrm{out}$ (dashed), 
and corresponding 
$U_\textrm{in}$ vs $U_\textrm{eff}$ (dotted), 
where each exhibits a strongly linear behaviour.
%
For $U_\textrm{out}$ we calculate a 
slope$=-0.64(2)$, 
and for $J_\textrm{out}$ we get a 
slope$=0.486(4)$, 
for which the resulting profile for 
$U_\textrm{eff}=U_\textrm{out}-J_\textrm{out}$ 
has a  slope of $-1.12(2)$.

%
\begin{figure}[th!]
\centering
\includegraphics[height=0.494\textwidth]{images/NiO_uin_vs_uout.pdf}
\caption[Profiles of  $U_\textrm{out}$, $J_\textrm{out}$,  $U_\textrm{eff}$ vs $U_\textrm{in}$ for NiO]
{Curves, with accompanied error bars, 
for $U_\textrm{out}$ (solid), 
$J_\textrm{out}$ (dashed), 
and $U_\textrm{eff}$ (dot-dashed), 
with respect to the applied $U_\textrm{in}$ for NiO. 
%
From the linear fits we compute 
$U^{(1)}=4.1(1)$~eV, 
$U^{(2)}=10.5(5)$~eV, 
$U^{(3)}=6.7(1)$~eV, 
with $J_\textrm{out}=5.93(7)$~eV 
at $U_\textrm{in}=10.5$~eV.}
\label{fig:NiO_uin_vs_uout}
\end{figure}
%

%
The derived values for $U$ 
for each self-consistency condition, 
along with the associated $J_\textrm{out}$ 
and interaction kernel $F_J$ 
at the corresponding $U_\textrm{in}$, 
are presented in the left hand side of 
Table~\ref{table:NiO_uout_values}.
%
In the right hand side of Table~\ref{table:NiO_uout_values}, 
we present the same values for 
the self-consistency conditions 
applied to $U_\textrm{eff}$ curve.

\begin{table*}[th!]
\centering
\begin{tabular}{lrrr|rrr}
\hline\hline
Condition	&$U$ (eV)	&$J_\textrm{out}$ (eV)&$F_J$ (eV)&$U_\textrm{eff}$ (eV)&$J_\textrm{out}$ (eV)&$F_J$ (eV)\\	
\hline
$U^{(1)}$	&4.1(1)	&2.82(4)	&0.78(4)	&2.7(1)	&2.17(4)	&0.80(4)\\
$U^{(2)}$	&10.5(5)	&5.93(7)	&0.69(7)	&5.2(2)	&3.36(6)	&0.77(7)\\
$U^{(3)}$	&6.7(1)	&0.84(3)	&0.84(3)	&5.8(1)	&0.84(3)	&0.84(3)\\
\hline\hline
\end{tabular}
\caption[Summary of calculated $U$, $J$ and $F_J$ values for each self-consistency condition applied to NiO.]{
The values determined from $U_\textrm{out}$ (left) for each self-consistency scheme 
applied to NiO 
with the corresponding $J_\textrm{out}$ and interaction kernel $F_J$.
%
On the right hand side are the same conditions 
applied to the $U_\textrm{eff}$ curve.}
\label{table:NiO_uout_values}
\end{table*}

The calculated values for $U^{(1)}$ and $U^{(3)}$ 
here seem quite reasonable 
and fall in range of the values 
calculated in other studies~\cite{PhysRevB.44.943,PhysRevB.60.10763,PhysRevLett.102.226401,PhysRevB.62.16392,PhysRevB.71.035105}.
%
This lends credence to 
the selection of one of these values 
as a self-consistency scheme 
in absence of the arguments 
made in the previous section.
%
In comparison, 
the value of $U^{(2)}$ is much larger 
than the necessary correction, 
which, in isolation, over-corrects  
the SIE by double-counting 
the average subspace interaction.

Meanwhile, 
the corresponding $J_\textrm{out}$ values 
span a relatively large range 
for the $U_\textrm{in}$. 
%
For instance, 
the $J_\textrm{out}$ related 
to $U^{(1)}$ and $U^{(3)}$ seem quite plausible, 
and are very similar to previous values~\cite{PhysRevB.44.943,PhysRevB.60.10763,PhysRevLett.102.226401,PhysRevB.62.16392}, 
but the $J_\textrm{out}$ for $U^{(2)}$ 
is certainly much too large.
%
Since $J_\textrm{out}$ scales linearly 
with the applied $U_\textrm{in}$ 
(c.f. Eq.~\eqref{eq:j_self_consistent_interaction2})
it may be more appropriate to pair the $U_\textrm{out}$ values 
with the corresponding interaction terms $F_J$, 
plotted in Fig.~\ref{fig:NiO_interactions}, 
which are all $<1$~eV.
%
Moreover, 
we will also consider the effective $U-J$ curve 
acquired from the pairs of $U$, $J$ that arise from 
each self-consistency point on the $U_\textrm{out}$ curve.

We emphasise, however,
 that these are not necessarily the same 
values that arise from the self-consistency conditions  
applied to the $U_\textrm{eff}$ curve itself.
%
Applying the self-consistency schemes 
to the $U_\textrm{eff}$ curve yields different values, 
which also all seem very plausible.
% 
However, in this analysis, we will 
restrict our investigation to  
$U^{(2)}_\textrm{eff}$.
%as we can confidently predict that 
%$U^{(1)}_\textrm{eff}$ is too small.
%
Finally, 
we shall pair $U^{(2)}_\textrm{eff}$ 
with the corresponding $J_\textrm{out}$ and $F_J$, 
where these parameters will be applied 
explicitly to unlike spin interactions.


%%%%%%%%%%%%%%%%%%%%%%%%%%%%%%%%
From the data in Fig.~\ref{fig:NiO_uin_vs_uout}, 
we may also derive the corresponding interaction kernels 
$F_\textrm{Hxc}$ (circles) and  $F_J$ (squares), 
and the average of the like-spin and unlike-spin interactions
$F^{\sigma\sigma}$ (up triangles) 
and $F^{\sigma\bar{\sigma}}$ (down triangles), 
which are presented in 
Fig.~\ref{fig:NiO_interactions}.
%
We see here that the largest interactions 
occur between un-like spins 
$F^{\sigma\bar{\sigma}}$, 
which diminish from 7.5~eV to 6~eV 
over the range of $U_\textrm{in}$.
% 
This is similar in magnitude to $F^{\sigma\sigma}$, 
as we earlier predicted, 
which takes values between 6~eV to 4.5~eV 
over the same range.
%
The average of the two returns 
the interaction kernel $F_\textrm{Hxc}$, 
which then varies from 6.5~eV to 5~eV 
and corresponds to the $U$ values 
in Table~\ref{table:multi_electron_results} 
that give the best agreement with experiment 
according to Fig.~\ref{fig:NiO_compare}.

Consequently, 
half the difference between 
$F^{\sigma\bar{\sigma}}$  and $F^{\sigma\sigma}$ yields $F_J$, 
which is much smaller by comparison ($\approx$0.7~eV), 
but compares well to $J$ determined in other works 
(see Table~\ref{table:multi_electron_results}) 
but particularly with cRPA~\cite{PhysRevB.81.245113,PhysRevB.87.165118,zhang2017dft+,PhysRevB.82.045108,PhysRevB.96.045137,PhysRevB.85.045132}.
%
It changes very little with respect to $U_\textrm{in}$, 
between 0.8-0.6~eV,
and is of the correct order of 
exchange parameters typically 
used in spin-exchange correction 
(see Table~\ref{table:multi_electron_results}).

Finally, 
we observe that the slopes for 
$F_\textrm{Hxc}$, $F^{\sigma\sigma}$ and $F^{\sigma\bar{\sigma}}$ 
are approximately equal  
and are calculated to be $\approx-0.14$.
%
This corresponds to very little variation in these interactions, 
no more than $\approx1.5$~eV, 
over a range of 10~eV for $U_\textrm{in}$ 
and indicates a high resistance to SIE correction.

\begin{figure}[th!]
\centering
\includegraphics[height=0.494\textwidth]{images/NiO_interactions.pdf}
\caption[Profiles of $F_\textrm{Hxc}$, $F_J$,  $F^{\sigma\sigma}$, 
$F^{\sigma\bar{\sigma}}$ vs $U_\textrm{in}$ for NiO]
{Curves, with accompanied error bars, 
for the interaction kernels 
$F_\textrm{Hxc}$ (circles) and $F_J$ (squares), 
as well as the average like and unlike-spin interactions 
$F^{\sigma\sigma}$ (up triangles) and $F^{\sigma\bar{\sigma}}$ (down triangles), 
with respect to the applied $U_\textrm{in}$ for NiO.}
\label{fig:NiO_interactions}
\end{figure}


We performed each calculation 
from the selection presented above, 
in which the applied $U$ and $J$
parameters are those that feature in 
the original DFT+$U$+$J$ functional 
in Eq.~\eqref{eq:dft+u+j_functional}.
%
%We also perform one calculation 
%with the spin-minority term 
%referenced in Ref.~\cite{PhysRevB.84.115108} 
%for with $U^{(2)}+J_\textrm{min}^{(2)}$.
%
Values from the $U_\textrm{eff}$ curve, however, 
along with the accompanying $J$ or $F_J$, 
are used instead in the 
modified DFT+$U$+$J$ functional 
in Eq.~\eqref{eq:dft+u+j_functional2} 
since the $U-J$ scaling has already taken 
place\footnote{Equivalently, 
one could use Eq.~\eqref{eq:dft+u+j_functional} 
with parameters $(U_\textrm{eff}+J)$ and $J$.}.

In the following section, 
we discuss a subset of these calculations in detail 
and present the associated DOS 
plots\footnote{There are some redundancies, however, e.g., $U^{(3)}+J^{(3)}=U^{(3)}+F_J^{(3)}$, 
and $(U-J)^{(3)}=U_\textrm{eff}^{(3)}$.}.
%
In particular we compare 
the $p-s$ band gap $E_{p-s}$, 
the $p-d$ transition energy $E_{p-d}$ 
(quantified by the position of the peak in the conduction band), 
and the magnetic moment $m$ 
to the experimental measurements.
%
The remaining DOS plots are 
available in Appendix~\ref{ch:self_consistency_appendix}.

We also compute the total 
normalised root-mean-square (NRMS) error 
of these results compared to the experimental values, 
and rank the schemes based on this overall metric 
in Fig.~\ref{fig:NiO_compare}.
%
Here, the shaded regions denote 
approximate margins of error,
which for $E_{p-s}$ and  $E_{p-d}$
is taken to be $\pm0.5$~eV, 
and for for the magnetic moment 
the boundary is represented by 
the two experimental values.
%
A comprehensive summary 
of the results is presented (not in order of performance) 
in Table~\ref{table:multi_electron_results}, 
in which we have highlighted the five schemes 
with the best overall agreement in bold font.


\newpage
\begin{table*}[th!]
\centering
\resizebox{\textwidth}{!}{%
\begin{tabular}{lrrrrr}
\hline\hline
Calculation &$U$ (eV)  	&$J$ (eV)	 &$E_{p-s}$ (eV)&$E_{p-d}$ (eV)&$m$ ($\mu$B)\\
\hline
Experiments~\cite{Hufner1992,0953-8984-11-7-002,PhysRevB.43.14674,PhysRevLett.53.2339,alperin1962nio,PhysRevB.27.6964}		
		& -		& -		& 3.1&4.3	&1.64-1.9\\[0.25em]
Refs.~\cite{PhysRevB.44.943,PhysRevB.60.10763,PhysRevLett.102.226401}	
				& 8.0		& 0.95	& 3.1-3.4	& -	& 1.56-1.74 \\[0.25em]
Ref.~\cite{PhysRevB.62.16392}& 5.0 & 0.95 & 2.8 & -& 1.73 \\[0.25em]
Ref.~\cite{PhysRevB.71.035105}& 4.6 & 0.0 & 2.7 &- & 1.7 \\[0.25em]
\hline
PBE						& 0.0		& 0.0 	&1.66	&2.27	& 1.37\\[0.25em]
{\bf PBE+$U^{(1)}$}				& 4.1 	& 0.0		&2.96	&3.65	& 1.53\\ [0.25em]
PBE+$U^{(2)}$				& 10.5 	& 0.0		&3.06	&5.84	& 1.73\\ [0.25em]
{\bf PBE+$U^{(3)}$}				& 6.7 	& 0.0		&3.05	&4.44	& 1.62\\ [0.25em]
PBE+$U^{(1)}$+$J^{(1)}$		& 4.1		&2.82	&1.52	&3.67	& 1.68\\[0.25em]
PBE+$U^{(2)}$+$J^{(2)}$		& 10.5	& 5.9		&0.63	&6.87	& 1.86\\[0.25em]
PBE+$U^{(3)}$+$J^{(3)}$		& 6.7		& 0.84	&2.95	&4.87	& 1.66\\[0.25em]
{\bf PBE+$U^{(1)}$+$F^{(1)}_J$}	& 4.1		& 0.78	&2.86	&3.97	& 1.58\\[0.25em]
PBE+$U^{(2)}$+$F^{(2)}_J$	& 10.5	& 0.69	&3.01	&6.22	& 1.75\\[0.25em]
%{PBE+$U^{(3)}$+$F^{(3)}_J$}& 6.7	& 0.84	&2.95	&4.87	& 1.66\\[0.25em]
PBE+$(U-J)^{(1)}$			& 1.3		& 0		&2.17	&2.78	& 1.43\\[0.25em]
PBE+$(U-J)^{(2)}$			& 4.6		& 0		&2.99	&3.80	& 1.54\\[0.25em]
{\bf PBE+$(U-J)^{(3)}$}		& 5.9		& 0		&3.01	&4.19	& 1.59\\[0.25em]
{\bf PBE+$U_\textrm{eff}^{(2)}$}	& 5.2		& 0		&3.01	&3.97	& 1.57\\[0.25em]
%{PBE+$U_\textrm{eff}^{(3)}$}	& 5.9		& 0	&2.95	&4.87	& 1.66\\[0.25em]
PBE+$U_\textrm{eff}^{(2)}$+$J_\textrm{eff}^{(2)}$ 
						& 5.2 	& 3.48	&2.55	&6.66	& 1.80\\[0.25em]
PBE+$U_\textrm{eff}^{(2)}$+$f_{J,\textrm{eff}}^{(2)}$ 
						& 5.2		& 0.84	&2.94	&4.63	& 1.64\\[0.25em]
\hline\hline
\end{tabular}
}
\caption{Summary of DFT+$U$+$J$ parameters for NiO, 
from this work and similar studies,  
such as the $U$ (eV), $J$ (eV), 
$p-s$ band gap $E_{p-s}$ (eV), $p-d$ transition energy $E_{p-d}$ (eV), 
and magnetic moment $m$ ($\mu$B) 
compared to experimental data. 
%
The five schemes with the best overall agreement 
with experiment are given in bold.}
\label{table:multi_electron_results} 
\end{table*}


\newpage
\begin{figure}[th!]
\centering
\begin{subfloat}[]{
\includegraphics[height=0.65\textheight]{images/NiO_bandgap_compare.pdf}
}
\end{subfloat}
\begin{subfloat}[]{
\includegraphics[height=0.65\textheight]{images/NiO_magmom_compare.pdf}
}
\end{subfloat}
\begin{subfloat}[]{
\includegraphics[height=0.65\textheight]{images/NiO_dpeak_compare.pdf}
}
\end{subfloat}
\caption[Ranking of all self-consistency schemes for NiO]
{Ranking of all self-consistency schemes according 
to the total normalised root-mean-square error 
with respect to the experimental measurements for 
$E_{p-s}$~(eV), $m$~($\mu$B), 
and  $E_{p-d}$~(eV).
%
Shaded regions denote 
approximate margins of error, 
which is given by $\pm0.5$~eV 
for $E_{p-s}$ and  $E_{p-d}$, 
and by the experimental 
values for $m$.}
\label{fig:NiO_compare}
\end{figure}
\newpage


% COMPARISON OF SELF-CONSISTENCY SCHEMES
\subsection{Comparison of self-consistency schemes}

Before delving into the calculation details, 
let us examine the broader performance of 
the schemes against each other, 
as depicted in Fig.~\ref{fig:NiO_compare}, 
with regards to replicating the experimental results.
%
Here, the shaded regions illustrate heuristic 
(and ultimately arbitrary) 
margins of error for the experimental values, 
which are $3\pm0.5$~eV for the $p-s$ band gap 
and $4\pm0.5$~eV for the $p-d$ transition energy.
%
The shaded region for the magnetic moment, 
meanwhile, is bounded by the experimental values 
1.64~\cite{alperin1962nio}~$\mu$B and 1.9~\cite{PhysRevB.27.6964}~$\mu$B, 
respectively\footnote{However, 
based on the relative scatter of these results, 
and assuming a consistent level of accuracy 
between all three properties, 
one would be lead to believe the magnetic moment for NiO is $\approx$1.6~eV. 
If this were the case, $U_\textrm{eff}^{(2)}$ would give the best overall agreement.}. 
%
We find that the band gap is the easiest to reproduce  
since it is the result of a dispersive $s$ state, 
which is unperturbed (ignoring screening effects) 
by DFT+$U$+$J$ potentials.
%
In contrast, the magnetic moment is the most 
difficult and requires large $U$ or $J$ to attain 
at the expense of the other two properties, 
which is most often $E_{p-d}$.
%
In general, we can produce two of 
the three properties within the error 
bars\footnote{$U^{(3)}$ is the exception, 
however it does not provide the 
best overall performance.}.

We find the best eight calculations provide 
very good agreement with experiment.
%
In fact, all of these calculations invoke $U$ values 
in the range 4.1~eV to 6.7~eV 
(which, as we have seen, 
is the approximate range spanned by 
$F^{\sigma\sigma}$)
and only a small $J$, if any.

The next four calculations, 
between $(U-J)^{(1)}$ and $U^{(2)}+F_J^{(2)}$, 
quickly deteriorate in accuracy.
%
In these instances, 
the results are poor because a $U$ is applied 
that is either too small, 
as is the case for  $(U-J)^{(1)}$  and $U^{(1)}+J^{(1)}$, 
which underestimates the band gap, 
or too large, i.e., $U^{(2)}$, 
which pushes the conduction band peak too high in energy.

The next best is the PBE calculation, 
which underestimates all three quantities significantly 
yet, surprisingly, is not at the bottom of the list.
%
Indeed, the worst calculations 
are those that invoke the largest $J$ values, 
which have a detrimental effect on the 
calculation results regardless of the $U$.

Let us now inspect some calculations 
and DOS plots in detail.


%PBE and PBE+U3
\subsubsection{PBE and PBE+$U^{(3)}$} 
To compare the following results to the PBE DOS, 
we refer the reader to Fig.~\ref{fig:NiOpbedos}.
%
Since the conventional $U$ is computed 
without any applied $U_\textrm{in}$, 
as in the previous Chapter, 
it is very similar to $U^{(3)}$ 
such that it is sufficient to regard 
Fig.~\ref{fig:NiO6.7dos} 
as that pertaining to PBE+$U^{(1)}=6.7$~eV.
% 
Let us now look toward examining 
some of the remaining schemes, 
including $U^{(1)}$ and  $U^{(2)}$.

%PBE+U1
\subsubsection{PBE+$U^{(1)}$} 

The DOS computed from $U^{(1)}=4.1$~eV 
is plotted in Fig.~\ref{fig:NiO.4.1.dos}, 
and shows reasonable agreement with experiment 
in terms of the results 
for the band gap, transition energy and 
magnetic moment.
% 
It is not surprising then that this encouraging result
would lead users to choose this condition.
%
However, we argue that this selection 
is founded on more of an empirical, 
rather than mathematical, footing 
and in light of the evidence presented in preceding sections,  
we may discard this scheme  
in favour of better alternatives.

\begin{figure}[th!]
\centering
\includegraphics[height=0.494\textwidth]{images/NiO.4.1.dos.pdf}
\caption[NiO DOS with $U=4.1$~eV and $J=0$~eV]
{The species-resolved density-of-states of NiO 
with half-width Gaussian smearing of 0.1 eV 
and the zero-energy set to the Fermi level (eV) for  
PBE+$U^{(1)}=4.1$~eV.
%
The lines depict    
Ni majority-spin (green), 
Ni minority-spin (blue), 
O (red), 
total (black), 
and Hubbard total (dashed grey)}
\label{fig:NiO.4.1.dos}
\end{figure}



%PBE+U2
\subsubsection{PBE+$U^{(2)}$}

If we instead opt to use the self-consistency condition 
proven to be the best for a one-electron system, 
i.e., $U^{(2)}=10.5$~eV, 
and neglect the double-counting of the interaction 
when extended to multiple electrons,  
the resulting DOS is that presented in Fig.~\ref{fig:NiO.10.5.dos}.
%
Here, the large $U$ parameter 
clearly exceeds the necessary correction 
and causes the conduction band peak 
to migrate to 6~eV above the Fermi level.
%
If we attempt to mitigate the correction with a $J$, 
the corresponding DFT+$U$+$J$ calculation, 
shown in Fig.~\ref{fig:NiO.10.5.J.dos}, 
now pulls the conduction peak below the Fermi level 
and leaves behind sparse conduction states 
that are not observed in experiments at all.
%
%This result is a direct consequence of a 
%$J$ parameter that is much too large 
%for its direct use in the second term 
%of Eq.~\ref{Eq:dft+u+j_functional}.
%
Finally, 
if we include the supplementary term 
relating to orbital exchange in 
Eq.~7 of Ref.~\cite{PhysRevB.84.115108}, 
(PBE+$U^{(2)}+J_\textrm{min}^{(2)}$)
it makes matters worse still by breaking 
the spin-symmetry of the DOS 
and rendering it ferromagnetic.
% 
This is shown in Fig.~\ref{fig:NiO.10.5.J.minority}, 
and the comprehensive failure by all three approaches 
leaves us no choice but to abandon this scheme entirely.

\begin{figure}[th!]
\centering
\includegraphics[height=0.494\textwidth]{images/NiO.10.5.dos.pdf}
\caption[NiO DOS with $U=10.5$~eV and $J=0$~eV]
{The species-resolved density-of-states of NiO 
with half-width Gaussian smearing of 0.1 eV 
and the zero-energy set to the Fermi level (eV) for 
PBE+$U^{(2)}=10.5$~eV.}
\label{fig:NiO.10.5.dos}
\end{figure}


%PBE+U3+J3
\subsubsection{PBE+$U^{(3)}$+$J^{(3)}$}

Let us now consider augmenting 
the original PBE+$U^{(3)}$ calculation presented in 
Fig~\ref{fig:NiO6.7dos}, 
with the corresponding $J=0.84$~eV.
%
Depicted in Fig.~\ref{fig:NiO.6.7.J.dos} 
is the resulting DOS, 
in which there is qualitatively little change 
in comparison to its predecessor.
%
Some minor change of character 
in the valence states is observed 
between -4~eV and -2~eV, 
however, this is unlikely 
to affect the overall chemical behaviour significantly.
%
We therefore conclude that $J<1$~eV 
will not induce any significant 
changes in calculation results.

\begin{figure}[th!]
\centering
\includegraphics[height=0.494\textwidth]{images/NiO.6.7.J.dos.pdf}
\caption[NiO DOS with $U=6.7$~eV and $J=0.84$~eV]
{As in Fig~\ref{fig:NiO.4.1.dos}. 
The DOS for PBE+$U=6.7^{(3)}$~eV+$J^{(3)}=0.84$~eV.}
\label{fig:NiO.6.7.J.dos}
\end{figure}


%PBE+Ueff2
\subsubsection{PBE+$U_\textrm{eff}^{(2)}$}
Finally, the DOS for $U_\textrm{eff}=5.2$~eV is shown in 
Fig.~\ref{fig:NiO.5.2.dos}, 
and returns a qualitatively accurate profile 
based on current experimental measurements.
%
The band gap is found to be 3~eV 
while the $d$-peak correctly occurs at 4~eV, 
and the magnetic moment is 1.57~$\mu$B, 
which are all in excellent agreement 
with experimental values.

Moreover, 
the constituent states of the 
valence band from $-1.5$~eV 
up to the Fermi level are 
32\% Ni and 68\% O, 
in comparison to the conduction band
whose makeup as far as 5~eV 
is 88\% Ni and 12\% O, 
which correctly categorise NiO 
as a charge transfer type semi-conductor.
%
This self-consistency scheme therefore reproduces 
both qualitative and quantitative experiment results 
and shares this status with 
the PBE+$(U-J)^{(3)}$ calculation 
(shown in Fig.~\ref{fig:NiO.5.9.dos}). 
%
Given that the two $U$ values implemented 
differ only by $0.7$~eV, 
and give qualitatively similar results, 
we consider their treatments equivalent 
for our current purposes.

\begin{figure}[th!]
\centering
\includegraphics[height=0.494\textwidth]{images/NiO.5.2.dos.pdf}
\caption[NiO DOS with $U_\textrm{eff}=5.2$~eV]
{As in Fig~\ref{fig:NiO.4.1.dos}. 
The DOS for PBE+$U_\textrm{eff}=5.2$~eV.}
\label{fig:NiO.5.2.dos}
\end{figure}


We have by now demonstrated that 
our motivations for selecting $U_\textrm{eff}^{(2)}$ 
as the appropriate correction scheme are not only well 
founded on comprehensive mathematical analysis, 
but now strongly supported by numerical calculations 
that prove the same.
%
We have also shown that the conventional 
$U$ parameter may also be 
sufficient in some circumstances, 
and may perform even better when replaced by $U-J$.
%
We expect however, 
that self-consistency schemes will become essential 
when removing the SIE in systems that transition 
from a metallic to an insulating state 
when treated with DFT+$U$.
%
This kind of electronic behaviour 
may increase the differences between 
$U^{(1)}$, $U^{(2)}$ and $U^{(3)}$, 
as featured in Fig.~\ref{fig:NiO_interactions}.

Finally, we direct the reader to the remaining DOS figures 
presented in Appendix~\ref{ch:self_consistency_appendix}, 
which describe the other schemes in Table~\ref{table:multi_electron_results} 
not discussed here.
% 




%REVISITING H2
\section{A self-consistent $J$ applied to H$_2$}

Now that we have a firm handle 
on the self-consistent calculation of $U$ parameters, 
let us apply these lessons to calculate  
$J$ in the dissociation limit of H$_2$.
%
For every internuclear distance considered over the binding curve, 
we performed a $J_\textrm{in}$ vs $J_\textrm{out}$ 
according to Eq.~\eqref{eq:j_self_consistent_interaction2}, 
for which $U_\textrm{in}=0$ and negative $J_\textrm{in}$ 
values were applied in the modified DFT+$U$+$J$ functional 
in Eq.~\eqref{eq:dft+u+j_functional2}.
%
{
The negative  $J_\textrm{in}$ ensures 
that the system does not spontaneously polarise 
when a magnetic perturbation is applied.}
%
Across the binding curve, 
we found that the absolute magnitude of $J_\textrm{in}$ 
required to ensure this stable equilibrium 
increased with the bond-length.
% 

In Fig.~\ref{fig:H2_jin_vs_jout} 
we present two sample 
$J_\textrm{in}$ vs $J_\textrm{out}$ profiles 
corresponding to the limits of the binding curve 
at bond-lengths of 1~a$_0$ and 6~a$_0$, respectively.
%
{
For 1~a$_0$, 
the profile has an enormous 
slope of $-14.0\pm0.1$~eV$^{-1}$ 
which renders an extremely large 
$J^{(3)}=154.0\pm0.2$~eV, 
as well as more reasonable, 
but still over calculated, 
$J^{(1)}=10.3\pm0.1$~eV, 
$J^{(2)}=11.0\pm0.1$~eV.
%
These values are, without question, 
much too big for this bonding regime.
%
It is evident that 
the large subspace overlap near the equilibrium bond-length, 
coupled with the breakdown of a weakly interacting 
subspace with its bath, 
contributes to the calculation of 
excessively large linear-response parameters.

Similarly at 6~a$_0$ bond-length we also get 
highly accurate\footnote{Here, the errors are $\sim10^{-4}$~eV.} 
(and negative) 
$J^{(1)}=-1.9$~eV, 
$J^{(2)}=-3.0$~eV, 
$J^{(3)}=-5.2$~eV, 
with a slope of -1.72
which are much more reasonable indeed 
and shows that the calculated $J_\textrm{out}$ values 
change sign over the binding curve.}
%
{Moreover, 
we see the Coulson-Fischer point at $\approx3.5$~a$_0$ 
where the measured $J_\textrm{out}$ passes through zero.
%
This serves as a graphical illustration for the 
necessity of a $J_\textrm{in}$, 
especially after this point.
% 
A negative $J_\textrm{out}$ suggests that the 
system prefers an open-shell configuration 
but it is held in an unstable equilibrium by the spin-symmetry.
%
The energy difference between the two states 
is exactly the SCE (c.f. Fig.~\ref{fig:H2_sce}).}

\begin{figure}[th!]
\centering
\subfloat[]{
\includegraphics[height=0.45\textwidth]{images/H2_jin_vs_jout_1.0}
\label{fig:H2_jin_vs_jout_1.0}}
\quad
%
\subfloat[]{
\includegraphics[height=0.45\textwidth]{images/H2_jin_vs_jout_6.0}
\label{fig:H2_jin_vs_jout_6.0}}
%
\caption[$J_\textrm{in}$ vs $J_\textrm{out}$ calculations 
for H$_2$ at 1~a$_0$ and 6~a$_0$ bond-length]
{$J_\textrm{in}$ vs $J_\textrm{out}$ calculations for H$_2$, 
with error bars, 
at 
\subref{fig:H2_jin_vs_jout_1.0} 
1~a$_0$
\subref{fig:H2_jin_vs_jout_6.0} 
6~a$_0$ bond-length 
illustrating each of the 
self-consistency schemes 
$J^{(1)}$ (triangle), 
$J^{(2)}$ (diamond), 
and $J^{(1)}$ (circle).
%
We find a highly-linear profile across 
all bond-lengths for this system 
with very small errors despite 
only three data points.
%
Between these bond-lengths we see that 
the $J_\textrm{out}$ changes sign.}
\label{fig:H2_jin_vs_jout}
\end{figure}


%CORRECTING SCE WITH J
\subsection{Correcting SCE with $J$}

If we proceed to plot the various $J$ schemes 
against the estimated $J_\textrm{int}$ required to correct the total-energy 
(which is incidentally negative for all bond-lengths), 
some very interesting behaviour emerges, 
as shown in Fig.~\ref{fig:H2_jout_schemes}.
%
Unsurprisingly, 
all three schemes substantially overestimate 
the required $J$ for short bond-lengths, 
in which $J^{(1)}$ attains a maximum of 154~eV.
%
This is reminiscent of a similar issue
that occurred with H$_2^+$, 
albeit much worse, 
and is also due to imperfect population analysis 
caused by occupancy 
double-counting and charge spillage,
as well as the breakdown of the 
subspace-bath separation. 

Beyond the Coulson-Fischer point, however, 
and toward the dissociation limit, 
the $J$ values quickly decrease 
and attain much more reasonable values.
%
While $J^{(3)}$ is now underestimated, 
and $J^{(1)}$ is still slightly too large, 
it is $J^{(2)}$ that converges to the correct values 
and gives remarkable agreement with 
$J_\textrm{int}$ toward dissociation.

\begin{figure}[th!]
\centering
\includegraphics[height=0.494\textwidth]{images/H2_jin_vs_jout}
\caption[Calculated $J_\textrm{out}$ schemes for dissociating H$_2$]
{The estimated best $J$ value,
$J_\textrm{int}$ (solid),  for correcting the total-energy
SCE in H$_2$, shown with the
$J^{(1)}$ (dashed), 
$J^{(2)}$ (dotted), and 
$J^{(3)}$ (dot-dashed) values 
with error bars.
%
All schemes give values that are much too large 
at short bond-lengths. 
%
Beyond the Coulson-Fischer point 
$J^{(2)}$ converges on $J_\textrm{int}$ 
with excellent agreement.}
\label{fig:H2_jout_schemes}
\end{figure}
%


In Fig.~\ref{fig:H2_dft+jout} 
we plot the PBE+$J^{(2)}$ total-energy 
resulting from this scheme 
in comparison to the full configuration interaction (FCI) 
total-energy presented earlier.
%
We see that around equilibrium 
the total-energy has been considerably over-estimated, 
as expected from the large $J$ values 
computed at these bond-lengths.
%
After the Coulson-Fischer point, however, 
the PBE+$J^{(2)}$ total-energy quickly 
converges on the FCI total-energy.
%
The inset figure further illustrates 
that this remarkable agreement is within $0.05$~eV.

\begin{figure}[th!]
\centering
\includegraphics[height=0.494\textwidth]{images/H2_dft+jout}
\caption[Binding energy of H$_2$ with PBE+$J^{(2)}$]
{The H$_2$ binding  curve 
calculated with FCI (red),
PBE (orange), 
and PBE+$J^{(2)}$ (blue).
%
The reference energy for each curve 
is that of two isolated H atoms calculated with 
the corresponding XC functional.
% 
In the dissociated limit (inset),  
the $J^{(2)}$ result tends asymptotically to the exact one, 
which suggests that $J_\textrm{out}$ alone is sufficient to correct the SCE.}
\label{fig:H2_dft+jout}
\end{figure}
%
This result seems to suggest that 
$J_\textrm{out}$ alone is sufficient to correct the SCE, 
that is 
%
\begin{equation}
J_\textrm{out}=0 
\quad\Rightarrow\quad
J_\textrm{in}=2F_J^{\hat{P}}
=\frac{1}{2}\left(F^{\uparrow\uparrow}-F^{\uparrow\downarrow}
+F^{\downarrow\downarrow}-F^{\downarrow\uparrow}\right), 
\end{equation}
%
and that our earlier concern for the presence 
of a factor of two in this equation 
is therefore unsubstantiated.

Indeed, 
this result seems to indicate that there is no need to consider $U_\textrm{out}$, 
and that the $J_\textrm{eff}=0$ condition leads to the wrong result.
%
It is, as of yet, unclear 
why the self-consistency conditions 
we require to cancel the spurious curvatures 
differ between $U$ and $J$, 
but it presents a possible avenue for future work.
%
Nonetheless, 
we have provided an encouraging, preliminary demonstration 
that DFT+$J$, 
when provided with suitably calculated $J$, 
particularly at self-consistency, 
may be a remedy for SCE, 
albeit only for large enough bond-lengths 
and the appropriate population analysis. 
%
Acquiring the same correction for short bond-lengths 
is a feature intended for future development.



%DO WE NEED UEFF AND JEFF?
\subsection{Will $U_\textrm{eff}$ and $J_\textrm{eff}$ improve H$_2$?}
{
As a final proposition, 
we consider if the results presented above 
may be improved by instead considering 
$U_\textrm{eff}$ and $J_\textrm{eff}$, 
especially in the near-bonding regime 
where the $J$ values are too large. 
%
Given this observation,
%$J_\textrm{in}$ and $J_\textrm{out}$ profiles 
%in Fig.~\ref{fig:H2_jin_vs_jout},
one may argue that we are encountering 
a similar problem to that discussed around 
Eq~\ref{eq:u_self_consistent_interaction1}, 
where the absence of considering $J_\textrm{out}$ 
resulted in a $U_\textrm{out}$ that over-corrected the interaction.
%
Indeed, in this instance, 
our self-consistency condition $J_\textrm{out}=0$
alludes to a similar scenario 
%
\begin{equation}
J_\textrm{out}=0 
\quad\Rightarrow\quad 
J_\textrm{in}=2F_J^{\hat{P}}
=\frac{1}{2}\left(F^{\uparrow\uparrow}-F^{\uparrow\downarrow}
+F^{\downarrow\downarrow}-F^{\downarrow\uparrow}\right),
\end{equation}
%
where the factor of two here raises concern.}

We might then suggest 
adding to this term the computed $U_\textrm{out}$  
from the same calculations, 
in a similar manner to how we calculated 
$U_\textrm{eff}$ for NiO (c.f. Fig.~\ref{fig:NiO_uin_vs_uout}).
%
For a closed-shell system, however, 
computing $U$ from magnetic perturbations is very challenging, 
if not impossible 
since the resulting potentials in the 
1$s$ subspace will conserve the net charge, 
such that $U$ cannot be reliably 
computed\footnote{In fact, we tried this approach 
with the $J$ calculations but we found enormous error bars 
on the $U$ values}.

{
We instead calculated the $U$ values for dissociating H$_2$ 
from the usual uniform perturbation $\alpha=\pm0.05$~eV  
applied to the subspace of one of the H atoms 
(as we previously showed in Fig.~\ref{fig:H2_dft+u_energy2}).
%
We determined the $U$ value self-consistently 
at 2.8~a$_0$ in the near bond-length regime, 
to illustrate that a non-self-consistent $U$ 
provides an adequate approximation 
to the fully self-consistent  $U^{(2)}$.
%
Indeed, 
the $U_\textrm{in}$ vs $U_\textrm{out}$ 
curve depicted in Fig.~\ref{fig:H2_uin_vs_uout}, 
gives $U^{(2)}=4.68\pm0.05$~eV 
and $U^{(3)}=4.38\pm0.02$~eV. 
%
\begin{figure}[th!]
\centering
\includegraphics[height=0.494\textwidth]{images/H2_uin_vs_uout.pdf}
\caption[The self-consistent $U$ profile for H$_2$ at 2.8~a$_0$]
{The self-consistency calculation 
for the $U$ parameter at 2.8~a$_0$ along 
the binding curve of H$_2$.
%
From this we compute 
$U^{(1)}=2.26\pm0.02$~eV,
$U^{(2)}=4.68\pm0.05$~eV, 
and $U^{(3)}=4.38\pm0.02$~eV, 
which are very accurately determined 
and show that 
$U^{(3)}\approx U^{(2)}$.}
\label{fig:H2_uin_vs_uout}
\end{figure}
%
We therefore refrain from calculating 
the $U$ self-consistently at all bond-lengths, for convenience, 
and proceed with the non-self-consistent $U$ values.}

{
The effective Hubbard parameters 
resulting from this procedure, 
approximated  by   
$U_\textrm{eff}=U^{(2)}-J^{(2)}\approx U^{(3)}-J^{(2)}$ 
and $J_\textrm{eff}=U^{(2)}+J^{(2)}\approx U^{(3)}+J^{(2)}$ 
are shown in Fig.~\ref{fig:H2_ueff_jeff}.
%
\begin{figure}[th!]
\centering
\includegraphics[height=0.494\textwidth]{images/H2_jeff.pdf}
\caption[Approximated  $U_\textrm{eff}$ and $J_\textrm{eff}$ for dissociating H$_2$]
{The approximated $J_\textrm{eff}$ (dotted) 
and $U_\textrm{eff}$ (dot-dashed)  
compared to the required  
$J_\textrm{int}$ (solid) and $U_\textrm{int}$ (dashed), 
separately estimated to correct the total-energy in dissociating H$_2$.
%
These effective quantities do not compare 
to those required and so provide no additional advantage 
over $J^{(2)}.$}
\label{fig:H2_ueff_jeff}
\end{figure}
%
Here we observe 
that these effective parameters 
do not compare at all to the 
$U$ or $J$ values required to correct the total-energy 
(with either DFT+$U$ or DFT+$J$), 
and therefore suggest no reason to 
use them in place of $J^{(2)}$.
%
It is therefore not necessary, or advised, it seems, 
to consider $U_\textrm{eff}$ and $J_\textrm{eff}$ 
when treating SCE in H$_2$, 
wherein $J_\textrm{out}$ provides the best possible correction.
%
The precise reasons for this outcome are 
not yet clear and a more 
thorough investigation of this result 
remains an open prospect for the future.}


%CHOICE OF SUBSPACE PROJECTORS
\section{Choice of subspace projectors}
\label{sec:subspace_projectors}
\edit{The choice of subspace projectors $\{\varphi_m\}$ 
plays a crucial role in the both 
the calculation and application of 
the $U$ and $J$ parameters.
%
Throughout this dissertation  
we utilise static pseudoatomic orbitals (PAO) 
that are generated from the occupied KS state 
of the neutral atom with the underlying XC functional.
%
Other projectors that may be chosen include 
hydrogenic wave functions~\cite{PhysRevB.71.035105,:/content/aip/journal/jcp/133/11/10.1063/1.3489110},
maximally-localised Wannier functions~\cite{PhysRevB.77.085122,PhysRevB.74.235113},
and linear-muffin tin orbitals~\cite{PhysRevB.30.4734,PhysRevB.57.1505,PhysRevB.44.943,PhysRevB.48.16929} (LMTO), 
which may also be optimised self-consistently~\cite{PhysRevB.82.081102}.}


\edit{The DFT+$U$ method operates under the assumption 
that the localised charged encapsulated by the projectors 
is only weakly interacting with the charge in the surrounding electron bath.
%
Furthermore, 
we require that projectors are sufficiently 
well separated such that the 
overlap matrix between adjacent sites 
$O_{mm'}^{IJ}=\bra{\varphi^I_m}\hat{\rho}\ket{\varphi^J_{m'}}$ 
is minimal.
%
However, 
when the on-site projectors do not sufficiently capture 
all of the localised charge,  
or where the overlap between adjacent sites $O_{mm'}^{IJ}$ is large, 
the inappropriate population analysis 
can result in incorrect calculations of $U$ and $J$.}

\edit{
These effects are clearly evident    
in the calculation of $U$ in 
Fig.~\ref{fig:h2+_uout_vs_r} 
and $J$ in Fig.~\ref{fig:H2_jout_schemes}.
%
For short bond-lengths 
there is significant spatial overlap between 
the 1$s$ projectors on both H atoms 
where the charge contained in $O_{mm'}^{IJ}$ 
represents a large fraction of $\textrm{Tr}[\hat{n}_{mm'}^I]$.
%
Therefore, 
any perturbation applied to the targeted subspace, 
e.g., uniform or magnetic, 
will also inadvertently affect the charge contained 
in the other subspace, i.e., the bath, 
and the interaction cannot be considered weak.
%
Thus, 
calculating $U$ (or $J$) in this regime  
also takes into account the interactions 
present in the adjacent subspace 
and exaggerates the corresponding response functions.}

\edit{
Indeed, 
we see that the negative $U$ values 
calculated in Fig.~\ref{fig:h2+_uout_vs_r} 
at short bond lengths are lower than what is required 
by as much as 4~eV.
%
On the other hand, 
those calculated between 
3~a$_0$ and 6~a$_0$ 
are too large, 
but the error is relatively smaller   
since the overlap is less extreme.
%
In Fig.~\ref{fig:H2_jout_schemes} 
the $J^{(2)}$ values calculated before 
the Coulson-Fischer point are 
also much too large.
%
In the dissociation limit in both examples, however, 
where the projector overlap is minimal, 
the calculated $U$ and $J$ values are 
in very reasonable agreement 
with what is required.}
%

\edit{
It is therefore clear 
that the degree of overlap 
between subspace projectors $O_{mm'}^{IJ}$ 
strongly influences the calculation of $U$ and $J$ 
and must be considered 
in systems where such overlapping is prevalent.
%
However, 
rather than reconstructing the formulas 
for the Hubbard parameters 
to explicitly account for $O_{mm'}^{IJ}$, 
a more effective solution may be achieved 
by redefining the projectors themselves.
%
We therefore propose to construct the projectors 
to span all the subspaces that belong to a 
specific group $\mathcal{G}$, 
characterised by a certain 
species, spin-orientation, location etc.
%
In this scheme, 
which we have dubbed the 
{\it unified sites} approach (US), 
the group projector is defined as follows    
%
\begin{equation}
\hat{P}_{mm'}^\mathcal{G}=\sum_{I,J}\ket{\varphi^I_m}\bra{\varphi^J_{m'}}
\quad\mbox{with}\quad
I,J\in \mathcal{G},
\end{equation}
%
and is a sparse matrix that 
spans the subspace combining all the atoms.
%
Thus, uniform and magnetic perturbations, 
as well as the on-site interactions that determine $U$ and $J$, 
are contained within the unified subspace 
and circumvent the introduction of 
explicit double-counting corrections.
%
Hence, 
the overlap between adjacent sites may be implicitly   
taken into account both in the calculation and application 
of the Hubbard parameters.
%
We will discuss this feature further 
in our conclusions in Chapter~\ref{ch:conclusions}.}


%CONCLUSION
\section{Conclusion}

In this Chapter, 
we have extended our linear-response method developed in 
Chapter~\ref{ch:calculating_hubbard_u} 
to incorporate the self-consistent calculation of $U$ and $J$,  
in which only quantities calculated 
from ground-state densities are required.
% 
This formalism simplifies the analysis of 
parameter self-consistency schemes considerably and, 
at least for this specific method, 
a clear best choice of self-consistency criterion 
for one electron emerged, $U^{(2)}$, 
which has been explored relatively 
little in the literature to date. 
%
In stringent calculations of the dissociated limit of the 
system affected by SIE alone H$_2^+$, 
in which DFT+$U$ operates under ideal conditions, 
we are able to directly observe that the method
corrects the SIE in the total-energy 
very precisely, as foreseen in Ref.~\cite{PhysRevLett.97.103001}.
%
It does so entirely from first-principles 
when the $U^{(2)}$ scheme is used.

Moreover, 
this method readily extends to the exchange parameter $J$, 
which is required in the multi-electronic extension to NiO  
to facilitate the correct self-consistency condition 
$U_\textrm{eff}=0$.
%
For this condition, the computed values for the  
band gap, transition energy and magnetic moment 
were in excellent agreement with experimental data.

Our analysis also shows that the comparison of 
thermodynamically relevant DFT+$U$
quantities such as the total-energy 
between dissimilar systems demanding different  first-principles
$U$ parameters is, at least, well defined.
%
Indeed, this comparison evidently becomes one
between purely ground-state properties 
when applied with parameter self-consistency, 
but there may well be other circumstances 
in which this also holds true.

Furthermore, 
the self-consistency formalism can 
be used to enhance the results of other methods 
as we shall see in the next Chapter.
%
The procedure then extends the DFT+$U^{(2)}$ 
full SIE correction of the H$_2^+$
total-energy to the highest occupied eigenvalue, 
approximately enforcing Koopmans' condition.
%
We also recommend the use of a more complicated 
criterion, the previously proposed $U^{(3)}$, 
particularly for density-non-self-consistent methods such as 
post-processing DFT+DMFT.


The self-consistent scheme was also extended 
to calculate the exchange parameter $J$ 
in dissociating H$_2$.
%
For this system we found it not only advantageous, 
but necessary, to calculate $J$ self-consistently 
in order to correctly ameliorate the SCE  
beyond the Coulson-Fischer point 
and into the dissociation limit. 
%
In this instance we found that 
the PBE+$J^{(2)}$ total-energy 
gave remarkable agreement with 
the FCI total-energy in the dissociation limit, 
despite drastically over-estimating it 
near equilibrium.
%
{
Surprisingly, 
$U_\textrm{eff}$ and $J_\textrm{eff}$
did not coincide with the necessary values, 
which warrants further analysis into the 
disparity between the self-consistency 
regimes for $U$ and $J$.}

To properly account for SIE or SCE in the total-energy, 
both across the bond-length range of H$_2^+$ and in general, 
we note that one would need to fully take into account 
the effects of subspace charge spillage, overlap and double-counting, 
possibly through the use of 
Wannier functions~\cite{PhysRevB.77.085122} 
generated self-consistently with the DFT+$U$
electronic structure~\cite{PhysRevB.82.081102}.
%
At least as important for correcting SIE in the 
bonding regime, perhaps, is the necessity 
to overcome the breakdown of the 
single-site approximation.
%
For this, the account of inter-subspace SIE 
offered by the multi-site method such as 
DFT+$U$+$V$~\cite{0953-8984-22-5-055602}, 
or our proposed US approach, 
is a promising avenue for  investigation.

Finally, 
in Appendix~\ref{ch:self_consistency_appendix}, 
we present self-consistent calculations of $U$ for 
Ni(CO)$_4$ and Cr$_2$O$_3$, 
which were also investigated as part of this work. 




