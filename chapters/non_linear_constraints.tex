
\lett{W}{hile the construction} of viable, 
\edit{explicit density functionals 
free of pathologies such as SIE 
is extremely challenging~\cite{doi:10.1021/cr200107z}, 
we discussed in Chapter~\ref{ch:self_interaction_error} 
{some of the} significant progress made in 
the development of implicit functionals 
toward this {goal}. 
%
%Examples included methods
%that correct SIE on a one-electron basis
%according to variationally optimised definitions, 
%such as generalised PZ SIC  
%approaches~\cite{PhysRevB.23.5048,PhysRevLett.65.1148,
%:/content/aip/journal/jcp/140/12/10.1063/1.4869581,PhysRevB.93.165120}, 
%%in which much progress has recently been made
%%by generalising to complex-valued
%%orbitals~\cite{PhysRevA.85.062514,
%%PhysRevLett.108.146401,
%%:/content/aip/journal/jcp/137/12/10.1063/1.4752229}, 
%and those addressing Koopmans compliance~\cite{PhysRevB.82.115121,
%PhysRevB.90.075135}.
%
Furthermore, 
we demonstrated in 
Chapters~\ref{ch:calculating_hubbard_u}~\&~\ref{ch:self_consistent_hubbard} 
how the DFT+$U$ method~\cite{QUA:QUA24521} 
may be utilised as an effective and computationally efficient, 
even linear-scaling~\cite{PhysRevB.85.085107}, 
treatment for many-body SIE.
%
However, 
as we illustrated in Fig.~\ref{fig:h2+_dft+u_eigenvalue}, 
it is patently unable to provide 
any sort of correction to the Kohn-Sham eigenvalue, 
which is the relevant quantity for, e.g.,
calculating first ionisation energies, 
heterogeneous catalysis~\cite{Bliem1215}, 
and 
charge transfer energies~\cite{PhysRevB.88.165112,
PhysRevB.93.165102,
doi:10.1021/ct0503163,
doi:10.1021/jp061848y,
doi:10.1021/jp204962k,
doi:10.1021/jp103153a,
:/content/aip/journal/jcp/125/16/10.1063/1.2360263,
doi:10.1021/jp912049p,
doi:10.1021/jp106989t,
doi:10.1063/1.3190169,
PhysRevB.77.115421,
Kubas14jcp,
Kubas15pccp,
doi:10.1021/acs.chemrev.7b00086}.
%
This non-compliance of Koopman's condition~\cite{KOOPMANS1934104,PhysRev.123.420,PhysRevB.82.115121,PhysRevB.90.075135} 
corresponds to a short-coming in the 
linear term affecting the potential, 
which in turn 
exposes a significant limitation in the 
applicability of the DFT+$U$.}

\edit{As a solution to this problem, 
we propose the construction of 
a generalised DFT+$U$ functional 
comprising separate $U$ parameters 
for the linear and quadratic density terms, 
thereby allowing for an adaptive double-counting correction.
%
Our goal is that DFT+$U$ may be simultaneously applied 
to two important sources of error, 
namely the many-body SIE and 
non-compliance with Koopman's condition, 
which have long hindered the reliability 
of electronic structure calculations.
%
Moreover, 
the additional degree of freedom provided by two parameters 
raises the possibility of developing 
a self-contained, self-consistency scheme for calculating both parameters.
%
Hence, the challenge central to this Chapter 
is the development of such a scheme 
that is comparable in efficiency and accuracy 
to the self-consistent linear-response 
approach described in the previous Chapters.}


%However, a considerable degree of care 
%is needed to calculate the Hubbard $U$ parameters, 
%the methods of which are varied 
%and can sometimes pose numerical 
%challenges~\cite{:/content/aip/journal/jcp/133/11/10.1063/1.3489110,QUA:QUA24521}.
\edit{
We begin 
by first exploring the self-contained calculation of 
conventional Hubbard $U$ parameters 
by means of an automated variational extremisation approach.
%
Indeed, 
the automated correction of subspace SIE 
is a compelling possibility in itself, 
and one that warrants a rigorous investigation 
as such a mechanism  
would be extremely useful for many practitioners    
and expedient in the context of high-throughput materials informatics~\cite{jain2011high,setyawan2010high,curtarolo,Curtarolo2012218,PhysRevX.5.011006}.
%
This procedure may be in fact possible 
if we can effectively define the state 
that is free of SIE in terms of 
relevant constraints on the ground-state density.
%
We therefore propose to exploit 
the efficiency of the optimisation methods encapsulated by 
constrained density-functional theory 
(cDFT)~\cite{PhysRevLett.53.2512,Sit2007107,doi10.1021/cr200148b}, 
to directly and automatically enforce SIE correction 
with the constraint-like form of the DFT+$U$ functional, 
for which the $U^I$ resemble the Lagrange multipliers
of constraints on the eigenvalues of $\hat{n}^{I \sigma}$}.


\edit{
We shall demonstrate, however,  
that cDFT is fundamentally incompatible with 
constraints beyond linear order, 
{and that this prohibits the satisfaction 
of exact constraints targeting SIE interactions}.
%
Fundamentally,  
we show that one cannot excite a system afflicted with SIE 
into a state that reliably exhibits less, 
without breaking a physical symmetry 
or introducing auxiliary parameters.}

\edit{
However, 
we show that a two-parameter method is 
sufficient to precisely cancel the SIE 
and attain the exact subspace occupancy 
or exact eigenvalue 
when the free energy is considered instead.
%
This finding illustrates that a generalised DFT+$U$ 
functional is effective in treating 
both the total-energy and ionisation potential 
of a one-electron system, 
entirely from first principles.
%
Moreover, 
we derive formulae for estimating 
these parameters which depend only 
on ground-state DFT quantities 
and the self-consistent Hubbard $U$ 
calculated from linear-response in the previous Chapters.}
%

%%
%cDFT formalises and automates the use of 
%self-consistent~\cite{PhysRevB.94.035159}
%penalty functionals in DFT 
%to access excited states, 
%and imposes them as exact constraints 
%by locating the lowest-energy excited state 
%compatible with the underlying functional

%
%Constrained density-functional theory 
%(cDFT)~\cite{PhysRevLett.53.2512,Sit2007107,doi10.1021/cr200148b} 
%formalises and automates the use of self-consistent~\cite{PhysRevB.94.035159}
%penalty functionals in DFT,
%and imposes them as exact constraints 
%by locating the lowest-energy excited state 
%compatible with the underlying functional.
%% 
%The KS potential is supplemented 
%by an external, constraint potential, 
%parameterised by a set of 
%{\it Lagrange multipliers}, 
%which are then optimised {\it in situ}.
%
%{
%cDFT is often used to access excited states, 
%which are typically beyond the scope 
%of Kohn-Sham DFT calculations, 
%and is sometimes effective 
%for treating SIE indirectly~\cite{PhysRevLett.97.028303,doi10.1021/cr200148b}, 
%in which these states are parameterised 
%by expectation values of the density, 
%or density-dependent quantities, 
%within a localised region.}
%
%It is effective for treating SIE indirectly 
%in systems comprising well-separated fragments, 
%where it may be used to break physical symmetries 
%and explore the integer-occupancy states 
%at which SIE is typically 
%reduced~\cite{PhysRevLett.97.028303,doi10.1021/cr200148b}.
%

%{
%The demonstrable utility of cDFT 
%for ameliorating SIE in its own right, 
%combined with the constraint-like functional form of DFT+$U$ - 
%where the $U^I$ resemble the Lagrange multipliers
%of constraints on the eigenvalues of $\hat{n}^{I \sigma} $ - 
%therefore suggests the potential use of the 
%optimisational methods encapsulated by cDFT 
%to directly and automatically enforce 
%the objective of DFT+$U$.}
%
%This automated correction of 
%systematic errors in approximate functionals, 
%such as subspace SIE, 
%is a compelling possibility, 
%and one that warrants a rigorous investigation.

%Regardless, we propose circumventing this problem 
%by expanding the constraint into terms comprising 
%powers of the target variable,
%which may be interpreted either as distinct constraints 
%each with its own Hubbard $U$ type Lagrange multiplier, 
%or as the components of a generalised DFT+$U$ functional.
%%
%The latter approach prevails 
%in our tests on a model one-electron system, H$_2^+$,
%in that it readily recovers the exact total-energy
%while constraints preserving the symmetric 
%treatment of the density fail to do so.
%%
%This motivates the construction of 
%a generalised DFT+$U$ functional 
%that circumvents the limitation of the 
%linear term in original construction to 
%reconcile Koopmans' condition 
%and the total-energy concurrently.
%
%For this, 
%we outline a practical, approximate scheme 
%by which the required pair of Hubbard parameters, 
%denoted as $U_1$ and $U_2$, 
%may be calculated from first-principles DFT quantities.


%BACKGROUNG TO CDFT
\section{Constrained-DFT}
\label{sec:cdft}

Constrained density-functional theory (cDFT)~\cite{doi10.1021/cr200148b}
is a generalisation of DFT, 
in which the total-energy functional is augmented by 
one or more {constraints}.
%
{
These constraints operate on 
the expectation values of 
charge or spin-density  
(or their sums, differences or moments), 
which are enforced by a set of 
{\it Lagrange multipliers} $\{V_\textrm{c}\}$.}

The total-energy of a constrained system $W$, 
in which the total charge density in some region
(in our case the localised subspace) is constrained, 
is given as follows
%
\begin{align}
&W[n,V_\textrm{c}]=E_\textrm{DFT}[n]
+V_\textrm{c}E_\textrm{c}[n] \nonumber \\[0.5em]
\quad\mbox{where}\quad 
&E_\textrm{c}[n]=\left(\sum_\sigma \int d\br\ w_\textrm{c}^\sigma(\br)n(\br) - N^\sigma_\textrm{c}\right).
\end{align}
%
The {\it weight function}  $w_\textrm{c}^\sigma(\br)$
specifies the region in which the constraint 
enforces the target charge $N^\sigma_\textrm{c}$.

Although constraints have long played 
a significant role in the interpretation of physical processes, 
the constraint-approach originally gained traction 
in the DFT community with the work of 
Dederichs~\cite{PhysRevLett.53.2512} 
and other contributors~\cite{PhysRevB.44.13319,
PhysRevLett.56.2407,
doi:10.1021/j100132a040,
JCC:JCC12,
PhysRevLett.94.036104,
PhysRevB.75.115409,
Behler:947885}, 
whose combined efforts facilitated 
the calculation of states not directly accessed 
by the unconstrained exchange-correlation functional, 
such as charge transfer energies, 
defect states, 
and magnetic impurities.
%
However these calculations were computationally 
intensive to perform as they required 
the manual scanning over the space of Lagrange multipliers 
until the desired state was achieved.

The seminal work of 
Van Voorhis {\it et al.}~\cite{PhysRevA.72.024502,
:/content/aip/journal/jcp/125/16/10.1063/1.2360263,
:/content/aip/journal/jcp/140/18/10.1063/1.4862497,
doi:10.1021/ct0503163,
doi:10.1021/jp061848y,
:/content/aip/journal/jcp/127/16/10.1063/1.2800022,
PhysRevB.94.035159}, 
on which the contemporary application of cDFT is now based, 
later formalised an {optimisation} approach 
toward the solution of constraints, 
in which they established 
that the constraint is satisfied at an extremum of the total-energy 
with respect to the Lagrange multipliers $V_\textrm{c}$
%
\begin{equation}
\frac{dW}{dV_\textrm{c}}=0 
\quad\Rightarrow\quad 
E_\textrm{c}[n]=0
\label{eq:dwdv}
\end{equation}
%
and that the extremum corresponds to a maximum, 
guaranteed by a negative-definite curvature, 
%
\begin{equation}
\frac{d^2W}{dV_\textrm{c}^2}\leq 0 
\quad\mbox{for all}\quad
V_\textrm{c},\ N_\textrm{c}.
\label{eq:d2wdv2}
\end{equation}
%
{
However, 
it was recently remarked by 
O'Regan and Teobaldi~\cite{PhysRevB.94.035159}, 
that the assumption of {stationary} orbitals, 
which contributes to the derivation of Eq.~\eqref{eq:d2wdv2} 
is incomplete when 
extended to variational procedures\footnote{See Eq.~6 in Ref.~\cite{PhysRevA.72.024502}
vs Eq.~14 in Ref.~\cite{PhysRevB.94.035159}.}.
%
%and the condition which must be strictly enforced 
%is that of {\it orthornormalised} KS orbitals, 
%i.e, $\braket{\psi^*_{i\sigma}}{\psi_{j\sigma}}=\delta_{ij}$.
%
Under this generalisation, 
the constraint is still satisfied at the energy maximum as in Eq.~\eqref{eq:dwdv}, 
but with the introduction of the inverse dielectric response $\epsilon^{-1}$ 
in the expression for Eq.~\eqref{eq:d2wdv2}.
% 
This means that the curvature is no longer 
{\it guaranteed} to be negative-definite for multiple constraints 
(thereby precluding the existence of a unique maximum) 
but is proven to be so for single constraints, 
or those that are not ill-posed.}


{
cDFT has  proven extremely useful in many applications 
over the years}, including 
charge transfer calculations~\cite{PhysRevB.88.165112,
PhysRevB.93.165102,
doi:10.1021/ct0503163,
doi:10.1021/jp061848y,
doi:10.1021/jp204962k,
doi:10.1021/jp103153a,
:/content/aip/journal/jcp/125/16/10.1063/1.2360263,
doi:10.1021/jp912049p,
doi:10.1021/jp106989t,
doi:10.1063/1.3190169,
PhysRevB.77.115421,
Kubas14jcp,
Kubas15pccp,
doi:10.1021/acs.chemrev.7b00086}, 
electronic coupling parameters~\cite{doi:10.1063/1.3507878,Kubas14jcp,Kubas15pccp}, 
and in calculating of Hubbard~$U$ 
parameters in DFT+$U$~\cite{PhysRevB.44.943,PhysRevB.74.235113,PhysRevB.67.153106,PhysRevB.39.9028}.
%
cDFT is also practiced in the correction of static correlation error 
to construction efficient, minimal basis sets for configuration interaction 
calculations~\cite{:/content/aip/journal/jcp/127/16/10.1063/1.2800022,
:/content/aip/journal/jcp/140/18/10.1063/1.4862497,
:/content/aip/journal/jcp/130/3/10.1063/1.3059784,
:/content/aip/journal/jcp/133/6/10.1063/1.3470106}.
% 
For a comprehensive account of the cDFT method, 
we refer the reader to 
Refs.~\cite{PhysRevLett.53.2512,
PhysRevA.72.024502,
PhysRevB.94.035159,
doi10.1021/cr200148b,
doi:10.1021/acs.chemrev.7b00086}.

The broad application of cDFT to date 
has focused primarily on constraints 
linear in density-dependent quantities 
while there have been sparse references
to non-linear constraints in the prevailing
literature~\cite{PhysRevLett.79.1337,PhysRevB.59.12173,GIRAUD197023}, 
particularly in the realm of nuclear physics~\cite{FLOCARD1973433,Staszczak2010,PhysRevC.78.024313}.
%
To our knowledge, however, 
a formal solution to non-linear constraints 
on the electronic density was not proposed in any,  
hence motivating our interest in their application.

%FALURE OF NON-LINEAR CONSTRAINTS
\section{Non-linear constraints}
\label{sec:non_linear_constraints}

To investigate the efficiency  
of non-linear constraints for targeting SIE, 
let us reconsider the one-electron system H$_2^+$, 
which we showed in the previous Chapters 
to be highly amenable to DFT+$U$ type corrections.
%
The simplest conceivable constraint functional
that targets SIE assumes a quadratic form
%
\begin{equation}
C_2 =\sum_{I\sigma} ( N^{I\sigma} - N^{I\sigma}_\textrm{c} )^2 
\quad\mbox{with}\quad
N^{I\sigma}  = \textrm{Tr} [ \hat{n}^{I\sigma} ], 
\end{equation}
%
where $N^{I\sigma}$ is the total occupancy of 
the error-prone subspace $I$, 
and $N^{I\sigma}_\textrm{c}$ is its targeted value, 
for spin $\sigma$.
%
Although this constraint is a 
functional of subspace total occupancies, 
rather than the occupancy eigenvalues as in DFT+$U$, 
it is an unimportant distinction for single-orbital sites.
%
For $M$ identical subspaces with  
$N^{I\sigma}_\textrm{c} = N_\textrm{c}$ for all $I$ and $\sigma$, 
the total-energy of the system is given by
%
{
\begin{equation}
W[n,V_\textrm{c}] = E_\textrm{DFT}[n] + M V_\textrm{c}\,C_2[n]
\end{equation}}
where $V_\textrm{c}$ is the common 
cDFT Lagrange multiplier.
%
This gives rise to a constraining potential of the form
%
\begin{equation}
\hat{v}^{I\sigma}_\textrm{c} = 
2 V_\textrm{c} \sum_I ( N^{I\sigma} - N_\textrm{c} ) \hat{P}^{I } 
\end{equation}
%
which depends explicitly 
on the violation of the constraint, i.e., 
it is attractive for $N^{I\sigma}<N_\textrm{c}$ 
and repulsive for $N^{I\sigma}>N_\textrm{c}$.
%
This, in turn, 
generates a corrective 
interaction-kernel given by
%
\begin{equation}
\hat{f}_\textrm{c}=\frac{\delta \hat{v}^{I\sigma}_\textrm{c}}{\delta\hat{n}}= 2 V_\textrm{c}\, \sum_I \hat{P}^{I }  \hat{P}^{I} 
\label{eq:cdft_kernel}
\end{equation}
%
which acts to modify the energy-density profile.
%
In fact, the interaction-kernel $\hat{f}_\textrm{c}$
is identical to that generated by DFT+$U$
when $V_\textrm{c} = - U^I / 2$, 
and thus motivates 
our initial choice of a quadratic constraint.
%
Moreover, 
following the procedure prescribed in 
Ref.~\cite{PhysRevB.94.035159} 
for the self-consistent cDFT problem,
the Hellmann-Feynman theorem 
provides that the first energy derivative 
with respect to the Lagrange multiplier 
is simply the constraint functional 
%
\begin{equation}
\frac{d W}{d V_{\textrm{c}}} = C_2 = (N^{I\sigma}-N_\textrm{c})^2,
\label{eq:energy_deriv1}
\end{equation}
%
so that the total-energy 
$W \left( V_\textrm{c}  \right)$ 
always finds a stationary point 
when the constraint is satisfied, i.e., $C_2 = 0$.
%

{
In order to assess the potency 
of this functional to correct SIE 
in a one-electron model, 
we investigate the H$_2^+$ molecule}
at an intermediate bond-length of 4~a$_0$, 
using the same calculation parameters 
outlined in Chapter~\ref{ch:self_interaction_error}.
%
{
At this bond-length, 
the overlap of the two PBE 1$s$ orbitals   
yields an intermediate double-counting 
in the subspace total occupancy of 24\%}. 
%
A physically intuitive target occupancy of
$N_\textrm{c} = 0.5$~e per atom is 
chosen\footnote{{Henceforth in this analysis, 
we shall suppress the site and spin indices for brevity.}}, 
which requires a repulsive constraint 
and, therefore, a positive $V_\textrm{c}$.
%
However, the same qualitative outcome 
can be composed from the selection of any 
$N_\textrm{c} \ne N_\textrm{DFT}$, 
where $N_\textrm{DFT}$ 
is the unperturbed ground-state subspace density.

The total-energy of the system 
as a function of the Lagrange multiplier $V_\textrm{c}$ 
is illustrated in Fig.~\ref{fig:h2+_lagrange_multiplier}, 
in which it is clearly demonstrated 
that the $C_2$ constraint cannot be enforced, 
as was similarly remarked by Staszczak in Ref.~\cite{Staszczak2010}, 
since $W \left( V_\textrm{c} \right)$ 
fails to attain a stationary point.
%
This is further emphasised by the exponential decay 
of the total-energy derivatives, 
the first three of which are also depicted.

\begin{figure}[th!]
\centering
\includegraphics[height=0.494\textwidth]{images/h2+_lagrange_multiplier}
\caption[Total-energy of constrained H$_2^+$ vs Lagrange multiplier]
{The total-energy of constrained H$_2^+$ at 4~a$_0$ bond-length, 
with a target occupancy of $N_\textrm{c}=0.5$~e per atom 
against the cDFT Lagrange multiplier $V_\textrm{c}$. 
%
Also shown is the constraint functional 
$C_2=(N-N_\textrm{c})^2$,
averaged over the two atoms, and its
first and second derivatives, 
which all rapidly diminish with $V_\textrm{c}$.}
\label{fig:h2+_lagrange_multiplier}
\end{figure}



Indeed, the key to understanding 
the failure of the  $C_2$ constraint
lies in the rapid {decline} in the 
self-consistent cDFT response 
functions~\cite{PhysRevB.94.035159} 
$d^m N/d V_\textrm{c}^m\sim d^{m+1} W/d V_\textrm{c}^{m+1}$, 
the first three of which are 
depicted in Fig.~\ref{fig:h2+_response_functions}.
%
This results in diminishing returns 
in regard to the efficacy of the constraining potential 
as $V_\textrm{c}$ is increased.
%
In other words, 
as the constraint asymptotically 
approaches satisfaction with
increasing $V_\textrm{c}$, 
the density becomes less and less 
responsive to to the perturbation 
incurred by $V_\textrm{c}$, 
and so never attains the target value.

\begin{figure}[th!]
\centering
\includegraphics[height=0.494\textwidth]{images/h2+_response_functions.pdf}
\caption[Interacting response functions for H$_2^+$ vs Lagrange multiplier]{
The magnitudes of the  
interacting density response 
$\chi=\mathrm{Tr} \left[ \frac{d N}{d \hat{v}_\textrm{ext}}  \hat{P} \right]$ 
%calculated by means of Eq.~\eqref{eq:first_order_resp}, 
and  cDFT response functions $d^m N / d V_\textrm{c}^m$, 
calculated from a polynomial fit to the
average subspace occupancy of H$_2^+$ at 4~a$_0$. 
%as in Fig.~\ref{fig:h2+_lagrange_multiplier}. 
%
The responses diminish as we asymptotically 
approach constraint satisfaction,
while the occupancy, 
and hence $\chi$, 
tends to a constant value.}
\label{fig:h2+_response_functions}
\end{figure}


\subsection{Proof of the inapplicability of non-linear constraints}

{
The incapability of the quadratic constraint 
to attain a stationary point for this simple system 
presents a surprising result.
%
We are therefore motivated to investigate whether 
non-linear constraints of a general form
%
\begin{equation}
C_n = ( N - N_\textrm{c} )^n
\quad\mbox{with}\quad
n\in \{\mathds{N}; n\geq2\}, 
\end{equation}
%
are also unable to be satisfied 
for a target occupancy 
$N_\textrm{c} \ne N_\textrm{DFT}$, 
or if this outcome is specific 
to quadratic constraints only.}
%
To preface this inquiry, 
it is provided that the $n=0$ case is trivial, 
and $n=1$ returns a conventional linear constraint.
%
Moreover, 
the constraint is ill-defined for $n<0$
since the total-energy would diverge 
upon constraint satisfaction, 
and $C_n$ becomes imaginary for non-integer $n$
when  $( N - N_\textrm{c})$ is negative.
%
We shall, therefore, 
limit our discussion to 
constraints of integer order $n \ge 2$.

{
To assess the ease of accessing the stationary points 
sought by non-linear constraints 
we must investigate the nature of 
the derivatives of the total-energy
$W\left( V_\textrm{c} \right)$.
%
We begin this analysis with the first derivative, 
which simply returns the constraint itself 
%
\begin{equation}
\frac{dW}{dV_\textrm{c}}=C_{n},
\end{equation}
%
as in Eq.~\eqref{eq:energy_deriv1}.
%
Hence, the first derivative is trivially equal to zero 
at the stationary point.}
%
We proceed then to the second derivative, 
which follows directly from the first and is 
%
\begin{align}
\frac{d^2W}{dV_\textrm{c}^2}= \frac{d C_n}{dV_\textrm{c}} ={}& 
 n\left(N-N_\textrm{c}\right)^{n-1}\frac{dN}{dV_\textrm{c}} 
 = nC_{n-1} \frac{dN}{dV_\textrm{c}}. 
\label{eq:energy_deriv2}
\end{align}
%
It is constructive from here on
to express the total-energy derivatives in terms of the 
subspace-projected interacting response function 
%
\begin{equation}
\chi = \mathrm{Tr} \left[  \frac{d N}{d \hat{v}_\textrm{ext}}  \hat{P} \right], 
\end{equation}
%
since this object 
- unlike the cDFT response functions - 
is independent of the form of the constraint, 
where the external potential is denoted by 
%
\begin{equation}
\hat{v}_\textrm{ext}
%= \frac{\delta E_\textrm{ext}}{\delta\hat{\rho}}
=\hat{v}_\textrm{c}
=V_\textrm{c}\frac{\delta C_n}{\delta\hat{\rho}}
=n  V_\textrm{c} \left( N - N_\textrm{c} \right)^{n-1} \hat{P}.
\end{equation}
%
An expression for the 
cDFT response function 
$dN / d V_\textrm{c}$
may then be derived 
by means of the chain rule via 
the external potential, as follows
%
\begin{align}
\frac{d N}{d V_\textrm{c}}
&=\mathrm{Tr} \left[ \frac{d N}{d \hat{v}_\textrm{ext} } 
\frac{d \hat{v}_\textrm{ext} }{d V_\textrm{c} } \right] =\mathrm{Tr} \left[ \frac{d N}{d \hat{v}_\textrm{ext} } 
\frac{d}{d V_\textrm{c} }\left(n  V_\textrm{c}\left( N - N_\textrm{c} \right)^{n-1} \hat{P}\right) \right] \nonumber  \\[1em]
&=\mathrm{Tr} \left[ \frac{d N}{d \hat{v}_\textrm{ext} }  \hat{P} \right]
\frac{d}{d V_\textrm{c} }\left(n  V_\textrm{c}C_{n-1}\right) 
%&=n \chi \frac{d}{d V_\textrm{c}} \left( V_\textrm{c}\,  C_{n-1}\right) \nonumber \\
=n \chi\left(C_{n-1}+V_\textrm{c}\,\frac{dC_{n-1}}{dV_\textrm{c}}\right) \nonumber  \\[1em]
&=n \chi\left(C_{n-1}+V_\textrm{c}\,(n-1)(N-N_\textrm{c})^{n-2}\frac{dN}{dV_\textrm{c}}\right) \nonumber   \\[1em]
&=n \chi\left(C_{n-1}+V_\textrm{c}\,(n-1)C_{n-2}\frac{dN}{dV_\textrm{c}}\right).
\label{eq:first_order_resp_derivation}
\end{align} 
%
Thus, simplifying the terms 
in the last step yields 
the first cDFT response function
%
\begin{equation}
\frac{d N}{d V_\textrm{c}}
=\frac{n \chi C_{n-1}}{ 1 - n \left( n-1 \right) \chi V_\textrm{c} C_{n-2}}.
\label{eq:first_order_resp}
\end{equation}
%
%which can be verified numerically.
%
Substituting Eq.~\eqref{eq:first_order_resp} 
into Eq.~\eqref{eq:energy_deriv2} 
then gives the second energy derivative 
in terms of the constraint and 
subspace interacting response
%
\begin{equation}
\frac{d^2W}{dV_\textrm{c}^2}
=\frac{\chi n^2 C_{n-1}^2}{ 1 - n \left( n-1 \right) \chi V_\textrm{c} C_{n-2}}.
\label{eq:energyderive2_full}
\end{equation}
%

For any valid stationary point,
where $N=N_\textrm{c}$,  
the constraint is satisfied 
by construction: $C_n = 0$.
%
Conversely, 
the Lagrange multiplier $V_\textrm{c}$,  
the interacting response $\chi$, 
and the energy derivatives beyond first order 
must all remain finite to distinguish  
the unique inflection point.
%
Inspecting Eqs.~\eqref{eq:first_order_resp}~\&~\eqref{eq:energyderive2_full} 
again, we see that the numerators 
for the response 
$d N / d V_\textrm{c}$ 
and energy curvature 
$d^2 W / d V_\textrm{c}^2$
vanish at the stationary point since 
%
\begin{equation}
C_{n-1}=(N-N_\textrm{c})^{n-1} = 0 
\quad\mbox{for}\quad
n\geq 2.
\end{equation}
%
The latter is therefore not 
a stationary point discriminant, 
and we must inspect higher-order derivatives 
instead.
%
Continuing from Eq.~\eqref{eq:energy_deriv2}, 
the third derivative of the total-energy is\footnote{In general, 
the energy derivative of order $m$  involves cDFT
response functions up to order $m-1$, 
and positive integer powers of $( N - N_\textrm{c})$, 
which vanish for all $n\neq m-1$, 
but do not diverge.} 
%
\begin{align}
\frac{d^3 W }{ d V_\textrm{c}^3 } ={}&
n  \left(n-1\right) C_{n - 2} 
\left( \frac{d N}{d V_\textrm{c} }  \right)^2 +
 n C_{ n - 1} \frac{d^2 N}{d V_\textrm{c}^2 }.
\end{align}
%
which depends on the both the first 
and second-order responses.
%
Since we have shown the former vanishes 
at the stationary point, 
it remains to be proven for the latter.

Beginning with the last line in   
Eq.~\eqref{eq:first_order_resp_derivation}, 
the second-order response may be derived thence.
%
We begin by evaluating the 
first term in the derivative of the expansion:
%
\begin{align}
T_1=\frac{d}{dV_\textrm{c}}\left[n\chi C_{n-1}\right]
&=n\frac{d\chi}{dV_\textrm{c}}C_{n-1}+n\chi\frac{C_{n-1}}{V_\textrm{c}}\nonumber \\[1em]
&=n\frac{d\chi}{dV_\textrm{c}}C_{n-1}+n(n-1)\chi C_{n-2}\frac{dN}{dV_\textrm{c}}.
\end{align}
%
The second term is then given by
%
\begin{align}
T_2&=n(n-1)\frac{d}{dV_\textrm{c}}\left[\chi V_\textrm{c}C_{n-2}\frac{dN}{dV_\textrm{c}}\right] \nonumber \\[1em]
%
&=n(n-1)\left[\frac{d\chi}{dV_\textrm{c}}V_\textrm{c}C_{n-2}\frac{dN}{dV_\textrm{c}}
+ \chi C_{n-2}\frac{dN}{dV_\textrm{c}}\right.+\left.\chi V_\textrm{c}\,\frac{dC_{n-2}}{dV_\textrm{c}}\frac{dN}{dV_\textrm{c}}+\chi V_\textrm{c}\,C_{n-2}\frac{d^2N}{dV_\textrm{c}^2}\right] \nonumber  \\[1em]
%
&=n(n-1)\frac{dN}{dV_\textrm{c}}\left[\frac{d\chi}{dV_\textrm{c}}V_\textrm{c}C_{n-2}
+ \chi C_{n-2}
+\chi V_\textrm{c}\,(n-2)C_{n-3}\frac{dN}{dV_\textrm{c}}\right] \nonumber \\
&\qquad\qquad\qquad\qquad +n(n-1)\chi V_\textrm{c}\, C_{n-2}\frac{d^2N}{dV_\textrm{c}^2} \nonumber  \\[0.75em]
&=T_2^\prime + n(n-1)\chi V_\textrm{c}\, C_{n-2}\frac{d^2N}{dV_\textrm{c}^2},
\end{align}
%
where in the last step we have grouped terms 
depending on $dN/dV_\textrm{c}$ 
into the term $T_2^\prime$.
%
Gathering the terms from 
$T_1$ and $T_2$, 
the second-order response function 
may be constructed as 
%
\begin{align}
%\frac{d^2N}{dV_\textrm{c}^2}&= T_1+T_2^\prime + n(n-1)\chi V_\textrm{c}\, C_{n-2}\frac{d^2N}{dV_\textrm{c}^2}\nonumber \\
%\nonumber \\
%\Rightarrow\quad
\frac{d^2N}{dV_\textrm{c}^2} &= \frac{T_1+T_2^\prime}{1-n(n-1)\chi V_\textrm{c}\, C_{n-2}\frac{d^2N}{dV_\textrm{c}^2}},
\label{second_order_resp}
\end{align}
%
where the numerator is given by
%
\begin{align}
\label{second_order_resp_numerator}
T_1+T_2^\prime &= 
n\frac{d\chi}{dV_\textrm{c}}C_{n-1}   \\[0.5em]
&+n(n-1)\frac{dN}{dV_\textrm{c}}\left[\frac{d\chi}{dV_\textrm{c}}V_\textrm{c}C_{n-2}\right.
+ \left. 2\chi C_{n-2} +\chi V_\textrm{c}\,(n-2)C_{n-3}\frac{dN}{dV_\textrm{c}}\right] \nonumber, 
\end{align} 
%
and depends explicitly on the terms 
$C_{n-1}$ and $dN/dV_\textrm{c}$, 
which, as we have seen, 
vanish at the stationary point $N=N_\textrm{c}$ 
for $n\geq 2$.
%
Furthermore, 
any potentially non-zero terms, 
such as $C_{n-2}$ or 
$C_{n-3}$\footnote{For $n=3$, $C_{n-3}=C_0=1$ and the result is still valid.}, 
are contained in the square brackets, 
which are pre-multiplied by 
the vanishing $dN/dV_\textrm{c}$.
%
Therefore, 
the second-order response 
$d^2 N / d V_\textrm{c}^2 $ 
and, by extension 
$d^3 W / d V_\textrm{c}^3 $, 
both vanish at stationary points for all $n \ge 2$, 
due to the vanishing $C_{n-1}$
in the first term in the numerator, 
and the vanishing 
$dN/dV_\textrm{c}$ 
coupled to all remaining terms.


In general, 
the cDFT response function 
$d^m N / d V_\textrm{c}^m$ 
comprises terms proportional to 
response functions of lower order, 
in addition to mixed terms composed of 
the interacting response function $\chi$
and its derivatives 
$d^{m-1} \chi / d V_\textrm{c}^{m-1}$, 
and constraint terms $C_{n}$, 
such that 
%
\begin{align}
\frac{d^mN}{dV_\textrm{c}^m}\equiv&
\frac{d^mN}{dV_\textrm{c}^m}\left[\left\{\frac{d^{m-k}N}{dV_\textrm{c}^{m-k}}\right\},
\left\{\frac{d^{m-1-k}\chi}{dV_\textrm{c}^{m-1-k}}\right\},
\left\{C_{n-k}\right\}\right]=0, \quad \nonumber \\[0.5em]
&\forall\  m \quad \mbox{with}\quad
k=\{1,\cdots,m-1\}.
\end{align}
%
It is then always the case 
that potentially finite terms, 
such as $\chi$ and its derivatives, 
or non-divergent powers of $(N-N_\textrm{c})$, 
are coupled to response functions, 
or the vanishing constraint $C_{n-1}$.
%
It follows, 
that response functions of all orders 
vanish as we approach a stationary point, 
as illustrated in Fig.~\ref{fig:h2+_response_functions}, 
since each depends successively on 
vanishing lower-order response functions, 
beginning with the first: $dN/dV_\textrm{c}$.

Then, 
since each term in the $m^{\textrm{th}}$ 
energy derivative is always
proportional to non-divergent powers of 
$(N-N_\textrm{c})$ 
and response functions of (at most) order $m-1$, 
the same logic applies, 
and all energy derivatives tend to zero as well, 
as depicted in Fig.~\ref{fig:h2+_lagrange_multiplier}, 
thereby proving the conjecture.
%
This serves as an inductive proof 
that non-linear constraints targeting SIE, 
of the form $C_n$, 
cannot be enforced.
%


%TWO CONSTRAINTS
\subsection{Construction of separable constraints}
\label{sec:two_constraints}

In light of the failure of non-linear constraints 
to attain a stationary point, 
one possible option remains: 
to re-cast the $C_2$ functional targeting SIE  
into a more viable form 
comprising separate linear and quadratic constraints.
%
For a particular site, 
this new constraint functional is constructed 
by expanding the terms in $C_2$ as follows
%
\begin{align}
C_2&=(N-N_\textrm{c})^2\nonumber \\
&=N^2+N_\textrm{c}^2-2NN_\textrm{c}\nonumber\\
&=N^2+N_\textrm{c}^2-2NN_\textrm{c}+2N_\textrm{c}^2-2N_\textrm{c}^2\nonumber\\
&=N^2-N_\textrm{c}^2+2N_\textrm{c}(N_\textrm{c}-N)\nonumber\\
&=-2N_\textrm{c}(N-N_\textrm{c})-(N_\textrm{c}^2-N^2).
\end{align}
%
Invoking separate Lagrange multipliers 
in this reformulation 
may then afford the system 
an additional degree of freedom 
to satisfy the individual constraints.

It is instructive to write this result 
in the notation of DFT+$U$.
%
By invoking a small  change of variables, 
we arrive at the constraint energy 
for all sites (neglecting spin)
%
\begin{align}
E_{C_2}&=\sum_{I} \frac{U_1}{2} \left(N^I-N_\textrm{c}\right) 
 + \sum_{I} \frac{U_2}{2} \left(N^{ 2}_\textrm{c}- \left(N^{I}\right)^{2}\right)\nonumber \\[0.5em]
& \mbox{where}\quad
 U_1=-2N_\textrm{c}V_\textrm{c}
 \quad\mbox{and}\quad
 U_2=-V_\textrm{c}.
\label{eq:dft+nu}
\end{align}
%
The vanishing response problem
can now be circumvented by 
decoupling the linear and quadratic
Hubbard-like parameters 
$U_1$ and $U_2$, 
and interpreting them 
as separate Lagrange multipliers.
%
The corrective potential is then modified to 
%
\begin{equation}
\hat{V}_{U_1,U_2}=\sum_{I\sigma}\sum_{mm'}\ket{\varphi^I_m}\left(\frac{U_1^I}{2}-U_2^I\,\hat{n}^{I\sigma}_{mm'}\right)\bra{\varphi^I_{m'}}, 
\label{eq:corrective_potential}
\end{equation}
%
so that the characteristic occupancy eigenvalue
dividing an attractive from a repulsive  potential 
is changed from $1/2$ to $U_1/ 2 U_2 $.
%
Here, the $U_2$ parameters are responsible 
for correcting the interaction  
and for any modification to the gap.
%
The $U_1$ parameters, meanwhile, 
may be used to adjust 
the linear dependence of the energy
on the subspace occupancies, 
and thereby refine eigenvalue derived
properties such as the ionisation potential (IP).


Fig.~\ref{fig:h2+_u1_u2_total_energy} 
shows the total-energy $W$ contour plot 
of the H$_2^+$ system as before 
against the $U_1$ and $U_2$
defined in Eq.~\eqref{eq:dft+nu}.
%
The subspace target occupancy is 
$N_\textrm{c}=N_\textrm{exact}=0.602$~e, 
where the population of each of the
two PBE $1s$ orbital subspaces  
was calculated using the exact functional.
%
We see that the total-energy is 
maximised along the heavy white line 
at $\sim 0.92$~eV below the exact energy, 
for which the constraint is satisfied 
for non-unique pairs of $(U_1,U_2)$.

\begin{figure}[th!]
\centering
\includegraphics[height=0.494\textwidth]{images/h2+_u1_u2_total_energy_new.pdf}
\caption[Constraint energy landscape of H$_2^+$ vs Lagrange multipliers $U_1$ and $U_2$]{
The constrained energy of H$_2^+$ at 4~a$_0$ 
relative to the exact total-energy, 
against $U_1$and $U_2$,  
with target occupancy $N_\textrm{c} = N_\textrm{exact}$.
%
The constraint is satisfied at the  
total-energy maximum along the 
solid white line, 
which lies $\sim 1$~eV below 
the exact energy.
%
The ionisation potential (IP) is exact  
along the thick yellow line.
%
The linear-response Hubbard $U_2$, 
together with the  $U_1$,  
needed to recover the 
exact subspace density and maximise the total-energy, 
are shown using thin dashed lines.}
\label{fig:h2+_u1_u2_total_energy}
\end{figure}


To understand why the total-energy 
stationary points are always degenerate, 
and hence why the occupancy condition 
fails to distinguish a unique 
$\left( U_1,U_2 \right)$,
we must conduct a second derivative test 
on the Hessian of the constraint 
functional\footnote{Total-derivatives are used here
to indicate that self-consistent
density response effects are included. 
The Hubbard parameters
remain independent variables.}, 
given by
%
\begin{equation}
H_{ij}=\frac{d^2 W}{d U_i d U_j}.
\end{equation}
%
It directly follows that the determinant of ${\bf H}$ is
conveniently calculated
in terms of the response functions
%
\begin{align}
|{\bf H}| 
&=\frac{1}{2}\left|\begin{array}{cc}
	dN/dU_1 & -dN^2/dU_1 \\
	dN/dU_2 & -dN^2/dU_2 \\
	\end{array}\right| 
=\frac{1}{2}\left|\begin{array}{cc}
	dN/dU_1 & -2NdN/dU_1 \\
	dN/dU_2 & -2Nd/NdU_2 \\
	\end{array}\right| \nonumber \\[0.75em]
&=N\left(\frac{dN}{dU_1}\frac{dN}{dU_2}-\frac{dN}{dU_1}\frac{dN}{dU_2}\right)=0,
\end{align}
%
since 
%
\begin{equation}
\frac{dW}{dU_1}=\frac{1}{2}(N-N_\textrm{c})
\quad\mbox{and}\quad
\frac{dW}{dU_2}=\frac{1}{2}(N_\textrm{c}^2-N^2).
\end{equation}
%
Given an inconclusive second derivative test, 
this implies a vanishing energy curvature 
along the lines which contain 
a constant corrective potential and, 
hence, no unique maximum, 
as shown by Fig.~\ref{fig:h2+_u1_u2_total_energy}.

Suppose we may identify a unique maximum by  
directly calculating the $U_2$ parameter first, 
and then optimising the total-energy along $U_1$.
%
The self-consistent linear-response $U$ 
calculated at this bond-length is $4.84$~eV. 
%
If we intuitively set $U_2 = U$, 
based on the correspondence between 
the interaction corrections in both approaches, 
then an associate $U_1 = 3.16$~eV 
is required to recover the exact subspace density.
%
These values are indicated by the intersecting 
white dashed lines in Fig.~\ref{fig:h2+_u1_u2_total_energy}.
%
However, the maximum constrained total-energy 
is still bounded by the choice of target occupancy
and remains $\sim 1$~eV below the exact energy 
unless supplemented with an additional energy correction.

Alternatively, 
if we select another plausible target occupancy  
$N_\textrm{c} = 0.511$~e 
the constrained energy can be tuned 
to reach a maximum at the exact energy, 
however it requires an unphysical 
$U_1= U_2 \approx -444$~eV 
to do so.

Finally, 
the line on which the IP is exact  
intercepts $U_1 = 0$~eV 
at the linear-response $U_2 \approx U$, 
echoing the `SIC' double-counting 
correction proposed in Ref.~\cite{PhysRevB.76.033102}.
%
However, 
the IP is also irreconcilable with the exact total-energy, 
for any $U_1$ and $U_2$, 
unless subsidised with an energy correction as well.
%
We conclude, therefore, 
that a ground-state susceptible to SIE 
cannot be excited in a systematic way 
to a state that is free from SIE 
by means of exact constraints 
on the density, 
without breaking a physical symmetry.
% 
In other words, 
the total-energy cannot typically be corrected of SIE 
in this scheme by altering the density alone, 
and a non-zero energy correction term is typically required.

%GENERALISED DFT+U FORMULA
\section{A generalised DFT+{$U$} formula}
\label{sec:generalised_dft+u}

We have heretofore shown that 
non-linear constraints 
targeting the SIE of a one-electron system 
cannot restore the exact total-energy 
by means of automated methods.
%
Moreover, 
we have observed that attaining 
the exact total-energy or eigenvalue 
from a linear combination of 
occupancy-dependent  constraints 
is not feasible without 
a supplemental energy term.
%
Motivated by these findings,  
we return toward the development 
of a generalised DFT+$U$ functional - 
one that can simultaneously correct 
the SIE and restore Koopmans' condition. 
%
{The strategy we elect to explore 
in the remainder of this Chapter then  
is that of computing the $U_1$ and $U_2$ parameters 
from first-principles from  linear-response methodology.}

%We demonstrated in Fig.~\ref{fig:h2+_dft+u_eigenvalue} 
%that a persistent deficiency 
%in the Hubbard functional 
%lay in restoring Koopmans compliance, 
%which we affiliated with 
%an insufficiency in the linear term 
%in relation to amending the eigenvalue.
%

Returning again to Eq.~\eqref{eq:dft+nu}, 
let us omit the target density dependence 
(by setting $N^I_\textrm{c}=0$) 
and extract the free-energy~\cite{PhysRevA.72.024502} 
while adapting the treatment of 
the subspace occupancy to 
multi-orbital sites and 
neglecting inter-eigenvalue terms 
by resetting 
$N\to \textrm{Tr}[\hat{n}]$, 
in the spirit of DFT+$U$.
%
By the above modifications, 
we arrive at the generalised DFT+$U$ functional,
given by
%
\begin{align}
E_{U_1,U_2} = \sum_{I \sigma} \frac{U^I_1}{2} 
\mathrm{Tr} \left[ \hat{n}^{I \sigma} \right]  
-  \sum_{I \sigma}  \frac{U^I_2}{2}
  \mathrm{Tr} 
\left[ (\hat{n}^{I \sigma})^{ 2} \right], 
\label{eq:dft+u1+u2}
\end{align}
%
where the conventional DFT+$U$ functional of 
Eq.~\eqref{eq:dft+u_functional} is recovered 
by simply setting $U^I_1=U^I_2$\footnote{
We also note a resemblance between Eq.~\eqref{eq:dft+u1+u2} 
and the three-parameter DFT+$U\alpha\beta$ functional
proposed in Ref.~\cite{dabo2008towards}, 
however the functional forms are quite different.}.

The supplemental energy correction 
we previously sought in 
order to reconcile the eigenvalue 
is hereby provided by the 
linear term in Eq.~\eqref{eq:dft+u1+u2}, 
the total-energy generated by which is shown in 
Fig.~\ref{fig:h2+_u1_u2_free_energy} 
against $U_1$ and $U_2$ as before.
 %
The zero of energy is again 
set to the exact energy $E_\textrm{exact}$, 
the intercept of which with 
the PBE+$U_1$+$U_2$ energy is 
indicated by the heavy dashed line.
%
The thin dashed lines then indicate 
the previously calculated Hubbard 
$U_2 = 4.84$~eV, 
and the corresponding
$U_1 = 4.44$~eV 
required to recover it.
%
We note these are remarkably similar to 
the $U$ value needed by a 
traditional DFT+$U$ calculation 
($U_1=U_2 = 3.85$~eV) 
to do the same.

\begin{figure}[th!]
\centering
\includegraphics[height=0.494\textwidth]{images/h2+_u1_u2_free_energy_new.pdf}
\caption[DFT+$U_1$+$U_2$ free energy landscape of H$_2^+$]{
The free energy of H$_2^+$ at 4~a$_0$ 
relative to the exact total-energy, 
obtained from the generalised DFT+$U$ 
functional in Eq.~\eqref{eq:dft+u1+u2}
against $U_1$and $U_2$,  
with target occupancy $N_\textrm{c} = N_\textrm{exact}$.
%
The thick orange line is the 
exact energy intercept, 
and the thin dashed lines show 
the linear-response $U_2$ 
together with the corresponding $U_1$ 
needed to recover the exact energy.
%
The solid white line 
indicates where the exact subspace
occupancy is recovered,
as in Fig.~\ref{fig:h2+_u1_u2_total_energy}.
}
\label{fig:h2+_u1_u2_free_energy}
\end{figure}

The heavy white line, 
indicates where the exact 
subspace occupancy is acquired, 
as in Fig.~\ref{fig:h2+_u1_u2_total_energy}, 
which intersects with the exact total-energy 
(orange dashed line) 
at  $U_1=5.73$~eV and  $U_2=6.98$~eV,
at which point 
$E_\textrm{exact}$ and $N_\textrm{exact}$
are simultaneously attained.
%
At the same point, however, 
the KS eigenvalue 
$\varepsilon_\textrm{DFT}$
lies $\sim 2.8$~eV above $\varepsilon_\textrm{exact}$.
%
This highlights that an accurate total-energy 
at a particular occupancy 
may coincide with an inaccurate ionisation energy, 
and vice versa,
as was recently shown in detailed analyses 
of the residual SIE in hybrid 
functionals~\cite{PhysRevB.94.035140}
and in DFT+$U$ itself~\cite{:/content/aip/journal/jcp/145/5/10.1063/1.4959882}, 
at fractional total occupancies.

However, 
the generalised DFT+$U$ functional 
does in fact enable the simultaneous correction of the IP 
(yellow line in  Fig.~\ref{fig:h2+_u1_u2_total_energy}) 
and the total-energy at 
$U_1=10.03$~eV and $U_2=14.18$~eV.
%
This preliminary finding is very encouraging
and in the following section, 
we will endeavour to invoke 
these exact conditions 
across the entire dissociation of H$_2^+$.
%
We have shown, however, 
that the generalised DFT+$U$ formula 
may provide the correction of any two of 
the exact energy, eigenvalue, or occupancy,  
but not all three simultaneously.
%
This {highlights} the inherent limitation 
of this generalised functional going forward.


%CALCULATING U1 AND U2
\subsection{Non-self-consistent estimates for {$U_1$} and  {$U_2$}}
\label{sec:calculating_u1+u2}

We have thus far been able 
to effectively estimate  
the $U_1$ and $U_2$ values 
required to attain the 
exact total-energy and IP 
by exploring the energy landscape and 
taking reasonable extrapolations.
%
This procedure, however, 
is not practical to follow for all bond-lengths, 
and a more efficient means of computing 
$U_1$ and $U_2$ must be employed.
%
Fortuitously, 
in a one-electron case such as H$_2^+$, 
the degrees of freedom afforded by 
the two Hubbard parameters  
allow us to simultaneously 
resolve the exact total-energy and IP 
using simple linear algebra.

As demonstrated by the  
occupancy curves in Fig.~\ref{fig:h2+_occ}, 
there is a negligible charge and kinetic screening effect
upon applying a $+U$ correction.
%
We may therefore estimate the required 
$U_1$ and $U_2$ values non-self-consistently  
by simply supplementing the PBE total-energy 
(Fig.~\ref{fig:h2+_total_energy})
with the DFT+$U_1$+$U_2$  functional, 
and the ionisation potential 
(Fig.~\ref{fig:h2+_eigenvalues}) 
with the corresponding correction potential 
given in Eq.~\eqref{eq:corrective_potential}.
%
We may then fit these curves 
against the exact total-energy profile 
with respect to the 
$U_1$ and $U_2$ parameters, 
to acquire reliable estimates.

Here, we have also used 
the convenient feature of H$_2^+$ 
that the PBE $1s$ orbital 
subspace projectors closely match
the KS orbitals in spatial profile, 
and almost exactly so at dissociation.
%
Consequently, the red and blue open circles in
Fig.~\ref{fig:h2+_koopmans} 
illustrate the values of 
$U_1$ and $U_2$ required to perform this task, 
as a function of internuclear distance, 
which at first seem very large but, 
nonetheless, deliver the desired result.

\begin{figure}[th!]
\centering
\includegraphics[height=0.494\textwidth]{images/h2+_koopmans.pdf}
\caption[Estimated $U_1$ and $U_2$ parameters to restore 
exact total-energy and Koopmans' condition in dissociating H$_2^+$]{
Generalised Hubbard $U_1$ and $U_2$ parameters 
(red, blue) estimated to recover the exact total-energy 
and Koopmans' condition in dissociating H$_2^+$ 
using Eq.~\eqref{eq:U1U2}.
%
Open circles show the corresponding 
quantities estimated from linear regression 
of the exact total-energy and the PBE+$U$ total-energy. 
%
The Koopmans $U_K$ is also shown (orange), 
which dominates $U_1$ at short bond-lengths.
}
\label{fig:h2+_koopmans}
\end{figure}

Let us now attempt to derive formulae to estimate 
$U_1$ and $U_2$  
from ground-state DFT quantities.
%
Upon the application of a 
conventional DFT+$U$ correction, 
stiffening in the subspace response is typically 
observed~\cite{doi:10.1021/jp070549l,PhysRevLett.97.103001}, 
as demonstrated by the comparable 
occupancies in Fig.~\ref{fig:h2+_occ} 
for $U=0$~eV and $U=8$~eV.
%
It is therefore reasonable to construct 
a charge non-self-consistent first-principles
calculation scheme for  $U_1$ and $U_2$, 
e.g., for use in refining DFT+$U$ calculations in 
order to approximately enforce Koopmans' condition, 
and we may ask if we can conveniently 
estimate $U_1$ and $U_2$ 
using intrinsic quantities of a PBE calculation, 
without explicit access to the exact values.
%
It would also be advantageous to establish a relationship 
between $U_1$ and $U_2$ 
and the linear-response $U$.
%
In this section, 
we shall therefore outline a procedure 
to estimate the charge non-self-consistent 
$U_1$ and $U_2$ parameters for H$_2^+$ 
without relying on the fitting of curves 
with the exact energy profile post-process.

%
For $M$ equivalent one-orbital subspaces 
contributing to the energy band responsible for 
the eigenvalue  
$\varepsilon_{\textrm{DFT}}$, 
the exact total-energy and eigenvalue   
may be approximated from 
their corresponding PBE quantities 
as follows 
%
\begin{align}
\label{eq:e_approximated}
E_{\textrm{exact}} 
&\approx E_{\textrm{DFT} } + M\left( \frac{U_1}{2} N_{\textrm{DFT}} - \frac{U_2}{2} N_{\textrm{DFT}}^2\right)  \\[0.5em]
\varepsilon_{\textrm{exact}} 
&\approx \varepsilon_{\textrm{DFT}} + \frac{U_1}{2} -  U_2 N_{\textrm{DFT}}
\end{align}
%
with density non-self-consistent $U_1$ and $U_2$, 
where $N_\textrm{DFT} = \mathrm{Tr}  \left[ \hat{n}_\textrm{DFT} \right]$, 
and neglecting overlap and spillage.


Let us now suppose that we have calculated 
a conventional $U$ which 
reconciles the total-energy reasonably, 
so that
%
\begin{equation}
E_{\textrm{exact}} \approx E_{\textrm{DFT}} +  M\frac{U}{2} \left( N_{\textrm{DFT}} - N^2_{\textrm{DFT}} \right).
\label{eq:dft+u_approximated}
\end{equation}
%
 Finally, we wish to exactly satisfy Koopmans' condition:
%
\begin{equation}
\varepsilon_{\textrm{exact}} = E_\textrm{DFT} [N] - E_\textrm{DFT} [N-1],
\label{eq:koopmans_approximated}
\end{equation} 
%
where the term $E[N-1]$ is the DFT-estimated 
total-energy of the ionised system, 
which in this case, is simply 
the ion-ion energy $E_\textrm{ion-ion}$.
 
Thus, combining Eqs.~\eqref{eq:e_approximated}~-~\eqref{eq:koopmans_approximated}, 
where only convenient, approximate 
DFT quantities have been used, 
we arrive at 
%
\begin{align}
U_1 &{}\approx  U \left( 1 - N_{\textrm{DFT}} \right) 
\left( 2 - M N_{\textrm{DFT}}  \right) 
+ U_K \nonumber \\[0.5em]
\mbox{and}\quad 
N_\textrm{DFT}U_2 &{}\approx  U \left( 1 - N_{\textrm{DFT}} \right) 
\left( 1 - M N_{\textrm{DFT}}  \right) + U_K , 
\label{eq:U1U2}
\end{align}
%
where we have defined the `Koopmans  $U$' as follows 
%
\begin{equation}
U_K =  2 \left(\varepsilon_\textrm{DFT} - \left\{E_\textrm{DFT}[N] - E_\textrm{DFT}[N-1]\right\}\right).
\end{equation} 
%
This term is the part of the correction that 
effectively imposes Koopmans' condition, 
primarily affecting $U_1$.
%
These results correspond remarkably 
well with the exact values required, 
as shown by the solid lines in Fig.~\ref{fig:h2+_koopmans}, 
and confirms the validity of approximating  
a negligible density response.

The interdependence between the parameters 
for any value of $M$ 
%
\begin{equation}
U_1 - U_2 N_\textrm{DFT} \approx U \left( 1 - N_\textrm{DFT} \right)
\end{equation}
%
also reveals the role
of $U$ in characterising the degree of 
splitting between $U_1$ and $U_2$.
 %
Moreover, 
enforcing Koopmans' condition in H$_2^+$ 
pushes both $U_1$ and $U_2$ 
up to considerably higher values
than are commonplace~\cite{QUA:QUA24521} 
in conventional DFT+$U$.
%
In Fig.~\ref{fig:h2+_koopmans}, 
we see that 
$U_1$ is dominated by $U_K$ at short bond-lengths, 
where $U_1 \approx N_\textrm{DFT} U_2$, 
before ultimately falling off to 
the average of $U$ and $U_2$ in
the fully dissociated limit.
%
Over the same range, 
$U_K$ composes 75\% - 50\% of $U_2$.






%RECALCULATING H2+ CURVE
\subsection{Application to H$_2^+$}
\label{sec:h2+_with_dft+u1+u2}

Using the estimated $U_1$ and $U_2$ values in Eq.~\eqref{eq:U1U2}, 
we performed density non-self-consistent 
DFT+$U_1$+$U_2$ calculations on H$_2^+$ 
using the non-self-consistent formulae 
for the PBE total-energy 
and occupied KS eigenvalue 
given in 
Eqs.~\eqref{eq:dft+u_approximated}~\&~\eqref{eq:koopmans_approximated}.
%
To put the method on a first-principles footing, 
we invoke the self-consistent $U^{(2)}$ value, 
(calculated using the methods described 
in Chapters~\ref{ch:calculating_hubbard_u}~\&~\ref{ch:self_consistent_hubbard}) 
to determine $U_1$ and $U_2$.
%
The resulting corrections to the 
total-energy $E$ and eigenvalue $\varepsilon$ take the form 
%
\begin{align}
\Delta E = U^{(2)} (N - N^2)
\quad\mbox{and}\quad&
\Delta v_U =  \frac{U^{(2)}}{2} (1 -2N) - \frac{U_K}{2}, \nonumber \\[0.5em]
\mbox{where}\quad\quad
U_K/2= \varepsilon_\textrm{PBE}&-(E_\textrm{PBE}-E_\textrm{ion-ion}) .
\end{align}

Fig.~\ref{fig:h2+_koopmans_total_energy} 
illustrates the result of this simple technique, 
which simultaneously reconciles the 
total-energy and eigenvalue-derived 
binding curves with the 
exact result\footnote{
The total-energy here is that of PBE+$U^{(2)}$ 
given in Fig.~\ref{fig:h2+_pbe+uout}.} 
in Fig.~\ref{fig:h2+_total_energy}.
%

Although the correction is imprecise, 
since this is a non-self-consistent post-processing step, 
it nonetheless presents a compelling possibility 
toward extending DFT+$U$ 
to ameliorate the IP.
%
Indeed, in the dissociating limit, 
where changes to the occupied Kohn-Sham 
orbital are negligible and $N\to1/2$, 
we find that
%
\begin{align}
\varepsilon_{\textrm{PBE}+U_1+U_2} 
%&= \varepsilon_\textrm{PBE} + \Delta v_U  \nonumber \\
&= \varepsilon_\textrm{PBE} - ( E_\textrm{ion-ion} - E_\textrm{PBE} + \varepsilon_\textrm{PBE}) \nonumber \\
&= - ( E_\textrm{ion-ion} - E_{\textrm{PBE}+U_1+U_2} )\equiv - \textrm{IP}, 
\end{align}
%
which restores Koopmans compliance 
for a SIE correction strength $U$.


\begin{figure}[th!]
\centering
\includegraphics[height=0.494\textwidth]{images/h2+_koopmans_total_energy}
\caption[Total-energy and eigenvalue binding curves for H$_2^+$ calculated with generalised DFT+$U$ functional]{
Exact (red),
total-energy based $E_{\textrm{PBE}+U_1+U_2}$ (blue), 
and eigenvalue based $\varepsilon_{\textrm{PBE}+U_1+U_2}$ (orange)
binding curves for H$_2^+$, calculated
using DFT+$U_1$+$U_2$, 
which enables the simultaneous  
SIE correction of the total-energy and Kohn-Sham eigenvalue 
to precisely the same accuracy.}
\label{fig:h2+_koopmans_total_energy}
\end{figure}



%%POSSIBLE EXTENSIONS
%\subsection[{Possible extensions to the generalised DFT+$U$ scheme}]{Possible extensions to the generalised\break DFT+$U$ scheme}
%\label{sec:possible_extensions_dft+u1+u2}
%
%If the proposed scheme described above,  
%is applied to supplement an 
%existing DFT+$U$ calculation 
%that has already accurately 
%recovered the total-energy, 
%then it is sufficient to set $U = 0$~eV 
%in our approximate formulae of 
%Eq.~\eqref{eq:U1U2}.
%%
%This is because the SIE interaction has 
%already been treated by an appropriate $+U_0$ correction.
%%
%An approximately Koopmans-compliant  
%DFT+$U$ calculation may then be 
%performed by a simple change of parameters 
%%
%\begin{equation}
%(U_1,N_\textrm{DFT}U_2) \to \left(U_K,U_K\right), 
%\end{equation}
%%
%which only requires knowledge of 
%the ionised state $E[N-1]$.
%
%The scheme may also be generalised to 
%multi-orbital subspaces straightforwardly, 
%by replacing $N_\textrm{DFT}$ 
%with a sum over the eigenvalues of 
%$\hat{n}^I_\textrm{DFT}$, 
%given as follows 
%%
%\begin{equation}
%N_\textrm{DFT} \to \sum_{m}\left[\hat{n}^I_{mm'}\right]
%\quad\mbox{and}\quad
%N_\textrm{DFT}^2 \to \sum_{m}\left[\hat{n}^I_{mm'}\hat{n}^I_{m'm}\right], 
%\end{equation}
%%
%which ignores intra-subspace interactions 
%in the spirit of DFT+$U$.
%
%Finally, the approximation of constant $N_\textrm{DFT}$ 
%may be replaced by a linear-response approximation, 
%in terms of the response function $\chi$, 
%%
%\begin{equation}
%N_\textrm{DFT}\to \chi \textrm{Tr}[\hat{v}_\textrm{ext}\hat{P}]
%\end{equation}
%%
%or lifted entirely by means of
%a  parametrisation of the occupancies in terms of 
%$\textrm{Tr}\left[ \hat{v}_{U_1 U_2} \right] $ 
%and a numerical solution of 
%the resulting equations.


%CONCLUSION
\section{Conclusion}
%
To conclude, 
we have proven analytically, 
and with stringent numerical tests,
that non-linear constraints are incompatible 
with the correction of SIE with cDFT.
%
It is not possible, therefore, to automate 
systematic SIE corrections of the type 
provided by DFT+$U$ by means of cDFT, 
notwithstanding the great utility of the latter.


We demonstrated that progress can be made, however,  
by constructing separate constraints 
for the linear and quadratic density terms 
and decoupling the corresponding Lagrange multipliers.
%
As we have shown, 
the resultant non-uniqueness of stationary points 
of such a scheme requires the 
introduction of additional physics, 
such as linear-response theory for $U_2$.
%
%The additional freedom provided 
%by a third parameter, 
%admitting a constant energy shift as suggested
%in Ref.~\cite{dabo2008towards}, 
%could also make up the difference.

Our conclusion that 
a ground-state afflicted by SIE cannot be systematically
excited to a state that is less so, 
by means of constraints,
is reinforced by the fact that it is 
the target-independent free-energy, 
and not total-energy, 
that agrees with the exact energy.
%
Indeed, 
the free-energy of this functional, 
equivalent to a generalised DFT+$U$ energy, 
allows both the exact occupancy and energy 
to be recovered for a reasonable pair of $U$ parameters, 
which improves upon the performance of conventional DFT+$U$.
%
It also provides for the correction of the 
eigenvalue and the exact energy simultaneously, 
but at the forfeit of the exact density.

Nonetheless, 
we have found that the free-energy functional 
offers the capability of simultaneously 
correcting two central quantities in DFT, 
the total-energy and the highest occupied orbital energy.
%
This provides further motivation 
for our so-called `generalised DFT+$U$' functional, 
the results of which highlight the potential of 
the DFT+$U_1+U_2$ functional 
and its immediate compatibility with 
self-consistently calculated Hubbard $U$ parameters.

The approximate formulae for the required parameters, 
which may differ greatly from the familiar Hubbard $U$ 
(even in a one-electron system), 
offer a novel framework in which to further
develop double-counting techniques and first-principles 
schemes for the promising class of methods correcting SIE  
based on DFT+$U$~\cite{QUA:QUA24521,
PhysRevLett.97.103001,
:/content/aip/journal/jcp/145/5/10.1063/1.4959882,
:/content/aip/journal/jcp/133/11/10.1063/1.3489110}, 
as well as opening up possibilities for their diverse application.
%
For example, we envisage that SIE  correction schemes
of two or more parameters may also be useful 
for generalising the exchange fraction of
hybrid functionals~\cite{PhysRevB.94.035140}, 
and for DFT+$U$ type corrections of perturbative many-body 
approximations such as $GW$~\cite{PhysRevB.82.045108}, 
the deviation from linearity of which is somewhat 
analogous~\cite{PhysRevA.75.032505,PhysRevB.93.121115}.
%
More pragmatically, 
we expect the extra degree of freedom furnished by $U_1$ 
to be beneficial in cases where  the quadratic
approximation to the subspace-averaged
self-interaction does not remain valid 
all the way down to the ionised state, 
which is particularly relevant for H$_2^+$, 
since that is where the state corresponds 
to the low-density limit.
%


Interesting avenues for the 
immediate development of this method 
include its extension to multi-electron, heterogeneous,
and non-trivially spin-polarised systems, as well as to
perform self-consistency over the density 
and to lift the fixed-occupancy approximation, 
as outlined in Ref.~\cite{PhysRevB.94.220104}.
%
For compliance with Koopmans' condition, 
it seems unavoidable for now that data must be collected 
from both the approximate neutral and ionised systems
(the total-energy of which may be sufficient), 
to calculate $U_1$ and $U_2$.
%
In the manner in which we have performed it  here, 
non-self-consistent DFT+$U_1+U_2$ 
requires only one total-energy calculation 
at the ionised state, 
on top of the usual apparatus
of a linear-response DFT+$U$ calculation,  
in order to simultaneously 
(albeit approximately) 
correct the total-energy
and the HOMO eigenvalue for SIE.
%
In principle, a further refinement of the
method might entail the self-consistent
linear-response calculation of $U_1$ and $U_2$
separately for the neutral and ionised states.

Nonetheless, 
what is revealed in this Chapter is a convenient scheme 
for correcting the exact energy and eigenvalue, 
in conjunction with a $U$ 
calculated via self-consistent ground-state linear-response approach (or otherwise), 
for a system with negligible variance in 
average subspace occupancy 
under an applied $U$.
%
%In the following Chapter, 
%we will focus on the practical calculation 
%of Hubbard $U$ parameters 
%via linear-response in a direct-minimisation code, 
%which may be then put to use 
%in a generalised scheme described above,  
%or in conventional DFT+$U$ calculations.




%%EXTRA STUFF I DIDNT INCLUDE
%Despite the success of DFT+$U$  in SIE-correcting 
%the total-energy   using a suitably
%calculated $U$ value, 
%the fact remains that 
%it is incapable of simultaneously correcting the highest 
%occupied Kohn-Sham eigenvalue to minus the ionisation potential
%in compliance with Koopmans' condition,
%as indicated in Fig.~\ref{figure1b}. 
%%
%This issue has previously been  explored  
%in Ref.~\cite{:/content/aip/journal/jcp/145/5/10.1063/1.4959882}, 
%and by us in Ref.~\cite{PhysRevB.94.220104} 
%where we constructed a generalised, 
%two-parameter DFT+$U$ functional, 
%comprising separate parameters for the linear ($U_1$) 
%and quadratic ($U_2$) terms.
%%
%In fact, Eq. 9 of Ref.~\cite{PhysRevB.94.220104}
%indicates that if a symmetric system of two one-orbital subspaces
%(a very good approximation for H$_2^+$, with approximately
%constant subspace occupancies $N$) is Koopmans compliant
%(so that the Koopmans' $U_K = 0$), and it is then corrected 
%using DFT+$U$ for the SIE in the total-energy
%(it is possible for the interaction strength to be inaccurate,
%but for the system still to comply with Koopmans'  condition), then
%DFT+$U$ will act to spoil that condition unless  
%$U_1 = 2 U_2 ( N - N^2) / ( 1 - 2 N )$.

%Motivated by the great utility cDFT for correcting SIE by 
%promoting broken symmetry, 
%integer-occupancy configurations
%with well described by approximate 
%functionals~\cite{PhysRevLett.97.028303,doi10.1021/cr200148b},
%quadratic or higher-order constraints are
%required to change the  inter-electron interaction
%in DFT, and thus to systematically correct SIE.
%
%The gross disagreement of the occupied Kohn-Sham eigenvalue
%with the exact energy for all relevant calculations, 
%in any case, reveals a glaring failure of 
%DFT+$U$ type functionals 
%to simultaneously reconcile the energy and
%the ionisation potential with that of the exact system.
%%
%To our knowledge, the SIE of approximate DFT 
%has not previously been  simultaneously
%addressed for the total-energy and the occupied
%Kohn-Sham eigenvalue using a
%first-principles correction method of DFT+$U$ type,
%even for a  one-electron system such as this. 
%%
%
%%
%For the analysis and correction of spuriously self-interacting
%multi-reference systems,  we may
%learn much from the exact solution of minimal  
%models~\cite{doi:10.1063/1.4871875}.}
%
