
 \setlength\epigraphwidth{.55\textwidth}
 \renewcommand{\epigraphflush}{center}

\epigraph{\it All models are wrong but some are useful.}
{George E. P. Box~\cite{doi:10.1080/01621459.1976.10480949}}

\lett{T}{he pillars of modern} 
physics now comprise the well established 
fields of theory, experiment and simulation, 
where the latter addition 
has been made possible thanks to the 
technological advances 
%in micro-processing power 
of the 20\textsuperscript{th} century, 
characterised by Moore's law~\cite{5696765}. 
%
In the current `Age of Silicon'~\cite{sass1998substance} 
simulation and modelling 
have become indispensable tools 
that find cutting-edge application    
in the field of quantum physics, 
in particular for solving the complex 
many-body Sch{\"o}dinger equation~\cite{Trabesinger2012}.
%
Notwithstanding the significant progress made  
in micro-processors over the last five decades, 
it is only by the intelligent use of 
algorithms and suitable approximations  
that it is possible to quantitatively describe   
chemical processes 
and complex material properties  
with sufficient accuracy and efficiency.

Density-functional theory (DFT) conforms 
to all of these principles 
and is one of the most successful 
and highly acclaimed techniques 
in modern science. 
%
It affords a timely and accurate 
description of the ground-state properties 
of many atomic systems on an {\it ab inito} basis 
that is from an entirely deterministic standpoint  
with only minimal approximations 
and without the use of empirical models.
%
It enables the straightforward calculation 
of properties derived from the 
ground-state total-energy, 
which includes, but is not limited to, 
elastic moduli, 
atomic forces, 
electro-magentic responses, 
cohesive binding energies, 
local magnetic moments, 
geometrically optimised structures, 
and charge transfer energies~\cite{martin2004electronic}.
%
Its role in computational chemistry and 
materials science continues to flourish 
to this day.


Indeed, 
many systems pertinent to 
chemical, biological or technological applications, 
which typically comprise first-row transition metal or lanthanide ions, 
are often largely spatially disordered and electronically complex.
%
This special class of extensive system presents 
numerous challenges to DFT practitioners, 
as calculations on these so-called {\it strongly-correlated} materials   
often misrepresent, or directly contradict, experimental observations.
%

The first is simply a matter of 
computational economy; 
the study of these systems 
requires accurate large-scale simulation, 
which is unfortunately unattainable with current hardware.
%
On the one hand, 
this is invariably due to the availability of finite resources, 
while on the other we are confronted with the uneconomical 
compute time of conventional algorithms for large systems, 
which scale cubically with system size.
%
The design of linear-scaling methods, 
in which the computational cost 
scales linearly with system size, 
has provided considerable aid in this regard.

The second challenge pertains to the physically-relevant, 
yet poorly described, interactions between correlated electrons 
responsible for the exotic behaviour, 
which are demonstrably beyond the scope 
of the independent-particle approximation.
%
The established, and computationally efficient, 
DFT+$U$ method is now routinely applied to these systems 
to restore the absent physical character, 
whereupon the strength of the corrective potential 
is determined by a set of Hubbard $U$ parameters 
that must be determined.
%


%WHAT IS THE FOCUS OF THIS RESEARCH?
The focus of this dissertation concerns
the {\it ab initio} calculation of 
Hubbard parameters within linear-scaling DFT, 
and their subsequent application in an 
existing linear-scaling DFT+$U$ framework, 
in order to accurately describe strongly-correlated systems on a large-scale.
%
Moreover, 
the ultimate goal of this body of research  
is to direct the design of automated first-principles methods 
for the description of strongly-correlated systems.
% 
We intend our work to contribute valuably to  
techniques involving high-throughput materials informatics, 
in particular for the provision of a mechanism for 
computing thermodynamical quantities, 
by availing ourselves of the computational utility afforded by 
a combined self-contained, linear-scaling DFT+$U$ procedure.


%WHAT IS THE CENTRAL MESSAGE? 
Notwithstanding 
this ambitious undertaking 
towards ever-more accurate 
and versatile compuational techniques, 
as a minor player in this venture 
I am reminded of the words of 
the eminent statistician, George E. P. Box, 
whose famous aphorism 
prefaces this Chapter.
%
I believe its lesson fittingly applies 
in the present context of 
designing efficient {\it ab initio} practices, 
to which his words speak to our 
unending, insatiable pursuit to understand,
and ultimately replicate, nature.


%A procedure that is often employed 
%to ameliorate these inaccuracies, 
%when they may be associated with highly-localised electrons, 
%is the DFT+$U$ method.
%%
%DFT+$U$ intends to restore 
%the physical description of these 
%strongly-correlated systems 
%by applying a corrective potential 
%to the localised electrons, 
%the strength of which is determined 
%by the Hubbard $U$ parameters.
%%
%
%largely spatially disordered and electronically complex

%Moreover, 
%Kohn-Sham DFT concerns itself 
%only with the ground-state of the system, 
%and so is unable to reliably  
%access excited state properties.
%%
%DFT is often supplemented, 
%or even the starting point, 
%of additional techniques that 
%supplemented with additional techniques 
%such
%%

% OUTLINE
\section{Outline of dissertation}
We begin in Chapter~\ref{ch:qm_simulation} 
by providing an overview of the key developments 
that have contributed to the formulation of contemporary 
Kohn-Sham DFT, 
such as the Hohenberg-Kohn theorems, 
the Kohn-Sham equations, 
the construction of approximate exchange-correlation (XC) functionals, 
and the treatment of the ionic cores 
with the pseudopotential approximation 
and projector-augmented wave method.
%
We then motivate and outline 
the framework for enabling the computational expense 
to scale linearly with system size, 
which relies on the attenuation of non-local effects 
in a single-particle density matrix, 
and discuss some of the central features 
of the linear-scaling code {\sc ONETEP}, 
in which our methods are implemented.



In Chapters~\ref{ch:mech_prop_2d_mater}~\&~\ref{ch:elec_prop_2d_mater} 
we present the results of extensive 
calculations on the strained layers of 
phosphorus, arsenic, and antimony.
%
We compare our results to 
the available experimental data
for these materials 
and make qualitative predictions 
for the mechanical and electronic properties, 
which includes several electronic transitions 
and a number of states supporting ballistic conduction.

Following this discussion, 
we examine in detail the 
nature and origin of the 
self-interaction and static correlation errors 
in Chapter~\ref{ch:self_interaction_error}, 
and highlight the role they play   
in undermining the accuracy of 
many DFT calculations.
%
We discuss some of the notable methods 
developed over the years to effectively treat 
the inaccuracies affiliated with these errors, 
and emphasise the importance of 
%determining accurate first-ionisation energies 
restoring compliance with Koopmans' theorem.
%  
In particular, 
we outline the construction 
of the DFT+$U$ correction functional 
intended to treat the self-interaction error  
when it stems from highly-localised electrons.
%
Finally, we briefly discuss a similar 
error arising from static-correlation of degenerate states, 
and suggest a possible mechanism for its treatment  
within the full DFT+$U$+$J$ framework.

We then proceed to discuss the {\it ab initio} calculation of 
Hubbard $U$ parameters in 
Chapter~\ref{ch:calculating_hubbard_u}, 
in which we address various 
methods utilised heretofore, 
in particular the popular linear-response approach 
conceived by Coccocioni and de Gironcoli.
%
We motivate and develop a 
variational approach modified from this 
method to be applied more conveniently 
in codes that employ total-energy direct-minimisation  
(as opposed to density self-consistency) 
to locate the DFT ground-state.
%
We test this method on a variety of systems
and consider the similarities and differences 
between the two approaches.
%
Finally, 
we devise an equivalent scheme 
to calculate the exchange parameter $J$, 
and highlight some of the importance consequences 
of computing interaction parameters 
as ground-state quantities.
%particularly at their self-consistency.
%

Our variational linear-response approach 
provides a convenient format  
to investigate the somewhat esoteric  
practice of parameter self-consistency 
in Chapter~\ref{ch:self_consistent_hubbard}, 
for which numerous plausible 
schemes have been proposed.
%
In doing so, 
we perform extensive calculations 
on dissociating H$_2^+$ 
to identify the self-consistency scheme 
that provides the appropriate 
correction to one-electron self-interaction, 
and then generalise the method 
to treat multi-electronic systems 
using rock-salt NiO as a case study.
%
Moreover, 
we provide an original formulation 
for a self-consistent $J$ 
and use this to correct 
the static-correlation error 
in dissociating H$_2$ 
beyond the Coulson-Fischer point.

\edit{
Finally, 
in Chapter~\ref{ch:non_linear_constraints} 
we focus on the development of a 
generalised DFT+$U$ functional 
that can simultaneously target the 
total-energy SIE and the restoration of Koopmans' condition, 
which represents an original and novel construct.
%
We first approach this challenge by investigating 
if the automated correction of systems prone to SIE  
is feasible within the framework of 
constrained-density-functional theory (cDFT).
%
However, we provide a rigorous proof 
and stringent numerical calculations  
that verify that the non-linear constraints, 
including those required to enforce this condition, 
are impossible to satisfy.
%
Rather than adding to the multitude of available approaches, 
here we make progress by limiting the scope for 
further proliferation of methods based on this strategy.}

\edit{
Nonetheless, 
we show that a generalised DFT+$U$ functional, 
wherein the Hubbard parameters 
may be determined from a 
self-consistent linear-response $U$ parameter, 
is capable of correcting both the total-energy 
and eigenvalue of a one-electron system 
to the same precision.
%
This work opens the door to 
new advances and possibilities 
using a new class of generalised DFT+$U$ functionals.}


We conclude in Chapter~\ref{ch:conclusions} 
with a synopsis of the major findings of this work, 
and suggest possible new directions for future research.
%
Finally, some appendices are provided 
to complement the primary text, 
which are intended to 
advance the discussion and 
provide further clarity, when necessary, 
with supplemental figures, tables, and derivations.




%ASSOCIATED PUBLICATIONS
\newpage
\section{Associated publications}

The Chapters that arise entirely, or in part, 
from previous publications 
or works intended to be published 
are as follows
\vspace{3em}

\hspace{17pt}\makebox[\textwidth][c]{%
\begin{minipage}{0.8\textwidth}

\begin{itemize}
\setlength{\itemsep}{-5pt}

\item[Chapter~\ref{ch:mech_prop_2d_mater}:]
Damien Hanlon, Claudia Backes, {\it et al.} 
``Liquid exfoliation of solvent-stabilized few-layer black phosphorus for applications beyond electronics." \newline
{\it Nature Communications} 6, 8563 (2015).\newline
DOI:~\href{http://europepmc.org/articles/PMC4634220}{10.1038/ncomms9563} 
\\[1em]
%
\item[Chapters~\ref{ch:mech_prop_2d_mater}~\&~\ref{ch:elec_prop_2d_mater}:]
Glenn Moynihan, Stefano Sanvito, and David D. O'Regan. 
``Strain-induced Weyl and Dirac states and direct-indirect gap transitions in group-V materials." \newline
{\it 2D Materials} {\bf 4} 045018 (2017).\newline
DOI:~\href{https://doi.org/10.1088/2053-1583/aa89d2}{10.1088/2053-1583/aa89d2} 
\\[1em]
%
\item[Chapters~\ref{ch:calculating_hubbard_u}~\&~\ref{ch:self_consistent_hubbard}:]
Glenn Moynihan, Gilberto Teobaldi, and David D. O'Regan. 
``A self-consistent ground-state formulation of the first-principles Hubbard U." \newline 
In preparation. \href{https://arxiv.org/abs/1704.08076}{arXiv:1704.08076}~{\bf [cond-mat.str-el]} 
\\[1em]
%
\item[Chapter~\ref{ch:non_linear_constraints}:]
Glenn Moynihan, Gilberto Teobaldi, and David D. O'Regan. 
``Inapplicability of exact constraints and a minimal two-parameter generalization to the DFT+ U based correction of self-interaction error." \newline 
{\it Physical Review B} {\bf 94}, 220104(R) (2016).\newline
DOI:~\href{http://link.aps.org/doi/10.1103/PhysRevB.94.220104}{10.1103/PhysRevB.94.220104}
\end{itemize}
\end{minipage}}
