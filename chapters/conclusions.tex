
\lett{I}{t is the desire of} 
every postgraduate student  
to contribute meaningfully and fruitfully, 
in a manner however small, 
to the body of work cultivated   
by their mentor, colleagues, 
and predecessors.
%
In this dissertation, 
we have strived, as much as possible, 
to present a comprehensive and 
coherent account of our modest research, 
for both posterity and the hope 
that it may contribute to new paths forward.
%
We now end with 
a brief summary on the central findings of each Chapter 
and a discussion on some potential avenues 
for further research based on our research.


%SYNOPSIS
\section{Synopsis}

We began our discussion in Chapter~\ref{ch:qm_simulation} 
by outlining the chronology of 
significant developments that gave rise to Kohn-Sham DFT, 
which is now  routinely applied in contemporary research 
in chemistry and materials science.
%
We highlighted the important consequences 
of utilising approximate exchange-correlation functionals, 
which includes, but is not limited to, 
the qualitative mischaracterisation 
of strongly-correlated systems 
due to the onset of self-interaction (SIE) 
and static correlation (SCE) errors.
%
Moreover, 
we placed a special emphasis on 
the exploitation of the near-sightedness of 
quantum mechanical interactions~\cite{PhysRevLett.76.3168}, 
which contributed to the development of 
linear-scaling methods.
%

In Chapters~\ref{ch:mech_prop_2d_mater}~\&~\ref{ch:elec_prop_2d_mater} 
we presented the results of a 
comprehensive set of DFT calculations 
on the puckered phase of 
phosphorus, arsenic, antimony, 
and their few-layer counterparts.
%
\edit{We discovered that 
the mechanical properties of these materials 
are well described by approximate DFT 
and corroborate experimental measurements.
%
This enables further predictions 
regarding the mechanical properties of these novel 2D materials, 
such as their use in nano-composite materials~\cite{PMID:26469634,LI201617}, 
to be made in confidence.}

%
The electronic properties, 
on the other hand, 
were found to be largely underestimated 
with respect to the experimental values, 
a result that is characteristic of the SIE.
%
Nonetheless, 
we expect general trends in electronic behaviour 
to be reliable and we made preliminary predictions 
of possible Weyl and Dirac states in many of these structures.
%
\edit{
Our preliminary results in this section 
showcase the possible exciting directions 
that developments in these materials 
may take the near future such as in 
high-mobility conductors~\cite{doi:10.1021/nn501226z,Kim723,doi:10.1063/1.4943548,Yao2013,Lu2016,Zhao2015,Lu2016,doi:10.1021/nl500935z}, 
battery materials~\cite{doi:10.1021/jp302265n,Sun2015}, 
transistors~\cite{Xia2014,Li2014} , 
and catalysts~\cite{doi:10.1021/jp508618t,doi:10.1021/acs.jpcLetters5b01094}.
%
Indeed, 
from our results we have shown that reasonable strains, 
which are experimentally accessible, 
provide a convenient mechanism 
to attain exotic electronic behaviour in 
the group-V materials without the need for doping.}


In Chapter~\ref{ch:self_interaction_error}, 
we explored the nature of the 
SIE and SCE in greater detail.
%
In particular, 
we discussed how they relate to violations of the 
physical conditions obeyed by the exact functional, 
namely piece-wise linearity and piece-wise constancy, 
which may be respectively understood from the perspectives of 
fractional occupancies and fractional spins.
%
We also discussed some historically successful 
procedures to ameliorate these conditions 
and recognised the noteworthy contributions 
of Perdew and Zunger~\cite{PhysRevB.23.5048}, 
and Dabo~\cite{PhysRevB.82.115121}.
%
In particular, 
we note the efforts of the latter to 
restore Koopmans' compliance, 
as well as the widely successful 
DFT+Hubbard $U$ (DFT+$U$) method~\cite{QUA:QUA24521}, 
the marriage of which is a central focus of this thesis.
%which has gained widespread popularity 
%as a means to correcting one-electron SIE 
%when it is attributed to highly localised electrons, 
%typically the 3$d$ or 4$f$ orbitals.

The calculation of Hubbard $U$ parameters from first-principles 
in a code utilising direct minimisation 
motivated the work discussed in Chapter~\ref{ch:calculating_hubbard_u}.
%
We outlined the challenges we faced 
with the popular linear-response method  
of Cococcioni and co-workers~\cite{PhysRevB.71.035105,PhysRevB.84.115108}
and our approach for modifying it to surpass them.

The calculated parameters were 
accompanied by small statistical errors 
and generalised easily to multi-electronic systems.
%
The DFT+$U$ calculations 
performed on various strongly-correlated systems, 
including bulk NiO and Cr$_2$O$_3$, 
yielded results that greatly improved upon 
the uncorrected DFT treatment 
without exception and were in line with qualitative 
and quantitative experimental data 
provided by colleagues.
%
This showed that our method was highly robust,  
readily applicable to a broad range of systems,  
and capable of competing with other state-of-the-art methods.}

\edit{A key advantage of our adapted method 
is that the response functions, 
and by extension the Hubbard $U$ and $J$ parameters, 
are computed as ground-state properties of the system, 
thereby elevating DFT+$U$ 
to the status of a self-contained, 
ground-state density-functional theory.
%
This was not the case beforehand 
and precluded the direct comparison of 
DFT+$U$ total energies needed for 
computing thermodynamic quantities, 
which were inherently ill-defined as a result.
%
The fact that the Hubbard $U$ parameter can now be 
regarded as an implicit functional of the ground-state density 
represents an important first step toward enabling this functionality 
and will be a welcome upgrade for many 
researchers in this field~\cite{PhysRevB.84.045115,doi:10.1021/cm702327g,PhysRevB.73.195107,PhysRevB.85.155208}.}

We also drew upon the observations we made in 
Chapter~\ref{ch:self_interaction_error} 
to propose a similar procedure 
for calculating the exchange parameter $J$.
%
\edit{The exchange parameters were also 
determined very accurately and 
were largely in line with values calculated 
in previous studies.
%
These results were greatly encouraging 
and provided a firm framework 
as we progressed to establishing 
self-consistency schemes for both parameters.}

Our desire for a framework enabling the 
direct comparison of DFT+$U$ total-energies 
to facilitate the calculation of thermodynamic quantities, 
was outlined in Chapter~\ref{ch:self_consistent_hubbard}.
%
We discussed how the comparability of total-energies 
depends on the condition that the applied $U$ (or $J$) 
is a functional of the DFT+$U$ ground-state density 
to which it is applied.
%
We therefore extended the variational linear-response method 
to allow the self-consistent calculation of $U$ and $J$.
%
Here, they became implicit functionals 
of both the ground-state density and input parameters 
and thereby provided a convenient mechanism 
to identify the self-consistency scheme 
that fully corrects the SIE.
%
The resulting output parameters 
were also highly accurately resolved and 
extensive investigation of the different 
schemes applied to NiO allowed us to 
identify the self-consistency schemes that 
provided the best agreement with experiment, 
which were also in line with our model predictions.
%

This approach was then used to correct the SCE  of H$_2$ 
in the dissociating limit with a self-consistently calculated $J$, 
which is the first instance that this parameter has been used in this manner.
%
\edit{
We showed that not only was 
a self-consistent $J$ capable 
of quantifying and removing the SCE 
in H$_2$ beyond the Coulson-Fischer point, 
but a self-consistent scheme 
was actually required to do so.
%
This confirmed our earlier empirical observation 
that both $J$ and the SCE relate 
to the total-energy curvature with respect to magnetisation 
and indicates that a combined correction 
of SIE and SCE may be possible.
%
However, more research is required to render 
the scheme applicable for short bond-lengths.}

\edit{Our findings in this section provide 
new and interesting insights 
toward the efficient first-principles correction 
of SIE and SCE.
%
Our scheme is not only accurate and transferable, 
but it is also applicable in direct minimisation 
and SCF codes alike and, therefore, 
has far reaching potential 
to be useful across many platforms and 
for a large variety of systems.}

We then motivated our development of 
a generalised DFT+$U$ functional 
that allowed for the simultaneous treatment  
of both SIE and non-compliance with Koopman's theorem 
in Chapter~\ref{ch:non_linear_constraints}.
%
We first demonstrated that the 
automatic removal of SIE in a system 
by means of enforcing non-linear constraints 
within constrained-DFT (cDFT) to be unfeasible, 
even for a one-electron system.
%
\edit{This presented a surprising impasse 
through which we make progress by 
instead limiting the directions of methods 
based on this strategy.}

%
Nevertheless, 
we showed that a two-parameter DFT+$U$ functional 
could correct both the total-energy and eigenvalue.
%
We then derived an approximate scheme 
to calculate the $U_1$ and $U_2$ parameters 
required to restore Koopmans' compliance  
and correct SIE in a one electron system 
with a generalised DFT+$U$ functional.
%
We provided the formulae with 
the best self-consistent $U$ values 
derived in Chapter~\ref{ch:self_consistent_hubbard}, 
and proceeded to correct the total-energy 
and eigenvalue for dissociating H$_2^+$ 
to precisely the same accuracy.

\edit{
This very promising result   
prompts the development of an 
application to multi-electronic systems 
where it would be useful for instance 
in heterogeneous catalysis 
and charge transfer calculations.}
%
%Supplementing this correction with a $J$ term 
%may then possibly enable the simultaneous correction 
%of three of the most fundamental and important sources of error 
%in electronic structure calculations 
%that is efficient, accurate and affordable.}
\edit{
Indeed, 
we expect that the novel developments 
made in this dissertation will contribute greatly towards the 
high-throughput calculation of 
thermodynamical quantities~\cite{doi:10.1021/cm702327g,CapdevilaCortada201558,PhysRevB.85.155208,PhysRevB.84.045115,PhysRevB.85.115104,PhysRevB.90.115105}, 
reaction barriers~\cite{curtarolo,Curtarolo2012218,PhysRevX.5.011006}, 
charge-transfer energies~\cite{martin2004electronic,PhysRevB.88.165112,PhysRevLett.95.146402,:/content/aip/journal/jcp/126/20/10.1063/1.2743004,doi:10.1021/jp9533077}, 
and formation enthalpies.
%
These quantities are central 
to the development of solutions to many of the challenges 
we currently face in the 21$^{\textrm{st}}$ century
in sustainable energy, 
transport, 
materials science, 
and engineering.
%
We have therefore shown that there still lies much scope in the development of DFT+$U$ 
and its utility in electronic structure methods has yet untapped potential   
to provide more valuable and exciting contributions toward these endeavours.}



%FUTURE WORK
\section{Future work}

Let us now briefly discuss 
some of the possible future directions of research 
based on the works presented in this dissertation.


%DFT+U1+U2 and Koopmans

It is clear that Koopmans' compliance 
remains a vexing challenge 
for the design of self-contained methods, 
even for very sophisticated 
exchange-correlation functionals 
and correction schemes.
%
On the basis of our results in Chapter~\ref{ch:non_linear_constraints}, 
a generalised functional, 
such as our DFT+$U_1$+$U_2$, 
presents an intriguing and very promising
avenue of exploration for future methods.
%

If the proposed scheme described earlier 
is applied to supplement an 
existing DFT+$U$ calculation 
that has already accurately 
recovered the total-energy, 
then it is sufficient to set $U = 0$~eV 
in our approximate formulae of 
Eq.~\eqref{eq:U1U2}.
%
This is because the SIE interaction has 
already been treated by an appropriate $+U_0$ correction.
%
An approximately Koopmans-compliant  
DFT+$U$ calculation may then be 
performed by a simple change of parameters 
%
\begin{equation}
(U_1,N_\textrm{DFT}U_2) \to \left(U_K,U_K\right), 
\end{equation}
%
which only requires knowledge of 
the ionised state $E[N-1]$.

The scheme may also be generalised to 
multi-orbital subspaces straightforwardly, 
by replacing $N_\textrm{DFT}$ 
with a sum over the eigenvalues of 
$\hat{n}^I_\textrm{DFT}$, 
given as follows 
%
\begin{equation}
N_\textrm{DFT} \to \sum_{m}\left[\hat{n}^I_{mm'}\right]
\quad\mbox{and}\quad
N_\textrm{DFT}^2 \to \sum_{m}\left[\hat{n}^I_{mm'}\hat{n}^I_{m'm}\right], 
\end{equation}
%
which ignores intra-subspace interactions 
in the spirit of DFT+$U$.

Finally, the approximation of constant $N_\textrm{DFT}$ 
may be replaced by a linear-response approximation, 
in terms of the response function $\chi$, 
%
\begin{equation}
N_\textrm{DFT}\to \chi \textrm{Tr}[\hat{v}_\textrm{ext}\hat{P}]
\end{equation}
%
or lifted entirely by means of
a  parametrisation of the occupancies in terms of 
$\textrm{Tr}\left[ \hat{v}_{U_1 U_2} \right] $ 
and a numerical solution of 
the resulting equations.
%
In this work, we have opened the door to 
the possibility of designing other generalised functionals, 
which may be motivated by the need for other specialised purposes 
that are beyond the scope of the traditional DFT+$U$ functional.
%
The possibilities for such will only become apparent with time, 
and will hopefully illuminate new paths in future research.

%Indeed, 
%we demonstrated in Chapter~\ref{ch:non_linear_constraints} 
%that the conventional rotationally invariant DFT+$U$ functional 
%is not equipped to enforce compliance with Koopmans' condition, 
%even when provided with an adequate 
%self-consistently calculated $U$.
%
%%The well-motivated, but potentially provocative, formulation of 
%A generalised DFT+$U$ functional, 
%%comprising Hubbard parameters $U_1$ and $U_2$, 
%therefore presents an intriguing possibility 
%to treat Koopmans' compliance as readily 
%as the band gap from within the framework of  
%well-established DFT+$U$ methods.


%INVESTIGATION OF JIN VS JOUT
On the basis of the modest success of 
our self-consistency schemes for $U$, 
we are obliged to investigate more thoroughly 
the extension to the self-consistent $J$.
%
A puzzling outcome was the 
facile agreement of the PBE+$J^{(2)}$ scheme 
with the H$_2$ total-energy at dissociation, 
where the expected $J_\textrm{eff}$ scheme 
was clearly inappropriate.
%
We are hence motivated to investigate 
the difference in the self-consistency calculation schemes 
presented for the $U$ and $J$ terms, 
in correcting the SIE and SCE, respectively.

Furthermore,  
the analysis of the 
$U_\textrm{in}$ vs $U_\textrm{out}$ 
and related procedures, 
raises many questions about the 
nature of the full 4-dimensional space described by  
$U_\textrm{in}$, $U_\textrm{out}$, 
$J_\textrm{in}$, and $J_\textrm{out}$.
%
We may ask, 
what are the self-consistency schemes 
arising from the intersections of each curve with the axes, 
in particular the location of the point where 
$U_\textrm{eff}=J_\textrm{eff}=0$.


%U SPIN POLARISED
The calculation of $U$ heretofore, 
has only considered uniform  
perturbations to both spins in a subspace.
%
An interesting generalisation of this, 
which is currently being pursued by our colleague, 
Edward Linscott, 
is the calculation of a Hubbard $U$ matrix 
indexed by spin instead of site.
%separate perturbation of each spin 
%while the potential of the other is kept fixed.
%
%In this method, 
%the Dyson equation defined in Eq.~\eqref{eq:dyson_equation2} 
%generalises to an operation of spin-indexed response matrices 
%(similar to the treatment of inter-site responses~\cite{PhysRevB.71.035105,0953-8984-22-5-055602}), 
%which result in a spin-indexed $U$ matrix. 
\begin{equation}
U^{\sigma\bar\sigma}=
\left(\begin{array}{cc}
F^{\uparrow\uparrow} & F^{\uparrow\downarrow} \\
F^{\downarrow\uparrow} & F^{\downarrow\downarrow}
\end{array}
\right)
\end{equation}
%
The advantage here is one could compute 
each (screened) spin interaction precisely, 
%namely, $F^{\uparrow\uparrow}$,
%$F^{\downarrow\downarrow}$,
%$F^{\downarrow\uparrow}$,
%$F^{\uparrow\downarrow}$, 
and no longer have to rely on averages.
%
An encouraging preliminary test 
shows that the method conveniently reduces 
to the average on-site $U$ value, 
at least for closed-shell systems, 
when averaged over all interactions in the $U$ matrix.


%UNIFIED SITES 
Another modification to DFT+$U$ 
that we discussed in section~\ref{sec:subspace_projectors},  
pertains to the definition of the subspace projectors.
%%
%In its current format, 
%the DFT+$U$ functional operates separately 
%on the individual Hubbard sites $I$, 
%defined by the projectors $\hat{P}^I$, 
%with a unique Hubbard $U^I$ parameter for each.
%
Our proposal to circumvent the inaccurate population 
analysis that arises in cases of strong subspace overlap 
is to redefine the projectors to span 
the subspaces that belong to a 
specific group $\mathcal{G}$, 
characterised by a certain properties 
in a so-called unified sites (US) approach.
%
This proposed scheme is reminiscent 
of the DFT+$U$+$V$ functional in Ref.~\cite{0953-8984-22-5-055602},  
which invokes an inter-site $V$ term, 
and is expected to be advantageous 
under conditions of strong subspace hybridisation, 
for example, 
in the near-bonding regimes for dissociating dimers.
%where the subspace-bare and bath-screened 
%on-site interactions computed for individual subspaces 
%are not a true reflection of the physical reality.
% 
Another possible application is 
in VO$_2$ where the $U$ parameter applied 
to the individual subspaces of each V atom 
is known to erroneously cause them to dimerise, 
and ultimately fails to produce a gap~\cite{PhysRevLett.108.256402}.
%
In the US approach, 
we would compute and apply a $U$ parameter 
that is bare to the {\it combined} V atom subspace 
with the intention to avoid the spurious dimerisation.
%
Our colleague, Evan Sheridan, 
is currently working on this approach 
and initial testing on VO$_2$ has been very promising.

%Moreover, 
%in the limit where the subspaces are actually well localised, 
%this scheme simply reduces to the conventional 
%separate treatment of each localised site $I$, 
%albeit with the same $U$ parameter.
%%
%On that note, 
%this procedure may also be a remedy 
%for correcting SIE in the bonding regime, 
%where the single-site approximation breaks down. 

%DFT+U ON 2D MATERIALS
Finally, 
building upon our exercise in two-dimensional materials, 
we propose some candidates 
that may benefit from a DFT+$U$ treatment 
in conjunction with a self-consistent $U$. 
%
Some nascent materials in this class  
include transition metal diborides~\cite{ANIE:ANIE201207972,C5NR04359J}
such as titanium diboride (TiB$_2$)~\cite{PhysRevB.90.161402,doi:10.1021/acs.chemmater.7b01433,PhysRevB.93.035401}, 
and magnesium diboride (MgB$_2$)~\cite{PhysRevB.80.134113,doi:10.1063/1.4876129,doi:10.1021/acs.inorgchem.6b01685},
as well as vanadium oxide (V$_2$O$_5$)~\cite{PhysRevB.91.125116,machej1991monolayer,PhysRevB.71.165437,PhysRevB.89.045109}, 
for which current literature does 
not feature DFT+$U$ calculations 
on the few-layer structures.




%\begin{align}
%\hat{U}&=\hat{\chi_0}^{-1}-\hat{\chi}^{-1}\\
%&=\left(\hat{\epsilon}\hat{\chi}\right)^{-1}-\hat{\chi}^{-1}\\
%&=\hat{\chi}^{-1}{\epsilon}^{-1}-\left(\hat{\chi}^{-1}\right)\\
%&=\hat{\chi}^{-1}\left({\epsilon}^{-1}-1\right)\\
%&=\left(\frac{\delta\hat n^{\sigma}}{\delta\hat v^{\bar\sigma}_{\text{ext}}}\right)^{-1}\left({\left(\frac{\delta\hat v_{\text{KS}}^{\sigma}}{\delta\hat v^{\bar\sigma}_{\text{ext}}}\right)}^{-1}-1\right)
%\end{align}

%\begin{align*}
%\hat{U}^{\sigma\bar\sigma}&=\left(\hat{\chi_0}^{-1}\right)^{\sigma\bar\sigma}-\left(\hat{\chi}^{-1}\right)^{\sigma\bar\sigma}
%\\
%	&=\left(\frac{d n^{\sigma}}{d V_{\text{KS}}^{\bar\sigma}}\right)^{-1} 
%	-\left(\frac{dn^{\sigma}}{dV_{\text{ext}}^{\bar\sigma}}\right)^{-1}
%\\
%	&=\left(\frac{d n^{\sigma}}{d V_{\text{ext}}^{\bar\sigma}}\frac{d V_{\text{ext}}^{\bar\sigma}}{d V_{\text{KS}}^{\bar\sigma}}+\frac{d n^{\sigma}}{d V_{\text{ext}}^{\sigma}}\frac{d V_{\text{ext}}^{\sigma}}{d V_{\text{KS}}^{\bar\sigma}}\right)^{-1} 
%	- \left(\frac{dn^{\sigma}}{dV_{\text{ext}}^{\bar\sigma}}\right)^{-1}\\
%\end{align*}




