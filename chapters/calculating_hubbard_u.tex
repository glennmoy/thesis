
\lett{T}{he effectiveness of methods} intended to 
correct inherent inaccuracies in approximate DFT, 
either by DFT+$U$, 
constrained-DFT~\cite{PhysRevLett.53.2512,Sit2007107,doi10.1021/cr200148b} 
or hybrid functionals~\cite{doi:10.1063/1.464304,PhysRevB.94.035140} etc., 
depends largely on the accurate calculation 
of parameters controlling the strength 
of the corrective potentials.
%
%Automated methods, 
%such as those employed by cDFT, 
%{optimally-tuned range-separated hybrids, 
%or Koopmans compliant functionals, 
%can readily determine these parameters 
%by optimisation algorithms}.
%%
%However, 
%the inability to extend this formalism 
%to the DFT+$U$ functionals targeting SIE, 
%{as we demonstrated to be the case in} the previous Chapter, 
%means that users must determine 
%these parameters for themselves  
%in advance of performing the primary calculation.
%
In this Chapter, 
we will describe 
the process of computing the Hubbard $U$ and $J$ 
parameters from first principles in {\sc ONETEP}, 
using an adaptation of 
the acclaimed linear-response (LR) approach 
of Cococcioni and de Gironcoli~\cite{PhysRevB.71.035105,PhysRevB.84.115108}.

To date, there exists in the literature 
a variety of methods for determining 
the Hubbard $U$ parameter 
in strongly-correlated systems.
%
Originally, 
the Hubbard $U$ was determined via 
semi-empirical fitting to experimental data, 
which was
typically spectral~\cite{HERZBERG1972425,PhysRevB.70.125426,:/content/aip/journal/jcp/127/24/10.1063/1.2800015,Morgan20075034}, 
structural~\cite{PhysRevB.70.125426,PhysRevB.75.035115,:/content/aip/journal/jcp/127/24/10.1063/1.2800015}, 
or energetic~\cite{PhysRevB.73.195107,C1EE01782A}, 
and is a practice that continues in many studies to this day.

This method, 
while reasonably motivated, 
is rather unfavourable in practice 
as it interprets the Hubbard $U$ parameter 
as an adjustable variable 
and not as an intrinsic property of the system.
%
{
Moreover, 
tuning the $U$ parameter to reproduce 
the electronic or optical gaps exactly is precarious, 
since the KS eigenvalues are not intended for this purpose 
and there is no guarantee that the SIE 
will have been adequately resolved.}
%
This approach becomes particularly unviable when 
the experimental data is insufficient  
or difficult to measure.
%
If instead the Hubbard $U$ was calculated as a 
variational ground-state density-functional property, even implicitly, 
then DFT+$U$ becomes elevated 
to the status of a fully variational first-principles method. 
%

%cLDA
Early attempts to compute an \emph{ab initio} $U$ 
for use in the Anderson-type Hubbard functional 
in Eq.~\eqref{eq:hubbard1}, 
began with taking the difference in the energy eigenvalues of ionic states, 
for example~\cite{PhysRevB.43.7570} 
%
\begin{equation}
U\approx\varepsilon_{3d\uparrow}\left(\frac{N+1}{2};\frac{N}{2}\right)-\varepsilon_{3d\uparrow}\left(\frac{N+1}{2};\frac{N}{2}-1\right).
\end{equation}
%
From this perspective, 
the $U$ parameter is determined 
via a constrained-LDA (cLDA) 
calculation~\cite{PhysRevB.38.6650, PhysRevB.39.1708, PhysRevB.39.9028, PhysRevB.41.514, PhysRevB.43.7570, PhysRevB.44.943, PhysRevB.49.6736, PhysRevLett.53.2512, PhysRevB.71.045103}, 
where the correlated subspace occupancy is constrained 
while artificially decoupled from the surrounding atom.
%
This method remained popular 
for estimating Coulomb parameters for many years
but tended to drastically over-estimate the $U$ values 
because it neglected the screening effects
of the surrounding environment.

%RPA
The random-phase approximation 
(RPA)~\cite{PhysRevB.57.4364,PhysRevB.70.195104,PhysRevB.71.045103,PhysRev.85.338, PhysRev.92.626,0953-8984-12-11-307,PhysRevB.93.035133}, 
%formulated within the GW approximation~\cite{PhysRev.139.A796}, 
was later used and 
circumvented this by allowing for 
responses in both on-site and off-site electrons 
as well as permitting a frequency-dependent $U(\omega)$ 
to be calculated.
%
%cRPA
This later evolved into the 
{more accurate} constrained-RPA (cRPA) 
approach~\cite{PhysRevB.74.125106, PhysRevB.77.085122, PhysRevB.80.155134, PhysRevB.83.121101}, 
which excluded on-site transitions from 
contributing to the $U$.
%

Other methods for calculating $U$ include
the local density supercell impurity 
approach~\cite{PhysRevB.33.8896}
and the calculation of on-site Coulomb matrix elements 
from maximally-localised Wannier 
functions~\cite{PhysRevB.56.12847,PhysRevB.65.075103}.
%however these, and the preceding strategies, 
%will not be discussed in detail any further. 

	
%THE LINEAR-RESPONSE APPROACH
\section{Overview of the linear-response method}
\label{sec:linear_response_approach}

The linear-response (LR) approach to calculating $U$ 
was initially conceived by Pickett {\it et al}~\cite{PhysRevB.58.1201}, 
and later developed by Coccocioni and de Gironcoli~\cite{PhysRevB.71.035105} 
(who also extended the method the exchange parameter $J$~\cite{PhysRevB.84.115108}), 
as an inexpensive method to be 
implemented within the rotationally invariant 
DFT+$U$ functional given in 
Eqs.~\eqref{eq:dft+u_functional}~\&~\eqref{eq:dft+u+j_functional}.
%
In this formulation, 
a uniform external potential shift 
$\hat{V}_\textrm{ext} = \alpha \hat{P}$
is applied to the electrons in the localised subspaces 
and the $U$ parameter is determined 
from the corresponding response functions, 
including the effects of screening.


As noted by the authors in Ref.~\cite{PhysRevB.71.035105}, 
the SIE in strongly-correlated systems, 
representative of an open quantum system
 that does not interact with its bath for particle exchange, 
is illustrated by the spurious curvature 
in the total-energy  
with respect to subspace occupancy, 
which we discussed at length in Chapter~\ref{ch:self_interaction_error}.
%
Consequently, 
their interpretation of the Hubbard $U$ parameter 
is that of the measure of unphysical curvature exhibited in a system, 
which is in keeping with its implementation 
in Eq.~\eqref{eq:dft+u_functional}, 
given by (suppressing site and spin index)
%
\begin{equation}
\frac{d^2E_U}{dn^2}=-U.
\end{equation}
%
The authors in Ref.~\cite{PhysRevB.71.035105} 
make an important distinction 
between the {\it interacting} and {\it non-interacting} curvatures 
present in a system.
%
{
The latter is a superfluous feature 
that has nothing to do with the interacting curvature, 
and so must be removed  
to enable the reliable calculation of  $U$ values 
for both solids and molecules.}
%The latter is due to the hybridisation 
%of {\it non-interacting} KS orbitals in extented solids, 
%which produce physically meaningful fractional occupancies 
%in the subspace, 
%and should therefore not contribute 
%the calculation of $U$.
%
{
This observation was alluded to, 
but initially absent, 
in the original formulation~\cite{PhysRevB.58.1201}.}

Inspired by the cLDA approach, 
Cococcioni and de Gironcoli 
postulated that the spurious curvature 
can be accessed by cDFT calculations 
on the local subspace responsible for SIE, 
in accordance with target occupancies $\{q_I\}$ 
and Lagrange multipliers $\{\alpha_I\}$
%
\begin{equation}
E[\{q_I\}]=\min_{n({\br},\alpha_I)}\left\{E[n({\br})]+\sum_I\alpha_I(n_I-q_I)\right\}
\label{eq:coc_linear_response1}.
\end{equation}
%
In practice, however, 
it is more feasible to re-parameterise 
Eq.~\eqref{eq:coc_linear_response1} 
in terms of the Lagrange multipliers $\{\alpha_I\}$, 
and avoid performing cDFT calculations altogether 
% 
\begin{equation}
E[\{\alpha_I\}]=\min_{n({\br})}\left\{E[n({\br})]+\sum_I\alpha_I(n_I-q_I)\right\}.
\label{eq:coc_linear_response2}
\end{equation}
%
The second derivative of Eq.~\eqref{eq:coc_linear_response2} 
thus corresponds to a value of $U$ that is consistent with 
its definition as the curvature 
of the of total-energy with respect to occupation 
%
\begin{equation}
\frac{d^2E[\{\alpha_I\}]}{d q_Jd q_I}
=-\frac{d\alpha_I}{d q_J} = -\chi_{IJ}^{-1},
\label{eq:coc_linear_response3}
\end{equation}
%
where we have introduced the 
{\it interacting} response function $\chi$.
%
However, the expression in Eq.~\eqref{eq:coc_linear_response3} 
is inclusive of the curvature associated with 
the re-hybridisation of the KS orbitals 
in response to the applied potential.
%
We must therefore subtract this curvature, 
given in terms of the 
{\it non-interacting} response $\chi_0$
corresponding to the KS system $E_0$,
to ensure a $U$ value that 
accurately reflects the magnitude  
of on-site Coulomb {and XC} interaction.
%
Hence,  $U$ is evaluated 
according to the following Dyson equation 
%
\begin{equation}
U=+\frac{d^2E_0[\{q_I\}]}{d q_I^2}-\frac{d^2E[\{q_I\}]}{d q_I^2}
=(\chi_0^{-1}-\chi^{-1})_{II}.
\label{eq:dyson_equation1}
\end{equation}
%

The non-interacting response 
present in Eq.~\eqref{eq:dyson_equation1} 
is a, {hitherto}, ambiguous, yet crucial, quantity 
in determining the calculation of $U$.
% 
In Ref.~\cite{PhysRevB.71.035105}, 
it is interpreted as the reorganisation of charge 
subject to the potential shift on the subspace 
without allowing the self-consistent screening to take effect.
%
In practice then, 
it is then acquired after the first iteration in 
the self-consistent calculation; 
subsequent to the initial response of the density, 
but prior to the Hamiltonian being updated.


The LR method just described 
has become a widely popular approach~\cite{QUA:QUA24521,:/content/aip/journal/jcp/133/11/10.1063/1.3489110,0953-8984-22-5-055602,PhysRevB.75.035115,Lee2012,PhysRevLett.106.118501}
to calculating Coulomb interaction parameters 
as it allows the parameters to be 
evaluated from a small set of inexpensive DFT calculations.
%
The mathematical elegance 
and practical simplicity of the LR approach 
therefore presents an ideal foundation 
on which to construct the 
evaluation of $U$ parameters in 
a direct-minimisation code such as {\sc ONETEP}.
%
However, 
as we will discuss in the following section, 
this type of formalism can not be directly accommodated 
in codes that utilise direct-minimisation 
as a means to locate the KS ground-state, 
and an appropriate adaptation 
to the LR method must be made 
to generalise its application.


%linear-response IN ONETEP
\section{Construction of a variational linear-response approach}
\label{sec:variational_linear_response}

The most problematic aspect of extending the LR approach 
to\break direct-minimisation codes lies in the calculation of 
the non-interacting response function $\chi_0$.
%
Given that codes like these do not update 
the density and Hamiltonian iteratively\footnote{
That is, within nested loops in which  
the density is optimised while the Hamiltonian is kept fixed, 
followed by the optimisation of the Hamiltonian where the density is fixed.}, 
but do so simultaneously throughout the calculation, 
it is inefficient and extraneous 
to calculate $\chi_0$ 
in the fashion described above.
%
We are therefore motivated to seek 
a linear-response formalism for the 
Hubbard $U$ that is readily compatible with 
direct-minimisation DFT solvers, 
particularly for linear-scaling DFT+$U$~\cite{PhysRevB.85.085107}.
%
The principal challenge lies in the 
evaluation of the non-interacting response, 
the derivation of 
which we shall now outline.

%THE NON-INTERACTING RESPONSE
\subsection{The non-interacting response}

If we perturb the subspace 
of a given system by an external potential 
$\delta V_\textrm{ext}({\br})$, 
the resulting variation in the density is given, 
to first order, in terms of the response function 
$\chi(\br,\br')$, such that 
%
\begin{align}
&\delta n({\br})=\int \chi({\br}, {\br'}) \delta V_\textrm{ext}({\br'})\, d{\br'} \quad\mbox{where}\quad
\chi({\br}, {\br'})=\frac{\delta n(\br)}{\delta V_\textrm{ext}(\br')}.
\end{align}
%
For the same perturbation, 
we can define a second response function 
$\chi_\textrm{KS}({\br}, {\br'})$
in terms of the change in the 
KS potential $V_\textrm{KS}(\br)$
%
\begin{align}
&\delta n({\br})=\int \chi_\textrm{KS}({\br}, {\br'}) \delta V_\textrm{KS}({\br'})\, d{\br'}\quad\mbox{where}\quad
\chi_\textrm{KS}({\br}, {\br'})=\frac{\delta n(\br)}{\delta V_\textrm{KS}(\br')}.
\label{eq:non_interacting_response1}
\end{align}


Following the discussion in Chapter~\ref{ch:qm_simulation}, 
the change in KS potential may be 
conveniently expressed, to first order, as the sum 
of changes in its interacting constituents, namely the 
Hartree,  XC, and the external potential  
$\delta V_\textrm{KS} = \delta V_\textrm{Hxc} + \delta V_\textrm{ext}$.
%
Moreover, any perturbation to the potential 
may be expressed in terms of the interaction kernel, 
given by 
%
\begin{align}
&F_\textrm{Hxc}({\br'},{\br''})=\frac{\delta V_\textrm{Hxc}({\br'})}{\delta n({\br''})}, 
\nonumber \\[0.5em]
\mbox{such that}\quad &
\delta V_\textrm{Hxc}(\br')=\int F_\textrm{Hxc}({\br'},{\br''})\delta n({\br''})\, d{\br''}.
\label{eq:non_interacting_response2}
\end{align}

Combing Eqs.~\eqref{eq:non_interacting_response2}~\&~\eqref{eq:non_interacting_response1} 
then yields the variation in the density 
in terms of KS quantities 
%
\begin{align}
\delta n({\br})
&=\int \chi_\textrm{KS}({\br}, {\br'}) \left[\delta V_\textrm{Hxc}({\br'})+\delta V_\textrm{ext}({\br'})\right]\, d{\br'} \\[0.5em]
&=\iint \chi_\textrm{KS}({\br}, {\br'''}) F_\textrm{Hxc}({\br'''},{\br''})\delta n({\br''})\, d{\br''} 
\, d{\br'''}\
+\int \chi_\textrm{KS}({\br}, {\br'}) \delta V_\textrm{ext}({\br'})\, d{\br'}, \nonumber
\end{align}
%
where in the double integral 
we have relabeled the dummy variables 
$\br'\to \br'''$ 
for later convenience.
%
We can now equate this expression to 
Eq.~\eqref{eq:non_interacting_response1}, 
where the variation in the density is given 
in terms of the interacting quantities, 
and substitute the same relation for $\delta n({\br''})$  
in the double integral.
%
We then remove the integral over $d\br'$ 
and divide across by $\delta V_\textrm{ext}(\br')$ 
for brevity, thereby giving 
%
\begin{equation}
\chi({\br}, {\br'})= \chi_\textrm{KS}({\br}, {\br'}) +\iint \chi_\textrm{KS}({\br}, {\br'''}) F_\textrm{Hxc}({\br'''},{\br''})\chi({\br''}, {\br'})\,d{\br''} \, d{\br'''} 
\end{equation}
%
which relates 
$\chi$, $\chi_\textrm{KS}$ and $F_\textrm{Hxc}$ 
by a Dyson-like equation where 
we identify the non-interacting response
function as none other than the KS response
%
\begin{equation}
\chi_0(\br,\br)\equiv\frac{\delta n(\br)}{\delta V_\textrm{KS}(\br)}.
\end{equation}
%

This interpretation of $\chi_0$, 
in terms of the KS response to the perturbation, 
can now be readily computed 
in a variational procedure 
as easily as the interacting response $\chi$. 
%
Let us now proceed to constructing a definition of $U$ 
in accordance with the desired properties, 
which will require an appropriately projected interaction kernel. 
%



%KERNEL SELECTION
\subsection{Choice of projection for the interacting kernel}

In this work, 
our definition of $U$ for a given subspace 
is derived from  its original interpretation, given in 
Eq.~\eqref{eq:interaction_parameters}, 
i.e., that of the {net on-site Coulomb and XC} interactions 
acting within it.
%
More specifically, 
we seek terms coupling to the 
subspace density matrices in Eq.~\eqref{eq:dft+u_functional} 
up to order $( n_i^{I} )^2$, 
and therefore only consider 
an interacting kernel like
$\hat{f}_\textrm{int} = \delta^2 E_\textrm{int} / \delta \hat{\rho}^2$, 
and not terms related to 
$\hat{g}_\textrm{int} = \delta^3 E_\textrm{int} / \delta \hat{\rho}^3$, 
or higher.
%
Here, 
$E_\textrm{int}$ comprises comprises 
the Hartree, XC, and any other electronic interaction terms.

Crucially, 
we require only the interactions  
that arise due to density variations within a subspace 
to contribute to  
$\hat{f}_\textrm{int}$, 
and so an appropriate projection must be chosen.
%
Finally, our formulation of $U$ must comply with  
the convention of correcting the many-body SIE  
by subtracting the individual one-electron SIE  
of each eigenstate of the subspace density-matrix. 
%
It should therefore not scale extensively with 
the subspace eigenvalue count $\mathrm{Tr} [ \hat{P} ]$ 
and must be appropriately averaged.
%
In order to illustrate the above requirements, 
let us consider some candidate formulae for Hubbard 
$U$ parameters which do not meet them.


%TOTAL DERIVATIVES
\subsubsection{Simple curvature from total derivatives of the energy}

Immediately, we may rule out 
the formulation 
proposed by Pickett and co-workers~\cite{PhysRevB.58.1201}, 
in which the $U$ parameter is calculated 
from the curvature of the total-energy 
with respect to total occupancy\footnote{$\mathcal{U}$ 
is used to distinguish incorrect formulae 
from the correct one, which are denoted $U$.} 
%
\begin{align}
\mathcal{U}=\frac{d^2E}{dN^2}=-\frac{d\alpha}{dN}=-\chi 
\quad\mbox{where}\quad
N=\textrm{Tr}\left[\hat{n}\right].
\end{align}
% 
As discussed  in Ref.~\cite{PhysRevB.71.035105}, 
this term fails to account for the non-interacting 
contribution arising from hybridisation of charge 
and is thus an inappropriate starting point.

Consider instead the curvature of 
just the interaction energy   
%
\begin{equation}
\mathcal{U}=\frac{d^2 E_\textrm{int}}{ d  N^{2}}
\quad\mbox{where}\quad
E_\textrm{int}=E_\textrm{Hxc}\equiv E_H+E_\textrm{xc}.
\end{equation}
%
However, the reasons for its 
non-suitability are quite subtle, 
which relate to the fact that the Hellman-Feynman theorem 
cannot be applied to $E_\textrm{int}$ alone.
%
Therefore, the first total derivative of $E_\textrm{int}$ 
with respect to $N$ is given by 
%
\begin{equation}
\frac{dE_\textrm{int}}{dN} = \frac{\partial E_\textrm{int}}{\partial N} 
+ \textrm{Tr}\left[ \frac{\delta E_\textrm{int}}{\delta \hat{\rho}} \frac{d \hat{\rho}}{ d N} \right]
= \textrm{Tr}\left[ \hat{V}_\textrm{int}\frac{d \hat{\rho}}{ d N} \right],
\end{equation}
%
yielding a vanishing partial derivative 
(due to no explicit $N$-dependence in $E_\textrm{int}$), 
but also a term proportional to 
the interaction potential 
$\hat{V}_\textrm{int}$.
%
This term is screened with respect to the bath, 
since 
$d \hat{\rho} / d N$ 
is coupled to the external potential
$\hat{V}_\textrm{ext} = \alpha \hat{P}$.
%
The second total derivative
%
\begin{equation}
\frac{d^2 E_\textrm{int}}{dN^2}=\textrm{Tr}\left[\frac{d\hat{V}_\textrm{int}}{dN}\frac{d \hat{\rho}}{d N}+\hat{V}_\textrm{int}\frac{d^2 \hat{\rho}}{d N^2}\right]
\end{equation}
%
incorporates screening again, 
and the resulting twice-screened objects 
have no physical interpretation 
and so this formulation is also omitted.

%FUNCTIONAL DERIVATIVES
\subsubsection{Curvature from functional derivatives of the energy} 

One may then suggest taking the curvature of the 
interaction kernel projected onto the subspace instead, 
which is denoted by 
%
\begin{equation}
\mathcal{U} = \hat{P}\frac{\delta^2 E_\textrm{int}}{\delta \hat{n}^{ 2}} \hat{P}  = \hat{P} \hat{f}_\textrm{int} \hat{P}.
\end{equation}
%
However, 
this interaction is now bare of potentially 
substantial screening of density-matrix variations 
outside the subspace, 
and also makes an unsuitable 
starting point for measuring of many-body SIE.
%


This problem is the opposite, in a sense, to that of
$ d^2E_\textrm{int}/dN^2$, 
from which one may surmise the 
correct definition is an intermediate case, 
where screening effects due to the complement 
of the subspace at hand should be incorporated, but only once.
%
In other words, we require that $U$ 
should be screened with respect to the bath, 
yet remain bare within the subspace.
%
This motivates us to work not from the interacting energy,  
but  from the interacting potential, i.e., 
to start from the unscreened functional derivative
of the interacting energy, 
and to then differentiate that object by $N$.


% TOTAL DERIVATIVE OF UNSCREEND POTENTIAL
\subsubsection{Total derivative of the unscreened potential}

Since the functional derivative 
is defined as a generalised partial derivative, 
the interaction term in the KS potential 
%
\begin{equation}
\hat{V}_\textrm{int} = \frac{\delta E_\textrm{int}}{ \delta \hat{\rho}}
\end{equation}
%
 is bare of screening, 
 as is its subspace projection 
%
\begin{equation}
\hat{P} \hat{V}_\textrm{int} \hat{P} 
= \hat{P}\left( \frac{ \delta E_\textrm{int}}{ \delta \hat{n} }\right) \hat{P}.
\end{equation}
%
The object is then described by 
the average total derivative of 
the interacting potential, 
given by
%
\begin{equation}
\mathcal{U}=\textrm{Tr} \left[ \hat{P} \frac{d \hat{V}_\textrm{int}}{ d \hat{n}} \hat{P}\right] /
\textrm{Tr} \left[  \hat{P} \right]^{2}, 
\end{equation}
%
seems to fulfil many of the requirements 
of a valid Hubbard parameter, 
in that it is once-screened, 
non-extensive and subspace-averaged.

In practice, however, the screened kernel 
$ d \hat{V}_\textrm{int} / d \hat{n} $ 
is arduous to calculate,
even in orbital-free density-functional theory.
%
More importantly, 
it includes screening effects due to 
density-matrix fluctuations within the subspace, 
which makes it unsuitable for quantifying 
the bare subspace interaction we 
intend to correct with DFT+$U$.
%

This is easily remedied by instead 
taking the derivative with respect 
to the total occupancy $N$, 
and finally defining 
%
\begin{align}
U \equiv \frac{d V_\textrm{int} }{d  N},
 \quad \mbox{where}\quad
V_\textrm{int} \equiv \frac{ \textrm{Tr} [\hat{V}_\textrm{int} \hat{P}]}{ \textrm{Tr} [  \hat{P} ] }
\label{eq:u_definition}
\end{align} 
%
as the conveniently calculated, non-extensive,
subspace-averaged interaction potential.
%
To illustrate this interpretation, 
it is instructive to represent $U$ in terms of  
{approximations to the scalar {\it subspace-screened}} 
parallel and anti-parallel interactions, denoted 
$F^{\sigma\sigma}$ and $F^{\sigma\bar{\sigma}}$, 
respectively\footnote{{The distinction is made 
to avoid confusion with the subspace-bare interaction kernels 
denoted by $F^{\sigma\sigma'}(\br,\br')=\delta V^\sigma(\br)/\delta n^{\sigma'}(\br')$.}}.
%
Let us expand Eq.~\eqref{eq:u_definition} 
in terms of the {total} derivatives 
over the spin-dependent densities  
%
\begin{align}
U\equiv\frac{dV_\textrm{int}}{dN}&
%=\frac{dV_\textrm{int}}{dN}
=\frac{1}{2}\left(\frac{dV_\textrm{int}^\uparrow}{dN}+\frac{dV_\textrm{int}^\downarrow}{dN}\right)\nonumber \\[0.75em]
&=\frac{1}{2}\left(\frac{d V_\textrm{int}^\uparrow}{d n^\uparrow}\frac{dn^\uparrow}{dN}
+\frac{d V_\textrm{int}^\uparrow}{d n^\downarrow}\frac{dn^\downarrow}{dN}+
\frac{d V_\textrm{int}^\downarrow}{d n^\downarrow}\frac{dn^\downarrow}{dN}+
\frac{d V_\textrm{int}^\downarrow}{d n^\uparrow}\frac{dn^\uparrow}{dN}\right)\nonumber \\[0.75em]
&=\frac{1}{2}\left(F_\textrm{int}^{\uparrow\uparrow}\frac{dn^\uparrow}{dN}
+F_\textrm{int}^{\uparrow\downarrow}\frac{dn^\downarrow}{dN}+
F_\textrm{int}^{\downarrow\downarrow}\frac{dn^\downarrow}{dN}+
F_\textrm{int}^{\downarrow\uparrow}\frac{dn^\uparrow}{dN}\right)\nonumber \\[0.75em]
\Rightarrow F^{\hat{P}}_\textrm{Hxc}
&=\frac{1}{4}\left(F_\textrm{int}^{\uparrow\uparrow}+F_\textrm{int}^{\uparrow\downarrow}
+F_\textrm{int}^{\downarrow\downarrow}+F_\textrm{int}^{\downarrow\uparrow}\right). 
\label{eq:u_interaction}
\end{align}
%
In the last line we have defined the 
interaction kernel $F^{\hat{P}}_\textrm{Hxc}$ 
as the approximation to the screened, 
net, average electronic interaction 
acting within the subspace 
due to {\it all} spin interactions, 
using the expressions
%
\begin{align}
n^\uparrow=\frac{1}{2}(N+M)
\quad&\mbox{and}\quad
n^\downarrow=\frac{1}{2}(N-M)
\quad \Rightarrow\quad
\frac{dn^\uparrow}{dN}&=\frac{dn^\downarrow}{dN}=\frac{1}{2}.
\end{align}
%
Thus, 
$U$ may be similarly defined as the 
net, average electronic interaction 
acting within the subspace 
due to {\it all} spin interactions.
%
Finally, 
since the uniform potential $\alpha$ 
used in the linear-response method 
induces no density-variations 
within the subspace to first-order
except for the uniform shift, 
the screening within the subspaces is effectively suppressed, 
similar to the  cRPA 
approach~\cite{PhysRevB.74.125106,PhysRevB.77.085122,PhysRevB.80.155134}.


{In practice,
the external potential is applied by 
a uniform shift of magnitude $\alpha$ 
to all electrons in the subspace, 
such that 
%by the idempotency of $\hat{P}$, 
%
\begin{equation}
d V_\textrm{ext}=\frac{\textrm{Tr}[d \hat V_\textrm{ext} \hat P]}{\textrm{Tr}[\hat P]} 
=\frac{\textrm{Tr}[ d \alpha \hat P \hat P]}{\textrm{Tr}[\hat P]} = d \alpha.
\end{equation}}
%
Therefore, 
for a single-site model 
incorporating a scalar $U$, 
Eq.~\eqref{eq:u_definition}
 may be derived using the Dyson equation in 
Eq.~\eqref{eq:dyson_equation1} 
 as follows
%
\begin{align}
U\equiv \chi_0^{-1}-\chi^{-1}
=\left(\frac{dN}{dV_\textrm{KS}}\right)^{-1}-\left(\frac{dN}{d\alpha}\right)^{-1} 
= \frac{dV_\textrm{Hxc}}{dN}
=F_\textrm{Hxc}, 
\label{eq:dyson_equation2}
\end{align}
%
using the interacting and non-interacting response functions
%
\begin{align}
&\chi=\frac{d\alpha}{dN}
\quad\mbox{and}\quad
\chi_0=\frac{dV_\textrm{KS}}{dN}
\quad\mbox{where}\quad
V_\textrm{KS} \equiv \frac{\textrm{Tr}  [\hat{V}_\textrm{KS} \hat{P}]}{ \textrm{Tr}[ \hat{P} ]}. 
\end{align}
%


Hence, we have derived 
the relevant expression for the Hubbard $U$ 
to be determined from a variational procedure.
%
Moreover, 
the Dyson equation above in 
Eq.~\eqref{eq:dyson_equation2}, 
may be readily extended to calculate 
long-range, inter-subspace parameters $V$~\cite{0953-8984-22-5-055602}, 
and their corresponding Hubbard $U$ values, 
by constructing the matrix response functions~\cite{PhysRevB.58.1201}
$\chi^{IJ}$ and $\chi_0^{IJ}$.



%COMPARABILITY OF METHODS
\subsection[{Comparing the SCF and variational linear-response methods}]{Comparing the SCF and variational\break linear-response methods}

In this section, 
we shall digress for a moment to discuss 
the technical differences between the 
SCF LR method introduced in Ref.~\cite{PhysRevB.71.035105}, 
and the variational adaptation just described.

Insofar as the LR method has been applied 
in both scenarios, 
the SCF~\cite{PhysRevB.71.035105} 
and variational methods are identical 
in terms of their applied external perturbation $\alpha$, 
the use of the Dyson equation, 
and issues surrounding DFT+$U$ 
population analysis choice and convergence. 
%
They are also equally convenient for use 
with SCF-type codes, 
however, the variational approach 
is much more convenient  for 
use with direct-minimisation solvers.
%
They are also perfectly identical 
in their calculation of  the interacting response $\chi $, 
with the primary difference relating to the 
calculation of the non-interacting response $\chi_0$ 
(at least formally, but the mathematical 
differences are as yet unexplored).

In the SCF LR procedure, 
$\chi_0$ is calculated following the first step 
of the SCF cycle, 
which relies, by construction, 
upon a non-ground-state density 
for each given external parameter $\alpha$.
%
An important consequence of this practice is that 
the SCF non-interacting response $\chi_0=-dV_\textrm{ext}/dn_0$  
is calculated from the total-derivative 
of the external potential with respect to the non-converged density $n_0$, 
which does not correspond to the ground-state of the system.
{
Consequently, 
the SCF Hubbard $U$ is not 
a property that can be derived 
from the ground-state density 
as a function of the external potential, 
without reference to the eigensystem 
of the KS Hamiltonian.}
%a ground-state 
%property of the unperturbed ground-state
%density, but instead an excited-state property 
%dependent on the eigensystem 
%of the non-interacting KS Hamiltonian.
%
Thus, 
since the SCF Hubbard $U$ 
cannot be regarded as a purely ground-state density-functional property, 
the total-energies computed from the 
DFT+$U$ calculations are not necessarily so either.
%
This has important consequences 
for the comparability of total-energies 
between various crystallographic 
or molecular structures with differing 
Hubbard $U$ values,
and the validity of thermodynamic calculations 
based on DFT+$U$, 
which remains a challenging 
endeavour~\cite{doi:10.1021/cm702327g,CapdevilaCortada201558,PhysRevB.85.155208,PhysRevB.84.045115,PhysRevB.85.115104,PhysRevB.90.115105}.


Conversely, in the variational procedure above, 
{\it both} $\chi$ and $\chi_0$, 
given in Eq.~\eqref{eq:dyson_equation2}, 
are calculated at the end 
of the minimisation procedure 
from the same set of 
constrained ground-state 
densities defined by $\alpha$.
%
As a result, 
$\chi$, $\chi_0$, and the corresponding $U$ value 
are all derived from the ground-state density of the system.
%
This is a promising first step towards 
implementing DFT+$U$ with the 
calculation of thermodynamical properties 
and  we will return to address this issue further 
in the next Chapter.


%GLOBAL VS LOCAL CURVATURE
\subsection{The subspace contribution to SIE: global vs local curvatures}

A further point to address is the important distinction between  
the total many-body SIE of a system, 
and that stemming from a localised subspace.
%
The former pertains to the 
addition or removal of electrons in a system, 
while the latter is measured as a result 
of redistribution of charge in a system, 
albeit with the caveat that the local subspace 
weakly interacts with the surrounding environment.
%
In both cases, 
it relates to the deviation in 
piece-wise linear behaviour in total-energy, 
either with respect to total particle number, 
or subspace occupancy, respectively.

With the application of the DFT+$U$ functional, 
the aim is to correct the former as much as possible 
by treatment of the latter on a per-electron basis, 
and is assumed to be precise in the atomic limit, 
based on the conservation of total particle number 
and dominance of local interactions. 
%
It is therefore most effective when the 
subspace, or {\it local}, SIE composes    
a large portion of the {\it global} SIE, 
adopting the terminology from 
Ref.~\cite{:/content/aip/journal/jcp/145/5/10.1063/1.4959882}. 

Consider for a moment the local SIE 
arising from a strongly-correlated subspace, 
which may be computed from  
the integral of the interacting potential 
over the subspace occupancy 
up to its ground-state value $N_0$
%
\begin{equation}
E^\textrm{SIE}_\textrm{local}=\int_0^{N_0} dN'\, V_\textrm{int}(N').
\end{equation}
%
A change of variables allows us 
to compute this instead as the 
energy expended in 
depleting the subspace of charge 
from its ground-state
%
\begin{align}
E^\textrm{SIE}_\textrm{local}
&=-\int_0^\infty d\alpha\, V_\textrm{int}(N'(\alpha)) \frac{dN'}{d\alpha}\nonumber\\[0.5em]
&=\int_0^\infty d\alpha\, V_\textrm{int}(N'(\alpha))\left(\left.\frac{d^2E_\textrm{total}}{dN'^2}\right|_{N'(\alpha)}\right)^{-1},
\end{align}
%
where in the last line we have used the relation in 
Eq.~\eqref{eq:coc_linear_response3}, 
to connect the expression with the global 
curvature of the constrained total-energy $E_\textrm{total}$.

The curvature of $E^\textrm{SIE}_\textrm{local}$ 
in a single-site model, 
while not implemented in practice, 
is the term measured by $U$
%
\begin{equation}
\frac{d^2E^\textrm{SIE}_\textrm{local}}{dN^2}=U
\end{equation}
%
and not, as we have discussed, 
the curvature of the total-energy
%
\begin{equation}
\frac{d^2E_\textrm{total}}{dN^2}=-\chi^{-1}.
\end{equation}
%
It follows that the difference between these quantities 
returns the positive-definite non-interacting 
inverse response $\chi_0^{-1}$
%
\begin{align}
\frac{d^2(E_\textrm{total}-E^\textrm{SIE}_\textrm{local})}{dN^2}
&=-\chi^{-1}-U=-\chi_0^{-1} \geq 0\nonumber\\[0.5em]
\Rightarrow\quad&
\frac{d^2E_\textrm{total}}{dN^2}\geq \frac{d^2E^\textrm{SIE}_\textrm{local}}{dN^2}.
\end{align}
%
Thus, the global energy curvature 
with respect to subspace occupancy,
is shown to be at least 
as large as the local energy curvature, 
subject to particle number conservation.
%
The above result is heuristically analogous 
to the comprehensive study by Kulik {\it et al.} in 
Ref.~\cite{:/content/aip/journal/jcp/145/5/10.1063/1.4959882}, 
in which it was rigorously proven that 
a $+U$ correction, 
while intended to mitigate the local SIE, 
cannot, at the very least, increase the global SIE.

%ONE-ELECTRON TEST CASE: H2+
\subsection{A one-electron test case: H$_2^+$}

To illustrate the efficacy of the 
variational LR approach described above, 
let us consider its application 
in treating the SIE present in the 
binding energy of H$_2^+$.
%
Calculations to determine the $U$ values 
were performed using the 
calculation parameters outlined in 
Chapters~\ref{ch:qm_simulation}~\&~\ref{ch:self_interaction_error}.
% 
The external potential $\alpha$ 
was varied between $\pm 0.1$~eV 
in intervals of $0.05$~eV 
and applied to one atom only, 
while the other acted as the electronic bath.
%
The $U$ values were then computed 
according to Eq.~\eqref{eq:u_definition} 
for all bond-lengths,  
as shown in Fig.~\ref{fig:h2+_u_values}, 
and compared to the values required to correct 
the PBE total-energy $U_\textrm{int}$,  
which were estimated from the linear interpolation 
of the various PBE+$U$ binding energies  
presented in Fig.~\ref{fig:h2+_dft+u_total_energy} 
with the exact binding energy.
%

We see from Fig.~\ref{fig:h2+_u_values} 
that the calculated $U$ parameters  
span a large range of values between -5~eV and 8~eV 
and broadly match the behaviour of $U_\textrm{int}$, 
particularly in the dissociation limit.
%
For short bond-lengths, however,
the comparison breaks down 
due to large subspace overlap, 
charge spillage and inappropriate population analysis.
%
Remarkably, 
we also see that $U$ attains negative values for $r<2$~a$_0$, 
which implies that the charge in this regime is over-localised 
according to the response functions.

Furthermore, 
the calculations to determine $U$ 
were found to remain highly linear 
across all bond-lengths 
for this particular system and methodology.
%
A typical example of such a calculation, 
at a bond-length of 4~a$_0$,
is presented in the inset of Fig.~\ref{fig:h2+_u_values}, 
which shows a clear linear relationship 
between $V_\textrm{int}$ and $N$ 
with only 5 data points.
%
The slope of the resulting linear fit 
$dV_\textrm{int}/dN$, 
corresponding to the variational LR formula for $U$, 
results in $U=4.840\pm 0.006$~eV, 
which is accompanied by very small fitting errors.  
%
%
%\begin{figure}[th!]
%\centering
%\includegraphics[height=0.494\textwidth]{images/h2+_alpha.pdf}
%\caption[Example of variational linear-response calculation for $U$]
%{The interacting response $V_\textrm{int}$ 
%against total subspace occupancy $N$ 
%of the perturbed atom $H$ 
%at 4~a$_0$ bond-length. 
%%
%The variational linear-response $U=dV_\textrm{int}/dN$ 
%results here in $U=4.840\pm0.006$~eV.}
%\label{fig:h2+_alpha_calculation}
%\end{figure}
%
\begin{figure}[th!]
\centering
\includegraphics[height=0.494\textwidth]{images/h2+_u_values.pdf}
\caption[Variational linear-response $U$ values calculated for dissociating H$_2^+$]
{Variational linear-response $U$ values 
calculated for dissociating H$_2^+$
compared to the $U_\textrm{int}$ values 
required to correct the PBE binding energy 
estimated from interpolating with the 
exact binding energy.}
\label{fig:h2+_u_values}
\end{figure}
 
%
{We then applied the calculated $U$ values to both atoms and, 
maintaining a symmetric charge density},
we computed the PBE+$U$ binding curve shown in 
Fig.~\ref{fig:h2+_pbe+u} below.
%
{
Here,
we see a marked improvement over the bare PBE curve, 
for which the DFT+$U$ functional 
resolves the total-energy to appreciable accuracy across all bond-lengths 
and provides qualitative agreement with the exact curve, 
thereby demonstrating that DFT+$U$ is an effective scheme 
for the treatment of one-electron SIE.}

{
Crucially, the improvement continues into the dissociation limit, 
at which point the two 1$s$ orbitals, 
used to define the Hubbard subspaces, 
overlap and spill charge minimally, 
such that the population analysis becomes ideal.
%
Furthermore, 
the weakening inter-atomic interaction becomes 
increasingly ideal for the application of DFT+$U$ 
in this limit.
%
This shows that our method for calculating $U$ values 
corresponds sufficiently with the necessary correction.}


\begin{figure}[th!]
\centering
\includegraphics[height=0.494\textwidth]{images/h2+_pbe+u.pdf}
\caption[PBE+$U$ binding curve of H$_2^+$ with linear-response $U$]
{The calculated PBE+$U$ binding curve of H$_2^+$ (blue dashed)
with $U$ calculated from variational linear-response, 
compares well to the exact total-energy (red), 
and shows a remarkable improvement over the bare PBE curve (orange, dashed) 
in the dissociation limit.}
\label{fig:h2+_pbe+u}
\end{figure}
% 
%We observe, however, 
%that the equilibrium bond-length is reduced 
%in these calculations.
%


Our results show, therefore, 
that DFT+$U$, 
when supplied with the $U$ values calculated from 
variational linear-response, 
is capable of adequately correcting 
the total-energy SIE of a one-electron system 
under ideal population-analysis conditions, 
and is the first explicit confirmation  
of such to our knowledge.
%
It is then clear that DFT+$U$ 
provides an efficient and effective correction to the
SIE in the total-energy, 
as discussed in detail in 
Refs.~\cite{PhysRevLett.97.103001,:/content/aip/journal/jcp/133/11/10.1063/1.3489110,:/content/aip/journal/jcp/145/5/10.1063/1.4959882}.
%
In the following section, 
we shall apply our variational approach 
to more challenging, multi-electronic systems.

%APPLICATION TO MULTI-ELECTRONIC SYSTEMS
\section{Application to multi-electronic systems}

The treatment of SIE in dissociating H$_2^+$ 
has so far served the important purpose
of testing our variational method 
to ensure the intended behaviour 
and adequate treatment of one-electron SIE.
%
We now wish to extend this methodology  
to consider multi-electronic systems 
and certify that the variational LR method 
is robust and applicable to ameliorating many-body SIE.
%

%H2
\subsection{Dissociating H$_2$}

We shall first return to the H$_2$ molecule, 
which is the simplest multi-electronic structure to consider, 
and demonstrate that DFT+$U$ is unsuitable 
to correct the static-correlation error (SCE) 
characteristic of the system.

As we discussed in section~\ref{sec:static_correlation_error}, 
it may be possible to formulate a viable correction 
to the SCE exhibited in dissociating H$_2$ 
if DFT+$U$ is provided with the appropriate (negative) $U$ values. 
%
To this end, 
and as a demonstration of the inapplicability of 
DFT+$U$ in correcting SCE, 
we calculated the $U$ parameters along 
the dissociating curve, as with H$_2^+$, 
for $\alpha=\pm0.05$~eV, 
and recalculated the total-energy 
with the DFT+$U$ functional, 
presented in Fig.~\ref{fig:H2_dft+u_energy2}.

From this plot it is very clear that 
the $U$ values act counter to the intended result, 
and that DFT+$U$ increases the error in the total-energy.
%
The reason for this is because $U$ is a measure 
of SIE manifest in spurious {\it positive} curvature 
deviating from piece-wise linear behaviour, 
whereas SCE is exhibited by a {\it negative} curvature, 
deviating from piece-wise constancy 
with respect to total spin magnetisation.
%
The calculated $U$ values are positive 
and therefore inapplicable to 
{systems dominated by} SCE in this regard, 
thereby illustrating that DFT+$U$ cannot be 
applied in an {\it ad hoc} manner.
%
{
Let us now consider a molecule 
for which the error {\it is} dominated by SIE}.

\begin{figure}[th!]
\centering
\includegraphics[height=0.494\textwidth]{images/H2_dft+u_energy2}
\caption[PBE+$U$ binding curve of H$_2$ with linear-response $U$]
{Binding curve of H$_2$ 
calculated with CI (red) and 
PBE (orange) and PBE+$U$ (blue).
%
The $U$ values calculated with variational LR
exacerbate the SCE for all bond-lengths, as expected, 
and are therefore an inappropriate remedy for SCE.}
\label{fig:H2_dft+u_energy2}
\end{figure}


%Ni(CO)4
\subsection{Molecular Ni(CO)$_4$}

{
The following work was conducted in collaboration with 
Okan Karaca Orhan, Fiona McCarthy, 
Mark McGrath, and Thomas Wyse Jackson, 
whom we  thank for their valuable contributions.}

Let us now turn our attention to 
a transition metal complex nickel tetracarbonyl (Ni(CO)$_4$), 
which we selected based on its simple molecular structure 
centred around a Ni atom with 
almost full 3$d$ subshells~\cite{orhan2017tddft+}.
%
This ensures the dominant error on the Ni atom 
is attributed to SIE rather than 
SCE~\cite{PhysRevB.77.115123,cohen2008insights}.

The relaxed geometry for Ni(CO)$_4$ 
was acquired from Ref.~\cite{doi:10.1063/1.437911}, 
and entails a tetragonal structure 
with equal bond angles of 109.48$^\circ$
between all Ni and C atoms, 
as shown in Fig.~\ref{fig:nico4_structure}.
%
The Ni--C bond-length is 1.85~\AA\ , 
while the C--O bond-length is slightly 
shorter at 1.15~\AA\ .

Calculations were performed 
with a hard ($2.10$~a$_0$ cutoff) 
norm-conserving pseudopotential~\cite{PhysRevB.41.1227}, 
an LDA was used for the XC functional, 
and open boundary conditions~\cite{:/content/aip/journal/jcp/110/6/10.1063/1.477923} 
were employed.
%
The kinetic energy cutoff was 900~eV, 
9 NGWFs were chosen for the Ni atom 
and 4 each for the C and O atoms 
with a cutoff radius of $16$~a$_0$.
%
The HOMO-LUMO gap in Ni(CO)$_4$, 
as measured by experiment~\cite{doi:10.1021/ja00274a073}, 
is $4.83$~eV, 
while the gap determined from a PBE calculation with $U=0$
is $3.87$~eV, 
which is $\sim1$~eV below the experimental value. 

To calculate $U$, 
{in which the atomic positions are fixed},
the potential shift $\alpha$ 
is chosen to be on the order of approximately $1-2\%$ 
of the unperturbed interacting potential $V_\textrm{int}$.
%
This ensures as broad a range of perturbations 
as possible are applied, 
while remaining in the linear-response regime.
%
Here, 
$\alpha$ was varied between $\pm0.1$~eV 
in intervals of $0.05$~eV, 
from which 
we calculated $U=10.8\pm0.1$~eV quite accurately, 
as shown below in Fig.~\ref{fig:nico4_alpha}.
%
Taken in isolation, this value seems excessively large, 
however, we argue that by perturbing the charge density 
in such a spatially confined molecule, 
it is unsurprising that the response is larger than usual. 
%
Indeed, the HOMO-LUMO 
gap with the +$U$ correction applied in this case 
yields a relatively modest increase to $4.56$~eV, 
but it is much more agreeable with 
the experimental figure and 
is accurate to within 6\%.
%
{
Furthermore, under this correction, 
the occupancy of the Hubbard subspace 
only increased from 9.09~e to 9.31~e.}
%
The correct treatment of SIE in this simple molecule, 
using our variational method, 
is therefore an encouraging preliminary result.

\begin{figure}[th!]
\centering
\subfloat[]{
\includegraphics[height=0.45\textwidth]{images/nico4.5.jpg}
\label{fig:nico4_structure}}
\quad
%
\subfloat[]{
\includegraphics[height=0.45\textwidth]{images/nico4_alpha.pdf}
\label{fig:nico4_alpha}}
%
\caption[Ni(CO)$_4$ molecular structure and linear-response calculation]
{\subref{fig:nico4_structure} 
The tetragonal Ni(CO)$_4$ molecular structure.
\subref{fig:nico4_alpha} 
The interacting response $V_\textrm{int}$ 
against subspace occupancy $N$, 
the slope of which gives $U=10.8\pm0.1$~eV.}
\end{figure}


%NiO
\subsection{Bulk NiO}
\label{sec:NiO_calculating_u}
We shall investigate our first bulk structure, 
nickel oxide (NiO), 
which is a particularly challenging system for approximate DFT
and one that has been the subject of repeated testing 
for novel methods such as 
periodic unrestricted Hartree-Fock theory~\cite{PhysRevB.50.5041}, 
the self-interaction corrected local density
approximation~\cite{PhysRevLett.65.1148},
the GW approximation~\cite{PhysRevLett.74.3221},
LDA + DMFT~\cite{PhysRevB.74.195114} 
and first-principles methods for
the original calculation of the Hubbard $U$ parameter 
via linear-response~\cite{PhysRevB.71.035105,PhysRevB.58.1201}.
%
It has a rock-salt crystal structure, 
as shown in Fig.~\ref{fig:NiO_structure}, 
with a lattice constant of 4.16~\AA\ ~\cite{PhysRevB.57.1505} 
and a N\'eel temperature of 523~K~\cite{PhysRevB.30.4734},
below which it is an anti-ferromagnetic type-II insulator 
with alternating magnetic moments, 
ranging from 1.64~$\mu$B~\cite{alperin1962nio} 
to 1.9~$\mu$B~\cite{PhysRevB.27.6964}, 
between adjacent Ni atoms,
as measured by neutron diffraction. 

DFT+$U$ has also been successfully 
employed with NiO~\cite{PhysRevB.48.16929,PhysRevB.57.1505,PhysRevB.71.035105,PhysRevB.58.1201,PhysRevB.62.16392} 
when supplied with a sufficient $U$ parameter.


\begin{figure}[th!]
\centering
\subfloat[]{
\includegraphics[height=0.425\textwidth]{images/NiO.2.jpg}
\label{fig:NiO_structure}}
\quad
%
\subfloat[]{
\includegraphics[height=0.425\textwidth]{images/NiO_alpha.pdf}
\label{fig:NiO_alpha}}
%
\caption[NiO crystal structure and linear-response calculation]
{\subref{fig:NiO_structure} 
The NiO rock-salt crystal structure.
\subref{fig:NiO_alpha} 
The interacting response $V_\textrm{int}$ 
against subspace occupancy $N$, 
the slope of which gives $U=6.7\pm0.1$~eV.}
\end{figure}

The majority of studies to date 
have observed a band gap 
of $\sim$ 4~eV arising from 
charge-transfer-type $p-d$ transitions~\cite{PhysRevB.43.14674,PhysRevLett.53.2339}, 
however recent 
BIS spectral data~\cite{0953-8984-11-7-002}, 
supported by calculations invoking 
GW quasiparticle~\cite{PhysRevB.71.193102,doi:10.1080/00018739400101495} 
and screened-exchange LDA~\cite{0953-8984-25-16-165502}, 
report a highly dispersive 
$s$-like state as low as 3~eV
that accounts for very weak optical 
absorption~\cite{Hufner1992,PhysRevB.79.045118,PhysRevB.62.16392}.
%


Spin-polarised calculations of NiO were performed   
using a norm-conserving PBE pseudopotential 
with a PBE exchange-correlation functional, 
a sample input file is available in 
Appendix~\ref{ch:input_files}.
%
A plane-wave energy cutoff of 1170~eV 
was chosen with a $4\times4\times4$ supercell 
with $\Gamma$-point sampling, 
and an NGWF cutoff radius of 11~$a_0$ 
such that total-energy convergence was achieved to 
within $2\times10^{-6}$ Ha per atom.
%
Values of $\alpha$ between $\pm0.2$~eV 
in intervals of $0.1$~eV were 
applied here,
%\footnote{{Applying $\alpha=\pm0.01$, 
%as in H$_2^+$, was too small a perturbation 
%to produce a sufficient signal to noise ratio 
%in the changes to the subspace occupancy, 
%and it consequently produced large errors 
%in the linear fits.}}, 
for which we calculated $U=6.4\pm0.1$~eV, 
as shown in Fig.~\ref{fig:NiO_alpha}.
%
This is within reasonable 
range of the $U$ values used in other studies, 
which vary between 4.6~eV and 
8~eV~\cite{PhysRevB.71.035105,
PhysRevB.62.16392,
PhysRevB.44.943,
PhysRevB.60.10763,
PhysRevLett.102.226401}
%
The parameters and results of this study, 
as well as those from various other DFT+$U$ works, 
are summarised in Table~\ref{table:multi_electron_results} 
in Chapter~\ref{ch:self_consistent_hubbard}
along with the experimental data.


Unsurprisingly, 
the PBE calculation 
qualitatively underestimates both the 
local magnetic moment, 
measured here to be 1.37~$\mu$B, 
and the band gap, which is 1.66~eV, 
in accordance with other studies~\cite{PhysRevB.23.5048}.
%
Moreover, 
it incorrectly attributes the 
valence band edge states to nickel 3$d$-orbitals,  
as indicated by the density of states (DOS) 
plot in Fig.~\ref{fig:NiOpbedos}.
%
{
Given that the system is fully compensated, 
we inspect only the spin-up channel, 
for which we depict the contributions 
from Ni spin-up sub-lattice (green), 
Ni spin-down sub-lattice (blue), 
the O atoms (red), 
and finally the total spin-up DOS (black).}
%
{
We note that in this, and subsequent, DOS profiles 
the conduction states beyond a few eV 
are likely to be unreliable~\cite{PhysRevB.84.165131}, 
which is due to the limited resolution provided by 
the valence-optimised NGWFs for states 
beyond the valence band edge.}

\begin{figure}[th!]
\centering
\includegraphics[height=0.494\textwidth]{images/NiO.pbe.dos.1.pdf}
\caption[Species-resolved DOS for NiO calculated with PBE]
{The species-resolved DOS for NiO 
with half-width Gaussian smearing of 0.1 eV 
and the zero-energy set to the Fermi level (eV) for 
unperturbed PBE.
%
{
The DOS is plotted for the up-spin 
channel stemming from   
Ni spin-up atoms (green), 
Ni spin-down atoms (blue), 
O atoms (red), 
and total (black)}.}
\label{fig:NiOpbedos}
\end{figure}


%
The DOS computed from applying 
$U=6.7$~eV to all Ni atoms in the crystal 
is presented in Fig.~\ref{fig:NiO6.7dos}, 
in which the band gap is increased to 3.04~eV.
%
Here, the grey dashed line, 
labeled as the Hubbard DOS, 
describes the DOS stemming from the localised subspace, 
i.e., the localised Ni 3$d$ states.
%
The conduction band edge 
(shown in the inset)
is assigned to a dispersive $s$-state, 
which is unaffected by the $+U$ correction, 
in line with experimental 
results~\cite{Hufner1992,PhysRevB.79.045118,PhysRevB.62.16392}.
%
This is evident from the lack of contribution 
of this state arising from the localised subspace, 
as indicated by the total Hubbard DOS.
%
The sharp peak in the conduction states  
lies 4.44~eV above the Fermi level 
and likely relates to the 
experimentally observed band gap.
%
The magnetic moment for this state 
is 1.62~$\mu$B and within reasonable 
range of the experimental values~\cite{alperin1962nio,PhysRevB.27.6964}.

Finally, the constituent states of the 
valence band from $-1.5$~eV 
up to the Fermi level are 
25\% Ni and 75\% O, 
whereas the conduction band is composed of 
90\% Ni and 10\% O 
up to 5~eV above the Fermi level.
%
These figures corroborate 
experimental observations~\cite{PhysRevB.43.14674,PhysRevLett.53.2339} 
that categorise NiO 
as a charge transfer type semi-conductor.
%
We therefore conclude that our 
computed Hubbard $U$ value, 
from variational LR, 
is also applicable to extensive solids 
and yields sufficient agreement 
with experimental data.


\begin{figure}[th!]
\centering
\includegraphics[height=0.494\textwidth]{images/NiO.6.7.dos.pdf}
\caption[Species-resolved DOS for NiO calculated with PBE+$U$]
{The species-resolved density-of-states of NiO 
with half-width Gaussian smearing of 0.1 eV 
and the zero-energy set to the Fermi level (eV) 
for $U=6.7$~eV.
%
The lines depict the same as in Fig.~\ref{fig:NiOpbedos}, 
with the Hubbard total given in dashed grey.
%Ni majority-spin (green), 
%Ni minority-spin (blue), 
%O (red), 
%total (black), and 
%
Inset illustrates the conduction band minimum 
is not attributed to the localised Hubbard states 
but a dispersive $s$-state, 
giving a band gap of 3.04~eV.
}
\label{fig:NiO6.7dos}
\end{figure}





%Cr2O3
\subsection{Bulk Cr$_2$O$_3$}

%
Finally, 
we shall consider chromium(III) oxide (Cr$_2$O$_3$), or {\it chromia}, 
which is a Mott-Hubbard material 
when computed with DFT+$U$ but, 
until now, has not benefited from 
the calculation of the Hubbard $U$ 
from LR approach.
%
The following work was conducted with 
Emma Norton and Karsten Fleischer, 
and provides a welcome opportunity 
to showcase the first novel application 
of our variational LR method.

The search for a viable $p$-type 
transparent semiconducting oxide (TCO) 
has seen growing interest in 
magnesium-doped chromia Mg:Cr$_2$O$_3$, 
which has demonstrated potential as 
such~\cite{doi:10.1063/1.3638461,0953-8984-28-12-125501,PhysRevB.91.125202,PhysRevB.93.115302}.
%
The host matrix, Cr$_2$O$_3$, 
possesses a corundum crystal structure with lattice 
constants $a=b=4.96$~\AA\  and $c=13.60$~\AA\  
in the $R\bar{3}c$ space group~\cite{Gloege1999L917}.
%
The N\'eel temperature is 307~K~\cite{PhysRevB.15.4451,YACOVITCH19771126}, 
below which it is an antiferromagnetic insulator 
with alternating magnetic moments of 
2.76~$\mu$B, 
{determined via neutron diffraction}~\cite{doi:10.1063/1.1714118},
on each Cr atom along the $c$-axis.

The excitation mechanism in chromia 
is deemed to be between that of a charge-transfer and 
Mott-Hubbard type~\cite{CATTI19961735,PhysRevB.70.125426} 
with an optical band gap of 
3.0-3.4~eV~\cite{FISCHER19712455,doi:10.1063/1.1702871,doi:10.1063/1.1714118,HOLT1997201,PhysRevB.91.125202}.
%
However, several 
experiments~\cite{FISCHER19712455,HOWNG198075,PhysRevB.91.125202,doi:10.1021/jp404230k} 
have observed transition energies at 2.1~eV and 2.7~eV, 
but no computational study to date has 
successfully reproduced them, 
despite their being experimentally 
attributed to spin-allowed 
transitions~\cite{PhysRevB.57.9586}.

\begin{figure}[th!]
\centering
\subfloat[]{
\includegraphics[trim={7.8cm 0 6.8cm 0},clip,height=0.494\textwidth]{images/cr2o3.4.jpg}
\label{fig:cr2o3_structure}}
\quad
%
\subfloat[]{
\includegraphics[height=0.494\textwidth]{images/cr2o3_alpha.pdf}
\label{fig:cr2o3_alpha}}
%
\caption[Cr$_2$O$_3$ crystal structure and linear-response calculation]
{\subref{fig:cr2o3_structure} 
The Cr$_2$O$_3$ corundum crystal structure.
\subref{fig:cr2o3_alpha} 
The interacting response $V_\textrm{int}$ 
against subspace occupancy $N$, 
the slope of which gives 
$U=2.8\pm0.1$~eV.}
\end{figure}


Chromia has been the focus of 
many DFT+$U$ computational studies to date,
the first of which was performed 
by Catti {\it et al.}~\cite{CATTI19961735} 
using unrestricted Hartree-Fock (UHF), 
in which they calculated an excessively large 
band gap of $\sim$15~eV.
%
Progress later came when 
Dobin and co-workers~\cite{PhysRevB.62.11997} 
performed LSDA calculations 
but found an underestimated band gap of 1.5~eV.
%
The first attempt to address the poorly resolved 
electromagnetic behaviour came when 
Rohrbach {\it et al}~\cite{PhysRevB.70.125426,C0JM03852K} 
invoked the DFT+$U$ functional 
of Dudarev and Liechtenstein~\cite{PhysRevB.56.4900} 
with $U=5$~eV and $J=1$~eV, 
determined via fitting to experimental data.
%
Nevertheless, 
the band gap and magnetic moment they calculated, 
2.6~eV and 3.01~$\mu$B, respectively, 
were in much better agreement with experiment 
and have been widely used since~\cite{doi:10.1021/jp5039943,Mosey2009287,doi:10.1021/jp5039943}.

Mosey and co-workers~\cite{PhysRevB.76.155123,doi:10.1063/1.4865780,doi:10.1063/1.2943142,Mosey2009287}, 
were the first to calculate $U$ from {\it ab initio} methods 
and used UHF to calculate $U=3.3$~eV and $J=0.1$~eV, 
which they also implemented in Dudarev's 
DFT+$U$ functional~\cite{PhysRevB.56.4900}.
%
The band gap they resolved was 2.9~eV 
while the magnetic moment was 2.9~$\mu$B, 
which were in reasonable agreement with experiment.
%

Shortly thereafter, 
Shi {\it et al.}~\cite{PhysRevB.79.104404}, 
in pursuit of resolving the magnetic properties,
determined $U\approx4$~eV, for a fixed $J=0.58$~eV,  
from the full-potential linear augmented plane-wave (FLAPW) 
method described in Ref.~\cite{blugel2006computational}.
%
They implemented the 
rotationally invariant DFT+$U$ functional~\cite{PhysRevB.52.R5467,0953-8984-9-35-010,PhysRevB.71.035105} 
and computed a 2.88~eV band gap 
and 3.26~eV magnetic moment.
%
In more recent years, 
$U$ and $J$ values have been 
determined via fitting 
to experimental data~\cite{PhysRevB.73.195107,0953-8984-28-12-125501,Wang20121422,doi:10.1063/1.4970882,doi:10.1021/acs.jpcc.6b05575,PhysRevB.87.195124}, 
for which a range of values have been derived.
%
A summary of the parameters and results of this study, 
as well as those from other selected DFT+$U$ reports, 
and the experimental data 
are presented in Table~\ref{table:cr2o3_summary} 
in Chapter~\ref{ch:self_consistent_hubbard}.

For this study, 
spin-polarised calculations were performed using the\break 
projector-augmented wave (PAW) method 
and an LDA for the XC functional. 
%
A plane-wave energy cutoff of 1100~eV 
was chosen with a $3\times3\times2$ supercell, 
$\Gamma$-point sampling, 
and a NGWF cutoff radius of 12~$a_0$ 
such that total-energy  convergence was achieved to 
within $2\times10^{-5}$ Ha per atom.
%
Values of $\alpha$ between $\pm0.2$~eV 
in intervals of $0.1$~eV were again applied, 
for which we calculated $U=2.8\pm0.1$~eV, 
as shown in Fig.~\ref{fig:cr2o3_alpha}.
%
To our knowledge, this is the first
instance of the Hubbard $U$ calculated 
from LR (variational or otherwise) for this material. 
%

{
We again plot the DOS for the spin-up sub-lattice 
and present the contributions from 
Cr spin-up atoms (green), 
Cr spin-down atoms (blue), 
O atoms (red), 
and the total spin-up DOS (black).}
%
As expected, 
the traditional LDA calculation underestimates 
the local magnetic moment, 
here calculated as 2.30~$\mu$B, 
and the band gap, which is 1.27~eV, 
in line with Ref.~\cite{PhysRevB.62.11997}.
%
Fig.~\ref{fig:cr2o3pbedos} demonstrates that 
the valence and conduction band edge states 
are heavily attributed to the Cr $3d$-states.

\begin{figure}[th!]
\centering
\includegraphics[height=0.494\textwidth]{images/cr2o3.pbe.dos.1.pdf}
\caption[Species-resolved DOS for Cr$_2$O$_3$ calculated with LDA]
{The species-resolved DOS of Cr$_2$O$_3$ 
calculated with LDA, 
with half-width Gaussian smearing of 0.1 eV 
and the zero-energy set to the Fermi level (eV).
%
The lines depict    
Ni majority-spin (green), 
Ni minority-spin (blue), 
O (red), 
and total (black)}
\label{fig:cr2o3pbedos}
\end{figure}


The LDA+$U$ calculation on the other hand, 
provided with $U=2.8$~eV on all Cr atoms, 
returns a DOS depicted in Fig.~\ref{fig:cr2o3.2.8.dos}, 
in which the band gap is increased to 2.10~eV, 
and a magnetic moment of 2.56~$\mu$B, 
which are both in excellent agreement 
with experimental observations~\cite{FISCHER19712455,HOWNG198075,PhysRevB.91.125202,doi:10.1021/jp404230k,doi:10.1063/1.1714118}.
%
Excitations are now expected to be governed by 
$d-d$ like-spin transitions up to 2.8~eV above the Fermi level.
%at which point $d-d$ spin-transition excitations 
%dominate thereafter.
%
The constituent states of the 
valence band from $-1.5$~eV 
up to the Fermi level are, 
on average, 
70\% Cr and 30\% O, 
whereas the conduction band is composed of 
80\% Cr and 20\% O 
up to 4~eV above the Fermi level, 
which confirms the Mott-Hubbard 
character~\cite{PhysRevB.79.104404}.

\begin{figure}[th!]
\centering
\includegraphics[height=0.494\textwidth]{images/cr2o3.2.8.dos.pdf}
\caption[Species-resolved DOS for Cr$_2$O$_3$ calculated with LDA+$U$]
{The species-resolved DOS of Cr$_2$O$_3$, 
calculated with LDA$+U=2.8$~eV
with half-width Gaussian smearing of 0.1 eV 
and the zero-energy set to the Fermi level (eV).
%
The lines depict the same as in Fig.~\ref{fig:cr2o3pbedos}, 
with the Hubbard total given in dashed grey.}
\label{fig:cr2o3.2.8.dos}
\end{figure}


%COMPARISON TO PHOTOELECTRON MEASUREMENTS
\subsubsection{Comparison to spectroscopic measurements}
{
Our experimentalist colleagues, 
Emma Norton and Karsten Fleischer, 
obtained high resolution (0.58~eV)
X-ray and ultraviolet photoelectron 
spectroscopy measurements, 
termed XPS and UPS respectively, 
which probe the valence DOS of bulk chromia.
%
Furthermore, 
ultraviolet-visible spectroscopy (UV-Vis) data 
were provided,
which measures the optical absorption 
according to the method described in 
Ref.~\cite{PhysRevB.91.125202}.
%
In this section, 
we compare their measurements 
with computational data.}
%
A 0.29~eV half-width Gaussian smearing 
was applied to all LDA+$U$ DOS 
given in Fig.~\ref{fig:cr2o3.2.8.dos}, 
in order to provide a consistent comparison 
to the all experimental data.

Plotted in Fig.~\ref{fig:cr2o3.vb.xps.ups.total} 
are the Gaussian smeared 
total (solid red line) 
and Hubbard (blue solid line) DOS 
along with the XPS (dashed orange line) 
and UPS (dashed blue line) measurements.
%
In order to assist in their comparison, 
we have normalised the DOS plots   
such that the integral over the binding energy $E$ 
returns the total number of electrons in the unit cell (432).
%
From the graph it is clear that the total DOS 
reproduces very well the large peak in the XPS 
between -2.5~eV and 0~eV.
%
Furthermore, 
between -8~eV and -2.5~eV, 
the total DOS qualitatively reproduces 
the behaviour in the UPS data.
%
Meanwhile, 
the Hubbard DOS broadly follows 
the behaviour of the XPS across all energies, 
and accounts for approximately 
50\% of the total contribution. 

\begin{figure}[th!]
\centering
\includegraphics[height=0.494\textwidth]{images/cr2o3.vb.xps.ups.total.pdf}
\caption[Normalised XPS and UPS measurements of Cr$_2$O$_3$ compared to LDA+$U$
total DOS]
{Comparison of 
XPS (orange, dashed) and UPS (blue, dashed) 
measurements of Cr$_2$O$_3$
against the total (red, solid) and Hubbard (blue, solid) 
DOS calculated with LDA+$U$.
%
All data are normalised such that 
the integral under the curves equals 
the number of electrons in the unit cell.
%
A Gaussian smearing of 0.29~eV 
has been applied to the DOS 
to ensure consistent comparison.
%
The total DOS reproduces the peak in the XPS, 
and matches the broad trend in the UPS 
for lower binding energies very well.
%
{The Hubbard DOS,
which is attributed to the 
localised states, contributes} 
$\sim$50\% toward the XPS data 
across all energies.}
\label{fig:cr2o3.vb.xps.ups.total}
\end{figure}

To further aid this analysis, 
in Fig.~\ref{fig:cr2o3.vb.xps.ups.species}, 
we present the species-resolved DOS 
in comparison to the XPS and UPS data.
%
Here, we observe that the 
profile of the Cr up DOS (green solid line) 
contributes heavily to the sharp 
peak in the XPS data at $\approx$-2~eV.
%
Moreover, 
the trend in the UPS data 
seems to be reliably matched 
by that of the O DOS (solid red line), 
{from 0~eV}, 
where there exists a small peak, 
to the large spectral feature present 
from {-2~eV to -8~eV.}
%
The Cr down DOS (blue solid line), however, 
does not contribute appreciably to the total DOS.
%which seems to suggest that 
%the Cr atoms are highly magnetised.

\begin{figure}[th!]
\centering
\includegraphics[height=0.494\textwidth]{images/cr2o3.vb.xps.ups.species.pdf}
\caption[Normalised XPS and UPS measurements of Cr$_2$O$_3$ 
compared to LDA+$U$ species-resolved DOS]
{Comparison of 
XPS (orange, dashed) and UPS (blue, dashed) 
measurements of Cr$_2$O$_3$
against the 
Cr up (green, solid), 
Cr down (blue solid), 
and O (red, solid) 
DOS calculated with LDA+$U$.
%
All data are normalised and 
a Gaussian smearing of 0.29~eV 
has been applied to the DOS.
%
The Cr up DOS contributes heavily to the sharp 
peak in the XPS data at high binding energies.
%
The O DOS reproduces the broad trend 
in the UPS data at all energies.
%
The Cr down DOS contributes very little.
}
\label{fig:cr2o3.vb.xps.ups.species}
\end{figure}

{
Finally, 
in Fig.~\ref{fig:cr2o3.cb.absorption}, 
we present the absorption spectrum 
truncated to 5~eV above the conduction band edge, 
which has been normalised 
such that the integral under the curve is equal to one.
%
The absorption spectrum 
may be approximated from the joint DOS 
(assuming fixed optical matrix elements)
%
\begin{equation}
\alpha(\omega)\sim \int d\omega'\,\rho_v(\omega-\omega')\rho_c(\omega'), 
\end{equation}
%
and we plot the joint DOS 
evaluated from the summed 
convolution of the species-resolved 
valence and conduction DOS, 
given by $\rho_v(\omega)$ and $\rho_c(\omega)$, respectively.
%
This has been similarly normalised 
to aid in the visual comparison.}

{
Although the predictive capabilities 
of the calculated joint DOS is limited 
(see our above comment regarding 
conduction states)
we nonetheless observe some similar features 
present in both profiles.
%
For instance the onset of absorption 
coincides at $\approx$~2~eV, 
whereupon the peak values at 5~eV 
are of similar magnitude.
%
Furthermore, 
the excitation present at $\approx$~2.5~eV 
corresponds to a similar feature 
in the joint DOS.
}

\begin{figure}[th!]
\centering
\includegraphics[height=0.494\textwidth]{images/cr2o3.cb.absorption.pdf}
\caption[Normalised absorption spectrum of Cr$_2$O$_3$ 
compared to LDA+$U$ joint DOS]
{Comparison of 
UV-Vis (dashed) measurements of Cr$_2$O$_3$ 
of the optical absorption 
against the joint DOS calculated with LDA+$U$.
%
All data are normalised and 
a Gaussian smearing of 0.29~eV 
has been applied to the DOS.
}
\label{fig:cr2o3.cb.absorption}
\end{figure}

{
To produce a more reliable comparison 
between the joint DOS and UV-vis data, 
one could incorporate more NGWFs 
in the calculation to provide more conduction states, 
or optimise the energy with respect to the conduction states 
(with which it would be cumbersome to converge the total-energy), 
however, neither approach is guaranteed to significantly improve results.}

Regardless, 
these combination of the present results 
provide further support for the validity of 
our variational LR method,  
as we have now reliably calculated 
both qualitative and quantitative 
electronic properties to reasonably precision, 
without invoking fitting of any kind.


%CALCULATING J
\section[Calculating $J$ using the variational linear-response method]{Calculating $J$ using the variational \break linear-response method}
\label{sec:calculating_j}	

Let us now extend our variational approach 
to the calculation of the Hund's exchange parameter $J$, 
which is utilised in Eq.~\eqref{eq:dft+u+j_functional} 
to penalise anti-parallel spins occupying the same site 
and thereby encourages magnetic ordering.
%
While the full expression given 
in Eq.~\eqref{eq:dft+u+j_functional} 
is not frequently employed, 
it is common practice~\cite{PhysRevB.79.035103}, 
dating back to the Dudarev model~\cite{PhysRevB.57.1505}, 
to combine $U$ and $J$ to create 
an effective Coulomb repulsion between like-spins only 
$U_\textrm{eff}=U-J$ 
for use in Eq.~\eqref{eq:dft+u_functional}.

Similar to $U$, 
the $J$ term has historically been determined 
by fitting~\cite{PhysRevB.70.125426}, 
or calculated by 
cLDA~\cite{PhysRevB.44.943}, 
LAPW~\cite{Cai200989,0295-5075-69-5-777,PhysRevB.79.104404}, 
linear muffin-tin orbitals (LMTO)~\cite{PhysRevB.44.943}, 
linear combination of atomic orbitals (LCAO)~\cite{0953-8984-9-3-005}, 
or by configuration interaction (CI)~\cite{PhysRevB.63.214520,Calzado2001}.
%
Computed $J$ values are typically 
$\approx$1~eV in magnitude~\cite{PhysRevB.44.943} 
and are often absorbed into $U$, 
or sometimes ignored altogether.

Himmetoglu and co-workers 
in Ref.~\cite{PhysRevB.84.115108} 
formulated an extension to the LR approach outlined in section~\ref{sec:linear_response_approach}, 
in which $J$ is calculated from 
the {\it magnetic} response functions 
and is related to the curvature of the energy 
with respect to magnetisation 
$M=n^\uparrow-n^\downarrow$, given by
%
\begin{equation}
J \sim \frac{\partial^2E}{\partial M^2}.
\label{eq:j_curvature}
\end{equation}
%

Let us outline the formulation required 
to calculate the $J$ term from a variational approach
in the same spirit as we did for $U$ 
in section~\ref{sec:variational_linear_response}, 
by computing the total-energy curvature 
with respect to magnetisation.
%
We start by defining a constrained energy 
parameterised by a target magnetisation $M_I$, 
enforced by a set of Lagrange multipliers $\{\beta_I\}$, 
whereby 
%
\begin{equation}
E[\{\beta\}]=\min_{n({\br})}\left\{E[n({\br})]
+\sum_I \beta_I(n_I^\uparrow-n_I^\downarrow-M_I)\right\}.
\label{eq:coc_linear_response4}
\end{equation}
%
For a single site model, 
the difference between the 
spin up and spin down potentials 
in this system is then defined by the spin-splitting $\beta_I$ 
as follows
%
\begin{align}
\hat{V}^\uparrow-\hat{V}^\downarrow
=\frac{\delta E}{\delta \hat{\rho}^\uparrow}&-\frac{\delta E}{\delta \hat{\rho}^\downarrow}
= \hat{V}_\textrm{ext}^\uparrow-\hat{V}_\textrm{ext}^\downarrow 
= 2\beta\hat{P}\nonumber \\[0.5em]
\Rightarrow\quad 
\beta&=\frac{1}{2}\left( {V}_\textrm{ext}^\uparrow-{V}_\textrm{ext}^\downarrow\right), 
\end{align}
%
where $\hat{P}$ is the usual projection operator and 
$V_\textrm{ext}^\sigma\equiv \textrm{Tr}\left[\hat{V}_\textrm{ext}^\sigma\hat{P}\right]$.

{In keeping with the subspace-bare potential 
prescribed in section~\ref{sec:variational_linear_response}, 
we take the partial derivative of 
Eq.~\eqref{eq:coc_linear_response4}
with respect to $M$ which, 
according to the Hellman-Feyman theorem 
presented in Ref.~\cite{PhysRevB.94.035159}, 
yields} 
%
\begin{equation}
\frac{\partial E}{\partial M}=-\beta 
= -\frac{1}{2}\left(V_\textrm{ext}^\uparrow-V_\textrm{ext}^\downarrow \right).
\end{equation}
%
The total derivative of this expression follows to give 
the inverse of the bath-screen interacting 
on-site magnetic response $\chi_M$
%
\begin{equation}
\frac{d}{dM}\left(\frac{\partial E}{\partial M}\right)
=-\frac{d\beta}{dM} = -\chi_M^{-1}
= -\frac{1}{2}\frac{d}{dM}\left(V_\textrm{ext}^\uparrow-V_\textrm{ext}^\downarrow \right). 
\end{equation}
%
The non-interacting equivalent is then 
computed as before  
%
\begin{equation}
%\left(\frac{d^2E}{dM^2}\right)_0 = 
\chi_{M,0}^{-1}
= \frac{1}{2}\frac{d}{dM}\left(V_\textrm{KS}^\uparrow-V_\textrm{KS}^\downarrow \right), 
\end{equation}
%
such that we may define $J$ 
as the difference between interacting 
and non-interacting magnetic responses as before, 
giving the net, subspace-averaged exchange interaction 
%
\begin{equation}
J=\chi_{M,0}^{-1}-\chi_M^{-1}
=-\frac{1}{2}\frac{d(V_\textrm{int}^\uparrow-V_\textrm{int}^\downarrow)}{d(n^\uparrow-n^\downarrow)}.
\label{eq:j_definition}
\end{equation}
%

Moreover, 
we may expand the expression for $J$ 
in terms of the spin-dependent interactions, 
as we did for $U$, as follows
%
\begin{align}
J
&=-\frac{1}{2}\left(\frac{dV_\textrm{int}^\uparrow}{dM}
-\frac{dV_\textrm{int}^\downarrow}{dM}\right) \nonumber\\[0.75em]
&=-\frac{1}{2}\left(\frac{d V_\textrm{int}^\uparrow}{d n^\uparrow}\frac{dn^\uparrow}{dM} 
+\frac{d V_\textrm{int}^\uparrow}{d n^\downarrow}\frac{dn^\downarrow}{dM} 
-\frac{d V_\textrm{int}^\downarrow}{d n^\downarrow}\frac{dn^\downarrow}{dM} 
-\frac{d V_\textrm{int}^\downarrow}{d n^\uparrow}\frac{dn^\uparrow}{dM} \right) \nonumber\\[0.75em]
&=-\frac{1}{2}\left(F^{\uparrow\uparrow}\frac{dn^\uparrow}{dM} 
-F^{\uparrow\downarrow}\frac{dn^\downarrow}{dM} 
+F^{\downarrow\downarrow}\frac{dn^\downarrow}{dM} 
-F^{\downarrow\uparrow}\frac{dn^\uparrow}{dM} \right)\nonumber \\[0.75em]
\Rightarrow F^{\hat{P}}_\textrm{J}&=-\frac{1}{4}\left(F_\textrm{int}^{\uparrow\uparrow}-F_\textrm{int}^{\uparrow\downarrow}
+F_\textrm{int}^{\downarrow\downarrow}-F_\textrm{int}^{\downarrow\uparrow}\right).
\label{eq:j_interaction}
\end{align}
%
In the last line we have defined the 
interaction kernel $F^{\hat{P}}_\textrm{J}$ 
as the average {\it difference} between subspace-screened 
parallel and anti-parallel spin interactions, 
using 
%
\begin{align}
n^\uparrow=\frac{1}{2}(N+M)
\quad&\mbox{and}\quad
n^\downarrow=\frac{1}{2}(N-M)
\quad \Rightarrow\quad
\frac{dn^\uparrow}{dM}=-\frac{dn^\downarrow}{dM}=\frac{1}{2}.
\end{align}


The connection between the definitions 
of $U$,  presented in Eq.~\eqref{eq:u_interaction}, 
and $J$ now becomes clear, 
where the former is a measure of 
the average on-site interaction between all spins, 
and the latter is the average on-site difference  
between parallel and anti-parallel spin interactions.
%
{
We also remind the reader 
that the DFT+$U$ functional, 
presented in Eq.~\eqref{eq:dft+u_functional}, 
only explicitly affects like-spin interactions.}
%
Thus, the practice of incorporating 
$U-J$ now makes sense 
as a correction for the average like-spin interaction, 
as seen by the role of the DFT+U functional 
in Eq.~\eqref{eq:dft+u_functional}, 
%
\begin{equation}
U^I_\textrm{eff}=U^I-J^I
=\left(F^{\hat{P}}_\textrm{Hxc}-F^{\hat{P}}_\textrm{J}\right)^I
=\frac{1}{2}\left(F^{\uparrow\uparrow}_\textrm{int}+F^{\downarrow\downarrow}_\textrm{int}\right)^I.
\label{eq:u-j_interaction}
\end{equation}

%CORRECTING SCE IN H2
\subsection{Correcting static correlation error in H$_2$}

{
While SCE has not been directly treated by DFT+$J$, 
at least to our present knowledge, 
the association of $J$ with 
the curvature of the constrained energy 
with respect to magnetisation 
in Eq.~\eqref{eq:j_curvature},  
alludes to an intriguing prospect.}
%
If we consider Fig.~\ref{fig:H2_sce}, 
we observe that the SCE is expressed  
by a deviation from piece-wise constancy~\cite{doi:10.1063/1.2987202} 
with respect to total spin magnetisation 
in the form of a spurious curvature.
% 
{
We therefore surmise that SCE 
may be effectively treated 
by an energy correction
that vanishes at integer values 
of the spin-density and 
contributes a non-zero correction otherwise, 
which may be expressed in the form}
%
\begin{equation}
E_J=\frac{1}{2}\sum_{I\sigma}J^In^{I\sigma}(n^{I\sigma}-1).
\end{equation} 
%
This expression is precisely the 
double-counting term present 
in Eq.~\eqref{eq:double_counting} 
(up to a minus sign).
%
It therefore begs the question 
if a DFT+$J$ functional can be used effectively to this end.

We applied the variational method 
to calculate $J$ in spin-polarised calculations 
of dissociating H$_2$, 
with the intention of treating the SCE.
%
{
The resulting values were again 
calculated with excellent precision, 
within $10^{-3}$~eV}, 
where it transpired that for short bond-lengths, 
an excessively high exchange term was computed, 
e.g, 
$J\approx$150~eV at 1.0~a$_0$, 
due to the large responses induced 
by strongly overlapping 1$s$ orbitals.
%
These values quickly decayed with 
the internuclear separation, however, 
as the orbital overlap reduced 
and measured $J=4.3$~eV at 2.8~a$_0$.
%

However, beyond the 
Coulson-Fischer point~\cite{doi:10.1080/14786444908521726} 
at $\approx$3.2~a$_0$,  
$J$ can not be computed since the magnetic potential, 
given by $V^\uparrow_\textrm{KS}-V^\downarrow_\textrm{KS}$, 
switches from a stable to an unstable equilibrium.
%
\edit{This is illustrated below in Fig.~\ref{fig:H2_vplots} 
where we have computed the potential at 1~a$_0$, 
which clearly exhibits a stable equilibrium with respect to 
the magnetisation of a H atom.
%
Meanwhile, 
the system is clearly unstable at 4.8~a$_0$.
%
Hence, any magnetic perturbation 
that is applied beyond $\approx$3.2~a$_0$ 
immediately causes the system to fully polarise, 
wherein we depart from the linear-response regime,
thereby making a calculation of $J$ infeasible with this method.}
%
\begin{figure}[th!]
\centering
\subfloat[]{
\includegraphics[height=0.45\textwidth]{images/H2_vplot1}
\label{fig:H2_vplot1}}
\quad
%
\subfloat[]{
\includegraphics[height=0.45\textwidth]{images/H2_vplot2}
\label{fig:H2_vplot2}}
%
\caption[Stable and unstable magnetic potentials for H$_2$ at 1~a$_0$ and 4.8~a$_0$ bond-length]
{The stable and unstable magnetic potentials for H$_2$ at  
\subref{fig:H2_vplot1} 
 1~a$_0$ and
\subref{fig:H2_vplot2} 
4.8~a$_0$ bond-length 
illustrating the transition from 
a stable to an unstable equilibrium 
beyond the Coulson-Fischer point.}
\label{fig:H2_vplots}
\end{figure}


\edit{To effectively calculate $J$ for larger bond-lengths, 
and toward the dissociation limit, 
it will be necessary to supplement 
the system with an external potential 
to impose a stable equilibrium 
and mitigate this tendency to polarise.
%
The potential we will require 
to do this is none other than the second term in 
Eq.~\eqref{eq:dft+u+j_functional}, 
or just Eq.~\eqref{eq:dft+j_functional}, 
in conjunction with a {\it negative} $J$.
%
Crucially, this additional potential 
can penalise the polarisation and therefore  
increase the slope of the magnetic potential  
with respect to magnetisation 
to ensure a stable equilibrium once more.}

{This proposed strategy  
likens to that of the incipient practice of 
calculating self-consistent Hubbard $U$ 
parameters~\cite{PhysRevLett.97.103001,PhysRevB.81.245113,0953-8984-22-5-055602,cococcioni4}, 
which has aided in providing an improved 
description of many systems.
%
This method entails computing 
$U_\textrm{out}$ values in a system 
for incremental values of an applied $U_\textrm{in}$ 
until some pre-defined self-consistency condition has been met.
%
The self-consistency procedure will therefore feature prominently 
in our attempt to effectively calculate the $J$ parameter in dissociating H$_2$, 
and will be a central aspect of the following Chapter.}



%CALCULATING J FOR NiO
\subsection{Calculating J for NiO and Cr$_2$O$_3$}

For now, 
let us revisit NiO and Cr$_2$O$_3$ 
and calculate the $J$ term for these systems.
%
At the start of this section, 
we outlined how to calculate $J$ 
via magnetic perturbations of the form 
$\hat{V}^{I\sigma}_\textrm{ext}=\pm\beta\hat{P}^{I\sigma}$.
%
However, 
for spin-polarised systems, 
we may also calculate $J$ 
from the magnetic perturbations induced 
from the uniform perturbation 
$\hat{V}^I_\textrm{ext}=\alpha\hat{P}^{I}$, 
since nowhere in Eq.~\eqref{eq:j_definition} 
does $\beta$ feature explicitly.
%
%While the uniform perturbation does not generate 
%the minimal energy deviations 
%for these specific levels of magnetisation, 
%this indirect approach is sufficient 
%for the sake of convenience in our current calculations.
%
{
In other words, 
while the uniform external potential 
will not yield the lowest-energy states consistent 
with a given non-trivial magnetisation~\cite{PhysRevB.94.035159}, 
the magnetisation induced can nonetheless 
be used as a proxy.
%
This convenient workaround to evaluate $J$, 
from the same set of calculations we previously 
performed to determine $U$, 
approximates the result of the  
variational linear-response calculations 
we would otherwise have to perform.}

\subsubsection{NiO DFT+$U$+$J$ results}

Using the same set of calculations from the previous section, 
we determined $J=0.84\pm0.01$~eV 
and applied it in conjunction with $U=6.7$~eV 
in the full  DFT+$U$+$J$ functional 
in Eq.~\eqref{eq:dft+u+j_functional}.
%
In comparison to the previous DFT+$U$ calculation, 
the resulting band gap reduced slightly to 2.95~eV, 
while the magnetisation increased to 1.66~$\mu$B.
%
{
We again refer the reader to the summary of results 
presented in Table~\ref{table:multi_electron_results}.}

In Fig.~\ref{fig:NiO6.7Jdos} we present 
the resulting DOS, 
where the combined $U+J$ treatment 
{has increased the contribution} of Hubbard states 
in the valence bands between 
-4~eV and -1.75~eV, 
which now {feature} as much as 
the O states.
%
Moreover, 
there seems to be no qualitative difference 
in the conduction bands, 
however they have migrated upwards in energy, 
where the peak now lies at 
4.87~eV above the Fermi energy.
%
The profile of the dispersive $s$-states, 
meanwhile, seems relatively unaffected.

\begin{figure}[th!]
\centering
\includegraphics[height=0.494\textwidth]{images/NiO.6.7.J.dos.pdf}
\caption[Species-resolved DOS for NiO calculated with PBE+$U$+$J$]
{The species-resolved density-of-states of NiO 
for $U$=6.7~eV and $J$=0.84~eV.
%
The lines depict the same as in Fig.~\ref{fig:NiO6.7dos}.
%
The $J$ term has marginally decreased the band gap, 
and increased the magnetisation.
%
The conduction bands, 
along with deep valence states, 
have migrated upwards in energy, 
where the peak now occurs at 4.87~eV, 
while the dispersive $s$-state remains unaffected.
}
\label{fig:NiO6.7Jdos}
\end{figure}

% Cr2O3 DFT+U+J results
\subsubsection{Cr$_2$O$_3$ DFT+$U$+$J$ results}

Similarly, 
we calculated $J=0.435\pm0.004$~eV 
and applied it with $U=2.8$~eV from the same calculations 
in the DFT+$U$+$J$ functional.
%
%
Here 
the band gap is 2.06~eV 
and the magnetic moment is 2.59~$\mu$B
which only provides a marginal  
improvement in the latter  
compared to the DFT+$U$ results 
presented in Fig.~\ref{fig:cr2o3.2.8.dos}.

\begin{figure}[th!]
\centering
\includegraphics[height=0.494\textwidth]{images/cr2o3.2.8.J.dos.pdf}
\caption[Species-resolved DOS for Cr$_2$O$_3$ calculated with PBE+$U$+$J$]
{As in Fig.~\ref{fig:cr2o3.2.8.dos}. 
The species-resolved DOS of Cr$_2$O$_3$, 
calculated with LDA+$U=2.8$+$J=0.44$.
%
There exists little discernible difference 
between these two regimes.}
\label{fig:cr2o3.2.8.J.dos}
\end{figure}

{
We therefore conclude that the $J$-term  
computed in this alternative fashion, 
when implemented in the DFT+$U$+$J$ functional, 
does not produce any new qualitative features.
%
In fact, 
the band gaps, magnetic moments 
and DOS profiles remained largely comparable 
between the different regimes.
%
But without the relevant and experimental data, 
or the explicit evaluation of $J$ from magnetic perturbations, 
it is difficult to precisely ascertain the difference.
%
However, 
given that $J$ is essentially free to compute from the $U$ calculations, 
we maintain that there is still an advantage to be gained 
from computing it, 
such as the knowledge of its approximate magnitude.}

{
Finally, in Table~\ref{table:cr2o3_summary}, 
we present a summary of the results for Cr$_2$O$_3$
obtained in this and other studies, 
in comparison to experimental data.}
\newpage 

\begin{table}[th!]
\centering
\begin{tabular}{lrrrr}
\hline\hline
Calculation 	&$U$ (eV) &$J$ (eV)&$E_g$ (eV)&$m$ ($\mu$B)\\
\hline
Experiment	& -		& - 		& 2.1$^\textrm{\cite{PhysRevB.91.125202}}$
									&2.76$^\textrm{\cite{doi:10.1063/1.1714118}}$ \\
Refs.~\cite{PhysRevB.70.125426,C0JM03852K,doi:10.1021/jp5039943,Mosey2009287,doi:10.1021/jp5039943}	
			& 5.0		& 1.0		& 2.6		& 3.01\\
Ref.~\cite{PhysRevB.76.155123,doi:10.1063/1.4865780,doi:10.1063/1.2943142,Mosey2009287}	
			& 3.3 	& 0.1 	& 2.9 	& 2.9 \\
Ref.~\cite{PhysRevB.71.035105}
			& 4.0 	& 0.58 	& 2.88 	&3.26\\
\hline
LDA			& 0.0		& 0.0 	&1.27	&2.30\\
LDA+$U$		& 2.8 	& 0.0		&2.10	&2.56\\ 
LDA+$U$+$J$	& 2.8 	& 0.44	&2.06	&2.59\\ 
\hline\hline
\end{tabular}
\caption{Summary of experimental and 
computational results of Cr$_2$O$_3$, 
including band gap $E_g$ (eV) 
and magnetic moment $m$ ($\mu$B), 
and the $U$ (eV), $J$ (eV), where applicable.}
\label{table:cr2o3_summary} 
\end{table}


%CONCLUSION
\section{Conclusion}

{
In this Chapter, 
we developed a modification to the linear-response method  
for calculating the Hubbard $U$ and Hund's $J$ parameters 
from a variational approach, 
such that it is equally applicable in 
direct-minimisation and SCF codes alike.
%
We outlined the technical similarities and 
stated the key differences between the two methods, 
which amounted to the evaluation of the 
non-interacting response function 
from the ground-state density.
%
The variational framework thus enables 
DFT+$U$ to be placed 
on a first-principles footing within the context 
of direct-minimisation DFT solvers.
%
This prospect has important consequences for the 
calculation of DFT+$U$  total-energy differences 
{featured in high-throughput calculations~\cite{jain2011high,setyawan2010high,curtarolo,Curtarolo2012218,PhysRevX.5.011006}, 
phase diagrams~\cite{PhysRevB.87.115111,PhysRevB.84.045115,doi:10.1021/cm702327g}, 
heterogeneous catalysis~\cite{Bliem1215},
thermodynamical properties~\cite{PhysRevB.75.035109,PhysRevB.78.075125,PhysRevB.75.035115,1674-1056-17-4-035,doi:10.1021/cm702327g,CapdevilaCortada201558,PhysRevB.85.155208,PhysRevB.84.045115,PhysRevB.85.115104,PhysRevB.90.115105}, 
and spin-triplet splittings~\cite{Korshunov2004,PhysRevB.55.12829,PhysRevB.77.045118}}. 
}

{
We tested our method extensively on dissociating H$_2^+$ 
and successfully corrected the total-energy curve to remarkable accuracy, 
thereby demonstrating for the first time that DFT+$U$ is capable of correcting one-electron SIE under ideal conditions.
%
We proceeded to calculate highly accurate $U$ values 
for multi-electronic systems, Ni(CO)$_4$, NiO, and Cr$_2$O$_3$, 
with which we achieved broad agreement 
with experimental measurements of the 
HOMO-LUMO and optical gaps, 
magnetic moments, 
and DOS profiles. 
%
For both NiO, and Cr$_2$O$_3$, 
we also successfully calculated a $J$ term 
from the same set of calculations 
that produced a marginal difference.
%  
Finally, we compared our numerical results for  Cr$_2$O$_3$,
with high-resolution XPS, UPS and absorption measurements 
obtained from experimental colleagues, 
for which our calculations successfully 
reproduced the qualitative features.}

{
The successful employment of our variational method  
in testing conventional and novel structures 
therefore demonstrates that it 
is versatile, robust and effective.
%
It enables the reliable reproduction 
of both quantitative and qualitative 
features in solids and molecules 
that systemically suffer from many-body SIE.}

{
However,  
we illustrated with dissociating H$_2$ that DFT+$U$ is an inappropriate 
correction scheme for systems dominated by SCE.
%
We postulated, however, that a correction of the form presented 
by  DFT+$J$ may be an adequate remedy.
%
Unfortunately, this hypothesis could not be confirmed, 
since stable magnetic perturbations were intractable 
beyond the Coulson-Fischer point.
%
We proposed a possible solution to this problem   
that consists of supplementing the subspace orbitals 
with a $J_\textrm{in}$ correction 
in order to calculate a $J_\textrm{out}$.
%
This goal, and the pursuit of 
a self-contained mechanism for performing 
thermodynamics within the DFT+$U$ framework, 
provides the motivation for the 
development of a self-consistency scheme, 
presented in the next Chapter.}

%{
%The practice of calculating of interaction parameters 
%in the presence of an applied $U$ 
%is a relatively recent idea that bears 
%significant theoretical and conceptual consequences.
%%
%In the next Chapter, 
%we will incorporate the facility to calculate  
%self-consistent parameters 
%into the variational linear-response framework, 
%and investigate the various self-consistency criteria.
%%
%Of particular interest to our cause is what effect, if any, 
%will an applied parameter have 
%on the evaluation of the other, 
%since the simultaneous calculation of both $U$ and $J$ 
%has, to date, not been explored.
%%
%These questions will be the subject of the next Chapter.}






%
%In the SCF LR method it is calculated 
%from the density after the first self-consistency cycle, 
%which we have argued is an excited state of the system 
%and therefore not conducive to the calculation of 
%ground-state parameters.
%
%In contrast, 
%we define the non-interacting response function 
%as that derived from the Kohn-Sham potential, 
%which is calculated as a strict ground-state property 
%in conjunction with the interacting response, 
%thereby elevating the $U$ parameter to a ground-state property.
%
%This method puts DFT+$U$ on a first-principles footing
%within the context of direct-minimisation DFT solvers, 
%even the linear-scaling schemes  
%of the type now routinely used to simulate systems which
%are simultaneously spatially and electronically 
%complex~\cite{skylaris2005introducing,hine2009linear,hine2010linear,Gillan200714,0953-8984-14-11-303,0953-8984-14-11-302,0953-8984-20-6-064208,Genovese2011149,PhysRevB.72.045121,doi:10.1063/1.2841077}
%%%
%
%%
%Moreover, 
%we discussed the importance of the correct kernel projection, 
%to ensure the desired properties in the calculated parameters, 
%and the contribution of subspace SIE to the global error.
%%

%We also expressed the $U$ and $J$ parameters 
%in terms of the net average 
%and net difference of spin-dependent interaction kernels, 
%respectively, 
%subsequent to which we highlighted the physical significance 
%of the effective $U$ parameter, 
%given by $U-J$, 
%in concerning like-spin interactions only.
%
%In the case of the bulk systems, 
%we were able to reliably restore 
%the qualitative features of the 
%species-resolved DOS.
%
%For NiO, 
%we restored the charge-transfer behaviour, 
%magnetisation, and, 
%furthermore, successfully calculated a $J$ term 
%from the same set of calculations.
%
%In the case of chromia, meanwhile, 
%the computed valence DOS 
%compared very well 
%with high-resolution XPS and UPS measurements 
%obtained from experimental colleagues, 
%where the magnetisation and band gap 
%were also accurately determined.

