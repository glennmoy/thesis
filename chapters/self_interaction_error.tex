%THE SELF INTERACTION ERROR

\lett{I}{n the preceding Chapters}, 
we extensively explored 
a set of calculations on 
puckered, two-dimensional materials, 
and classified broad trends in 
electro-mechanical properties 
while making comparison 
to experimental and computational data.
%
However, specific quantities, 
particularly those derived 
from electronic band structures, 
were found to be 
significantly miscalculated and, 
in some instances, 
in direct contradiction to experiments.
%
These erroneous findings are not just limited 
to the group-V materials but 
are a {pervasive} feature of all DFT 
electronic structure calculations 
when performed with {commonplace}  
exchange-correlation (XC) functionals.

In spite of the numerous, 
yet justifiable, approximations 
introduced in the formulation of Kohn-Sham DFT 
in Chapter~\ref{ch:qm_simulation}, 
none affect the qualitative reliability of calculations 
more so than the approximation to the XC functional.
%
Although DFT is, 
in principle, an exact theory, 
the semi-empirical determination 
of the XC functional used in practical calculations 
fails to fully account for 
the nuanced many-body effects 
found in various chemical environments.
%
The infamous 
many-body self-interaction error (SIE)~\cite{PhysRevB.23.5048,doi:10.1063/1.476859,doi:10.1063/1.463297,seminario1996recent,PhysRevB.56.16021}, 
or delocalisation error~\cite{cohen2008insights,doi:10.1021/cr200107z,PhysRevLett.100.146401}, 
and static correlation error (SCE)~\cite{seminario1996recent,PhysRevLett.87.133004,doi:10.1063/1.1589733,cohen2008insights,doi:10.1063/1.2987202,doi:10.1021/cr200107z,PhysRevB.77.115123,doi:10.1021/ct8005419,PhysRevA.88.030501,PhysRevA.85.042507}
are extensively documented.
%
They are responsible for the systematic inaccuracies in 
electronic and optical band gaps~\cite{QUA:QUA560280846}; 
charge transfer excitation energies~\cite{doi:10.1021/jp9533077}; 
chemical reaction barriers~\cite{JOHNSON1994100,JURSIC1996603,BAKER199553};
binding energies~\cite{doi:10.1063/1.463297,doi:10.1021/jp972378y};
and electric field responses~\cite{:/content/aip/journal/jcp/119/21/10.1063/1.1630011}.
%
In fact, the SIE is arguably the largest obstacle  
{preventing DFT from attaining 
the superior level of accuracy 
possible in quantum chemistry calculations}. 
%
Understanding the origin of the SIE 
is therefore crucial to the 
construction of better XC functionals 
and the development of practical remedies.

%
In this Chapter, 
we shall describe the nature 
and origin of the SIE and SCE 
as they relate to the violation of conditions 
on the exact XC functional 
pertaining to fractional occupancies 
and fractional spins, respectively~\cite{cohen2008insights}. 
%
We shall employ the examples of the 
dihydrogen cation H$_2^+$ 
and hydrogen molecule H$_2$ 
as test models
in order to {understand}  
the effect of these errors 
in strongly-correlated systems. 

Moreover, 
we will outline corrective procedures developed 
to ameliorate these errors, 
{beginning with the formative 
work of Perdew and Zunger~\cite{PhysRevB.23.5048}, 
in constructing    
self-interaction corrections.
%
We will also emphasise  
the efforts of Dabo and co-workers~\cite{PhysRevB.82.115121,PhysRevB.90.075135,PhysRevLett.114.166405,nguyen2017koopmans} 
to restoring compliance with 
the generalised Koopmans' condition~\cite{KOOPMANS1934104,PhysRev.123.420,PhysRevB.82.115121,PhysRevB.90.075135}.}
%
{
We will then discuss 
the establishment of the widely successful 
DFT+$U$ (DFT + Hubbard $U$) method~\cite{PhysRevB.43.7570,PhysRevB.44.943,
PhysRevB.48.16929,PhysRevB.50.16861,PhysRevB.52.R5467,PhysRevB.57.1505,
PhysRevB.58.1201,PhysRevB.71.035105,QUA:QUA24521}
for treating one-electron SIE, 
which finds its roots in the 
treatment of strongly-correlated Mott-Hubbard systems, 
drawing from the work of 
Anisimov~\cite{PhysRevB.43.7570,PhysRevB.44.943,PhysRevB.48.16929,PhysRevB.50.16861}, Dederichs~\cite{PhysRevB.41.514,PhysRevB.50.16861,PhysRevB.49.6736}, and Gunnarsson~\cite{PhysRevB.39.1708,PhysRevB.43.7570,PhysRevB.37.9919,0295-5075-7-2-013,PhysRevLett.65.1148,PhysRevLett.68.1900}, 
and later revised by Liechtenstein and Dudarev~\cite{0953-8984-9-4-002,PhysRevB.52.R5467,PhysRevB.56.4900,0953-8984-9-35-010}, 
and Himmetoglu and Cococcioni~\cite{PhysRevB.71.035105,PhysRevB.84.115108}.
%
We find that DFT+$U$ may be well placed 
to correct one-electron SIE, 
when supplied with a sufficient $U$ parameter 
and under the appropriate population analysis.
%
However, it fails entirely 
to restore Koopmans' compliance.
%
Finally, we propose that 
the generalised DFT+$U$+$J$ functional 
may potentially be used to correct the SCE.}

%FRACTIONAL CHARGES
\section{Fractional charges}
\label{sec:fractional_charges}

%%%
Perdew~\cite{PhysRevLett.49.1691}, 
Zhang~\cite{doi:10.1063/1.476859}, 
and Ayers~\cite{Ayers2008}, 
cumulatively demonstrated that the SIE 
is best understood 
from the perspective {of} non-integer particle numbers 
 $E(N+q)$.
%
This formalism stems from 
the well-known failure of the XC energy 
$E_\textrm{xc}[n]$ 
to sufficiently cancel 
the Hartree energy $E_H[n]$ 
in fractionally occupied orbitals~\cite{PhysRevLett.100.146401}.
%
A somewhat heuristic argument 
can also be made by noting that the Hartree energy 
in Eq.~\eqref{eq:classical_hartree_energy} 
is incorrectly non-zero for one electron, 
as noted by Fermi and Amaldi~\cite{fermi1934mem}.
% 
The exact energy of a 
fractional-charge system  
was proven in Ref.~\cite{PhysRevLett.49.1691} 
to be a piece-wise linear interpolation
between the adjacent integer values (Fig.~\ref{fig:h2+_sie})
%
\begin{equation}
E(N+q)=(1-q)E(N)+qE(N+1) 
\quad\mbox{for}\quad
0\leq q\leq 1. 
\label{eq:linearity_condition}
\end{equation}
%
Deviations from this {\it linearity condition} 
are, therefore, indicative of a residual SIE 
that is affiliated with a spurious curvature   
in the total-energy vs particle number profile, 
as illustrated in Fig.~\ref{fig:h2+_sie}.

 \begin{figure}[th!]
 \centering
 \includegraphics[height=0.494\textwidth]{images/h2+_sie.pdf}
 \caption[SIE of a H atom as the deviation from piece-wise linearity]
 {The calculated SIE of a H atom 
 is shown by the deviation 
 of the PBE total-energy (dashed) 
 from the correct piece-wise linearity (solid)
 between integer particle number.}
 \label{fig:h2+_sie}
 \end{figure}
%%%


The concept of non-integer electrons  
makes very little sense from a chemical perspective, 
as no fractionally-charged systems exist in nature, 
yet they provide a useful contrivance  
for the diagnosis of XC functionals.
%and are indirectly accessible in systems 
%comprising {an odd number of} electrons.
%
The study of SIE, therefore, 
is often {constructed} from the dissociation
of homo-nuclear diatomic ions.
%namely H$_2^+$, He$_2^+$, and F$_2^+$.
%
{While H$_2^+$ is not a strongly-correlated system, 
it is the simplest 
one-electron system lacking in any  
extraneous multi-reference 
and static correlation error effects, 
and thus permits a systematic analysis 
of the SIE relevant only to 
the underlying XC functional.
%
Moreover, 
the exact binding curve can be calculated  
with the same code and pseudopotential 
by omitting contributions 
from the Hartree and XC potentials, 
and therefore allows a direct 
comparison between energies.

The binding curve of H$_2^+$, 
calculated in {\sc ONETEP}
with a hard ($0.65$~a$_0$ cutoff) 
norm-conserving pseudopotential~\cite{PhysRevB.41.1227}, 
{a PBE XC functional}, 
$10$~a$_0$ NGWF cutoff radius, 
and open boundary conditions~\cite{:/content/aip/journal/jcp/110/6/10.1063/1.477923}, 
is presented in Fig.~\ref{fig:h2+_total_energy}.
%
{We note that any attempt to extend our 
bond-length interval beyond $8.5$~a$_0$, 
in this and subsequent calculations, 
resulted in numerical instabilities 
due to the near-degeneracy of 
the Kohn-Sham $\sigma$ and $\sigma^*$ 
eigenstates, and shown here are  the results only
of well-converging calculations.}


\begin{figure}[th!]
\centering
\includegraphics[height=0.494\textwidth]{images/h2+_total_energy.pdf}
\caption[Binding energy curve of H$_2^+$]
{Binding energy of H$_2^+$ 
calculated with exact (solid) and PBE (dashed) XC functionals.
%
{The reference energy for each curve 
is that of the  isolated H atom calculated with 
the corresponding XC functional.}
%
The significant failure of PBE functional 
is evident in the dissociation limit.}
\label{fig:h2+_total_energy}
\end{figure}
%
%In the dissociation limit, 
%the physical solution 
%is a linear combination of states 
%composed of alternately filled H atoms 
%%
%\begin{equation}
%\Psi\sim 1s_A + 1s_B
%\end{equation}

\edit{The occupancy of each H atom, 
calculated respectively from the exact and PBE 
1$s$ orbital subspace projectors, 
closely match the KS orbitals 
and are reasonably well described 
as seen in Fig.~\ref{fig:h2+_occ}.
%
The subspace occupancy acquires values greater 
than one half due to the overlap of the orbitals, 
which results in double-counting of the occupancy.
%
Furthermore, 
when the KS density is not 
completely enclosed by the subspace orbitals,  
particularly for the exact calculations, 
the resulting charge spillage causes 
the subspace occupancy to be less than one half.}

\begin{figure}[th!]
\centering
\includegraphics[height=0.494\textwidth]{images/h2+_occupancy.pdf}
\caption[Occupancy of H atom in dissociating H$_2^+$]
{Average total occupancy of the H atoms in dissociating H$_2^+$ 
calculated with exact (solid) and PBE (dashed) 
and PBE+$U$ (open circles) XC functionals.
%
The population analysis was performed on the 
1$s$ orbital subspace projectors, 
which closely match the KS orbitals.}
\label{fig:h2+_occ}
\end{figure}
%

The spurious repulsion between the symmetrically charged ions 
is then {caused} by the {semi}-local 
PBE XC functional (since $E_\textrm{xc}[n]\leq 0$),
which significantly lowers the total-energy.
%
The superfluous repulsion 
imposes artificial electron delocalisation, 
manifested by the curvature 
of the total-energy vs occupancy profile, 
shown in Fig.~\ref{fig:h2+_sie}, 
and {is} the reason 
for an incorrect total-energy.
%
%In fact, it is clear that 
%the SIE is most potent where 
%the occupancy is $1/2$, 
%in the dissociation limit, 
%and least concerning where the occupancy 
%is near $1$ at short bond-lengths.
%
Moreover, 
the importance of fractional charges 
in chemical processes such as 
charge transfer, molecular binding 
and spectroscopic properties,  
{alludes to} the effect of SIE 
in these scenarios as well,   
for which many remedies 
have been proposed.


%ONE ELECTRON SIC
\section{One-electron self-interaction correction}
\label{sec:one_electron_sic}

Over the years, 
a number of methods have been devised 
for correcting many-body SIE~\cite{doi:10.1021/ja8087482,
doi:10.1063/1.3269030,
doi:10.1063/1.3269029,
PhysRevLett.100.146401,
RevModPhys.80.3,
doi:10.1021/ct8005419,
PhysRevB.67.125109,
PhysRevB.71.035105,
PhysRevB.71.205210,
doi:10.1063/1.2387954,
0953-8984-19-10-106206,
PhysRevB.78.125116,
PhysRevB.77.155106,
doi:10.1063/1.452616,
PhysRevB.82.115121}, 
many of which take direct inspiration 
from that proposed by Perdew and Zunger 
(PZ)~\cite{PhysRevB.23.5048}.
%
PZ first laid out a {formalism for} a 
self-interaction correction (SIC) 
based on the individual removal of 
one-electron SIE 
from fully-occupied spin-polarised 
KS-orbitals 
$\{\psi_{i\sigma}\}$, 
with occupancies 
$n_{i\sigma}=\left|\psi_{i\sigma}\right|^2$, 
via a correction term to the total-energy 
given by
%
\begin{align}
E_\textrm{PZ}&=E_\textrm{DFT}+\sum_{i\sigma}\Delta_{i\sigma}^\textrm{SIC} \nonumber\\[0.5em]
\mbox{with}\quad
\Delta_{i\sigma}^\textrm{SIC}
&=E_H[n_{i\sigma}]+E_\textrm{xc}^\textrm{approx}[n_{i\sigma},0].
\end{align}
%
{Here, $E_\textrm{xc}^\textrm{approx}[n_{i\sigma},0]$ 
is the approximate exchange energy functional 
where one spin-channel 
has an occupation $n_{i\sigma}$ 
and the other is empty.}
%
This corrective approach, 
based on the removal of 
single-orbital SIE from a system, 
became a popular precept for 
the treatment of many-body SIE~\cite{doi:10.1063/1.2204599,doi:10.1063/1.2176608}.
%for which a formal mathematic description 
%remains elusive.
%
The PZ SIC 
produced remarkable 
improvements for single atoms, 
but tended to over-correct the SIE 
in polyatomic systems, 
leading to inaccurate binding energies 
and under-estimated bond-lengths~\cite{PhysRevA.55.1765,PhysRevB.82.115121,doi:10.1021/cr200107z}.
%
This result prompted   
the development of several modifications~\cite{doi:10.1063/1.2387954,doi:10.1063/1.2176608,doi:10.1063/1.2204599,doi:10.1063/1.2566637} 
that scaled down the correction term 
by a factor $\alpha$
%
\begin{equation}
E_{\alpha PZ}=E_\textrm{PZ}+\alpha\left(E_\textrm{PZ}-E_\textrm{DFT}\right), 
\end{equation}
%
which yielded modest improvements.
%
Regardless, 
{an initial} drawback 
of orbital-dependent SIC, 
such as those used by PZ, 
is that the Hamiltonians are no longer invariant 
under unitary transformations, 
and so precluded 
the treatment of periodic systems,
such as crystals and extensive molecules~\cite{RevModPhys.80.3}.
%
{
Later work by 
Svane and Gunnarsson~\cite{PhysRevB.37.9919,0295-5075-7-2-013,PhysRevLett.65.1148,PhysRevLett.68.1900}, 
and Szotek and Temmerman~\cite{PhysRevB.47.4029,SZOTEK199119,PhysRevB.47.11533,PhysRevB.52.R14316,PhysRevLett.79.3970,PhysRevB.71.205109}, 
extended the method to include the treatment of solids.}


%KOOPMANS COMPLIANCE
\section{Koopmans' compliance}
\label{sec:koopmans_condition}

%The SIE is not restricted to 
%the miscalculation of total-energies 
%but can also adversely effect 
%other quantities such as 
%ionisation energies and band gaps.
%
The basis of all chemical reactions 
relies on the addition and removal of electrons 
in a system.
%
The chemical potential\footnote{$\mu$ 
also acts as the Lagrange multiplier 
conserving particle number 
in Eq.~\eqref{eq:hk_total_energy_functional}}
$\mu$ quantifies the difficulty 
of this process 
and is defined as the derivative of 
the total-energy $E(N)$
with respect to particle number $N$ 
for a fixed external potential $v$
%
\begin{equation}
\mu_N=\left(\frac{\partial E(N)}{\partial N}\right)_v.
\end{equation}
%
For exact functionals, 
%where the linearity condition in Eq.~\eqref{eq:linearity_condition} holds, 
the chemical potential assumes 
constant values between 
adjacent integer particle numbers 
(shown in Fig.~\ref{fig:h2+_sie}), 
given by the first ionisation energy $I_N$ 
and electron affinity $A_N$, respectively, 
for a neutral atom
%
\begin{equation}
\mu_N=\left\{\begin{array}{c} 
-I_N=E(N)-E(N-1)\\
-A_N=E(N+1)-E(N)
\end{array}\right. .
\end{equation}
%
Mulliken's electronegativity 
is defined as the average chemical potential 
at a given occupancy 
%
\begin{equation}
\chi_N=\frac{I_N+A_N}{2} = \frac{E(N-1)-E(N+1)}{2}
\end{equation}
%
and dictates the tendency 
of the atom 
to gain or lose electrons 
in chemical reactions.
%
Meanwhile, 
the fundamental gap 
for extended solids 
is the difference of these quantities 
%
\begin{align}
\epsilon_N^\textrm{gap}&=I_N-A_N\nonumber\\
				&=[E(N-1)-E(N)]-[E(N)-E(N-1)]\nonumber\\
				&=E(N-1)+E(N+1)-2E(N)
\end{align}
%
and plays a key role in 
charge transfer, 
electronic structure 
and the energetics 
of defects and surfaces.
%
In molecules, 
the corresponding quantity is 
the Pearson hardness~\cite{parr1983absolute}.
%

%Cohen {\it et al.}~\cite{PhysRevB.77.115123} 
%showed that the extension of this procedure to the KS DFT formalism, 
%under an in-variational ground-state potential $v_\textrm{gs}$,
%modified the particle number $\delta n_i$ 
%in the frontier ground-state orbits $\{\psi_i^\textrm{gs}\}$
%%
%\begin{equation}
%\mu=\left(\frac{\partial E[\rho(\br)]}{\partial N}\right)
%=\left(\frac{\partial E[\{\psi_i^\textrm{gs},n_i\}]}{\partial n_i}\right)_{\{\psi_i^\textrm{gs}\}}=\varepsilon_f, 
%\end{equation}
%%
%namely, 
%the highest occupied molecular orbital (HOMO) 
%and the lowest unoccupied molecular orbital (LUMO).
%%
%Invoking Janak's theorem~\cite{PhysRevB.18.7165} 
%then allows for the 
%HOMO and LUMO KS eigenvalues 
%to be directly interpreted 
%as the chemical potentials $\mu_N$.


A critical, 
and perhaps defining, 
characteristic of a system free from SIE is 
its compliance with {the generalised} Koopmans'  
theorem~\cite{KOOPMANS1934104,PhysRev.123.420,PhysRevB.82.115121,PhysRevB.90.075135}, 
which states that, 
{for a given system,} 
the energy of the 
highest occupied molecular orbital (HOMO) 
is precisely the negative of the ionisation energy 
$\epsilon_\text{HOMO}=-I$.
%
In a one-electron system, 
such as  H$_2^+$, 
this implies that the total-energy and the 
occupied Kohn-Sham eigenvalue $\varepsilon$
should differ only by the ion-ion energy.
%BY VIRTUE OF EQUATION BLAH IN CHAPTER 2
%
Thus, the binding curve of H$_2^+$ 
should be equivalently accessible 
by calculating either the total-energy directly, 
or by using the occupied eigenvalue 
with the expression  
$E = \varepsilon + E_\textrm{ion-ion}$.
%
Fig.~\ref{fig:h2+_eigenvalues} 
illustrates the strikingly poor 
compliance of KS DFT with 
Koopmans' condition and is, 
yet another, 
well-documented 
consequence of SIE~\cite{PhysRevB.82.115121,
doi:10.1063/1.3269030,
doi:10.1021/ct8005419,
PhysRevB.90.075135,
PhysRevB.81.205209}.
%
%The KS eigenvalues have no physical basis really
%
%Nonetheless, 
%the KS HOMO is, in fact, 
%physically well established~\cite{PhysRevLett.49.1691} 
%but much of the physical reproducibility 
%relies heavily on the quality of the XC functional utilised, 
%in particular its ability to uphold 
%the linearity condition in Eq~\eqref{eq:linearity_condition}, 
%the failure of which implies a SIE.
%
%On the other hand, 
%the KS 
%lowest unoccupied molecular orbital (LUMO)  
%energy  
%bears no physical meaning at all~\cite{PhysRevB.77.115123} - 
%even when computed with an `exact' functional -  
%since the conventional DFT procedure 
%is concerned only with occupied orbitals.
%%
%This is resoundingly 
%apparent in Fig.~\ref{fig:h2+_eigenvalues} 
%where the LUMO eigenvalues of both 
%the PBE and exact functionals 
%fail to establish a physically meaningful, 
%or even well-behaved, profile.
%%
%This result serves as a sobering  
%reminder of the practical limitations of KS DFT 
%to the study of ground-state properties alone.

\begin{figure}[th!]
\centering
\includegraphics[height=0.494\textwidth]{images/h2+_eigenvalues.pdf}
\caption[Eigenvalue-derived binding energy of H$_2^+$]
{Binding energy of H$_2^+$ 
derived from the KS occupied HOMO eigenvalue 
calculated with exact (solid) and PBE (dashed) XC functionals, 
relative to total-energy of an isolated H atom 
{with the respective functional}.
 %
The inability of the PBE functional to replicate 
the exact HOMO is another effect of the SIE.}
 \label{fig:h2+_eigenvalues}
 \end{figure}
%%%

Motivated by the explicit restoration 
of Koopmans' condition 
as a remedy for SIE,  
Dabo~{\it et al}~\cite{PhysRevB.82.115121,PhysRevB.90.075135,PhysRevLett.114.166405,nguyen2017koopmans} 
constructed a PZ-type SIC, 
termed $\alpha$NK.
%
Instead of removing single-orbital SIE, 
the $\alpha$NK correction to the total-energy 
is a measure of the 
unphysical occupation dependence 
of the orbital eigenvalue energies 
in the frozen-orbital approximation
%
\begin{align}
&E_{\alpha\textrm{NK}}=E_\textrm{DFT}+\alpha_\textrm{NK}\Pi_{i\sigma}^u(f) \nonumber \\[0.5em]
\mbox{with}\quad&
\Pi_{i\sigma}^u(f)= \int^{f_{i\sigma}}_0\ df' \left[\epsilon_{i\sigma}^u(f)-\epsilon_{i\sigma}^u(f')\right], 
\end{align}
%
{for which they define their 
generalised Koopmans' condition as 
%
\begin{equation}
\Pi_{i\sigma}^u(f)= 0
\quad\mbox{where}\quad
0\leq f\leq 1.
\end{equation}}
%
Here, the orbital screening is accounted for 
by the coefficient $\alpha^\textrm{NK}$.
%
The $\alpha$NK method 
offers significant improvement 
in atomic and molecular 
ionisation energies, 
photoemission energies, 
binding energies 
and molecular bond-lengths, 
avoiding the over-binding tendency of PZ~\cite{PhysRevB.82.115121}, 
{and has recently been extended 
to solids~\cite{nguyen2017koopmans}.}
%


%STRONGLY CORRELATED SYSTEMS
\section{Strongly correlated systems}
\label{sec:strong_correlation}

In addition to the practical application of 
a single-particle SIC in atoms and finite molecules, 
the emergence of SIE in extended systems, 
comprising first-row transition-metals or lanthanide ions, 
warrants a similarly measured treatment.
%
Indeed, 
many of these compounds 
boast considerable technological and biological significance 
in photovoltaics~\cite{Kasirga2012,PhysRevB.90.165142}, 
high-temperature superconductors~\cite{Dagotto257},
energy storage~\cite{ADMA:ADMA200903328,Zhou2016}, 
and biochemistry~\cite{doi:10.1021/jz3004188,C2CP41836C}, 
where much of the chemically-relevant 
interactions concern the 
highly-localised $3d$ or $4f$ valence electrons.
%
These states tend to exhibit strong correlation 
as a result of their spatial confinement, 
and are hence very much outside the scope  
of approximate XC functionals.

While there is no strict definition 
for strongly correlated systems, 
they are typically characterised  
by a correlation energy comparable 
with the kinetic or Hartree energies  
$E_c[n]\sim U_H[n]$ or $E_c[n]\sim T_s[n]$. 
%
{
Meanwhile, 
Rahm and Hoffman have 
recently proposed a useful metric for 
quantifying different types of chemical bonds~\cite{doi:10.1021/jacs.5b12434},
which may prove to be a plausible measurement  
of correlation energy in solids.}
%Rahm and Hoffman~\cite{doi:10.1021/jacs.5b12434}, 
%meanwhile, 
%have recently proposed a useful metric for 
%quantifying different types of chemical bonds, 
%via a parameter $Q$.
%%
%They find that the absolute value of $Q$ 
%is directly proportional to 
%the correlation energy present in the bonds 
%when compared against a series of similar molecules.
%%
%Going forward, 
%this $Q$-term may prove to be a plausible measurement  
%of correlation energy in solids.}

As a result of invoking 
approximate XC functionals, 
the incurred electron delocalisation 
often produces results 
that are both quantitatively and qualitatively 
inconsistent with experiments, 
including
{underestimated} local moments, 
{misclassified} Mott-Hubbard insulators 
and inaccurate electric field responses.
%
The treatment of the many-body SIE 
in these cases
is concerned with the restoration 
of piece-wise linearity 
in the total-energy functional 
with respect to the localised-orbital occupancy, 
much in the spirit of PZ SIC.
%
{
The modification of PZ-SIC 
to the treatment of solids 
was later enabled~\cite{PhysRevB.37.9919,0295-5075-7-2-013,PhysRevLett.65.1148,PhysRevLett.68.1900,PhysRevB.47.4029,SZOTEK199119,PhysRevB.47.11533,PhysRevB.52.R14316,PhysRevLett.79.3970,PhysRevB.71.205109,0953-8984-21-4-045604}, 
where Stengel and Spaldin~\cite{PhysRevB.77.155106} 
advocated the implementation of Wannier functions, 
and rotationally invariant correction functionals 
in general.}
%
In the following section, 
we will discuss {one such correction 
that is widely renowned, 
known as the} DFT+$U$ method, 
which seeks to correct SIE 
when it is attributed to 
such highly-localised regions.

%DFT+U METHOD
\section{The DFT+{$U$} method}
\label{sec:dft+u_method}

A widely established and 
computationally efficient~\cite{PhysRevB.85.085107} 
scheme for correcting SIE, 
when it may be appropriately 
attributed to highly-localised orbitals, 
is the DFT+$U$ 
method~\cite{PhysRevB.43.7570,PhysRevB.44.943,
PhysRevB.48.16929,PhysRevB.50.16861,PhysRevB.52.R5467,PhysRevB.57.1505,
PhysRevB.58.1201,PhysRevB.71.035105,QUA:QUA24521}, 
{which we have adopted in this dissertation   
as the preferred approach for correcting SIE}.
%
Initially formulated   
by Anisimov and co-workers~\cite{PhysRevB.43.7570,PhysRevB.44.943,PhysRevB.48.16929,PhysRevB.50.16861} 
from the tight-binding Anderson model~\cite{PhysRev.124.41}, 
its initial purpose was to restore  
the Mott-Hubbard effects 
absent in the LDA description of
transition-metal oxides.

In order to confine the electron charge-density 
to the localised-orbitals, 
in which it is assumed 
the correlated electrons interact more strongly 
compared to the electrons in the surrounding bath, 
the DFT total-energy 
is supplemented by an explicit 
on-site $d-d$ Coulomb energy term, 
given by
%
\begin{equation}
E_\textrm{Hub}=\frac{U}{2}\sum_{i\neq j,j}^{\textrm{sites}}n_in_j
{\quad\mbox{where}\quad
\hat{n}_i=\hat{c}_i^\dagger\hat{c}_i}.
\end{equation}
%
{
The $d$-shell occupancies $\{n_i\}$ 
are the number operators 
defined by the creation and annihilation operators 
on the subspace~\cite{PhysRev.124.41}, 
$\hat{c}_i^\dagger$ and $\hat{c}_i$, 
respectively.}
%
The on-site interactions described by $E_\textrm{Hub}$, however, 
are already present in an average sense 
in the total-energy $E_\textrm{DFT}$, 
stemming from the cumulative interaction 
between the total number 
of $d$ electrons present $N=\sum n_i$.
%
The proto-DFT+$U$ functional
was therefore constructed from the addition 
of subsidiary $d-d$ interaction energies 
to the KS total-energy functional,  
mitigated by the removal of the 
average on-site Coulomb energy 
in terms of the double counting correction $E_\textrm{dc}$
%
\begin{align}
E[n(\br)]&=E_\textrm{DFT}[n(\br)]+E_\textrm{Hub}-E_\textrm{dc}\nonumber\\[0.5em]
&=E_\textrm{DFT}[n(\br)]+\frac{U}{2}\sum_{i,j\neq i,}^{\textrm{sites}}n_in_j-\frac{U}{2}N(N-1).
\label{eq:hubbard1}
\end{align}
%
Here $n(\br)$ represents 
the conventional KS electron density. 
%
The resulting functional in Eq.~\eqref{eq:hubbard1} 
invariably awarded better 
experimental agreement to the calculation of  
transition-metal oxides~\cite{PhysRevB.44.943,PhysRevB.48.16929},   
and transition-metal impurities~\cite{PhysRevB.49.6736,PhysRevB.50.16861}.
%
Nonetheless, 
the functional was still orbital dependent  
and sensitive to the choice of basis 
representing the localised electrons.

A rotationally-invariant form
was later established 
by Liechtenstein {\it et al}~\cite{PhysRevB.52.R5467,PhysRevB.56.4900,0953-8984-9-35-010} 
in terms of the elements of 
a subspace occupancy matrix 
of the localised orbitals 
$\{n_{mm'}^{I\sigma}\}$,
given by  
%
\begin{equation}
%E_\textrm{DFT+$U$}[n(\br),\{n_{mm'}^{I\sigma}\}]
%=E_\textrm{DFT}[n(\br)]
E_U[\{n_{mm'}^{I\sigma}\}]=E_\textrm{Hub}[\{n_{mm'}^{I\sigma}\}]
-E_\textrm{dc}[\{n^{I\sigma}\}].
\label{eq:hubbard2}
\end{equation}
%
The occupancy matrix, 
for Hubbard site $I$, 
is constructed from 
the projection of the KS density matrix 
onto the manifold of 
localised orbitals 
$\{\varphi^I_m\}$, 
%also known as the 
%{\it Hubbard projectors},
%
\begin{equation}
n_{mm'}^{I\sigma}=\bra{\varphi^I_m}\hat{\rho}^{\sigma}\ket{\varphi^I_{m'}}
\quad\mbox{with}\quad
\hat{P}^I_{mm'}=\ket{\varphi^I_{m}}\bra{\varphi^I_{m'}}, 
\end{equation}
%
which also defines the idempotent 
Hubbard projection operator $\hat{P}^I$, 
where 
$\hat{P}^I=\hat{P}^{I2}$.
%
The localised subspaces are usually 
pre-defined for corrective treatment,
having been deemed responsible 
for the dominant SIEs on the basis of physical intuition
and experience, 
although a further level of 
self-consistency over subspaces 
is also possible~\cite{PhysRevB.82.081102}.

The Hubbard energy correction   
is then evaluated from the 
sum over the screened Coulomb interactions $V$ 
between all on-site orbitals, 
and is evaluated as follows
%
\begin{align}
E_\textrm{Hub}[\{n^{I\,\sigma}_{mm'}\}]
&=\frac{1}{2}\sum_{\{m\},\, I\sigma}\left\{V_{mm''m'm'''}^In^{I\sigma}_{mm'}n^{I\bar\sigma}_{m''m'''}\right.\nonumber \\[0.5em]
&+\left(V_{mm''m'm'''}^I- V_{mm''m'''m'}^I\right)\left.n^{I\sigma}_{mm'}n^{I\sigma}_{m''m'''}\right\}.
\label{eq:e_hub}
\end{align}
%
The terms in Eq~\eqref{eq:e_hub} 
respectively correspond to 
spin off-diagonal Coulomb repulsion, 
spin on-diagonal Coulomb repulsion, 
and spin on-diagonal exchange.
%
The Coulomb matrix elements 
of a given site $I$ 
are expressed in terms of 
renormalised Slater integrals 
%
%\bra{m,m''}\hat{V}_{ee}\ket{m',m'''}
%
\begin{equation}
V_{mm''m'm'''}^I=\sum_{k=0}^{2l}a_k(m,m',m'',m''')F^k 
\label{eq:slater_integrals}
\end{equation}
%
for angular momentum quantum number $l$ 
and magnetic quantum number $m$ 
where $\alpha_k$ are given by
%
\begin{equation}
a_k(m,m',m'',m''')=\frac{4\pi}{2k+1}\sum^{k}_{q=-k}\bra{lm}{Y}_{kq}\ket{lm'}\bra{lm''}{Y}^*_{kq'}\ket{lm'''}.
\label{eq:coefficients}
\end{equation}
%
\edit{The Slater integrals over the spherical harmonics 
are then defined as follows~\cite{slater1960quantum}
%
\begin{equation}
\bra{lm}{Y}_{kq}\ket{lm'} = \int Y^*_{lm}(\br)Y_{kq}(\br)Y_{lm'}(\br)\ d\br.
\end{equation}}

%
Following the procedure in 
Refs.~\cite{PhysRevB.71.035105}~\&~\cite{PhysRevB.84.115108},
to simplify matters,  
we concern ourselves only 
with the lowest-order non-zero polar terms 
between $m$ and $m'$, i.e.
$V_{mm'mm'}$ and $V_{mm'm'm}$.
%Is V^I_{mmm'm'} = 0 because or orthogonality between Ys?
%
Thus, the sum over interaction terms in  
Eq.~\eqref{eq:e_hub} 
may be approximated by the expansion 
%
\begin{align}
E_\textrm{Hub}%[\{n^{I\,\sigma}_{mm'}\}]
&\approx\frac{1}{2}\sum_{I\sigma}\sum_{mm'}\left\{V^I_{mm'mm'}n_{mm}^{I\sigma}n_{m'm'}^{I\bar\sigma}+V^I_{mm'm'm}n_{mm'}^{I\sigma}n_{m'm}^{I\bar\sigma}\right.\nonumber \\[0.5em]
&\left.+(V^I_{mm'mm'}-V^I_{mm'm'm})(n_{mm}^{I\sigma}n_{m'm'}^{I\sigma}-n_{mm'}^{I\sigma}n_{m'm}^{I\sigma})\right\}.
\label{eq:e_hub_reduced}
\end{align}
%
For $d$-electrons, 
the sum over $k$ 
in Eq.~\eqref{eq:slater_integrals} 
runs over 
$F^0$, $F^2$ and $F^4$ 
(and $F^6$ for $f$-electrons), 
hence, the average on-site 
Coulomb and exchange interactions 
are 
%
\begin{align}
&U^I=\frac{1}{(2l+1)^2}\sum_{m,m'}V_{mm'mm'}^I=F^0\nonumber\\[0.5em]
\mbox{and}\quad&
J^I=\frac{1}{2l(2l+1)}\sum_{m\neq m',m'}V_{mm'm'm}^I=\frac{F^2+F^4}{14}.
\label{eq:interaction_parameters}
\end{align}
%\begin{align}
%\bra{m,m'}V_{ee}\ket{m,m'}&=a_0(m,m,m',m')F^0+a_2(m,m,m',m')F^2+a_4(m,m,m',m')F^4\\
%					&=4\pi|\bra{lm}Y_{kq}\ket{lm'}|^2F^0\\
%					&=F^0
%\end{align}
%
%then summing this over the $2l+1$ orbitals
%\begin{equation}
%\sum_{m,m'}\bra{m,m'}V_{ee}\ket{m,m'}=\sum_{m=-l}^l\sum_{m'=-l}^l\bra{m,m'}V_{ee}\ket{m,m'}=(2l+1)^2F^0
%\end{equation}
%
Using these definitions, 
we construct the aforementioned 
double-counting correction 
in terms of the total subspace occupancy 
$N^{I\sigma}=\sum_{m} n^{I\sigma}_{mm}$
for site $I$ and spin $\sigma$,
%$E_\textrm{dc}$ as
%
\begin{align}
E_\textrm{dc}&=\frac{1}{2}\sum_{I}U^IN^{I}(N^{I}-1)-\frac{1}{2}\sum_{I\sigma}J^IN^{I\sigma}(N^{I\sigma}-1)\nonumber\\[0.5em]
&=\frac{1}{2}\sum_{I\sigma}\left\{(U^I-J^I)\left[N^{I\sigma}(N^{I\sigma}-1)\right]+U^IN^{I\sigma}N^{I\bar\sigma}\right\}.
\label{eq:double_counting}
\end{align}
%
Finally, substituting Eqs.~\eqref{eq:e_hub_reduced}~\&~\eqref{eq:double_counting} 
into Eq.~\eqref{eq:hubbard2}, 
we arrive at the full expression 
for the Hubbard correction 
%
\begin{align}
E_U
&=\frac{1}{2}\sum_{I\sigma}\sum_{mm'}\left\{V^I_{mm'mm'}n_{mm}^{I\sigma}n_{m'm'}^{I\bar\sigma}+V^I_{mm'm'm}n_{mm'}^{I\sigma}n_{m'm}^{I\bar\sigma}\right.\nonumber \\[0.5em]
&\left.+(V^I_{mm'mm'}-V^I_{mm'm'm})(n_{mm}^{I\sigma}n_{m'm'}^{I\sigma}-n_{mm'}^{I\sigma}n_{m'm}^{I\sigma})\right\}\nonumber \\[0.5em]
&-\frac{1}{2}\sum_{I\sigma}\left\{(U^I-J^I)\left[N^{I\sigma}(N^{I\sigma}-1)\right]+U^IN^{I\sigma}N^{I\bar\sigma}\right\}.
\end{align}
%
We may simplify this equation further 
by re-expressing the 
total occupancy $N^{I\sigma}$ 
in terms of the constituent matrix elements 
%
\begin{align}
E_U
&=\frac{1}{2}\sum_{I\sigma}\sum_{mm'}\left\{V^I_{mm'mm'}n_{mm}^{I\sigma}n_{m'm'}^{I\bar\sigma}+V^I_{mm'm'm}n_{mm'}^{I\sigma}n_{m'm}^{I\bar\sigma}\right.\nonumber \\[0.5em]
&\left.+(V^I_{mm'mm'}-V^I_{mm'm'm})(n_{mm}^{I\sigma}n_{m'm'}^{I\sigma}-n_{mm'}^{I\sigma}n_{m'm}^{I\sigma})\right.\nonumber \\[0.5em]
&-\left.(U^I-J^I)(n^{I\sigma}_{mm}n^{I\sigma}_{m'm'}-n^{I\sigma}_{mm'}\delta_{mm'})-U^I n^{I\sigma}_{mm}n^{I\bar\sigma}_{m'm'}\right\}, 
\end{align}
%
and grouping terms 
according to their dependence on such 
%
\begin{align}
E_U
&=\frac{1}{2}\sum_{I\sigma}\sum_{mm'}\left\{(V^I_{mm'mm'}-U^I)n_{mm}^{I\sigma}n_{m'm'}^{I\bar\sigma}\right.\nonumber \\[0.5em]
&+\left(V^I_{mm'mm'}-U^I+V^I_{mm'm'm}-J^I\right)n_{mm}^{I\sigma}n_{m'm'}^{I\sigma}\nonumber \\[0.5em]
&+(U^I-J^I)n^{I\sigma}_{mm}\delta_{mm'}\nonumber\\[0.5em]
&-(V^I_{mm'mm'}-V^I_{mm'm'm})n_{mm'}^{I\sigma}n_{m'm}^{I\sigma}\nonumber \\
&+\left.V^I_{mm'm'm}n_{mm'}^{I\sigma}n_{m'm}^{I\bar\sigma}\right\}.
\end{align}
%
Finally, we may simplify the expression again 
by replacing the orbital-dependent terms 
$V^I_{mm'mm'}$ and $V^I_{mm'm'm}$ 
by their respective scalar averages 
$U^I$ and $J^I$, 
which yields
%
\begin{align}
E_U
&=\frac{1}{2}\sum_{I\sigma}\sum_{mm'}\left\{(U^I-J^I)(n^{I\sigma}_{mm'}\delta_{mm'}-n_{mm'}^{I\sigma}n_{m'm}^{I\sigma})+J^In_{mm'}^{I\sigma}n_{m'm}^{I\bar\sigma}\right\}\nonumber\\[0.5em]
%&=\sum_{I\sigma}\sum_{m}\frac{U^I-J^I}{2}\left(n^{I\sigma}_{mm}-\sum_{m'}n_{mm'}^{I\sigma}n_{m'm}^{I\sigma}\right)+\frac{J^I}{2}\sum_{mm'}n_{mm'}^{I\sigma}n_{m'm}^{I\bar\sigma}
&=\frac{1}{2}\sum_{I\sigma}\left\{(U^I-J^I)\mbox{Tr}\left[\hat{n}^{I\sigma}-\hat{n}^{I\sigma}\hat{n}^{I\sigma}\right]
+J^I\textrm{Tr}\left[\hat{n}^{I\sigma}\hat{n}^{I\bar\sigma}\right]\right\},
\label{eq:dft+u+j_functional}
\end{align}
%
and completes the full 
DFT+$U$+$J$ functional 
for Hubbard site $I$ 
in terms of the spin-dependent 
occupancy matrices $\hat{n}^{I\sigma}$.


To simplify the utilisation and 
transparency of the Hubbard functional, 
many practitioners tend to ignore 
unlike-spin interactions 
by setting the exchange term $J^I$ to zero~\cite{PhysRevB.79.035103} 
(thereby omitting magnetic interaction), 
or equivalently,   
incorporating $J^I$ into the $U$-term 
to compose an effective parameter~\cite{PhysRevB.57.1505,PhysRevB.71.035105,PhysRevB.79.035103} 
$U^I_\textrm{eff}=U^I-J^I$~\footnote{Alternatively, it is not uncommon
for $J^I$ to be set to $\sim$1~eV.}, 
which simplifies 
Eq.~\eqref{eq:dft+u+j_functional} 
even more 
%
\begin{equation}
E_U[\{\hat{n}^{I\sigma}\}]=\sum_{I\sigma}\frac{U^I}{2}\textrm{Tr}\left[\hat{n}^{I\sigma}-\hat{n}^{I\sigma}\hat{n}^{I\sigma}\right], 
\label{eq:dft+u_functional}
\end{equation}
%
resulting in the functional 
most commonly  
implemented in the DFT+$U$ formalism.
%
The intended effect of 
Eq.~\eqref{eq:dft+u_functional} 
becomes more clear if it 
is represented in a basis 
that diagonalises the occupancy matrix 
%
\begin{equation}
\hat{n}^{I\sigma}v^{I\sigma}_i=q^{I\sigma}v^{I\sigma}_i
\quad\mbox{with}\quad
0 \leq q_i^{I\sigma} \leq 1,
\end{equation}
such that the correction becomes
%
\begin{equation}
E_U[\{\hat{n}^{I\sigma}\}]=\sum_{I\sigma}\sum_i\frac{U^I}{2}q^{I\sigma}(1-q^{I\sigma}) 
\label{eq:dft+u_diagonal},
\end{equation}
%
which is ensured because $\hat{n}$ is also idempotent\footnote{This is for the
same reason the Kohn-Sham density matrix $\hat{\rho}$ is, see section~\ref{sec:density_matrix}}.
%
Here, Eq.~\eqref{eq:dft+u_functional} 
is clearly imposing a penalty on 
non-integer occupancies 
in the localised orbitals $q^{I\sigma}$, 
the strength of which is 
tuned by the magnitude of the 
$U$ parameter, 
in order to restore the 
linear behaviour of the 
exact XC functional.
%
The correction thus favours 
orbitals that are either 
fully occupied 
$q^{I\sigma}=1$, 
or fully depleted 
$q^{I\sigma}=0$, 
and contributes a non-zero correction 
{to the potential when $q^{I\sigma}\neq1/2$}, 
which counteracts the SIE.

This effect can be further examined 
by constructing the DFT+$U$ potential 
%
\begin{align}
\hat{V}_U
=\frac{\delta E_U}{\delta\hat{\rho}}
&=\sum_{I\sigma}\frac{U^I}{2}\left(\hat{P}^I-2\hat{P}^I\hat{\rho}^\sigma\hat{P}^I\right)\nonumber \\[0.5em]
&=\sum_{I\sigma}\sum_{mm'}\frac{U^I}{2}\ket{\varphi^I_m}\left(\hat{1}-2\hat{n}^{I\sigma}_{mm'}\right)\bra{\varphi^I_{m'}}, 
\end{align}
%
which is repulsive for occupancies 
less than one half, 
and attractive for occupancies 
greater than one half, 
thus incentivising idempotency 
in the subspace occupancy matrix.
% 
This re-localisation process opens a gap 
on the order of $U$
between occupied and unoccupied 
KS states, 
thereby restoring the desired 
Mott-Hubbard behaviour.
%
In practice, however,  
the individual subspace occupancy matrices 
do not attain binary eigenvalues, 
and are only partially enforced, 
since it competes with the idempotency condition 
in the ground-state KS density-matrix $\hat{\rho}$, 
which is strictly enforced.
%

%ONE ELECTRON DFT+U
\subsection{A one-electron assessment of DFT+$U$}
\label{sec:one_electron_dft+u}

{
It has been repeatedly stated~\cite{PhysRevB.44.943,PhysRevB.48.16929,PhysRevB.71.035105,PhysRevB.71.041102,PhysRevB.73.195107,PhysRevB.75.035115,:/content/aip/journal/jcp/145/5/10.1063/1.4959882,PhysRevLett.97.103001}, 
or at least inferred, 
that DFT+$U$ provides a sufficient 
correction to one-electron SIE.
%
This claim, however, 
has not been explicitly verified to our knowledge, 
despite forming the basis for DFT+$U$ 
as an SIE correction scheme.
%
In Chapter~\ref{ch:calculating_hubbard_u}, 
we provide the first demonstration of this claim, 
but for now let us explore the 
effect of the DFT+$U$ functional on dissociating  H$_2^+$, 
which we saw provided an emphatic example of one-electron SIE, 
and investigate the effect of DFT+$U$ on this system.}

Following the same procedure 
described in section~\ref{sec:fractional_charges}, 
we calculated the binding curve using the DFT+$U$ 
functionality in {\sc ONETEP}~\cite{PhysRevB.85.085107}.
%
The $+U$ correction 
was applied simultaneously to each atom,
using a separate $1s$ pseudoatomic orbital (PAO) subspace
centred on each, 
generated from the occupied KS state 
of the pseudopotential for neutral hydrogen 
{with the PBE XC functional}.
%
The effect of the correction on 
the binding energy 
is evident in Fig.~\ref{fig:h2+_dft+u_total_energy}, 
where the Hubbard $U$ parameter required to correct the
PBE total-energy to the exact value, 
ranges over approximately $8$~eV from the fully bonded to 
dissociated limits.
 %
This result highlights the 
importance of the chemical environment, 
and not just the atomic species,
in determining the $U$ parameters, 
as was previously shown 
in Refs.~\cite{doi:10.1063/1.3660353,
doi:10.1063/1.4865831,PhysRevB.90.115105}.

 \begin{figure}[th!]
 \centering
 \includegraphics[height=0.494\textwidth]{images/h2+_dft+u_total_energy.pdf}
 \caption[Binding energy curve of H$_2^+$ with DFT+$U$]
 {Binding energy of H$_2^+$ 
 calculated with exact XC functional (solid) and DFT+$U$ with 
$U= 0,\ 4,\ 8$~eV (dashed), 
 relative to energy of isolated H atom.
 %
DFT+$U$ successfully resolves the SIE in the total-energy, 
albeit for a bond-length dependent Hubbard $U$.}
 \label{fig:h2+_dft+u_total_energy}
 \end{figure}


If we instead explore 
the efficacy of DFT+$U$ 
at restoring Koopmans' compliance, 
we can inspect the eigenvalue-derived 
total-energy as before, 
but where we now concern ourselves 
with the treatment of only the HOMO.
%
In Fig.~\ref{fig:h2+_dft+u_eigenvalue}, 
we observe the egregious failure 
of the DFT+$U$ scheme 
when charged with the latter condition.
%
The non-compliance with  
Koopmans' condition 
broadly increases both with 
bond-length and, 
to one's surprise, 
with the Hubbard $U$.
%
In fact, 
the effect of DFT+$U$ on 
the eigenvalue is lost entirely
in the dissociated limit, 
where both factors drive the 
subspace occupancy to $1/2$, 
at which point the $+U$ 
potential vanishes. 

 \begin{figure}[th!]
 \centering
 \includegraphics[height=0.494\textwidth]{images/h2+_dft+u_eigenvalue}
 \caption[Eigenvalue-derived binding energy of H$_2^+$ with DFT+$U$]
 {Binding energy of H$_2^+$ 
derived from the KS HOMO eigenvalue
 calculated with exact XC functional (solid) and DFT+$U$ with 
$U= 0, 4, 8$~eV (dashed), 
 relative to energy of isolated H atom.
 %
DFT+$U$ does not address the inaccuracy in the Kohn-Sham
eigenvalue, or non-compliance with Koopmans' condition.}
 \label{fig:h2+_dft+u_eigenvalue}
 \end{figure}
 
Across the binding curve, 
we see that the Hubbard $U$ 
required to enforce compliance
with Koopmans' condition, 
and that needed to attain the 
exact total-energy typically differ substantially.
 %
The facile correction of the 
total-energy of this system with DFT+$U$, 
albeit with a varying but reasonable $U$ value, 
is in sharp contrast with the 
qualitative {failure} to correct 
the occupied eigenvalue. 
%
Indeed, 
the {vanishing eigenvalue correction} at dissociation 
suggests an insufficiency in 
the linear term of the DFT+$U$ formalism, 
{the root of which lies in the 
symmetric treatment of the linear and quadratic 
terms in the double-counting correction  
in Eq.~\eqref{eq:double_counting}}.
%
Going forward, 
H$_2^+$ will serve as a 
useful tool for benchmarking 
the performance of 
the DFT+$U$ methodologies, 
since we have by now accumulated   
a commanding understanding  
of the behaviour of 
SIE and Koopmans' condition 
in this model system.
%and the expected compensation 
%of DFT+$U$ in those respects.


%STATIC CORRELATION ERROR
\section{Static Correlation Error}
\label{sec:static_correlation_error}

Another {important} source of error  
that also permeates approximate DFT calculations 
is the {\it static correlation error} (SCE).
%
The SCE is known to arise in DFT 
when performing calculations on systems with 
degenerate states~\cite{seminario1996recent,PhysRevLett.87.133004,doi:10.1063/1.1589733,cohen2008insights,doi:10.1021/cr200107z}, 
such as those found in 
strongly correlated systems, 
transition-metal oxides,
and chemical dissociation, 
where it plays an equally pivotal role as the SIE 
in their accurate description.
%
The SCE becomes particularly dominant in DFT 
when the single Slater determinant wave function  
is no longer a valid approximation.
%
For example, 
the closed-shell dissociation of 
H$_2$~\cite{doi:10.1063/1.1589733,doi:10.1063/1.1858371,
PhysRevLett.87.133004,
doi:10.1063/1.2747248,
cohen2008insights} 
is often used to illustrate SCE~\cite{cohen2008insights,doi:10.1021/cr200107z,doi:10.1063/1.2987202}, 
which we shall now explore.

We performed  non-spin-polarised 
calculations of the dissociating H$_2$ molecule 
in {\sc ONETEP} 
using the DFT+$U$ functional 
for $U=0,\ 2,\ 4$~eV, 
with the same calculation parameters 
as before in sections~\ref{sec:fractional_charges}~\&~\ref{sec:one_electron_dft+u}.
%
We then compared these results against 
highly accurate binding energies 
calculated in Ref.~\cite{sims2006high}, 
which in turn used the full configuration interaction (FCI) 
method described in Ref.~\cite{PhysRevA.4.908}.
%
Presented in Fig.~\ref{fig:H2_dft+u_energy} 
are the resulting binding curves 
where it is demonstrated that the 
conventional PBE functional 
performs reasonably well 
for short bond-lengths, 
but overestimates the total-energy 
in the dissociation limit.
%
Furthermore, 
positive $U$ values 
serve only to make matters worse 
by increasing the total-energy\footnote{
Although, it appears that 
selectively invoking negative $U$ values,  
which would instead {\it encourage} delocalisation, 
could correct the SCE 
by haphazardly increasing the level of SIE.}.
%
\begin{figure}[th!]
\centering
\includegraphics[height=0.494\textwidth]{images/H2_dft+u_energy}
\caption[Binding energy curve of H$_2$ with DFT+$U$]
{Binding energy of {closed-shell} H$_2$ 
calculated with FCI~\cite{sims2006high} (solid) and 
DFT+$U$ with $U=0,\ 2,\ 4$~eV (dashed), 
relative to energy of isolated H atoms, 
{with the respective functionals}.
%
Positive $U$ values exacerbate the SCE 
for all bond-lengths, in which case, 
negative $U$ values may be required.}
\label{fig:H2_dft+u_energy}
\end{figure}
%



% %%%%%%%%%%%%%
\edit{
To understand what is happening in this instance, 
let us consider the H$_2$ wave function 
as constructed in Hartree-Fock theory 
using molecular orbitals.
%
Around equilibrium, 
the  H$_2$  bonding and anti-bonding orbitals 
are constructed from a basis of 1$s$ atomic orbitals 
centred on the atoms labelled $A$ and $B$  
(up to a constant of normalisation) 
respectively given as follows 
%
\begin{align}
&{\sigma_{1s}(i,\alpha)}\sim \left({1s_A(i)}+{1s_B(i)}\right)\chi(\alpha)\\
\quad\mbox{and}\quad
&{\sigma_{1s}^*(i,\alpha)}\sim \left({1s_A(i)}-{1s_B(i)}\right)\chi(\alpha), 
\end{align}
%
where $\chi(\alpha)$ denotes the spin component 
of the wave function for electron $i$.
%
Thus, 
the ground-state $^1\Sigma^+_g$ wave function 
$\Psi_0$
may be built from a single Slater determinant 
of $\sigma_{1s}$ bonding orbitals 
with an anti-symmetric spin component 
%
\begin{align}
\Psi_{0}\sim 
%\left| \begin{array}{cc}
%{\sigma_{1s}(1,\uparrow)} & {\sigma_{1s}(1,\downarrow)}\\
%{\sigma_{1s}(2,\uparrow)} & {\sigma_{1s}(2,\downarrow)}
%\end{array}\right| 
{\sigma_{1s}(1)}{\sigma_{1s}(2)}\left[\chi(\uparrow)\chi(\downarrow)-\chi(\downarrow)\chi(\uparrow)\right].
\end{align}
%
If we now expand the spatial component of this expression  
we find that the wave function comprises 
terms attributed to both ionic and covalent states 
%
\begin{equation}
\Psi_{0}\sim 
\underbrace{1s_A(1)1s_A(2)+1s_B(1)1s_B(2)}_{\textrm{ionic}}
+\underbrace{1s_A(1)1s_B(2)+1s_A(2)1s_B(1)}_{\textrm{covalent}}, 
\end{equation}
%
in which the electrons simultaneously 
occupy one or both atoms, respectively.}

\edit{Around equilibrium, 
this single Slater determinant wave function 
produces a more or less reasonable approximation 
to the exact wave function and, 
depending on the level of theory, 
i.e., Hartree-Fock or DFT~\cite{doi:10.1021/cr200107z}, 
it determines the binding energy 
to a very reasonable accuracy 
as shown in Fig.~\ref{fig:H2_dft+u_energy}.}

\edit{
As the bond-length increases, however, 
this approximation clearly breaks down as the 
the ionic terms contribute as 
much to the construction of the wave function 
as the covalent terms.
%
In other words, 
the independent-particle approximation 
maintains that the two electrons have equal 
probability of occupying the same atom, 
even as the bond-length increases to infinity. 
%
Physically, we know this to be false 
as the electrons should be correlated 
in such a way that they avoid each other  
and localise on one atom each into dissociation.
%
The single Slater determinant approximation in Hartree-Fock 
therefore neglects a large component of 
the correlation that exists between the two electrons.}

\edit{
In the FCI, 
this correlation is explicitly considered 
as the wave function is built from a linear combination 
of all allowed single determinants 
$\Psi_{\textrm{CI}}=\sum_{I=0} a_I\Psi_I$, 
where the coefficients $\{a_I\}$ 
are optimised by a variational procedure.
%
In the case of the $^1\Sigma_g^+$ state of H$_2$,   
it suffices to consider only the bonding singlet state $\Psi_0$ 
and doubly-excited anti-bonding singlet state 
$\Psi_1$~\cite{frank1999introduction}, 
given by
%
\begin{align}
\Psi_{1}\sim 
%\left| \begin{array}{cc}
%{\sigma_{1s}^*(1,\uparrow)} & {\sigma_{1s}^*(1,\downarrow)}\\
%{\sigma_{1s}^*(2,\uparrow)} & {\sigma_{1s}^*(2,\downarrow)}
%\end{array}\right| 
{\sigma_{1s}(1)}^*{\sigma_{1s}(2)}^*\left[\chi(\uparrow)\chi(\downarrow)-\chi(\downarrow)\chi(\uparrow)\right].
\end{align}
%
Thus, the FCI wave function 
(expressed in terms of the 1$s$ orbitals only) 
is simply expressed in terms of the adaptive 
mixing of the ionic and covalent contributions
%
\begin{align}
\Psi_{\textrm{CI}}
&= a_0\Psi_0+a_1\Psi_1\nonumber \\
&= (a_0+a_1)\underbrace{(1s_A(1)1s_A(2)+1s_B(1)1s_B(2))}_{\textrm{ionic}}\nonumber \\
&\qquad+(a_0-a_1)\underbrace{(1s_A(1)1s_B(2)+1s_B(1)1s_A(2))}_{\textrm{covalent}}, 
\label{eq:ci_wavefunction}
\end{align}
%
which is optimised according to $a_0$ and $a_1$.
%
Around equilibrium, 
one finds that $a_1\approx 0$, 
while towards dissociation $a_1\approx -a_0$ 
such that the exact binding energy 
is calculated at all bond-lengths~\cite{doi:10.1021/cr200107z,sims2006high}.
%
The $\Psi_1$ state therefore introduces 
the so-called {\it left-right correlation} 
between the two electrons 
that is inherently absent in the 
single determinant construction.}
%


%%%%%%%%%%%%%%%%%%%

\edit{While electron correlation is entirely absent in HF theory, 
it is included to some extent in DFT via the PBE XC functional, 
which is used to generate Fig.~\ref{fig:H2_dft+u_energy}, 
in spite of the latter also utilising a single Slater determinant wave function.
%
Evidently, the PBE binding energy 
 still exhibits SCE 
and to understand its origins in DFT 
in terms of the {\it density} 
it is instructive to approach from the perspective of 
fractional spins~\cite{doi:10.1063/1.2987202,doi:10.1021/cr200107z,PhysRevB.77.115123,doi:10.1021/ct8005419,PhysRevA.88.030501,PhysRevA.85.042507}, 
similar to the diagnosis of SIE in terms of fractional occupancy.}

{Let us consider therefore H$_2$ 
hypothetically stretched to infinity, 
at which point each atom accommodates one electron in total 
that is equally split between spin-up and spin-down states.
%
This configuration, 
in which only the covalent terms 
in Eq.~\eqref{eq:ci_wavefunction} feature, 
must in principle be equal in energy 
to two fully spin-polarised H atoms at infinite bond-length, 
as proven in Ref.~\cite{doi:10.1063/1.2987202}.}
%

{Approximate XC functionals, 
such as the PBE functional in Fig.~\ref{fig:H2_dft+u_energy}, 
fail to comply with this so-called {\it constancy condition}, 
which states that, 
for fixed occupancy, 
the total-energy of a system  
with $g$ degenerate states 
is independent of the distribution 
of spin-density among those states.
%
The violation of this condition 
is thus responsible for the SCE 
and is expressed mathematically 
in Ref.~\cite{doi:10.1063/1.2987202} 
as follows 
%
\begin{align}
&E\left[\sum_{i=1}^g c_i\rho_i\right]=E[\rho_j]=E_0(N),\nonumber \\[0.5em]
\quad\mbox{with}\quad&
j=\{1,\cdots, g\}
\quad\mbox{and}\quad
\sum_{i=1}^g c_i=1, 
\end{align}
%
which is illustrated in Fig.~\ref{fig:H2_sce} 
to be maximised at half-spin occupancy.}
%


\begin{figure}[th!]
\centering
\includegraphics[height=0.494\textwidth]{images/H2_sce}
\caption[SCE of a H$_2$ molecule as the deviation from piece-wise constancy]
{The calculated SCE of a the H$_2$ molecule  
 is shown by the deviation 
 of the PBE total-energy (dashed) 
 from the correct piece-wise constant behaviour (solid)
 between integer spin-density .}
\label{fig:H2_sce}
\end{figure}

%
%Nonetheless, 
%it is sometimes possible to acquire 
%the exact total-energy in KS DFT  
%by invoking constraints that to break 
%the spin-symmetry of the orbitals 
%{(before the Coulson-Fischer point~\cite{doi:10.1080/14786444908521726})}
%using constrained-DFT~\cite{PhysRevB.94.035159}.
%%
%However, 
%this type of constrained-DFT calculation~\cite{doi10.1021/cr200148b}  
%(to be discussed further in the next Chapter)
%can be prohibitively expensive to perform, 
%and, not to mention, 
%would invariably return incorrect 
%time-averaged spin-densities, 
%which may be the quantity of interest.
%

\edit{As we have previously shown, 
the correction of SCE is possible 
with multi-determinant FCI wave functions.
%
While the recent work of Alavi {\it et al} 
have helped extend its application to 
quite large systems~\cite{PhysRevB.85.081103,Booth2012} 
it is nonetheless desirable 
to have at hand a first-principles 
DFT correction scheme.}
%
\edit{The unsuitability of the DFT+$U$ functional, 
while correctly targeting fractional occupancies, 
ultimately lies with the $U$ parameter itself, 
which is a measure of the curvature with respect 
to total occupancy and {\it not} magnetisation.
%
As such, it will not be conducive to correcting SIE 
and other options must be explored.}

%
Indeed, 
the $J$ parameter that couples occupancy matrices 
of opposite spin in Eq.~\eqref{eq:dft+u+j_functional} 
has been linked to linked to the curvature 
of the total-energy with respect to magnetisation 
in Ref.~\cite{PhysRevB.84.115108}, 
and acts to penalise non-magnetic 
ordering~\cite{PhysRevB.79.035103,PhysRevB.84.115108}.
%
Given this connection between $J$, the SCE, 
and the parabolic deviations from 
the constancy condition, 
as in Fig.~\ref{fig:H2_sce},
we postulate that $J$ may be the parameter we desire   
to correctly restore the H$_2$ total-energy, 
while retaining the physical spin-density.

Let us therefore restrict our treatment to 
un-like spin interactions only, 
for which the DFT+$J$ functional 
takes the form 
%
\begin{equation}
E_{DFT+J}
=E_\textrm{DFT}
+\sum_{I\sigma}\frac{J^I}{2}\textrm{Tr}\left[\hat{n}^{I\sigma}\hat{n}^{I\bar\sigma}\right].
\label{eq:dft+j_functional} 
\end{equation}
%
The spin-polarised calculations 
of the H$_2$ binding curve, 
for $J=0,\ 2,\ 4$~eV, 
are presented in Fig.~\ref{fig:H2_dft+j_energy}, 
in which it is evident 
that positive $J$ values again 
worsen the total-energy at all bond-lengths, 
albeit in a slightly different way 
to the DFT+$U$ functional in Fig.~\ref{fig:H2_dft+u_energy}.
%

\begin{figure}[th!]
\centering
\includegraphics[height=0.494\textwidth]{images/H2_dft+j_energy}
\caption[Binding energy curve of H$_2$ with DFT+$J$]
{Binding energy of {closed-shell} H$_2$ 
calculated with CI (solid) and 
DFT+$J$ with $J=0,\ 2,\ 4$~eV (dashed), 
relative to energy of isolated H atoms.
%
Positive $J$ values also worsen the SCE 
for all bond-lengths, but negative $J$ 
values may alleviate the error.}
\label{fig:H2_dft+j_energy}
\end{figure}

{This outcome can be understood 
when $J$ is interpreted as the negative of 
the energy curvature with respect to magnetisation~\cite{PhysRevB.84.115108},
in which case applying a positive $J$ will only increase 
the level of curvature present and 
consequently increase the error.}
%
Although a range of negative $J$ values
over the binding curve could hypothetically  
alleviate the SCE {by removing this spurious curvature}, 
we desire the means to calculate the 
necessary $J$ values from first-principles 
analogous to how the $U$ values may be 
computed to treat SIE.

Based on our discussion heretofore, however, 
we would argue that the SIE and SCE of a given system 
are likely to be intricately linked, 
as both relate to the violation of physical conditions 
pertaining to fractional orbital occupations.
%
Consequently, 
we are unable to determine, as of yet, 
exactly how the treatment of one error may affect the 
other~\cite{doi:10.1021/cr200107z,doi:10.1080/00268976.2010.507227}, 
e.g., favourably, adversely, or not at all as the case may be, 
prior to the analysis of the particular system.
%
This will become an important 
matter to consider when we pursue a ground-state 
that is simultaneous free of SIE and SCE.


%CONCLUSION
\section{Conclusion}
%

In this Chapter, 
we have outlined 
the origins of the most pervasive errors 
in approximate DFT, 
namely the self-interaction error (SIE) 
and static-correlation error (SCE), 
which may be respectively attributed 
to violations in the linearity and constancy conditions  
obeyed by exact XC functionals.
%
In particular, 
we have discussed how the SIE 
is manifest in a spurious 
curvature in the total-energy vs occupancy profile 
which is attributed to fractional occupancies.
%
%This invariably leads to 
%a many-body self-interaction 
%and consequent electron delocalisation.
%
In the same context, 
we explored how 
the incorrect energy-derivatives between integer occupancies 
is related to the 
non-compliance of Koopmans' condition 
of the HOMO eigenvalue.
%


We then outlined the motivation 
and subsequent development 
of the rotationally-invariant DFT+$U$+$J$ functional 
and its potential for effectively correcting 
single and many-body SIE 
in strongly-correlated systems, 
a claim which is widely accepted but 
has not been explicitly verified.
%
However, we found DFT+$U$ 
to be remarkably {unsuitable} for 
restoring Koopmans' compliance 
due to an {insufficient} linear-term.
%
{
It appears that DFT+$U$, 
in its current form, 
is limited to the treatment of either 
the total-energy or the eigenspectrum,
since it cannot achieve both for a one-electron system.}
%
The choice of the Hubbard subspace 
in the DFT+$U$ approach 
is also non-incidental but, 
for our purposes in this dissertation, 
we have restricted ourselves 
to using the well-known pseudoatomic orbitals (PSA).

Finally, 
we illustrated how SCE arises 
from using a single Slater determinant wave function 
and that explicit treatment of 
electron correlation found in the FCI 
are required for it to be adequately resolved.
%
In DFT, 
the SCE arises in systems with degenerate valence states, 
which tends to increase the total-energy, 
and is quantified in terms of 
the spurious curvature in the total-energy 
with respect to subspace magnetisation.

%An additional error, 
%which we have neglected to discuss, 
%is that proposed by by Burke {\it et al.}~\cite{PhysRevLett.111.073003}, 
%which is exhibited in systems 
%where the approximate XC functional 
%performs sufficiently well  
%but the error arises due to incorrect densities.
%%
%However this error shall not be discussed further.


To conclude, 
we have presented the 
motivation and theoretical framework 
for the DFT+$U$ method, 
which is known to be an efficient and highly 
effective many-body SIE correction scheme 
that now finds routine application 
in a diverse range of systems~\cite{QUA:QUA24521,
doi:10.1021/jz3004188,
PhysRevLett.113.086402,
doi:10.1021/jp3107809,
PhysRevB.93.085135}.
%
The versatility of the method 
grants it widespread applicability 
in a variety of software packages~\cite{PhysRevB.73.195107,PhysRevB.75.035115,PhysRevLett.97.103001}, 
including {\sc ONETEP}~\cite{o2011optimised,PhysRevB.83.245124,PhysRevB.85.085107,PhysRevB.85.193101,PhysRevB.82.081102,PhysRevB.94.220104}.
% 
Furthermore, it attains the status 
of an automated, first-principles
method when it is provided with calculated 
Hubbard $U$ parameters~\cite{0953-8984-9-4-002,
PhysRevB.58.1201,PhysRevB.71.035105,
PhysRevB.74.235113,QUA:QUA24521}
(particularly at their self-consistency~\cite{PhysRevB.93.085135,
PhysRevLett.97.103001,doi:10.1021/jp070549l}),
the calculation of which, 
from first-principles, 
will be addressed in the 
following Chapters.




