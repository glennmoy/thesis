%Introduction to 2D Materials

\lett{A}{ paradigm shift} 
in materials science occurred in 2004  
with the successful isolation of a 
stable sample of few-layer graphite 
by Novoselov and Geim~\cite{Novoselov666}.
%
Dubbed the `wonder material'~\cite{Geim1530,2058-7058-22-08-33},
graphene quickly captured the imaginations 
of researchers and the public~\cite{Geim2007,graphenetheguardian,doi:10.1021/cr900070d}, 
boasting almost limitless potential 
- from revolutionising electronics~\cite{Geim2007,RevModPhys.81.109} 
to water desalination~\cite{MISHRA201139} - 
and was envisioned to be 
the answer to our material woes.
%
The prowess of graphene 
lies in the extraordinary 
electronic and mechanical properties 
that it exhibits 
- most notably 
ballistic electron transport~\cite{RevModPhys.81.109}  
and unrivalled mechanical strength~\cite{PhysRevB.76.064120}.
%
However, enabling a viable band gap in graphene, 
without compromising on the 
conductive properties, 
has posed many challenges~\cite{2053-1613-1-2-020204}
and 
researchers have struggled to 
find a scalable, inexpensive 
method of producing quality samples~\cite{C5TA00252D}.

%
During the interim, 
focus has shifted to the 
characterisation and synthesis of other 
two-dimensional (2D) structures 
derived from their, 
primarily layered, bulk analogues,
such as
boron-nitride (BN)~\cite{PhysRevB.76.073103,R.2010,PENG201211,doi:10.1021/nl1022139}, 
molybdenum-disulfide (MoS\textsubscript{2})~\cite{doi:10.1021/nl903868w,RadisavljevicB.2011,ANIE:ANIE201000009} 
and, notably, black phosphorus (BP).
%
Like graphene, 
the appeal of these materials 
lies in the emergence of 
exotic electro-mechanical phenomena, 
and a rapid increase in surface-area, 
with the reduction of a degree of freedom  
along one dimension.
%
The technological 
impact of 2D materials 
is predicted to 
revolutionise existing practices 
and launch new innovations,
with wide-ranging potential applications in 
photovoltaics~\cite{Xie2016433,doi:10.1021/nl401544y,ADMA:ADMA200800366,doi:10.1063/1.3204698,Wang2012}, 
energy storage~\cite{ADMA:ADMA200602592,doi:10.1021/jp302265n,Sun2015,Bonaccorso1246501,doi:10.1021/nl800957b,ADMA:ADMA200903328,doi:10.1021/nl802484w}, 
composite materials~\cite{PMID:26469634,doi:10.1021/nl501617j},
and catalysis~\cite{ADMA:ADMA201204453,ADFM:ADFM201505380,ANIE:ANIE201307475}.
%
The successful simulation of these materials, 
in either a predictive or analytical capacity, 
is therefore an important component of their exploration, 
and DFT calculations have provided  
an extensive and valuable 
complement to experiments.
%

In the following two Chapters, 
we provide a comprehensive analysis of 
DFT calculations performed on puckered ($\alpha$-phase) 
phosphorus (P), arsenic (As) and antimony (Sb) 
in the monolayer, bilayer and bulk forms.
%  
We identify and compare the qualitative 
strain-related properties of each structure 
from a consistent set of calculations, 
thus treating each material on the same footing.

In this Chapter specifically, 
we compute the elastic properties, 
such as the Young's modulus $Y$; 
shear modulus $G$; 
bulk modulus $\mathcal{B}$; 
and Poisson ratio $\nu$ 
and present their in-plane and isotropic averages 
according to the Voigt-Reuss-Hill (VRH) averaging scheme, 
for which the relevant expressions are derived.
%
In the following Chapter, 
we will continue the discussion and 
explore the qualitative electronic properties 
that are predicted to occur at various 
levels of strain.
%
We shall observe that, 
while the mechanical behaviour 
of atomic systems is generally 
well reproduced by high-level DFT, 
the corresponding electronic properties, 
such as band structures and 
optical gaps, are typically not.
 

%RECENT ADVANCES IN BP
\section{Recent advances in black phosphorus}
\label{sec:recent_advances}
%
Two-dimensional black phosphorus (BP), 
or {\it phosphorene}, 
is one of several predicted stable allotropes of 
few-layer phosphorus~\cite{doi:10.1021/acs.nanolett.5b01041,PhysRevLett.113.046804,PhysRevLett.112.176802,doi:10.1021/nn5059248}; 
it has attracted considerable attention since its recent 
successful synthesis~\cite{Li2014,Xia2014,doi:10.1063/1.4868132,doi:10.1021/nn501226z,Ling14042015}, 
which is now possible with liquid phase exfoliation~\cite{C4CC05752J,PMID:26469634}.
%
The excitement behind BP 
is driven by the growing list of 
predicted technologically properties, 
which includes
a tuneable band-gap~\cite{PhysRevLett.112.176801,PhysRevB.90.085402,PhysRevB.91.235118,doi:10.1021/jp508618t,PhysRevB.91.115412,0957-4484-25-45-455703,doi:10.1021/nn501226z}, 
a negative Poisson's ratio~\cite{Jiang2014}, 
anisotropic conduction~\cite{doi:10.1021/nl500935z,C4CS00257A}, 
and linear dichroism~\cite{Qiao2014,PhysRevB.89.235319}.
%  
So far, we have seen experimentally verification of  
a high hole-mobility 
between $300-1000$~cm\textsuperscript{2}/Vs~\cite{doi:10.1021/nn501226z,PhysRev.92.580,doi:10.1063/1.4868132,Qiao2014}, 
considerable mechanical flexibility~\cite{doi:10.1063/1.4885215}, 
and a layer-dependent band gap~\cite{9909120720141024,PhysRevB.89.235319,Zant2014} ranging from 0.3~eV in bulk to 2.0~eV in the monolayer.
%
The anisotropic crystal structure of BP, 
as shown in Fig.~\ref{fig:phos_structure},
is largely responsible for its 
wide range of exotic
electro-mechanical properties, 
which are predicted to be strongly directional-dependent 
and highly responsive to 
mechanically strain~\cite{doi:10.1021/nl500935z,PMID:26469634}. 

\begin{figure}[th!]
%\begin{tabular}{cc}
%\begin{tabular}{c}
%\begin{subfloat}[]{
\centering
\includegraphics[height=0.494\textwidth]{images/crystal2.jpg}
  \label{fig:crystalstructure}
  %}
%\end{subfloat} 
%\end{tabular}
%&
%\begin{tabular}{c}
%\smallskip
%\begin{subfloat}[]{
%\centering
%\includegraphics[width=0.40\textwidth]{images/3dbz.jpg}
% \label{fig:bz}}
%\end{subfloat}
%\\
%\begin{subfloat}[]{
%\centering
%\includegraphics[height=0.494\textwidth]{images/buckled1.jpg}
%\label{fig:buckled}}
%\end{subfloat}
%\end{tabular}\\
%\end{tabular}
\caption[Crystal structure of orthorhombic black phosphorus]
{Top and side view of the 
2D orthorhombic puckered structure 
of the Cmca space group and D$_{2h}$ point group 
%(generated in VESTA~\cite{Momma:db5098}) 
with the primary vectors along the 
zigzag ($\vec x$) and puckered ($\vec y$) directions shown. 
The unit cell, 
given by the shaded region, 
is described by the lattice parameters $a$ and $b$ 
with the in-plane angle $\theta$ also defined.
%%
%b) 3D Brillouin zone with high-symmetry points 
%\{$\Gamma$, $X$, $S$, $Y$, $Z$\}.
%%
%c) Side-view of the buckled Sb state  
%at $\varepsilon_{yy}=-4\%$ compressive strain.
}
\label{fig:phos_structure}
\end{figure}

With this renewed interest in BP, 
attention has quickly turned to 
few-layer phases of the other pnictogen materials, namely 
arsenic (As)~\cite{PhysRevB.91.085423}, 
antimony (Sb)~\cite{doi:10.1021/acsami.5b02441}, 
bismuth (Bi)~\cite{PhysRevB.91.075429,doi:10.1021/nl502997v}, 
and their alloys~\cite{doi:10.1021/acs.nanolett.5b02227,doi:10.1021/acs.jpcc.5b07323,Xie2016433,ADMA:ADMA201501758,doi:10.1021/acs.jpcc.5b02096}.
%
In fact, many of the predicted strain-induced properties 
of these materials, such as 
direct-indirect band gap transitions~\cite{ANIE:ANIE201411246,PhysRevB.91.085423,doi:10.1021/acsami.5b02441,PhysRevB.90.085402,0957-4484-26-7-075701},
a negative Poisson's ratio~\cite{Jiang2014,1882-0786-8-4-041801},
as well as 
electronic~\cite{PhysRevLett.113.046804,PhysRevB.91.161404,ANIE:ANIE201411246}, 
structural~\cite{PhysRevB.92.064114}, 
and 
topological~\cite{Lu2016,PhysRevLett.115.186403,PhysRevB.93.195434,PhysRevB.91.195319} transitions, 
are already spurring emergent technologies in  
field-effect transistors~\cite{doi:10.1021/nn501226z,C4CS00257A},
gas sensors~\cite{doi:10.1021/acsnano.5b01961,doi:10.1021/jz501188k}, 
optical switches~\cite{PhysRevB.90.075434,1347-4065-28-11A-2104}, 
solar-cells~\cite{Xie2016433},
energy storage~\cite{ADMA:ADMA200602592,doi:10.1021/jp302265n,Sun2015},
 reinforcing fillers~\cite{PMID:26469634,doi:10.1021/nl501617j}, 
 and topological insulators~\cite{PhysRevLett.115.186403,Lu2016,PhysRevB.93.195434,PhysRevB.91.195319,Yao2013}.



\section{Methodology}
\label{sec:methodology1}
\subsection{Calculation details}
\label{sec:calc_details}
%
The calculations in the following two Chapters 
were performed with the 
{\sc QuantumEspresso} 
package~\footnote{QE is not linear-scaling and uses a plane-wave 
basis as discussed in section~\ref{sec:linear_scaling_dft}.}~\cite{0953-8984-21-39-395502}
using the Perdew-Burke-Ernzerhof (PBE)  form of 
the generalised-gradient approximation (GGA) 
exchange-correlation functional~\cite{PhysRevLett.77.3865}. 
%
An ultrasoft pseudopotential~\cite{PhysRevB.41.7892}
from the SSSP Library\footnote{http://theossrv1.epfl.ch/Main/Pseudopotentials}~\cite{kucukbenli2014projector} 
(with 5 valence electrons) was used 
to represent the core electrons.
%
Calculations in this Chapter 
were performed without spin-orbit coupling (SOC)
for convenience but it was later 
included to confirm some aspects of their electronic behaviour 
(see Chapter~\ref{ch:elec_prop_2d_mater}).
%
In all calculations, van der Waals (vdW) interactions 
were incorporated using the B97-D empirical 
dispersion correction functional~\cite{JCC:JCC20495}. 
%
In order to achieve 
an energy convergence of at least 1~meV/atom 
and force convergence of at least $1.3\times 10^{-4}$~eV/a$_0$, 
we found it sufficient to use a common 
plane-wave energy cutoff of 1100~eV
with `cold' Fermi-surface smearing~\cite{PhysRevLett.82.3296} 
of $10^{-4}$~K for all elements.
%
Meanwhile, 
the Brillouin zone sampling for bulk systems
was $15\times15\times15$, 
and $15\times15\times1$ 
for monolayers and bilayers.
%
Uniaxial and shear strains 
between $\pm$5\% were applied 
in increments of 1\% to the unit cell 
with internal relaxation subject 
to the same force convergence 
criterion as above.
%
A sample input file containing the 
preceding parameters is included in Appendix~\ref{ch:input_files}.

%
The elements of the stiffness matrix $C$ 
were computed from the gradients of the 
resultant stress-strain profiles 
$c_{ij} = \partial \sigma_i/\partial\varepsilon_j$, 
from which all elastic properties were derived.
%
In practice, however, 
the calculated stiffness tensors 
are not exactly symmetric 
due to numerical noise 
but we make them so by taking 
the average of $C$ and its transpose $C^T$ 
as the \emph{effective} stiffness tensor, 
which, henceforth, 
shall be refered to as $C$.
%
We begin our discussion with 
a brief overview of the the 
VRH scheme, 
which is a popular model 
used for computing effective isotropic 
elastic properties.  

%VOIGT-Reuss-HILL MODELS
\subsection{The Voigt-Reuss-Hill scheme}
\label{sec:vrh_scheme}
%
In order to effectively preserve, study and strain-engineer
few-layer nano-structures, 
such as BP~\cite{2053-1583-4-2-021032}, 
or graphene~\cite{R.2010},
the nano-flakes are typically 
deposited onto a suitable substrate.
% 
{
The macroscopic elastic properties 
that are measured by experiment 
are the result of microscopic 
interactions between 
the nano-flakes dispersed  
on or within a bulk medium.}
%
The theoretical calculation of 
these elastic properties 
requires an appropriate mixture model, 
such as the rule-of-mixtures (ROM)~\cite{askeland2011science}
or the Halpin-Tsai (HT)~\cite{doi:10.1177/002199836900300419,PEN:PEN760160512} models 
(see sections~\ref{sec:mixture_models}~\&~\ref{sec:comparison_to_exp}), 
which require the (typically averaged)
elastic properties of the interstitial nano-flakes.
%

In the theory of effective media, 
isotropic bulk properties are 
computed by averaging the stiffness tensor $C$ 
over all possible rotated reference frames~\cite{MULLEN19972247,hearmon1969elastic,0965-0393-7-6-301}, 
the approach for which is 
outlined in Appendix~\ref{sec:vrh_derivation}.
%
The result is called the Voigt average~\cite{ZAMM:ZAMM19290090104,cook1999advanced} 
and it gives isotropic averages for the bulk 
Young's modulus $Y_V$, 
and shear modulus $G_V$, 
given in Eq.~\eqref{eq:voigt1}.
%
The same scheme applied 
to the compliance tensor $S=C^{-1}$ results in 
the corresponding Reuss averages~\cite{ANDP:ANDP18892741206}, 
$Y_R$ and $G_R$, 
given in Eq.~\eqref{eq:reuss1}.
%
The Voigt scheme assumes that 
the material is subject to homogeneous strain 
and tends to produce over-estimated elastic constants.
%
Conversely, 
the Reuss scheme assumes homogeneous stress 
and it tends to under-estimate the elastic constants.
%
The Hill averages~\cite{0370-1298-65-5-307}, given by  
%
\begin{equation}
Y_{H}=\frac{Y_V+Y_R}{2}
\quad\mbox{and}\quad
G_{H}=\frac{G_V+G_R}{2},
\label{eq:hillaverage}
\end{equation}
%
are widely considered as reliable estimates of 
the actual physical values~\cite{0965-0393-7-6-301}
%
and from these, the isotropic Poisson's ratio $\nu_H$ and 
bulk modulus  $\mathcal{B}_H$ may also be calculated 
by means of 
%
\begin{equation}
\nu_H=\frac{Y_H}{2G_H}-1 
\quad\mbox{and}\quad
\mathcal{B}_H=\frac{Y_HG_H}{3(3G_H-Y_H)}.
\label{eq:poisson_ratio}
\end{equation}
%
The VRH approach described above 
is used in the present work to 
determine the isotropic averages of the 
Young's, shear and bulk moduli, 
and the Poisson ratio  
of the bulk structures using 
the computed elastic tensors.
%
We will now discuss how the above 
approach may be adapted 
to derive the relevant equations 
for the specific case 
of two-dimensional materials.

% Effective 2D averaging for composite materials
\subsection{In-plane Voigt-Reuss-Hill average}
\label{sec:in_plane_vrh_scheme}
%
If the interstitial nano-flakes in a bulk medium 
form high-quality planar sediments~\cite{PMID:26469634,PMID:23203296,Mahmoud20111534,ADFM:ADFM200801776,APP:APP38645,Young20121459,Yang2013,C4NR01208A}, 
the random orientation is observed
in the plane of the flakes 
and one must calculate 
isotropic-averages {\it in-plane}. 
%
Due to the weak vdW bonds 
between layers, 
strains related to out-of-plane directions 
can be consequently ignored, 
resulting in the reduced-stiffness tensor $C_\textrm{2D}$ 
as in Eq.~\eqref{eq:2dmatrix}.
%
In Appendix~\ref{sec:vrh_derivation}, 
we re-derive the angular dependence  
of the rotated tensor-elements 
$C_{ij}\left(\theta\right)$ and $S_{ij}\left(\theta\right)$
about the $\vec{z}$-axis 
as a function of the elements in the original 
reference frame, 
similar to the isotropic VRH scheme.
%  
The angular dependence of the in-plane  
elastic constants are then expressed, 
from the elements in Eq.~\eqref{eq:rot2dmatrix}, as     
%
\begin{equation}
\begin{aligned}
&Y_{V}\left(\theta\right)=\frac{C_{11}^2-C_{12}^2}{C_{11}},
\quad
&G_{V}\left(\theta\right)=C_{66},
\quad\quad
&\nu_{V}\left(\theta\right)=\frac{C_{12}}{C_{11}},
\\[0.75em]
&Y_{R}\left(\theta\right)=\frac{1}{S_{11}},
\quad
&G_{R}\left(\theta\right)=\frac{1}{S_{66}},
\quad\quad
&\nu_{R}\left(\theta\right)=-\frac{S_{12}}{S_{11}},
\end{aligned}
\label{eq:elasticconsts2d}
\end{equation}
%
with the Hill-average taken as in Eq.~\eqref{eq:hillaverage}.
%
The in-plane averages are then 
computed analogously 
by first integrating the elastic tensors 
$C_{ij}\left(\theta\right)$ and $S_{ij}\left(\theta\right)$ 
over $2\pi$, 
 before substituting back into Eqs.~\eqref{eq:elasticconsts2d}.

\subsection{Composite mixture models}
\label{sec:mixture_models}

The successful simulation of 
the macroscopic elastic behaviour
 of a composite material  
requires an appropriate mixture model 
that strongly depends on the interplay  
between the dispersed filler 
and the bulk matrix material. 
%
In addition to the overt chemical interaction 
between the two constituents, 
the macroscopic response 
can also rely on the shape of the filler -  
for example planar, fibre or particulate - 
a feature which is predominantly governed, 
in the nano-scale, 
by the crystal structure.
%
Here, we shall outline two simple models 
that could possibly describe a 
composite material composed of 
puckered group-V nano-flakes dispersed 
in a bulk polymer matrix.

\subsubsection{The Rule-of-Mixtures model}
%
The rule-of-mixtures (ROM) model~\cite{askeland2011science} 
invokes a simple weighted mean to determine 
the properties of a macroscopic composite material, 
which, for our purposes, is the Young's modulus $Y$.
%
In the absence of any formal mathematical description 
of the underlying interactions 
between the filler and matrix, 
it may be regarded as 
an empirical, first-order approximation 
to the composite behaviour. 
%
Similar to the VRH scheme, 
the ROM defines a composite 
lower-bound $Y^R_C$ 
and upper-bound $Y^V_C$ 
in terms of the Young's modulus of the filler $Y_f$, 
the Young's modulus of the matrix $Y_m$, 
and the filler volume-fraction\footnote{These are formulated under similar 
assumptions of homogeneous 
strain and stress, respectively.} $\phi$, 
where
%
\begin{equation}
Y^V_C=\phi Y_f+ (1-\phi)Y_m 
\quad\mbox{and}\quad
Y^R_C=\left(\frac{\phi}{Y_f}+\frac{1-\phi}{Y_m}\right)^{-1}.
\label{eq:rom_models}
\end{equation}
%
The true Young's modulus is thus 
predicted to lie somewhere between the two.
%
Furthermore, since the ROM model 
has no explicit dependence 
on filler shape or orientation, 
it is unclear how to strictly 
define the Young's modulus of the filler, 
i.e. as an isotropic or in-plane average, 
thus the model parameters 
are likely to represent broad estimations.


\subsubsection{The Halpin-Tsai model}

The equations of Halpin and Tsai (HT)~\cite{doi:10.1177/002199836900300419,PEN:PEN760160512} 
were developed as a simplified and extended form 
of earlier composite models 
derived by Hill~\cite{0370-1298-65-5-307} 
and Hermans~\cite{hermans1967koninklijke}.
%
Unlike the ROM model, 
the HT equations make explicit the dependence on 
the geometry and orientation of the filler, 
parameterised by the factor $\zeta$.
%
The particular alignment 
between filler and matrix
can potentially augment 
their interaction, 
and thus the efficient transfer of stress, 
due, at least in part, to a large filler aspect ratio 
that provides a larger interfacial region.
%
The HT equation for the 
Young's modulus of a composite material 
composed of a planar filler 
of length $l$ and thickness $t$ 
is given by 
%
\begin{equation}
Y_{HT}=Y_m\left(\frac{1+\zeta\eta\phi}{1-\eta\phi}\right)
\quad\mbox{with}\quad
\eta=\frac{Y_f/Y_m-1}{Y_f/Y_m+\zeta}
\quad\mbox{and}\quad
\zeta=\frac{2l}{t}.
\end{equation}
%
The HT equations provide 
an empirical extension to the ROM model, 
since the parameter $\zeta$ 
has no scientific basis 
and was determined 
by experimental fitting.
%
In fact, it can be shown 
that the ROM model can 
be derived from the HT model 
by taking the limiting values of $\zeta$.
%
The first limit corresponds to the 
transverse arrangement of fibres 
in a bulk matrix, 
i.e $\zeta \to 0$, such that  
%
\begin{align}
\lim_{\zeta\to0}E_{HT}(\zeta)&=Y_m\left(\frac{1}{1-\eta\phi}\right) =Y_m\left(\frac{1}{1-(1-\frac{Y_m}{Y_f})\phi}\right) \nonumber\\[10pt]
&=Y_m\left(\frac{1}{1-\phi+\frac{Y_m}{Y_f}\phi}\right)=\left(\frac{1-\phi}{Y_m}+\frac{\phi}{Y_f}\right)^{-1}, 
\end{align}
%
which reproduces exactly the 
(iso-stress) ROM Reuss model.

The second limit assumes 
the longitudinal arrangement of fibres 
within the matrix, 
i.e $\zeta \to \infty$, such that  
%
\begin{align}
\lim_{\zeta\to\infty}E_{HT}(\zeta)
&=\lim_{\zeta\to\infty}Y_m\left(\frac{\frac{Y_f}{Y_m}+\zeta+\zeta(\frac{Y_f}{Y_m}-1)\phi}{\frac{Y_f}{Y_m}+\zeta+(\frac{Y_f}{Y_m}-1)\phi}\right)\nonumber\\[10pt]
&=Y_m\left(1+\left\{\frac{Y_f}{Y_m}-1\right\}\phi\right) \quad\mbox{(by L'H\^opital's rule)}\nonumber \\[10pt]
&=Y_m(1-\phi)+Y_f\phi,
\end{align}
%
which is precisely the 
(iso-strain) ROM Voigt estimate.
%
Thus, for $0<\zeta<\infty$, 
an intermediate value 
between that of the 
Voigt and Reuss estimates 
of the Young's modulus 
is attained.

% 2D RESULTS 1
\section{Results}
\label{sec:mechanical_results}

In this section, 
we report the mechanical properties 
computed using plane-wave DFT, 
and compare them to the 
available experimental data.
%
We begin with the lattice parameters 
of the fully-relaxed structures, 
and then discuss 
the isotropically averaged bulk properties, 
as well as the elastic properties as a function of 
in-plane orientation.
%
Finally, we compare the 
aforementioned mixture models 
fitted to experimental data 
of a polymer-nano-flake composite, 
as described in our own Ref.~\cite{PMID:26469634}, 
and with this confirm our DFT results.

% LATTICE CONSTANTS
\subsection{Calculated lattice constants}
\label{sec:lattice_constants}

The lattice constants $\{a,\ b,\ c\}$ 
of the fully-relaxed structures 
are presented in Table~\ref{table:lattice_consts}, 
where, in the monolayer and bilayer cases,  
we quote the layer thickness $c^\prime$ 
instead of the unit cell height $c$.
%
Our computed lattice parameters  compare well 
with other recent theoretically predicted values~\cite{PhysRevB.92.064114,PhysRevB.86.035105,PhysRevB.94.205410,PhysRevB.91.085423,1882-0786-8-5-055201,doi:10.1021/acsami.5b02441,ANDP:ANDP201600152,Lu2016}
and the available experimental data~\cite{doi:10.1063/1.438523,doi:10.1080/14786437508229285}.
%
\begin{table}[th!]
\centering
%\resizebox{\textwidth}{!}{%
\begin{tabular}{lrrrr}
\hline\hline
&$a$ (\AA) &$b$ (\AA)&$c$ (\AA)\\
\hline
P\textsubscript{mono}	&4.57	&3.31	&2.11\\
P\textsubscript{bi}		&4.51	&3.31 	&7.34\\
P\textsubscript{bulk} 		&4.43	&3.32	&10.47\\
Ref.\cite{doi:10.1063/1.438523}&4.37& 3.31 	&10.47 \\
\hline
As\textsubscript{mono}	&4.70	&3.67	&2.39\\
As\textsubscript{bi}		&4.64	&3.69	&7.86\\
As\textsubscript{bulk} 	&4.56	&3.71	&10.94\\
Ref~\cite{doi:10.1080/14786437508229285}&4.47&3.65&11.0\\
\hline
Sb\textsubscript{mono}	&5.02	&4.23	&2.79\\
Sb\textsubscript{bi}		&4.88	&4.26	&8.83&\\
Sb\textsubscript{bulk} 	&4.73	&4.29	&2.09 \\
Ref~\cite{Barrett:a03833}	&		&4.3		&11.2\\
\hline\hline
\end{tabular}
%}
\caption{
Lattice parameters (\AA)  for monolayer, bilayer and bulk phases of P, As, and Sb 
compared to experimental data~\cite{doi:10.1063/1.438523,doi:10.1080/14786437508229285,Barrett:a03833} quoted in parentheses. 
For the monolayers and bilayers the layer thickness $c^\prime$ is given.
}
\label{table:lattice_consts}
\end{table}

For a given element, 
we find that the lattice parameter `$a$' 
along the puckered direction, 
shortens as the number of layers increases.
%
This is attributed to the increased 
vdW forces between layers 
leading to increased binding primarily 
in the softer puckered direction.

%ISOTROPIC BULK PROPERTIES
\subsection{Isotropic bulk properties}

The computed elements of the stiffness tensor
of each structure 
are presented in Table~\ref{table:cmatrix_data}, 
where those pertaining to bulk P 
compare well to experiments~\cite{doi:10.1143/JPSJ.55.1196,doi:10.1143/JPSJ.60.1612} 
and similarly computed values~\cite{PhysRevB.86.035105,Wang2015}.
%
For the elements related  
to in-plane strains 
$\{c_{11},\ c_{22},\ c_{66},\ c_{12}\}$
we observe the expected increase in stiffness 
as the layer number increases 
and for decreasing atomic number.
%
However, for the other elements 
related to out-of-plane and shear stresses
$\{c_{33},\ c_{44},\ c_{55},\ c_{23},\ c_{13}\}$
the stiffness is seen to, in general, increase 
with the atomic number.


The Hill-averaged bulk properties 
are presented in Table~\ref{table:3dhill_data}, 
which compare well to other DFT values~\cite{PhysRevB.86.035105,Wang2015}, 
while our calculated  
bulk modulus for bulk P (37.2~GPa) 
is also within reasonable range of 
 the experimental values
(32.32~\cite{doi:10.1063/1.438523} - 
36.02~GPa~\cite{doi:10.1080/08957958908201013}).
%
We also observe that the bulk properties 
remain largely comparable for all the species, 
but generally decrease from P to As to Sb 
(except for the Poisson's ratio and bulk modulus, 
which are largest for As).
%
While bulk P has the largest in-plane responses, 
As and Sb have larger out-of-plane 
and shear responses, 
which enable the net isotropic properties 
for all three species to remain comparable overall.


\begin{table}[th!]
\centering
\begin{tabular}{lrrrr}
\hline\hline
&$Y_H$ (GPa) &$G_H$ (GPa)&$\nu_H$&$\mathcal{B}_H$ (GPa)\\
\hline
P\textsubscript{bulk}	&61.1	&24.9	&0.23	&37.2\\
%, 32.32~\footnote{\label{f1:3dhill_data}Ref.~\cite{doi:10.1063/1.438523}}, 36.02~\footnote{\label{f2:3dhill_data}Ref.~\cite{doi:10.1080/08957958908201013}})
As\textsubscript{bulk} &60.0	&23.8	&0.26	&41.4\\
Sb\textsubscript{bulk}&52.5	&21.1	&0.24	&33.8\\
Ref.~\cite{PhysRevB.86.035105}&70.3&29.4&0.30	&38.5\\
\hline\hline
\end{tabular}
\caption{
The Hill-averaged 
Young's modulus $Y_H$,
shear modulus $G_H$ 
and bulk modulus $\mathcal{B}$ in GPa, 
and Poisson's ratio $\nu_H$ 
for bulk P, As and Sb 
compared to similarly calculated DFT values~\cite{PhysRevB.86.035105} 
and available experimental data~\cite{doi:10.1063/1.438523,doi:10.1080/08957958908201013}.
}
\label{table:3dhill_data}
\end{table}

% GRAPHS OF IN-PLANE ELASTIC PROPS
\subsection{In-plane elastic properties}
\label{sec:in_plane_elastic_props}

In Appendix~\ref{sec:vrh_derivation}, 
we re-derive the 
equations for the elastic properties as a 
function of the in-plane orientation angle $\theta$, 
defined in Fig.~\ref{fig:phos_structure}, 
as outlined in Ref.~\cite{jones1975mechanics}.
%
The mechanical properties 
as functions of orientation angle 
are presented 
for P in Fig.~\ref{fig:p_2d_in_plane_figs}, 
As in Fig.~\ref{fig:as_2d_in_plane_figs},
and Sb in Fig.~\ref{fig:sb_2d_in_plane_figs}, 
and  illustrate   
the Young's modulus $Y\left(\theta\right)$ 
and its average $\left<Y\left(\theta\right)\right>$,
the shear modulus $G\left(\theta\right)$,
and Poisson's ratio $\nu\left(\theta\right)$.
%
We summarise this data 
in Table~\ref{table:2d_extrema}.
%
In contrast to the isotropic averages for 
the bulk properties in Table~\ref{table:3dhill_data}, 
which remained largely similar for P, As and Sb, 
a much clearer trend across the species emerges 
once we have eliminated contributions from 
the out-of-plane and shear stresses.
%
In this instance, P clearly possesses superior 
in-plane mechanical strength in both moduli, 
which expectedly increase with increasing layers.
%
Meanwhile, As and Sb are largely similar in the monolayer, 
though less so in the bilayer and bulk phases, 
where they are stronger in the $\vec{x}$-direction.
%
In contrast,  
the Poisson's ratio tends to remain relatively consistent 
aside from generally decreasing with increasing 
number of layers.

%PHOSPHORUS FIGS
\begin{figure}[th!]
\begin{subfloat}[Monolayer P]{
\includegraphics[height=0.494\textwidth]{images/Pmono1.pdf}
  \label{fig:pmono1}}
\end{subfloat}
%
\begin{subfloat}[Bilayer P]{
\includegraphics[height=0.494\textwidth]{images/Pbi1.pdf}
  \label{fig:pbi1}}
\end{subfloat}
%
\begin{subfloat}[Bulk P]{
\includegraphics[height=0.494\textwidth]{images/Pbulk1.pdf}
  \label{fig:pbulk1}}
\end{subfloat}
%
\caption[Phosphorus in-plane mechanical properties]{
In-plane functions for P  
Young's modulus $Y\left(\theta\right)$ (blue)
and its isotropic average $\left<Y\left(\theta\right)\right>$ 
(red) in units of GPa; 
shear modulus $G\left(\theta\right)$ in GPa (red);
and Poisson's ratio $\nu\left(\theta\right)$ 
(scaled by 100).
}
\label{fig:p_2d_in_plane_figs}
\end{figure}


The anisotropy of the underlying crystal structure 
is clearly reflected in the mechanical profiles 
where the Young's modulus  
appears to have a 2-fold symmetry 
about the $x$-axis, 
in contrast to the 
shear modulus and Poisson's ratios, 
which display 4-fold symmetry about both the axes 
(except for the Poisson's ratio of P, 
which remains 2-fold symmetric).
%
Indeed, for a given species, 
the general shape of each profile 
appears to remain relatively consistent 
with respect to the number of layers, 
while the range of each 
quantity tends to increase.
%
This insight is advantageous 
in the strain-engineering of nano-flakes 
since one may forecast in advance the 
response of a material to in-plane strain, 
once the underlying profile 
and number of layers are known.


%ARSENIC FIGS
\begin{figure}[th!]
\begin{subfloat}[Monolayer As]{
\includegraphics[height=0.494\textwidth]{images/Asmono1.pdf}
  \label{fig:asmono1}}
\end{subfloat}
%
\begin{subfloat}[Bilayer As]{
\includegraphics[height=0.494\textwidth]{images/Asbi1.pdf}
  \label{fig:asbi1}}
\end{subfloat}
%
\begin{subfloat}[Bulk As]{
\includegraphics[height=0.494\textwidth]{images/Asbulk1.pdf}
  \label{fig:asbulk1}}
\end{subfloat}
%
\caption[Arsenic in-plane mechanical properties]{
In-plane functions for As 
Young's modulus $Y\left(\theta\right)$ (blue)
and its isotropic average $\left<Y\left(\theta\right)\right>$ 
(red) in units of GPa; 
shear modulus $G\left(\theta\right)$ in GPa (red);
and Poisson's ratio $\nu\left(\theta\right)$ 
(scaled by 50).
}
\label{fig:as_2d_in_plane_figs}
\end{figure}


Another interesting feature is that 
the extrema of the elastic functions 
do not necessarily coincide 
with the coordinate-axes.
%
For instance the Young's modulus 
for monolayer Sb is maximal at $22^{\degree}$.
%
Table~\ref{table:2d_extrema} summarises   
the global minima and maxima of each function 
and the angles at which they occur.
%
While most of the function extrema occur 
expectedly at $0^{\degree}$, $45^{\degree}$ or $90^{\degree}$, 
many are incident away from the coordinate-axes.
%
This result lends further insight into  
the mechanical anisotropy of the 
orthorhombic group-$V$ materials 
of the Cmca space group and D$_{2h}$ point group.
%

%The emergence of a negative Young's modulus 
%of $-2.1$~GPa in monolayer As (Fig.~\ref{fig:asmono1}), 
%at first glance, may give cause for concern.
%%
%It arises due to a negative Voigt estimate for 
%the Young's modulus at $90^{\degree}$ 
%where  
%$(c_{22}^2-c_{12}^2)/c_{22} = -8.9$~GPa 
%(since $c_{12}>c_{22}$), 
%which is larger in absolute magnitude 
%than the Reuss estimate at $90^{\degree}$ 
%given by $1/s_{22}=4.7$~GPa  
%and results in a net negative Hill-average.
%%
%In this instance, we surmise that 
%either the assumptions of the 
%Voigt model break down, 
%or the Hill-method
%is not universally appropriate 
%in arbitrary directions in-plane 
%and a more robust averaging 
%scheme must be employed.
%%
%Nevertheless, 
%the qualitative in-plane 
%functions and their isotropic averages 
%remain physically meaningful 
%and in general can provide valuable physical insight.


%ANTIMONY FIGS
\begin{figure}[th!]
\begin{subfloat}[Monolayer Sb]{
\includegraphics[height=0.494\textwidth]{images/Sbmono1.pdf}
  \label{fig:sbmono1}}
\end{subfloat}
%
\begin{subfloat}[Bilayer Sb]{
\includegraphics[height=0.494\textwidth]{images/Sbbi1.pdf}
  \label{fig:sbbi1}}
\end{subfloat}
%
\begin{subfloat}[Bulk Sb]{
\includegraphics[height=0.494\textwidth]{images/Sbbulk1.pdf}
  \label{fig:sbbulk1}}
\end{subfloat}
%
\caption[Antimony in-plane mechanical properties]{
In-plane functions for Sb 
Young's modulus $Y\left(\theta\right)$ (blue)
and its isotropic average $\left<Y\left(\theta\right)\right>$ 
(red) in units of GPa; 
shear modulus $G\left(\theta\right)$ in GPa (red);
and Poisson's ratio $\nu\left(\theta\right)$ 
(scaled by 50).
}
\label{fig:sb_2d_in_plane_figs}
\end{figure}


\clearpage
%COMPARISON TO EXP
\subsection{Polymer-nano-flake composite}
\label{sec:comparison_to_exp}
%
In our paper in Ref.~\cite{PMID:26469634}, 
we demonstrated that the 
controlled production of large quantities 
of few-layer BP can be achieved by 
liquid-phase exfoliation (LPE)~\cite{Nicolosi1226419}, 
in which the BP fragments are immersed in 
a solution of $N$-cyclohexyl-2- pyrrolidone (CHP), 
and separated into layers 
by sonication, or shearing.
%
The particular choice of CHP 
as a solution is two-fold: 
to lubricate the exfoliation process, 
and to guard the produced samples from oxidation~\cite{Li2014,Zant2014,doi:10.1021/nl5032293}, 
a problem that has been reported  
to hinder the large-scale production of BP. 
%
The advantage of sonication is 
not only the effective mass-production of samples, 
but the means to control the size-dispersion 
of the produced stock, 
which varied from 50~nm to 2500~nm.

Our study focused in particular on two batches:
one containing small samples 
with mean-length $L_s=130\pm111$~nm 
and one containing large samples 
with mean-length $L_l=2260\pm1000$~nm.
% 
A monolayer of BP was measured 
to be $2.06\pm0.02$~nm thick, 
which is a factor of 4 greater than the 
monolayer thickness estimated from 
the bulk value in Table~\ref{table:lattice_consts}, 
however this was acknowledged to be overestimated.
%
The samples were then determined to be 
8 layers thick on average, 
which corresponds to a bulk calculation
for our purposes.
%
The nano-flakes were dispersed in PVC, 
up to 0.3\% volume, 
and subject to mechanical stress to 
determine the composite PVC:BP 
Young's modulus.
%
The pristine PVC Young's modulus 
was measured to be $Y_m=511\pm118$~MPa.
%
The experimental data in this section was 
provided by the experiments 
conducted by our colleagues, 
Damien Hanlon, 
Conor Boland, and Johnny Coleman 
and may be found in Table~\ref{table:bp_exp_data}.

Since the ROM models 
in Eq.~\eqref{eq:rom_models} make
no distinction between flake sizes, 
they may be fit to all the data simultaneously, 
as shown in Fig.~\ref{fig:rom_plot}.
%
The upper-bound BP Young's modulus, 
given by the Voigt ROM model, 
is determined as  
$Y_f^V=184\pm 38$~GPa, 
and is in excellent agreement 
with the calculated DFT values.
%
In fact, the Young's modulus 
%lies precisely between the average in-plane 
%bulk modulus 
%$\langle Y_\textrm{bulk}\rangle=85.7$~GPa 
%(Table~\ref{table:cmatrix_data})
coincides precisely with the elastic stiffness in 
the zigzag direction $c_{11}=187.9$~GPa 
(Table~\ref{table:2d_extrema}), 
and, thus, supports our earlier assumption 
that the flakes are naturally aligned in-plane.
%
On the other hand, 
the Reuss model 
%returns a much lower result 
%on the order of the PVC modulus 
%$Y_f^R=0.537\pm 41$~GPa, 
%and 
cannot be feasibly extrapolated 
to a Young's modulus at 100\% volume fraction
since the fit quickly diverges 
beyond the limited range of the data.
%
Nonetheless, 
%the result trivially represents 
%the absolute lower-bound 
%of the BP Young's modulus 
%to be that of 
the PVC Young's modulus was determined 
by both models to be 
$Y_m^V=541\pm 44$~MPa, 
as expected.

\begin{figure}[th!]
\centering
\includegraphics[height=0.494\textwidth]{images/rom_plot.pdf}
\caption[PVC:BP Young's modulus fit to rule-of-mixtures model]{
Young's modulus of PVC:BP composite (MPa) 
against BP nano-flake volume fraction (\%)
fit to the Voigt (blue) and Reuss (red) ROM models.
}
\label{fig:rom_plot}
\end{figure}


The HT model, meanwhile, 
must accommodate the flake sizes 
according to the parameter $\zeta=2l/t$, 
which, for small flakes, is $\zeta_S=15.8$ 
and for large flakes is $\zeta_L=274.3$.
%
The models were then fit to the corresponding 
data where the matrix Young' modulus $Y_m=511$~MPa is kept fixed, 
and are shown by the solid lines in Fig.~\ref{fig:ht_plot}, 
keeping all other variables fixed.
%
The resulting Young's modulus
for the small flakes is 
$Y_S=117\pm71$~GPa, 
and for the large flakes is 
$Y_L=215\pm28$~GPa, 
which are very reasonable indeed 
and closely resemble the values we 
obtained in the VRH model.

\begin{figure}[th!]
\centering
\includegraphics[height=0.494\textwidth]{images/ht_plot.pdf}
\caption[PVC:BP Young's modulus fit to Halpin-Tsai model]{
Young's modulus of PVC:BP composite (MPa) 
against BP nano-flake volume fraction (\%) 
for large (orange) and small (green) flakes.
%
Halpin-Tsai models are shown for the 
large (blue) and small (red) flakes, 
where the parameters in parentheses indicate 
which variables are optimised in the models.
%
The solid line indicates the model where just $Y_f$ is optimised 
while the dashed lines indicate where $Y_f$ and $Y_m$ are optimised. 
%
}
\label{fig:ht_plot}
\end{figure}

Furthermore, 
if we allow the model 
to be simultaneously optimised according to 
$Y_f$ and $Y_m$ 
shown by the dashed lines in Fig.~\ref{fig:ht_plot}, 
we can find the best fit for $Y_m$ as well.
%
In this instance, 
the model parameters for the small flakes 
were calculated as 
$Y_S=140\pm148$~GPa 
and   
$Y_m=494\pm84$~MPa.
%
While $Y_S$ has a large associated uncertainty, 
$Y_m$ is at least close enough to the margin of 
experimental uncertainty.
%
Meanwhile, 
the model parameters for the large flakes, 
given by  
$Y_L=167\pm26$~GPa 
and 
$Y_m=622\pm41$~MPa
are slightly larger but have smaller uncertainties 
and are also broadly in line 
with all other results up to now.
%
%Remarkably, the large value for $\zeta_L$ 
%is also not too concerning 
%as it is an empirical parameter 
%and, thus, has no strict grounding 
%in the particular geometry 
%of the BP nano-flakes.
%%
%Moreover, its large value coincides 
%with the limiting regime, 
%i.e. 
%where $\zeta\to\infty$ 
%where the HT model is 
%equivalent to a Voigt ROM, 
%and, thus, supports the agreement 
%of the latter with our DFT calculations.

A summary of the model results 
is presented in Table~\ref{table:model_results}, 
where it is clear that the 
Voigt ROM and HT models 
are in very reasonable agreement
with experimental and theoretical data.
%
Moreover, 
with the various models we can estimate  
the Young's modulus of BP to be 
approximately between 
120~GPa and 200~GPa.
%
\begin{table}[th!]
\centering
%\resizebox{\textwidth}{!}
\caption{
ROM and HT model results 
for the Young's moduli of BP (GPa) and PVC (MPa), 
and $\zeta$ parameter.
%
Variables in parentheses 
are not optimised by the model.
}
\label{table:model_results}
\end{table}


\newpage
\section{Conclusion}
\label{sec:2d_conclusion1}
%
In this Chapter, 
we have quantified the qualitative mechanical properties 
of puckered 
phosphorus, arsenic and antimony, 
in the few-layer and bulk phases, 
from comprehensive 
DFT calculations.
%
We have found that these materials exhibit 
a broad range of elastic properties 
and in-plane behaviour 
that are reflective of the underlying 
anisotropic crystal structure, 
which potentially provide for widespread 
technological application.
%
Moreover, 
the calculated mechanical properties, 
from lattice constants to Young's moduli, 
are in excellent agreement  
with experimental observations.
%
We have provided further confirmation 
of these values  
from the experimental measurements 
of a PVC:BP composite material
and appropriate mixture models, 
which place the Young's modulus 
between 120~GPa and 200~GPa.

This work, 
detailing the macroscopic physical 
behaviour of 
mono-crystalline, alloyed,
or composite structures, 
is a typical example of 
the utility DFT provides.
%
While the predictive capability of DFT 
is largely dependent on 
the system under consideration, 
the quality of the pseudopotential 
and, crucially, 
the choice of exchange-correlation functional, 
quantities such as lattice constants 
and mechanical properties are generally 
well reproduced~\cite{MUSIC2016178}.
%
However, 
DFT suffers from inherent, systematic failures, which, 
as we shall discuss in the next Chapter, 
can lead to qualitative miscalculation 
of electronic properties, 
in particular the band gap.
%


