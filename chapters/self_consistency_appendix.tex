
%ALERNATIVE JOUT DERIVATION
\section{Alternative derivation for $U_\textrm{out}$ and $J_\textrm{out}$}

In this section we present an alternative derivation for $J_\textrm{out}$
when used in the original DFT+$U$+$J$ functional,
given by
%
\begin{equation}
E_{U+J}
=\sum_{I\sigma}\left\{\frac{(U^I-J^I)}{2}\mbox{Tr}\left[\hat{n}^{I\sigma}-\hat{n}^{I\sigma}\hat{n}^{I\sigma}\right]
+\frac{J^I}{2}\textrm{Tr}\left[\hat{n}^{I\sigma}\hat{n}^{I\bar\sigma}\right]\right\},
\end{equation}
%
in which the potential for $U$ remains the same, 
but $J$ now applies to both like and unlike-spins.
%
The potential for $J$ in this instance is then
%
\begin{equation}
\hat{V}^\sigma_{J}
=\frac{\delta E_{J}}{\delta \hat{n}^{I\sigma}}
=J^I\left(\hat{n}^{I\sigma}+ \hat{n}^{I\bar\sigma}-\frac{1}{2}\right)
\quad\Rightarrow\quad
{V}^\sigma_{J}=J\left(N-\frac{1}{2}\right),
\end{equation}
%
which we see is symmetric for both 
spin up and spin down potentials.
%
From this, we follow the same procedure as 
in Eqs.~\eqref{eq:u_self_consistent_interaction2} ~\&~\eqref{eq:j_self_consistent_interaction2} 
and find that 
%
\begin{align}
U_\textrm{out}&
=\frac{1}{2}\left(\frac{dV_\textrm{int}^\uparrow}{dN}+\frac{dV_\textrm{int}^\downarrow}{dN}
+\frac{dV_{U_\textrm{in}}^\uparrow}{dN}+\frac{dV_{U_\textrm{in}}^\downarrow}{dN}
+\frac{dV_{J_\textrm{in}}^\uparrow}{dN}+\frac{dV_{J_\textrm{in}}^\downarrow}{dN}\right)
\nonumber \\[10pt]
&=\frac{1}{2}\left(\frac{dV_\textrm{int}^\uparrow}{dN}+\frac{dV_\textrm{int}^\downarrow}{dN}\right)
-\frac{U_\textrm{in}}{2}\left(\frac{dN^\uparrow}{dN}+\frac{dN^\downarrow}{dN}\right)
+\frac{J_\textrm{in}}{2}\left(\frac{dN}{dN}+\frac{dN}{dN}\right)
\nonumber \\[10pt]
&=F_\textrm{Hxc}^{\hat{P}}(U_\textrm{in},J_\textrm{in})-\frac{U_\textrm{in}}{2}+J_\textrm{in}. 
\label{eq:u_self_consistent_interaction3}
\end{align}
%
Where $U_\textrm{out}$ now depends on $J_\textrm{in}$ 
instead of $J_\textrm{in}/2$.
%
Furthermore, 
we find with surprise that $J_\textrm{out}$ 
does not depend on $J_\textrm{in}$ at all 
since the contributions from the symmetric $J$ potentials now cancel:
%
\begin{align}
J_\textrm{out}&
=-\frac{1}{2}\left(\frac{dV_\textrm{int}^\uparrow}{dM}-\frac{dV_\textrm{int}^\downarrow}{dM}
+\frac{dV_{U_\textrm{in}}^\uparrow}{dM}-\frac{dV_{U_\textrm{in}}^\downarrow}{dM}
+\frac{dV_{J_\textrm{in}}^\uparrow}{dM}-\frac{dV_{J_\textrm{in}}^\downarrow}{dM}\right)
\nonumber \\[10pt]
&=-\frac{1}{2}\left(\frac{dV_\textrm{int}^\uparrow}{dM}-\frac{dV_\textrm{int}^\downarrow}{dM}\right)
+\frac{U_\textrm{in}}{2}\left(\frac{dN^\uparrow}{dM}-\frac{dN^\downarrow}{dM}\right)
+\frac{J_\textrm{in}}{2}\left(\frac{dN}{dM}-\frac{dN}{dM}\right)
\nonumber \\[10pt]
&=F_J^{\hat{P}}(U_\textrm{in},J_\textrm{in})+\frac{U_\textrm{in}}{2}. 
\label{eq:j_self_consistent_interaction3}
\end{align}

The linear combination 
of $U_\textrm{out}$ and $J_\textrm{out}$ 
results in the expression for $U_\textrm{eff}$ 
now depending explicitly on $J_\textrm{in}$, 
such that 
%
\begin{align}
U_\textrm{eff}&=U_\textrm{out}-J_\textrm{out} = \left(F_\textrm{Hxc}-F_J\right)-U_\textrm{in}+J_\textrm{in}\nonumber \\
J_\textrm{eff}&=U_\textrm{out}+J_\textrm{out} = \left(F_\textrm{Hxc}+F_J\right)+J_\textrm{in}.
\label{eq:uoutjout2}
\end{align}
%
However, 
invoking the self-consistency condition 
that completely removes the effective interactions, 
i.e., $U_\textrm{eff}=J_\textrm{eff}=0$, gives 
%
\begin{align}
U_\textrm{in}-J_\textrm{in} &= \left(F_\textrm{Hxc}-F_J\right)=\frac{1}{2}\left(F^{\uparrow\uparrow}+F^{\downarrow\downarrow}\right)\nonumber \\[10pt]
J_\textrm{in} &= -\left(F_\textrm{Hxc}+F_J\right)=-\frac{1}{2}\left(F^{\uparrow\downarrow}+F^{\downarrow\uparrow}\right).
\label{eq:ueffjeff2}
\end{align}
%
Here, the formula for $J$ is unchanged, 
whereas $U$ is now scaled by $J$, 
thereby producing an `effective' $U_\textrm{in}$ 
coupling same-spin interactions, 
which is exactly the formulation 
given in Eqs.~\eqref{eq:ueff}~\&~\eqref{eq:jeff}. 
%
Thus, the two derivations are equivalent 
and produce the same effect 
with this condition when provided 
with the corresponding $U$ and $J$ parameters.
\newpage


%SUPPLEMENTAL DOS PLOTS FOR NIO
\section{Supplemental DOS plots for NiO}


% Not great - band gap too low
\begin{figure}[th!]
\centering
\includegraphics[height=0.45\textwidth]{images/NiO.4.1.J.dos.pdf}
\caption[NiO DOS with $U=4.1$~eV and $J=2.82$]
{Although the PBE+$U^{(1)}$ calculation performed well, 
the addition of a relatively large $J$ here decreases 
the band gap. 
However, the magnetic moment is reasonable.}
\label{fig:NiO.4.1.J.dos}
\end{figure}

% Second worst result in the batch - U and J too high
\begin{figure}[th!]
\centering
\includegraphics[height=0.45\textwidth]{images/NiO.10.5.J.dos.pdf}
\caption[NiO DOS with $U=10.5$~eV and $J=5.9$]
{This calculation produces terrible results as the large $J$ pulls 
the conduction band peak below the Fermi energy 
and leaves behind depleted conduction states.}
\label{fig:NiO.10.5.J.dos}
\end{figure}
\newpage

% Pretty good - builds on lucky success of U1
\begin{figure}[th!]
\centering
\includegraphics[height=0.45\textwidth]{images/NiO.4.1.f.dos.pdf}
\caption[NiO DOS with $U=4.1$~eV and $F_J=0.78$]
{This calculation produces excellent agreement 
with the experimental measurements in this case,
 but it is generally not the scheme that is mathematically 
 proven to remove the SIE.}
\label{fig:NiO.4.1.f.dos}
\end{figure}



% Pretty good for gap and magmom but pushes peak up way too much
\begin{figure}[th!]
\centering
\includegraphics[height=0.45\textwidth]{images/NiO.10.5.f.dos.pdf}
\caption[NiO DOS with $U=10.5$~eV and $F_J=0.69$]
{This calculation, while producing the correct band gap 
and magnetic moment, has pushed the conduction band 
peak up too high to be in agreement with experiment.}
\label{fig:NiO.10.5.f.dos}
\end{figure}
\newpage

% Same as U3 + J3 above
%\subsubsection{PBE+$U^{(3)}$+$F_J^{(3)}$}
%\begin{figure}[th!]
%\centering
%\includegraphics[height=0.494\textwidth]{images/NiO.6.7.f.dos.pdf}
%\caption{U=6.7, fJ=0.84}
%\end{figure}


% Not great at all - U is too small 
\begin{figure}[th!]
\centering
\includegraphics[height=0.45\textwidth]{images/NiO.1.3.dos.pdf}
\caption[NiO DOS with $U-J=1.3$~eV]
{This calculation under-estimates all three properties 
and does not offer much improvement over the PBE 
as the effective $U$ value is much too small.}
\label{fig:NiO.1.3.dos}
\end{figure}


% Fairly good - U value is reasonable but mixes terms, i.e. Uin-Jout
\begin{figure}[th!]
\centering
\includegraphics[height=0.45\textwidth]{images/NiO.4.6.dos.pdf}
\caption[NiO DOS with $U-J=4.6$~eV]
{Similar to the PBE+$U^{(2)}$ calculation, 
the results produced here are in very good  
agreement with experiment as the $U$ value 
is approximately the correct magnitude.}
\label{fig:NiO.4.6.dos}
\end{figure}
\newpage

% Pretty much the same as Ueff2 = 5.2 
\begin{figure}[th!]
\centering
\includegraphics[height=0.45\textwidth]{images/NiO.5.9.dos.pdf}
\caption[NiO DOS with $U-J=5.9$~eV]
{This calculation produces the best overall agreement 
with the experimental data but is certainly within 
the margin of error achieved by the 
PBE+$U_\textrm{eff}^{(2)}$ calculation discussed earlier.}
\label{fig:NiO.5.9.dos}
\end{figure}


% Same as (U-J)^3 above
%\subsubsection{PBE+$U_\textrm{eff}^{(3)}$}
%
%\begin{figure}[th!]
%\centering
%\includegraphics[height=0.494\textwidth]{images/NiO.5.9.dos.pdf}
%\caption{Ueff=5.9}
%\end{figure}


% Third worst result result - J is too high
\begin{figure}[th!]
\centering
\includegraphics[height=0.45\textwidth]{images/NiO.5.2.J.dos.pdf}
\caption[NiO DOS with $U_\textrm{eff}=5.2$ and $J=3.48$~eV]
{While producing a magnetic moment exactly between 
the experimental measurements, 
the band gap and transition energy computed here 
are grossly incorrect.}
\label{fig:NiO.5.2.J.dos}
\end{figure}
\newpage

% Also ok - builds on Ueff2 nicely with J
\begin{figure}[th!]
\centering
\includegraphics[height=0.45\textwidth]{images/NiO.5.2.f.dos.pdf}
\caption[NiO DOS with $U_\textrm{eff}=5.2$ and $F_J=0.84$~eV]
{This calculation produces reasonable 
agreement with experiment but it is not the most effective.}
\label{fig:NiO.5.2.f.dos}
\end{figure}


% Bonkers result altogether - worst of the bunch
\begin{figure}[th!]
\centering
\includegraphics[height=0.45\textwidth]{images/NiO.10.5.J.minority.dos.pdf}
\caption[NiO DOS with $U=10.5$ and $J=5.94$~eV including spin minority term]
{This calculation produces the worst results 
by far, where the spin-symmetry of the system 
is broken rendering a ferromagnetic system.
%
The large $J$ here exacerbates 
the spin-minority term invoked from Ref.~\cite{PhysRevB.84.115108}.}
\label{fig:NiO.10.5.J.minority}
\end{figure}
\newpage


%SELF-CONSISTENCY FOR NI(CO)4
\subsection{Self-consistency calculations for Ni(CO)$_4$}

Here we present the 
$U_\textrm{in}$ vs $U_\textrm{out}$ 
curve computed for Ni(CO)$_4$ 
with non-spin-polarised calculations, 
such that the corresponding 
$J_\textrm{out}$ could not be determined.
%
We observe that 
$U^{(2)}\approx U^{(3)}\approx 11$~eV, 
and in fact, 
the HOMO-LUMO gap computed with  
PBE+$U^{(2)}$ is also 4.56~eV, 
while for PBE+$U^{(1)}$ it is 4.22~eV.
%
Hence there is little change 
in the frontier orbitals of Ni(CO)$_4$ 
upon application of relatively 
large $U$ corrections.
%
Nonetheless, 
the gap is in better agreement 
with the experimental value~\cite{doi:10.1021/ja00274a073} 
of $4.83$~eV.

\begin{figure}[th!]
\centering
\includegraphics[height=0.494\textwidth]{images/nico4_uin_vs_uout.pdf}
\caption[Self-consistency profile for Ni(CO)$_4$]
{Non-spin-polarised $U_\textrm{in}$ vs $U_\textrm{out}$ 
profile for Ni(CO)$_4$ with error bars.
%
From this we compute 
$U^{(1)}=5.43\pm0.01$~eV,
$U^{(2)}=10.94\pm0.03$~eV, 
and $U^{(3)}=10.80\pm0.01$~eV, 
for which we see $U^{(2)}\approx U^{(3)}$.}
\label{fig:nico4_uin_vs_uout}
\end{figure}
\newpage

%SELF-CONSISTENCY FOR Cr2O3
\subsection{Self-consistency calculations for Cr$_2$O$_3$}

Here we present the computed curves, 
with accompanied error bars, for 
$U_\textrm{out}$ (solid), 
$J_\textrm{out}$ (dashed), 
and  $U_\textrm{eff}$ (dotted) 
with respect to applied $U_\textrm{in}$ 
for Cr$_2$O$_3$, 
which are again very accurately determined.

\begin{figure}[th!]
\centering
\includegraphics[height=0.494\textwidth]{images/cr2o3_uin_vs_uout.pdf}
\caption[Self-consistency profile for Cr$_2$O$_3$]
{Curves, with accompanied error bars, 
for $U_\textrm{out}$ (solid), 
$J_\textrm{out}$ (dashed), 
and $U_\textrm{eff}$ (dot-dashed), 
with respect to the applied $U_\textrm{in}$ 
for Cr$_2$O$_3$. 
%
From the linear fits we compute 
$U^{(1)}=1.86\pm0.02$~eV,
$U^{(2)}=5.3\pm0.1$~eV, 
and $U^{(3)}=2.87\pm0.02$~eV.}
\label{fig:cr2o3_uin_vs_uout}
\end{figure}

Likewise, 
the evaluated interaction terms 
are presented in Fig.~\ref{fig:cr2o3_interactions}.
%
Here, 
$F^{\sigma\bar{\sigma}}$ varies between 3.5~eV to 2.5~eV, 
and $F^{\sigma\sigma}$ varies between 2.5~eV to 2.0~eV, 
which hints toward desired values of $U$.
%
The average on-site interaction $F_\textrm{Hxc}$ 
ranges from 3~eV to 2.5~eV, 
while $F_J$ remains approximately $0.5$~eV.

Since we discussed earlier that calculations 
are relatively unchanged when $J<1$~eV, 
we assume here that there will be very little difference between 
the DFT calculation evaluated with 
$U^{(2)}_\textrm{eff}=2.35\pm0.04$~eV, 
and that of a DFT+$U$+$J$ calculation 
using Eq.~\eqref{eq:dft+u+j_functional}
with $U^{(3)}=2.76$~eV and $J^{(3)}=0.435\pm0.004$~eV, 
since $(U-J)^{(3)}=2.32\approx U^{(2)}_\textrm{eff}$.
% 
We therefore refrain from 
providing excessive plots 
and instead refer the reader to 
the DOS of the latter presented previously in  
in Fig.~\ref{fig:cr2o3.2.8.J.dos}.

\clearpage
\begin{figure}[th!]
\centering
\includegraphics[height=0.494\textwidth]{images/cr2o3_interactions.pdf}
\caption[Profiles of $F_\textrm{Hxc}$, $F_J$,  $F^{\sigma\sigma}$
and  $F^{\sigma\bar{\sigma}}$ vs $U_\textrm{in}$ for Cr$_2$O$_3$]
{Curves, with accompanied error bars, 
for the interaction kernels 
$F_\textrm{Hxc}$ (circles) and $F_J$ (squares), 
as well as the average like and unlike-spin interactions 
$F^{\sigma\sigma}$ (up triangles) and $F^{\sigma\bar{\sigma}}$ (down triangles), 
with respect to the applied $U_\textrm{in}$ for Cr$_2$O$_3$.}
\label{fig:cr2o3_interactions}
\end{figure}




%\begin{figure}[th!]
%\centering
%\includegraphics[height=0.494\textwidth]{images/cr2o3.2.8.J.dos.pdf}
%\caption[Species-resolved DOS for Cr$_2$O$_3$ calculated 
%with $U=2.8$~eV and $J=0.44$~eV]
%{As in Fig.~\ref{fig:cr2o3.2.8.dos}. 
%The species-resolved DOS of Cr$_2$O$_3$, 
%calculated with PBE+$U=2.8$+$J=0.44$.}
%\label{fig:cr2o3.2.8.J.dos}
%\end{figure}
